\documentclass[11pt]{article}

\usepackage[a4paper,margin=1in]{geometry}
\usepackage{amsmath,amssymb}
\usepackage{graphicx}
\usepackage{hyperref}
\usepackage{booktabs}
\usepackage{physics}
\usepackage{caption}
\usepackage{subcaption}

\graphicspath{{../outputs/figures/}}

\hypersetup{
  colorlinks=true,
  linkcolor=blue,
  citecolor=blue,
  urlcolor=blue
}

\newcommand{\OA}{\textit{Origin Axiom}}
\newcommand{\epsfloor}{\varepsilon}


\title{Origin Axiom — Phase 3 (Mechanism): Non-cancelling Vacuum Toy Model}
\author{Origin Axiom Collaboration}
\date{}

\begin{document}
\maketitle

\begin{abstract}
This Phase~3 paper implements and tests a concrete toy mechanism for
a non-cancellation floor on a global amplitude observable.  We work
with a finite ensemble of complex modes
\(
  z_k(\theta) = \exp\!\bigl[i(\alpha_k + \sigma_k \theta)\bigr]
\)
with fixed phase offsets $\alpha_k$ and winding numbers
$\sigma_k \in \{1,2,3,4\}$, and define the baseline amplitude
\(
  A_0(\theta) = \bigl|\frac{1}{N}\sum_k z_k(\theta)\bigr|.
\)
On top of this ensemble we impose a simple floor,
\(
  A(\theta) = \max\bigl(A_0(\theta),\varepsilon\bigr),
\)
and study when this modification is both numerically well-behaved and
dynamically non-trivial.

The main contributions at this rung are: (i) an explicit, fully
specified toy configuration and $\theta$-scan that realise a genuine
\emph{binding regime}, in which the floor $\varepsilon$ is active on a
strictly between-zero-and-one fraction of the sampled grid; (ii)
numerical demonstrations that this binding regime produces measurable
global effects on the amplitude distribution (e.g.\ shifts in the
mean and a non-zero $L^2$ distance between $A_0$ and $A$); and (iii) a
set of reproducible diagnostics, including a separate measurement of
how often the ensemble approaches very small amplitudes and a
lightweight ``instability penalty'' built from these tails.

We make no claim that this toy mechanism corresponds to a physical vacuum, that the chosen floor $\varepsilon$ has any fundamental meaning, or that any particular value of $\theta$ is distinguished by Nature.  No corridor narrowing or parameter reduction is attempted at this rung.  The goal is strictly limited: to show, in a transparent and reproducible way, that a non-cancellation floor can be enforced in a controlled toy ensemble without collapsing into the pathologies of an always-binding or never-binding modification, and to prepare the ground for more physically motivated mechanisms in future rungs.
\end{abstract}

\section{Introduction}
\label{sec:introduction}

\subsection{Philosophical and Physical Motivation}
The question ``Why is there something rather than nothing?'' has persisted across philosophy and physics for millennia. Classical physics permits a stable vacuum state of absolute nothingness, yet quantum mechanics reveals pervasive vacuum fluctuations, suggesting a non-zero baseline energy density. This tension between classical stability and quantum non-triviality motivates the Origin Axiom: the universe is structurally forbidden from perfect global cancellation, enforced on complex scalar amplitudes $A(\mathcal{C})$ over the space of all possible configurations $\mathcal{C}$.

Formally, the axiom imposes a strict lower bound on the modulus:
\begin{equation}
|A(\mathcal{C})| > \epsilon > 0,
\label{eq:axiom_bound}
\end{equation}
where $\epsilon$ is a minimal positive floor, derived from quantum gravity considerations (simulated as $\sim 10^{-12}$ in normalized units for computational purposes). This bound prevents $|A| \to 0$, rendering absolute nothingness impossible.

Philosophically, the axiom implies that existence is not contingent but inherent—a structural bias against nothingness that favors complexity and differentiation. Intellectually, it offers a principled alternative to fine-tuning arguments: rather than invoking anthropic selection or multiverses, the axiom posits that the universe must exhibit non-zero structure at all scales, as enforced by the underlying configuration space geometry.

\subsection{The Unifying Phase \texorpdfstring{$\thetastar$}{theta*}}
To anchor the axiom in empirical physics, we introduce a unifying phase $\thetastar$ extracted from Standard Model flavor observables. $\thetastar$ is determined via joint $\chi^2$ minimization fits to the PMNS neutrino mixing matrix and CKM quark mixing matrix, yielding a fiducial value of 3.63 radians with a robust uncertainty band of $[2.18, 5.54]$ radians (detailed in the companion repository \texttt{origin-axiom-theta-star}).

This phase $\thetastar$ serves as a universal bridge between flavor phenomenology and scalar dynamics. Simulations demonstrate that $\thetastar$-dependent effects propagate consistently across scales—from neutrino masses and vacuum energy modulation to cosmological expansion, microstructure stability, baryogenesis, dark matter relics, and quantum gravity-derived parameters—suggesting a deep underlying unity rooted in the Origin Axiom.

\subsection{Unification Roadmap}
We demonstrate the axiom's unifying power through a systematic chain of simulations:

\begin{itemize}
    \item \textbf{Flavor bridge} (Section \ref{sec:flavor}): $\thetastar$-modulated seesaw mechanism reproduces PDG neutrino masses and mixing.
    \item \textbf{Vacuum shift} (Section \ref{sec:vacuum}): Microcavity models reveal $\sim$2.2\% modulation in energy shift $\Delta E(\thetastar)$.
    \item \textbf{Cosmological expansion} (Section \ref{sec:cosmology}): FRW evolution with $\Lambda(\thetastar)$ yields $\sim$1\% acceleration in scale factor growth.
    \item \textbf{Microstructure} (Section \ref{sec:microstructure}): 3D lattices with defects exhibit $\thetastar$-dependent stability.
    \item \textbf{Quantum gravity foundation} (Section \ref{sec:qg_epsilon}): Derivation of $\epsilon(\thetastar)$ from Planck-scale and holographic principles.
    \item \textbf{Standard Model compatibility} (Section \ref{sec:sm_integration}): Integration with electroweak symmetry breaking and gauge coupling modulation.
    \item \textbf{Baryogenesis} (Section \ref{sec:baryogenesis}): $\thetastar$-driven CP asymmetry yielding baryon asymmetry $\eta_B$.
    \item \textbf{Dark matter} (Section \ref{sec:dark_matter}): Defects as scalar candidates with computed relic density $\Omega_{\text{DM}} h^2$.
    \item \textbf{Experimental predictions} (Section \ref{sec:predictions}): $\thetastar$-modulated observables including $\theta_{13}$, CMB $\Delta T/T$, and $H_0$.
    \item \textbf{Synthesis} (Section \ref{sec:synthesis}): Full chain reveals aligned patterns and emergent phenomena.
\end{itemize}

The framework is fully reproducible via open-source code at \url{https://github.com/originaxiom/origin-axiom} and \url{https://github.com/originaxiom/origin-axiom-theta-star}, with git-tagged versions ensuring traceability.

\section{Mechanism design (skeleton)}
\label{sec:mechanism-design}

This section is a placeholder at Rung~1. It records the intended role
of the Phase~3 mechanism without yet committing to a specific model.

\begin{itemize}
  \item \textbf{State space:} a finite collection of degrees of freedom
        representing a toy vacuum configuration (to be specified).
  \item \textbf{Dynamics:} an update map or flow that, in the absence of
        a floor, would admit configurations with arbitrarily small net
        amplitude or residual.
  \item \textbf{Non-cancellation rule:} an explicit constraint that
        enforces $|A| \ge \epsfloor$ for a suitable global observable
        $A$, in the sense of the Phase~0 contract.
  \item \textbf{Observables:} a residual-energy proxy and a
        $\theta$-dependent diagnostic that can be exported as a
        theta-filter artifact.
\end{itemize}

Subsequent rungs will replace this section with a fully specified toy
model, including equations, update rules, and the definition of the
binding regime and binding certificate.

\section{Results: binding regime diagnostics}
\label{sec:phase3-results}

This section summarizes the numerical behavior of the toy vacuum under
the non-cancellation floor. We focus on two closely related
experiments:

\begin{enumerate}
  \item a \emph{baseline scan} of the unconstrained amplitude
        \(A_0(\theta)\) over \(\theta \in [0, 2\pi)\), used to define a
        quantile-based floor \(\epsfloor\); and
  \item a \emph{binding-certificate scan} in which the same grid and
        floor are used to compare \(A_0(\theta)\) with the
        floor-enforced amplitude \(A(\theta) = \max(A_0(\theta),
        \epsfloor)\).
\end{enumerate}

Both experiments work with a deterministic ``baseline\_v1'' vacuum
configuration as defined in Section~\ref{sec:phase3-mech}.

\subsection{Baseline scan and floor selection}

In the baseline scan we evaluate \(A_0(\theta)\) on a uniform grid of
\(N_\theta = 2048\) points covering \([0,2\pi)\). The results are
stored, for reproducibility, in
\begin{center}
  \texttt{phase3/outputs/tables/mech\_baseline\_scan.csv}
\end{center}
with summary diagnostics in
\begin{center}
  \texttt{phase3/outputs/tables/mech\_baseline\_scan\_diagnostics.json}.
\end{center}
From the empirical distribution of \(A_0(\theta)\) we extract a
quantile-based floor
\[
  \epsfloor \equiv Q_{0.25}\bigl(A_0(\theta)\bigr),
\]
i.e., the 25th percentile of the sampled amplitudes. For the baseline
configuration used here, the diagnostics report
\begin{align}
  \epsfloor &\approx 0.0251, \\
  \min A_0  &\approx 0.0092, \\
  \max A_0  &\approx 0.0577,
\end{align}
with a binding fraction \(f_{\mathrm{bind}} = 0.25\), meaning that the
floor is active on exactly one quarter of the sampled \(\theta\)
values.

This choice of \(\epsfloor\) ensures that the toy vacuum is in a
\emph{genuine binding regime}: there are regions where the floor is
active and regions where the unconstrained dynamics are visible, with
neither extreme dominating the grid.

\subsection{Binding-certificate scan and global diagnostics}

Using the same grid and the quantile-based floor, the
binding-certificate scan evaluates both \(A_0(\theta)\) and the
floor-enforced amplitude \(A(\theta)\) on each grid point. The results
are stored in
\begin{center}
  \texttt{phase3/outputs/tables/mech\_binding\_certificate.csv}
\end{center}
with summary diagnostics in
\begin{center}
  \texttt{phase3/outputs/tables/mech\_binding\_certificate\_diagnostics.json}.
\end{center}

Figure~\ref{fig:phase3-binding-profile} shows the profile of
\(A_0(\theta)\) and \(A(\theta)\) across the grid, together with the
floor level \(\epsfloor\). As expected, the floor clips the lower tail
of the amplitude distribution, leaving the upper tail unchanged.

\begin{figure}[t]
  \centering
  \includegraphics[width=\linewidth]{fig1_mech_binding_profile}
  \caption{Baseline binding profile for the Phase~3 toy vacuum. The
  unconstrained amplitude \(A_0(\theta)\) (solid curve) fluctuates
  above and below the quantile-based floor \(\epsfloor\) (dashed
  line). The floor-enforced amplitude \(A(\theta)\) (solid curve with
  plateaus) coincides with \(A_0(\theta)\) whenever
  \(A_0(\theta) \ge \epsfloor\) and is clipped to \(\epsfloor\) where
  \(A_0(\theta) < \epsfloor\).}
  \label{fig:phase3-binding-profile}
\end{figure}

The diagnostics report, for the baseline configuration,
\begin{align}
  \langle A_0 \rangle &\approx 0.0388,
  &
  \langle A \rangle &\approx 0.0407, \\
  \Delta_{\mathrm{mean}} &\equiv
    \langle A \rangle - \langle A_0 \rangle \;>\; 0,
  &
  \Delta_{L^2} &\equiv
    \bigl\| A - A_0 \bigr\|_2 \;>\; 0,
\end{align}
together with the same binding fraction \(f_{\mathrm{bind}} = 0.25\)
and the extrema of \(A_0(\theta)\) reported above. In other words, the
floor has a quantitatively non-trivial global effect: it shifts the
mean amplitude, modifies the variance, and induces a finite
\(L^2\)-distance between \(A\) and \(A_0\), all while preserving a
non-zero region where the unconstrained dynamics remain visible.

This behavior is exactly what is required for a Phase~0-style
binding certificate. The floor is neither a purely formal constraint
nor a numerically invisible perturbation; it acts as a genuine
dynamical ingredient in the toy vacuum, in a regime where its impact
can be summarized by simple, reproducible diagnostics.

\subsection{Auxiliary measure probe (non-binding)}
\label{sec:phase3-measure-probe}

To complement the baseline binding profile, we performed a simple,
non-binding measure probe on the same class of toy ensembles.  The
script \texttt{phase3/src/phase3\_mech/measure\_v1.py} samples many
independent random ensembles of phases and windings, computes the
baseline amplitude \(A_0(\theta)\) on a fixed grid, and records the
empirical distribution of \(A_0\) values across ensembles and
\(\theta\).

In the baseline configuration, the resulting distribution shows a
small but non-zero probability weight near \(A_0 \approx 0\).  For
example, the minimum observed value is of order \(10^{-5}\), while the
1\% quantile sits at \(A_{0,\mathrm{p01}} \approx 0.013\).  The
fractions of samples with \(A_0\) below a few illustrative thresholds
are approximately
\begin{align}
  \mathrm{Pr}[A_0 < 0.005] &\approx 1.6 \times 10^{-3}, \\
  \mathrm{Pr}[A_0 < 0.01 ] &\approx 6.3 \times 10^{-3}, \\
  \mathrm{Pr}[A_0 < 0.02 ] &\approx 2.4 \times 10^{-2}, \\
  \mathrm{Pr}[A_0 < 0.05 ] &\approx 1.5 \times 10^{-1}.
\end{align}
These numbers are purely illustrative and depend on the toy-mechanism
choices, but they make explicit that, even before a floor is enforced,
the detailed cancellation basin near \(A_0 = 0\) occupies only a small
fraction of the ensemble-level measure in this construction.

We emphasise that this probe does \emph{not} define a physical floor,
does \emph{not} alter the Phase~3 binding experiment, and does
\emph{not} introduce any new claims.  Its purpose is limited to making
transparent, in a fully reproducible way, how often the toy ensemble
approaches extremely small amplitudes under the current configuration.

\section{Limitations and outlook}
\label{sec:limitations}

Phase 4 is intentionally narrow in scope. Even once the mappings and
diagnostics are implemented, the phase will not claim:
\begin{itemize}
  \item a full derivation of cosmological parameters;
  \item a proof that the Origin Axiom is realised in nature; or
  \item a unique mechanism for connecting vacuum structure to FRW
        dynamics.
\end{itemize}

Instead, the goal is to provide a clean yes-or-no style test for a
specific question:

\begin{quote}
  Can the Phase 3 global-amplitude mechanism support scale-sane
  FRW-like behaviour, in at least one simple mapping family, without
  producing a degenerate or empty \(\theta\)-corridor?
\end{quote}

If the answer is ``no'' for all tested mapping families, Phase 4 will
record this as a structured negative result, signalling that either
the Phase 3 mechanism or the mapping strategy needs revision before
further unification attempts.


\appendix
\section*{Appendix A: Phase 4 claims table (draft)}
\label{app:phase4_claims_table}

Table~\ref{tab:phase4_claims} summarises the intended Phase 4 claims.
At this rung all entries are draft and non-binding.

\begin{table}[h]
  \centering
  \caption{Draft Phase 4 claims. Binding status will be updated once
  the phase is complete and audited.}
  \label{tab:phase4_claims}
  \begin{tabular}{llp{0.55\textwidth}}
    \toprule
    ID & Binding? & Summary \\
    \midrule
    C4.1 & no & Existence of at least one explicit mapping from the
                 Phase 3 global amplitude or residue into an FRW-like
                 or vacuum-energy-like observable with numerically
                 stable behaviour. \\
    C4.2 & no & Existence of a non-empty, non-trivial \(\theta\)-corridor
                 for at least one such mapping. \\
    C4.3 & no & Structured negative result if all tested mappings yield
                 empty or pathological corridors. \\
    \bottomrule
  \end{tabular}
\end{table}

\section{Reproducibility and gate levels}
\label{sec:phase3-reproducibility}

Phase~3 is implemented as a self-contained, reproducible unit inside
the \texttt{origin-axiom} repository. This appendix records the
filesystem layout, the gate script used to regenerate the canonical
artifact, and the minimal commands needed to reproduce the Phase~3
paper and figures.

\subsection*{Filesystem layout}

The Phase~3 tree is organised as follows:
\begin{itemize}
  \item \texttt{phase3/src/} --- source code for the Phase~3
    mechanism and diagnostics;
  \item \texttt{phase3/paper/} --- LaTeX sources for the Phase~3 paper,
    including \texttt{main.tex}, section stubs, and appendices;
  \item \texttt{phase3/workflow/} --- the Snakemake workflow
    driving the paper build, in
    \texttt{phase3/workflow/Snakefile};
  \item \texttt{phase3/outputs/} --- derived outputs produced by
    Phase~3 runs, including figures and the built paper;
  \item \texttt{phase3/artifacts/} --- canonical Phase~3 artifacts,
    including the versioned PDF used as an external reference.
\end{itemize}

At this rung the primary paper artifact is
\texttt{phase3/artifacts/origin-axiom-phase3.pdf},
with the corresponding build product in
\texttt{phase3/outputs/paper/phase3\_paper.pdf}.  The main figure used
in the text is stored as
\texttt{phase3/outputs/figures/fig1\_mech\_binding\_profile.pdf}.

\subsection*{Gate script and build pipeline}

Phase~3 uses a dedicated gate script
\texttt{scripts/phase3\_gate.sh} to orchestrate the build. The
level--A gate regenerates the Phase~3 paper and canonical artifact via
the Snakemake workflow in \texttt{phase3/workflow/Snakefile}.  From
the repository root, the human-facing entry point is:
\begin{quote}
\texttt{bash scripts/phase3\_gate.sh}
\end{quote}
which, at level~A, performs the following high-level steps:
\begin{enumerate}
  \item assembles the list of section and appendix files under
    \texttt{phase3/paper/};
  \item invokes Snakemake on \texttt{phase3/workflow/Snakefile};
  \item runs \texttt{latexmk} on \texttt{phase3/paper/main.tex} to
    produce \texttt{main.pdf};
  \item copies \texttt{main.pdf} to
    \texttt{phase3/outputs/paper/phase3\_paper.pdf} and
    \texttt{phase3/artifacts/origin-axiom-phase3.pdf}.
\end{enumerate}

The Snakemake rule \texttt{build\_phase3\_paper\_pdf} declares
\texttt{main.tex}, the section and appendix files, and
\texttt{Reference.bib} as its inputs, and produces both the
\texttt{phase3\_paper.pdf} output and the canonical
\texttt{origin-axiom-phase3.pdf} artifact.  This makes the LaTeX
dependencies explicit and allows Snakemake to detect when a rebuild is
necessary.

\subsection*{Assumptions and environment}

The Phase~3 paper build assumes:
\begin{itemize}
  \item a reasonably recent \LaTeX{} distribution (e.g.\ TeX Live~2025
    or similar) providing \texttt{latexmk}, \texttt{pdflatex}, and the
    standard packages used in Phase~0--4;
  \item a POSIX shell environment with \texttt{bash} and
    \texttt{make}-like tooling sufficient to run the gate script;
  \item Python and any libraries required by the Phase~3 mechanism
    code, for the production of figures and diagnostic tables.
\end{itemize}

At the present rung, Phase~3 defines a placeholder bibliography file
\texttt{phase3/paper/Reference.bib}, since the paper does not yet
make external-citation claims.  This file is still part of the
declared inputs for the build, so that the pipeline remains stable
when references are introduced at later rungs.

\subsection*{Reproducibility scope}

The Level--A gate guarantees that:
\begin{itemize}
  \item the Phase~3 paper builds successfully from the tracked
    \texttt{phase3/paper} sources and \texttt{Reference.bib};
  \item the canonical artifact
    \texttt{phase3/artifacts/origin-axiom-phase3.pdf} is
    regenerable from these sources via the Snakemake workflow; and
  \item the figures referenced in the text (in particular the
    binding-profile figure) are present under
    \texttt{phase3/outputs/figures/} and can be regenerated from the
    Phase~3 source code.
\end{itemize}

Higher gate levels, if introduced, would be expected to add explicit
checks on numerical outputs, diagnostic tables, and mechanism
parameters.  At this rung, however, the emphasis remains on ensuring
that the narrative and structural content of the Phase~3 paper is
fully reproducible from the repository state.

\subsection*{Auxiliary measure probe}

In addition to the main binding-profile experiment, Phase~3 includes a
non-binding auxiliary script
\texttt{phase3/src/phase3\_mech/measure\_v1.py} that probes the
empirical distribution of the baseline amplitude \(A_0(\theta)\) for a
large number of independently sampled toy ensembles.  This script does
not affect any of the main Phase~3 claims or floor definitions; it is
included solely to make the measure structure of the current toy
configuration explicit.

A typical invocation is:
\begin{verbatim}
  oa && python phase3/src/phase3_mech/measure_v1.py
\end{verbatim}
which writes:
\begin{itemize}
  \item a JSON diagnostics file
    \texttt{phase3/outputs/tables/phase3\_measure\_v1\_stats.json}
    containing basic summary statistics and quantiles of the
    \(A_0\) distribution; and
  \item a histogram CSV
    \texttt{phase3/outputs/tables/phase3\_measure\_v1\_hist.csv}
    with binned counts over \(A_0\).
\end{itemize}
The console output also prints a small selection of quantiles and
fractions below a few illustrative \(\varepsilon\) thresholds.  All of
these numbers are toy-model diagnostics and should be interpreted as
such; they are not promoted to binding corridor constraints or
physical scales at this rung.


\end{document}
