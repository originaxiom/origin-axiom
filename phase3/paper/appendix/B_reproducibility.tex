\section{Reproducibility and gate structure}
\label{app:phase3mech_repro}

This appendix documents how to reproduce the Phase~3 mechanism results
in a clean checkout of the \textit{Origin Axiom} repository. The
emphasis is on the toy vacuum configuration, the non-cancellation
floor, and the binding diagnostics; no external data or
\(\theta\)-filter artifact is introduced at this stage.

\subsection{Repository layout and environment}

The Phase~3 mechanism lives under
\texttt{phase3/} with a minimal Snakemake workflow and a small Python
package:

\begin{itemize}
  \item \texttt{phase3/src/phase3\_mech/}: toy vacuum mechanism and
        diagnostics scripts.
  \item \texttt{phase3/paper/}: LaTeX sources for the Phase~3
        mechanism paper.
  \item \texttt{phase3/outputs/}: tables and figures generated by the
        Phase~3 gate.
  \item \texttt{phase3/artifacts/}: the canonical Phase~3 PDF
        artifact \texttt{origin-axiom-phase3.pdf}.
  \item \texttt{scripts/phase3\_gate.sh}: convenience wrapper for the
        Phase~3 Snakemake workflow.
\end{itemize}

The numerical experiments are lightweight and require only a standard
Python 3 environment with NumPy and Matplotlib available. The LaTeX
build assumes a reasonably complete \texttt{latexmk}+\texttt{pdflatex}
installation.

\subsection{Gate invocation}

From the repository root, the Phase~3 mechanism gate is invoked as
\begin{verbatim}
bash scripts/phase3_gate.sh --level A
\end{verbatim}
which will:

\begin{itemize}
  \item run the baseline scan and binding-certificate diagnostics for
        the toy vacuum and non-cancellation floor;
  \item write the corresponding tables and diagnostics under
        \texttt{phase3/outputs/tables/}; and
  \item build the Phase~3 mechanism paper at
        \texttt{phase3/outputs/paper/phase3_paper.pdf} and copy it to
        \texttt{phase3/artifacts/origin-axiom-phase3.pdf}.
\end{itemize}

Higher gate levels may be introduced in later rungs if heavier
experiments or additional artifacts are added. At the present rung,
Level~A suffices to regenerate all declared Phase~3 mechanism
artifacts.

\subsection{Key artifacts and run diagnostics}

The following outputs constitute the core numerical artifacts for the
Phase~3 mechanism at this rung:

\begin{itemize}
  \item \texttt{phase3/outputs/tables/mech\_baseline\_scan.csv}: per-grid
        values of \(\theta\), \(A_0(\theta)\), \(A(\theta)\), and the
        binding mask for the quantile-selected floor \(\epsfloor\).
  \item \texttt{phase3/outputs/tables/mech\_baseline\_scan\_diagnostics.json}:
        summary diagnostics for the baseline scan, including
        \(\epsfloor\), \(\min A_0\), \(\max A_0\), the binding fraction
        \(f_{\mathrm{bind}}\), and basic quantiles of
        \(A_0(\theta)\).
  \item \texttt{phase3/outputs/tables/mech\_binding\_certificate.csv}:
        per-grid data for the binding-certificate experiment, again
        recording \(\theta\), \(A_0(\theta)\), \(A(\theta)\), and the
        binding mask.
  \item \texttt{phase3/outputs/tables/mech\_binding\_certificate\_diagnostics.json}:
        binding-certificate summary diagnostics, including the mean
        shift between \(A\) and \(A_0\), the \(L^2\) distance, and the
        binding fraction.
  \item \texttt{phase3/outputs/figures/fig1\_phase3\_mechanism\_binding.pdf}:
        a visual summary of the binding regime and the effect of the
        floor on the amplitude distribution.
\end{itemize}

The Phase~3 mechanism paper is built entirely from these artifacts and
the source code under \texttt{phase3/src/phase3\_mech/}. No manual
editing of generated tables or figures is required or assumed.

\subsection{Non-reproducible elements}

At this rung there are no stochastic components or external data
dependencies in the Phase~3 mechanism: all runs are deterministic
given the repository state. If future rungs introduce randomised
experiments or external data sources, they will be documented
explicitly together with the seeds, provenance, and additional gate
levels required to reproduce them.
