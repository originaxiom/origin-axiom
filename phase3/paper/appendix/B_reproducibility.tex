\section{Reproducibility and gate levels}
\label{sec:phase3-reproducibility}

Phase~3 is implemented as a self-contained, reproducible unit inside
the \texttt{origin-axiom} repository. This appendix records the
filesystem layout, the gate script used to regenerate the canonical
artifact, and the minimal commands needed to reproduce the Phase~3
paper and figures.

\subsection*{Filesystem layout}

The Phase~3 tree is organised as follows:
\begin{itemize}
  \item \texttt{phase3/src/} --- source code for the Phase~3
    mechanism and diagnostics;
  \item \texttt{phase3/paper/} --- LaTeX sources for the Phase~3 paper,
    including \texttt{main.tex}, section stubs, and appendices;
  \item \texttt{phase3/workflow/} --- the Snakemake workflow
    driving the paper build, in
    \texttt{phase3/workflow/Snakefile};
  \item \texttt{phase3/outputs/} --- derived outputs produced by
    Phase~3 runs, including figures and the built paper;
  \item \texttt{phase3/artifacts/} --- canonical Phase~3 artifacts,
    including the versioned PDF used as an external reference.
\end{itemize}

At this rung the primary paper artifact is
\texttt{phase3/artifacts/origin-axiom-phase3.pdf},
with the corresponding build product in
\texttt{phase3/outputs/paper/phase3\_paper.pdf}.  The main figure used
in the text is stored as
\texttt{phase3/outputs/figures/fig1\_mech\_binding\_profile.pdf}.

\subsection*{Gate script and build pipeline}

Phase~3 uses a dedicated gate script
\texttt{scripts/phase3\_gate.sh} to orchestrate the build. The
level--A gate regenerates the Phase~3 paper and canonical artifact via
the Snakemake workflow in \texttt{phase3/workflow/Snakefile}.  From
the repository root, the human-facing entry point is:
\begin{quote}
\texttt{bash scripts/phase3\_gate.sh}
\end{quote}
which, at level~A, performs the following high-level steps:
\begin{enumerate}
  \item assembles the list of section and appendix files under
    \texttt{phase3/paper/};
  \item invokes Snakemake on \texttt{phase3/workflow/Snakefile};
  \item runs \texttt{latexmk} on \texttt{phase3/paper/main.tex} to
    produce \texttt{main.pdf};
  \item copies \texttt{main.pdf} to
    \texttt{phase3/outputs/paper/phase3\_paper.pdf} and
    \texttt{phase3/artifacts/origin-axiom-phase3.pdf}.
\end{enumerate}

The Snakemake rule \texttt{build\_phase3\_paper\_pdf} declares
\texttt{main.tex}, the section and appendix files, and
\texttt{Reference.bib} as its inputs, and produces both the
\texttt{phase3\_paper.pdf} output and the canonical
\texttt{origin-axiom-phase3.pdf} artifact.  This makes the LaTeX
dependencies explicit and allows Snakemake to detect when a rebuild is
necessary.

\subsection*{Assumptions and environment}

The Phase~3 paper build assumes:
\begin{itemize}
  \item a reasonably recent \LaTeX{} distribution (e.g.\ TeX Live~2025
    or similar) providing \texttt{latexmk}, \texttt{pdflatex}, and the
    standard packages used in Phase~0--4;
  \item a POSIX shell environment with \texttt{bash} and
    \texttt{make}-like tooling sufficient to run the gate script;
  \item Python and any libraries required by the Phase~3 mechanism
    code, for the production of figures and diagnostic tables.
\end{itemize}

At the present rung, Phase~3 defines a placeholder bibliography file
\texttt{phase3/paper/Reference.bib}, since the paper does not yet
make external-citation claims.  This file is still part of the
declared inputs for the build, so that the pipeline remains stable
when references are introduced at later rungs.

\subsection*{Reproducibility scope}

The Level--A gate guarantees that:
\begin{itemize}
  \item the Phase~3 paper builds successfully from the tracked
    \texttt{phase3/paper} sources and \texttt{Reference.bib};
  \item the canonical artifact
    \texttt{phase3/artifacts/origin-axiom-phase3.pdf} is
    regenerable from these sources via the Snakemake workflow; and
  \item the figures referenced in the text (in particular the
    binding-profile figure) are present under
    \texttt{phase3/outputs/figures/} and can be regenerated from the
    Phase~3 source code.
\end{itemize}

Higher gate levels, if introduced, would be expected to add explicit
checks on numerical outputs, diagnostic tables, and mechanism
parameters.  At this rung, however, the emphasis remains on ensuring
that the narrative and structural content of the Phase~3 paper is
fully reproducible from the repository state.

\subsection*{Auxiliary measure probe}

In addition to the main binding-profile experiment, Phase~3 includes a
non-binding auxiliary script
\texttt{phase3/src/phase3\_mech/measure\_v1.py} that probes the
empirical distribution of the baseline amplitude \(A_0(\theta)\) for a
large number of independently sampled toy ensembles.  This script does
not affect any of the main Phase~3 claims or floor definitions; it is
included solely to make the measure structure of the current toy
configuration explicit.

A typical invocation is:
\begin{verbatim}
  oa && python phase3/src/phase3_mech/measure_v1.py
\end{verbatim}
which writes:
\begin{itemize}
  \item a JSON diagnostics file
    \texttt{phase3/outputs/tables/phase3\_measure\_v1\_stats.json}
    containing basic summary statistics and quantiles of the
    \(A_0\) distribution; and
  \item a histogram CSV
    \texttt{phase3/outputs/tables/phase3\_measure\_v1\_hist.csv}
    with binned counts over \(A_0\).
\end{itemize}
The console output also prints a small selection of quantiles and
fractions below a few illustrative \(\varepsilon\) thresholds.  All of
these numbers are toy-model diagnostics and should be interpreted as
such; they are not promoted to binding corridor constraints or
physical scales at this rung.
