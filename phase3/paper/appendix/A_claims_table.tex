\section{Phase 3 mechanism claims table}
\label{app:phase3mech_claims_table}

This appendix summarises the explicit claims and non-claims of the
Phase~3 mechanism as implemented in this paper. The focus is restricted
to the toy vacuum ensemble, the unconstrained global amplitude
\(A_0(\theta)\), the floor-enforced amplitude \(A(\theta)\), and the
binding diagnostics introduced in Section~\ref{sec:phase3-results}.

\subsection*{Claims}

\begin{center}
\begin{tabular}{llp{0.65\textwidth}}
\toprule
ID & Status & Description \\
\midrule
M3.1 &
Binding regime exists &
For the baseline configuration described in
Section~\ref{sec:phase3-mech}, there exists a strictly positive
floor \(\epsfloor > 0\) such that the toy vacuum is in a genuine
binding regime: the non-cancellation floor is active on a non-zero
fraction of the \(\theta\)-grid while leaving a non-trivial region
where \(A_0(\theta) > \epsfloor\), i.e.\ \(0 < f_{\mathrm{bind}} < 1\). \\[0.4em]
M3.2 &
Floor has a quantifiable global effect &
For the same configuration and floor, the floor-enforced amplitude
\(A(\theta)\) differs from the unconstrained amplitude
\(A_0(\theta)\) in a quantitatively non-trivial way. In particular,
the diagnostics reported in
\texttt{mech\_binding\_certificate\_diagnostics.json} show a strictly
positive mean shift and \(L^2\) distance between \(A\) and
\(A_0\), while maintaining numerical stability across the
\(\theta\)-grid. \\[0.4em]
M3.3 &
Reproducible code paths and artifacts &
The code paths that define the toy vacuum, \(A_0(\theta)\), the
non-cancellation floor, and the binding diagnostics are fully
specified in the repository, and the Phase~3 gate regenerates the
baseline-scan and binding-certificate artifacts in a clean checkout
using only the declared dependencies. \\
\bottomrule
\end{tabular}
\end{center}

\subsection*{Non-claims}

The following are explicitly \emph{not} claimed at this mechanism-only
stage:

\begin{itemize}
  \item Any identification of the toy vacuum with the real vacuum of
        our universe.
  \item Any claim that the chosen floor \(\epsfloor\) has direct
        physical meaning or matches an observed vacuum energy scale.
  \item Any claim that the present toy mechanism selects or narrows a
        physically distinguished \(\theta\) value or corridor.
  \item Any reduction of Standard Model free parameters, or any
        prediction for cosmological observables.
  \item Any assertion that this implementation is the unique or
        correct realisation of the Origin-Axiom non-cancellation
        principle.
\end{itemize}

These non-claims mirror the limitations discussed in
Section~\ref{sec:phase3-limitations}. Future rungs and phases will
need to connect the non-cancellation mechanism to external data and
the broader corridor architecture, or else conclude that the present
framework is not physically productive.
