\section{Phase 3 $\theta$-filter artifact (Phase 0 ledger interface)}
\label{sec:theta_filter_artifact}
Phase 3 emits a machine-readable $\theta$-filter artifact:
\path{phase3/outputs/theta_filter/phase_03_theta_filter.json}.
This file is the Phase 3 contribution to the Phase 0 corridor method: it reports an admissible set of $\theta$
values under an explicitly declared Phase 3 test suite, along with provenance sufficient for audit and reproduction.

\subsection{Declared test suite}
Phase 3 is an empirical calibration filter (not an OA-binding demonstration).
We therefore define a single test:
\begin{itemize}
  \item \textbf{fit\_compat\_interval:} $\theta$ is admissible if it lies within the declared fit interval
        reported in \path{theta_fit_summary.csv}.
\end{itemize}
This choice is intentionally conservative: it makes the corridor definition explicit and reproducible,
and it avoids parameter proliferation or implicit re-optimization beyond the one-parameter scan.

\subsection{Schema compliance}
The JSON conforms to the Phase 0 ledger interface: it declares a $\theta$ domain on $[0,2\pi)$,
provides an interval-form corridor representation, and includes a deterministic grid+pass array representation,
plus provenance bindings (commit/config/environment hashes when available).

\subsection{Ledger outcome in the current configuration}
The Phase~0 ledger applies the Phase~3 $\theta$-filter on top of the existing
Phase~0--2 filters and records the resulting combined corridor in
\path{phase0/phase_outputs/theta_corridor_history.jsonl}. In the baseline
configuration documented here, the ledger reports that the combined corridor is empty
once the Phase~3 filter is applied. This is logged as a negative test for the present
Phase~3 ansatz/target combination, and it is intentionally encoded so that future runs
with alternative ansatz choices or updated external targets can be compared against
this outcome.
