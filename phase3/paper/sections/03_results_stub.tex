\section{Results: binding regime diagnostics}
\label{sec:phase3-results}

This section summarizes the numerical behavior of the toy vacuum under
the non-cancellation floor. We focus on two closely related
experiments:

\begin{enumerate}
  \item a \emph{baseline scan} of the unconstrained amplitude
        \(A_0(\theta)\) over \(\theta \in [0, 2\pi)\), used to define a
        quantile-based floor \(\epsfloor\); and
  \item a \emph{binding-certificate scan} in which the same grid and
        floor are used to compare \(A_0(\theta)\) with the
        floor-enforced amplitude \(A(\theta) = \max(A_0(\theta),
        \epsfloor)\).
\end{enumerate}

Both experiments work with a deterministic ``baseline\_v1'' vacuum
configuration as defined in Section~\ref{sec:phase3-mech}.

\subsection{Baseline scan and floor selection}

In the baseline scan we evaluate \(A_0(\theta)\) on a uniform grid of
\(N_\theta = 2048\) points covering \([0,2\pi)\). The results are
stored, for reproducibility, in
\begin{center}
  \texttt{phase3/outputs/tables/mech\_baseline\_scan.csv}
\end{center}
with summary diagnostics in
\begin{center}
  \texttt{phase3/outputs/tables/mech\_baseline\_scan\_diagnostics.json}.
\end{center}
From the empirical distribution of \(A_0(\theta)\) we extract a
quantile-based floor
\[
  \epsfloor \equiv Q_{0.25}\bigl(A_0(\theta)\bigr),
\]
i.e., the 25th percentile of the sampled amplitudes. For the baseline
configuration used here, the diagnostics report
\begin{align}
  \epsfloor &\approx 0.0251, \\
  \min A_0  &\approx 0.0092, \\
  \max A_0  &\approx 0.0577,
\end{align}
with a binding fraction \(f_{\mathrm{bind}} = 0.25\), meaning that the
floor is active on exactly one quarter of the sampled \(\theta\)
values.

This choice of \(\epsfloor\) ensures that the toy vacuum is in a
\emph{genuine binding regime}: there are regions where the floor is
active and regions where the unconstrained dynamics are visible, with
neither extreme dominating the grid.

\subsection{Binding-certificate scan and global diagnostics}

Using the same grid and the quantile-based floor, the
binding-certificate scan evaluates both \(A_0(\theta)\) and the
floor-enforced amplitude \(A(\theta)\) on each grid point. The results
are stored in
\begin{center}
  \texttt{phase3/outputs/tables/mech\_binding\_certificate.csv}
\end{center}
with summary diagnostics in
\begin{center}
  \texttt{phase3/outputs/tables/mech\_binding\_certificate\_diagnostics.json}.
\end{center}

Figure~\ref{fig:phase3-binding-profile} shows the profile of
\(A_0(\theta)\) and \(A(\theta)\) across the grid, together with the
floor level \(\epsfloor\). As expected, the floor clips the lower tail
of the amplitude distribution, leaving the upper tail unchanged.

\begin{figure}[t]
  \centering
  \includegraphics[width=\linewidth]{fig1_mech_binding_profile}
  \caption{Baseline binding profile for the Phase~3 toy vacuum. The
  unconstrained amplitude \(A_0(\theta)\) (solid curve) fluctuates
  above and below the quantile-based floor \(\epsfloor\) (dashed
  line). The floor-enforced amplitude \(A(\theta)\) (solid curve with
  plateaus) coincides with \(A_0(\theta)\) whenever
  \(A_0(\theta) \ge \epsfloor\) and is clipped to \(\epsfloor\) where
  \(A_0(\theta) < \epsfloor\).}
  \label{fig:phase3-binding-profile}
\end{figure}

The diagnostics report, for the baseline configuration,
\begin{align}
  \langle A_0 \rangle &\approx 0.0388,
  &
  \langle A \rangle &\approx 0.0407, \\
  \Delta_{\mathrm{mean}} &\equiv
    \langle A \rangle - \langle A_0 \rangle \;>\; 0,
  &
  \Delta_{L^2} &\equiv
    \bigl\| A - A_0 \bigr\|_2 \;>\; 0,
\end{align}
together with the same binding fraction \(f_{\mathrm{bind}} = 0.25\)
and the extrema of \(A_0(\theta)\) reported above. In other words, the
floor has a quantitatively non-trivial global effect: it shifts the
mean amplitude, modifies the variance, and induces a finite
\(L^2\)-distance between \(A\) and \(A_0\), all while preserving a
non-zero region where the unconstrained dynamics remain visible.

This behavior is exactly what is required for a Phase~0-style
binding certificate. The floor is neither a purely formal constraint
nor a numerically invisible perturbation; it acts as a genuine
dynamical ingredient in the toy vacuum, in a regime where its impact
can be summarized by simple, reproducible diagnostics.

\subsection{Auxiliary measure probe (non-binding)}
\label{sec:phase3-measure-probe}

To complement the baseline binding profile, we performed a simple,
non-binding measure probe on the same class of toy ensembles.  The
script \texttt{phase3/src/phase3\_mech/measure\_v1.py} samples many
independent random ensembles of phases and windings, computes the
baseline amplitude \(A_0(\theta)\) on a fixed grid, and records the
empirical distribution of \(A_0\) values across ensembles and
\(\theta\).

In the baseline configuration, the resulting distribution shows a
small but non-zero probability weight near \(A_0 \approx 0\).  For
example, the minimum observed value is of order \(10^{-5}\), while the
1\% quantile sits at \(A_{0,\mathrm{p01}} \approx 0.013\).  The
fractions of samples with \(A_0\) below a few illustrative thresholds
are approximately
\begin{align}
  \mathrm{Pr}[A_0 < 0.005] &\approx 1.6 \times 10^{-3}, \\
  \mathrm{Pr}[A_0 < 0.01 ] &\approx 6.3 \times 10^{-3}, \\
  \mathrm{Pr}[A_0 < 0.02 ] &\approx 2.4 \times 10^{-2}, \\
  \mathrm{Pr}[A_0 < 0.05 ] &\approx 1.5 \times 10^{-1}.
\end{align}
These numbers are purely illustrative and depend on the toy-mechanism
choices, but they make explicit that, even before a floor is enforced,
the detailed cancellation basin near \(A_0 = 0\) occupies only a small
fraction of the ensemble-level measure in this construction.

We emphasise that this probe does \emph{not} define a physical floor,
does \emph{not} alter the Phase~3 binding experiment, and does
\emph{not} introduce any new claims.  Its purpose is limited to making
transparent, in a fully reproducible way, how often the toy ensemble
approaches extremely small amplitudes under the current configuration.
