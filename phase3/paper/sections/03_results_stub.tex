\section{Baseline binding experiment and floor diagnostics}
\label{sec:baseline_binding}

This section reports the first numerical experiments on the Phase~3
toy vacuum defined in Section~\ref{sec:mechanism_design}. The goal
is not yet to contact external observables, but to establish that the
non-cancellation floor can be enforced in a regime where it is both
active and diagnostically relevant for the global amplitude
\(\Amean(\theta)\).

\subsection{Baseline scan and quantile-based floor selection}

We begin by sampling the unconstrained amplitude
\(A_0(\theta)\) on a regular grid of
\(\theta \in [0,2\pi)\) with \(N_\theta = 2048\) points using the
baseline configuration from Section~\ref{sec:mechanism_design}. The
resulting grid values are written to
\begin{center}
  \texttt{phase3/outputs/tables/mech\_baseline\_scan.csv},
\end{center}
and summary diagnostics are stored in
\begin{center}
  \texttt{phase3/outputs/tables/mech\_baseline\_scan\_diagnostics.json}.
\end{center}

From the empirical distribution of \(A_0(\theta)\) we select the
non-cancellation floor \(\epsfloor\) as the 25th percentile,
\begin{equation}
  \epsfloor \equiv Q_{0.25}\!\bigl(A_0(\theta)\bigr),
\end{equation}
so that roughly one quarter of the grid points lie at or below the
floor. For the baseline run documented here, the diagnostics record
\(\min A_0(\theta) \approx 0.009\), \(\max A_0(\theta) \approx 0.058\),
and \(\epsfloor \approx 0.025\), with a binding fraction
\(f_{\mathrm{bind}} \equiv \mathbb{E}[\mathbf{1}_{A_0 < \epsfloor}]
\approx 0.25\). The toy vacuum is therefore in a genuine binding
regime: the floor neither freezes the dynamics (\(f_{\mathrm{bind}}=1\))
nor becomes inert (\(f_{\mathrm{bind}}=0\)).

\subsection{Binding certificate diagnostics}

To check that the floor changes the observable in a diagnostically
relevant way, we repeat the scan with \(\epsfloor\) enforced. Using
the same grid and configuration as above, we compute both the
unconstrained amplitude \(A_0(\theta)\) and the floor-enforced
amplitude
\begin{equation}
  A(\theta) \equiv \max\bigl(A_0(\theta), \epsfloor\bigr),
\end{equation}
together with a Boolean binding mask
\(\mathbf{1}_{\mathrm{bind}}(\theta) = \mathbf{1}_{A_0(\theta) < \epsfloor}\).
The per-grid results are written to
\begin{center}
  \texttt{phase3/outputs/tables/mech\_binding\_certificate.csv},
\end{center}
and summary diagnostics to
\begin{center}
  \texttt{phase3/outputs/tables/mech\_binding\_certificate\_diagnostics.json}.
\end{center}

From these we construct simple global diagnostics:
\begin{itemize}
  \item the mean amplitudes
        \(\mathbb{E}[A_0]\) and \(\mathbb{E}[A]\),
  \item the variances \(\mathrm{Var}(A_0)\) and \(\mathrm{Var}(A)\),
  \item the mean shift
        \(\Delta_{\mathrm{mean}} \equiv \mathbb{E}[A] - \mathbb{E}[A_0]\),
  \item an \(L^2\) distance
        \(\Delta_{L^2} \equiv \bigl(\mathbb{E}[(A - A_0)^2]\bigr)^{1/2}\),
  \item and the binding fraction
        \(f_{\mathrm{bind}} = \mathbb{E}[\mathbf{1}_{\mathrm{bind}}]\).
\end{itemize}
In the baseline run, the diagnostics confirm that
\(\Delta_{\mathrm{mean}} > 0\), \(\Delta_{L^2} > 0\), and
\(0 < f_{\mathrm{bind}} < 1\); the floor shifts the distribution
of amplitudes upward in a controlled but non-trivial way. This is
exactly the qualitative behavior required for a Phase~0--style
binding certificate: the non-cancellation floor is neither a purely
formal constraint nor a numerically invisible perturbation.

At this rung we stop short of constructing a full binding certificate
in the sense of Phase~0---where one would compare downstream
observables with and without the floor enforced. Here we only
establish that the toy vacuum admits a numerically well-behaved
regime in which the floor has a quantifiable, global effect on the
observable \(A(\theta)\). Later rungs will connect these diagnostics
to more structured tests and, ultimately, to a \(\theta\)-filter
artifact suitable for the Phase~0 ledger.
