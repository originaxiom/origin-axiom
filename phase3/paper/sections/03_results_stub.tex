\section{Baseline binding experiment}
\label{sec:baseline-binding}

With the toy vacuum and global amplitude defined in
Section~\ref{sec:mechanism-design}, the next step is to demonstrate that
there exist floor values \(\epsfloor > 0\) for which the non-cancellation
constraint is \emph{diagnostically active} rather than inert. In this
section we construct such a baseline binding regime and record it in
machine-readable form.

\subsection{Scan setup and floor selection}

We fix the baseline vacuum configuration \(\mathrm{cfg}_0\) provided by
\texttt{make\_vacuum\_config("baseline\_v1")}. For this configuration, we
scan the unconstrained amplitude \(A_0(\theta)\) on a uniform grid
\(\theta_k \in [0,2\pi)\) with
\begin{equation}
  \theta_k
  \;=\;
  \theta_{\min} + k\,\Delta\theta,
  \qquad
  \Delta\theta \equiv \frac{2\pi}{N_{\theta}},
\end{equation}
using the helper function
\texttt{scan\_amplitude\_unconstrained} in the Phase~3 mechanism package.

Rather than choosing \(\epsfloor\) by hand, we define it
\emph{relative to the observed distribution} of \(A_0(\theta)\):
\begin{equation}
  \epsfloor
  \;\equiv\;
  Q_{0.25}\bigl[A_0(\theta)\bigr],
\end{equation}
where \(Q_{0.25}\) denotes the 25th percentile of the sampled
\(A_0(\theta)\) values on the scan grid. By construction, this puts
\(\epsfloor\) strictly between the minimum and maximum of the
unconstrained amplitude and ensures that a non-zero fraction of the
grid points lie in the binding regime \(A_0(\theta) < \epsfloor\)
while the remainder remain unconstrained.

\subsection{Binding diagnostics and artifacts}

Given this choice of \(\epsfloor\), we rescan the same grid using the
floor-enforced amplitude
\begin{equation}
  A(\theta)
  \;=\;
  \max\bigl(A_0(\theta), \epsfloor\bigr),
\end{equation}
via the helper \texttt{scan\_amplitude\_with\_floor}. This returns
\begin{itemize}
  \item the grid of phase values \(\theta_k\);
  \item the unconstrained amplitudes \(A_0(\theta_k)\);
  \item the floor-enforced amplitudes \(A(\theta_k)\); and
  \item a boolean mask marking where the floor is active:
        \(\mathrm{bind}_k = \mathbf{1}[A_0(\theta_k) < \epsfloor]\).
\end{itemize}
From these we derive summary diagnostics,
\begin{equation}
  \min A_0, \quad
  \max A_0, \quad
  \text{and}\quad
  f_{\mathrm{bind}}
  \;\equiv\;
  \frac{1}{N_{\theta}}
  \sum_k \mathbf{1}[A_0(\theta_k) < \epsfloor],
\end{equation}
which quantify the extent to which the floor is active. A configuration
is considered to be in a \emph{binding regime} if
\(0 < f_{\mathrm{bind}} < 1\), so that the floor affects some but not
all grid points.

The per-grid data are written to
\begin{center}
  \texttt{phase3/outputs/tables/mech\_baseline\_scan.csv}
\end{center}
with columns \(\theta_k\), \(A_0(\theta_k)\), \(A(\theta_k)\), and the
binding mask. The corresponding diagnostics and quantiles are recorded
in
\begin{center}
  \texttt{phase3/outputs/tables/mech\_baseline\_scan\_diagnostics.json},
\end{center}
including \(\epsfloor\), \(\min A_0\), \(\max A_0\), the binding
fraction \(f_{\mathrm{bind}}\), and basic quantiles of the
unconstrained amplitude distribution.

For the baseline configuration used here, the diagnostics confirm that
\(\epsfloor\) lies strictly between the minimum and maximum of
\(A_0(\theta)\) and that \(0 < f_{\mathrm{bind}} < 1\); the toy
vacuum is therefore in a genuine binding regime. This establishes the
numerical foundation for a Phase~0--style binding certificate in
later rungs, where we will compare the behavior of observables with
and without the non-cancellation floor enforced.
