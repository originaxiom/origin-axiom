\section{Falsifiability and Failure Modes}
\label{sec:falsifiability}

Phase 3 is falsifiable in the narrow sense appropriate to an exploratory fit: the mapping from
\texorpdfstring{$\theta$}{theta} to the frozen flavor targets can fail to produce a stable, interpretable
interval.

\subsection{Explicit failure conditions}
We treat the Phase 3 extraction as \emph{invalid} if any of the following occur:
\begin{itemize}
  \item \textbf{Non-identifiability:} multiple disjoint \texorpdfstring{$\theta$}{theta} regions yield
        indistinguishably good minima (flat or multi-modal \texorpdfstring{$\chi^2$}{chi2} without a
        defensible interval).
  \item \textbf{Offset ambiguity:} the discrete offset sweep yields competing hypotheses with comparable
        minima, preventing a pre-registered baseline choice from being meaningful.
  \item \textbf{Target drift collapse:} a reasonable update of the external targets (future PDG/NuFIT
        releases) removes the minimum or shifts it outside any stable corridor overlap with other phases.
  \item \textbf{Ansatz fragility:} small, clearly stated modifications to the mapping family (within the
        declared ansatz class) destroy the existence of a stable minimum, indicating the result is an
        artifact of parameterization rather than a durable extraction.
\end{itemize}

\subsection{What Phase 3 can support downstream}
If Phase 3 passes the checks above, it supports only this downstream use:
the exported \texttt{phase\_03\_theta\_filter.json} interval can be compared by Phase 0 ledger tooling to
other phase constraints to test whether a corridor overlap emerges. No stronger inference is warranted.
