\section{Toy vacuum mechanism and global amplitude}
\label{sec:phase3_mech_design}

The Phase~3 mechanism is intentionally modest: we construct a simple,
deterministic ``toy vacuum'' whose only job is to expose a global amplitude
observable \(A_0(\theta)\) depending on a single phase parameter \(\theta\).
In later rungs we will enforce the Origin-Axiom non-cancellation floor on
this observable and study how the floor interacts with the toy dynamics.

\subsection{Ensemble definition}

We model the toy vacuum as an ensemble of \(N\) complex modes (phasors),
each labeled by an index \(k \in \{0,\dots,N-1\}\). For a given configuration
we assign to each mode a base phase \(\alpha_k\) and a sign
\(\sigma_k \in \{-1,+1\}\). The sequence \((\alpha_k, \sigma_k)\) is generated
once and for all from a fixed-seed pseudorandom generator, so that for a given
configuration identifier the ensemble is deterministic and reproducible.

Given a phase parameter \(\theta \in [0,2\pi)\), the mode-level contribution
is
\begin{equation}
  z_k(\theta) \;=\; \exp\bigl[i \, (\alpha_k + \sigma_k \theta)\bigr],
\end{equation}
and the global complex amplitude is the ensemble mean
\begin{equation}
  \bar{Z}(\theta) \;=\; \frac{1}{N} \sum_{k=0}^{N-1} z_k(\theta).
\end{equation}
Throughout this paper we write \(A_0(\theta) = |\bar{Z}(\theta)|\) for the
\emph{unconstrained} global amplitude.

At this rung we work with a single baseline configuration
\texttt{baseline\_v1}, defined in code by a fixed number of modes and a fixed
RNG seed (see the Phase~3 source in \texttt{phase3/src/phase3\_mech/}).
The intent is not to model any particular field, but to provide a simple,
well-controlled testbed on which the non-cancellation floor can act.

\subsection{Observable and the role of \texorpdfstring{$\theta$}{theta}}

The phase parameter \(\theta\) plays a purely structural role in Phase~3:
it is a single control knob that twists the ensemble of modes via the signs
\(\sigma_k\). For a fixed vacuum configuration, varying \(\theta\) induces
a family of amplitudes
\begin{equation}
  A_0(\theta) \;=\; \bigl| \bar{Z}(\theta) \bigr|.
\end{equation}

We do not, at this stage, assert that \(\theta\) is a physically interpreted
parameter (e.g. a Standard Model phase or a cosmological quantity). The
corridor method remains agnostic: later phases are allowed to impose
additional constraints on \(\theta\) (for example, by requiring overlap
with a flavor-sector fit or with a vacuum-energy diagnostic), but Phase~3
itself treats \(\theta\) only as a structural phase parameter.

\subsection{Non-cancellation floor (deferred to later rungs)}

The Origin Axiom asserts that the global amplitude cannot cross below a
strictly positive floor \(\epsfloor > 0\). In the language of this toy
vacuum, that means we will eventually enforce
\begin{equation}
  A(\theta) \;\ge\; \epsfloor \;>\; 0
\end{equation}
for all admissible configurations and values of \(\theta\), and we will
require a binding certificate demonstrating that the presence of the floor
changes the dynamics in a diagnostically relevant way.

However, this enforcement is postponed to later rungs. The present rung
defines only the unconstrained observable \(A_0(\theta)\) and the baseline
vacuum configuration on which the floor will act. This separation allows
us to compare, in a controlled way, the floor-enforced dynamics against the
unconstrained baseline in subsequent sections.
