\section{Mechanism design: toy vacuum and global amplitude}
\label{sec:phase3-mech}

The Phase~3 mechanism work introduces an explicit toy model of a ``vacuum''
ensemble and a global amplitude observable. The goal is not to claim a
realistic microscopic description of the quantum vacuum, but to define a
clean laboratory in which the non-cancellation floor can be stated,
implemented, and tested in a way that is compatible with the Phase~0
corridor/ledger infrastructure.

\subsection{Toy vacuum ensemble}

We model the vacuum as a finite ensemble of complex modes
\(\{z_k(\theta)\}_{k=1}^{N}\) with phase structure
\begin{equation}
  z_k(\theta) \;=\; \exp\!\bigl(i(\alpha_k + \sigma_k \theta)\bigr),
\end{equation}
where \(\alpha_k \in [0,2\pi)\) are fixed phase offsets and
\(\sigma_k \in \{1,2,3,4\}\) are small positive integers that control
how each mode winds as the global phase parameter \(\theta\) is varied.
For the present rung we use a single deterministic configuration
\begin{equation}
  \mathrm{cfg}_0 \;=\; \{\alpha_k, \sigma_k\}_{k=1}^{N},
\end{equation}
constructed by sampling \(\alpha_k\) and \(\sigma_k\) from simple
distributions with a fixed RNG seed. This makes all derived quantities
fully reproducible while still providing a non-trivial interference
pattern as \(\theta\) is scanned.

The toy vacuum is therefore specified by a configuration object
\texttt{VacuumConfig} (stored in memory and, if needed, on disk), which
collects the arrays of \(\alpha_k\) and \(\sigma_k\). Later rungs may
introduce additional configurations (e.g.\ higher-mode ensembles or
variants with different winding distributions) as part of robustness
checks, but all Phase~3 mechanism claims in this paper are scoped to
the baseline configuration \(\mathrm{cfg}_0\).

\subsection{Global amplitude observable}

Given a configuration \(\mathrm{cfg}_0\) and a value of \(\theta\), we
define the unconstrained global amplitude \(A_0(\theta)\) as the
modulus of the ensemble average
\begin{equation}
  A_0(\theta) \;=\; \Bigl\lvert \frac{1}{N}
  \sum_{k=1}^{N} z_k(\theta) \Bigr\rvert.
\end{equation}
In code, this is implemented as a function
\texttt{amplitude\_unconstrained(theta, cfg)} together with a grid
scanner \texttt{scan\_amplitude\_unconstrained(cfg, ...)} that evaluates
\(A_0(\theta)\) on a regular lattice of \(\theta\) values in
\([0,2\pi)\).

The observable \(A_0(\theta)\) plays two roles:
\begin{enumerate}
  \item It acts as the reference against which a floor-enforced
        amplitude will be compared in binding vs.\ non-binding regimes.
  \item It provides a concrete, non-trivial test bed for exploring how
        a global non-cancellation constraint interacts with an ensemble
        of interfering modes as \(\theta\) is varied.
\end{enumerate}
At this stage we make no claim that \(A_0(\theta)\) is directly
identifiable with a physical vacuum observable; it is a diagnostic
quantity in a controlled toy model.

\subsection{Non-cancellation floor and binding diagnostics}

The Origin Axiom asserts that the global amplitude cannot cross below a
strictly positive floor \(\epsfloor > 0\). In the language of this toy
vacuum, the floor-enforced amplitude \(A(\theta)\) is defined by
\begin{equation}
  A(\theta) \;=\; \max\!\bigl(A_0(\theta), \epsfloor\bigr),
\end{equation}
so that the unconstrained dynamics are recovered whenever
\(A_0(\theta) \ge \epsfloor\), and the floor becomes active only in the
sub-region where \(A_0(\theta) < \epsfloor\).

In code, this is implemented by the function
\texttt{amplitude\_with\_floor(theta, cfg, epsfloor)} together with a
grid-level helper \texttt{scan\_amplitude\_with\_floor} that returns:
\begin{itemize}
  \item the grid of \(\theta\) values,
  \item the unconstrained amplitudes \(A_0(\theta)\),
  \item the floor-enforced amplitudes \(A(\theta)\),
  \item a boolean mask indicating where the floor is active, and
  \item a small diagnostics dictionary with summary statistics.
\end{itemize}

The diagnostics are designed to support a binding certificate in the
sense of Phase~0. In particular, for a chosen \(\epsfloor\) and grid we
record:
\begin{itemize}
  \item \(\min_\theta A_0(\theta)\) and \(\max_\theta A_0(\theta)\),
  \item the fraction of grid points where \(A_0(\theta) < \epsfloor\),
  \item the value of \(\epsfloor\) itself.
\end{itemize}
A configuration is said to be in a \emph{binding regime} if the floor
is active on a non-zero fraction of the grid while still leaving
regions where the unconstrained dynamics are visible. This provides
the raw material for later rungs to construct an explicit binding
certificate: a quantitative demonstration that the presence of the
floor changes the dynamics in a diagnostically relevant way, rather
than being an inert constraint.

At this rung we do not yet fix a canonical value of \(\epsfloor\) or
tie the toy vacuum directly to cosmological observables. The purpose
is to define clean, separation-of-concerns interfaces: an
unconstrained amplitude \(A_0(\theta)\), a floor-enforced amplitude
\(A(\theta)\), and well-specified diagnostics that future rungs can
use to generate tables, figures, and ultimately the
\(\theta\)-filter artifact required by the Phase~0 contract.
