\section{Injection into the Phase 2 vacuum-residue mechanism}
\label{sec:phase3-injection}

This section explains how the Phase~3 best-fit phase $\hat{\theta}$ is used
as an input to the Phase~2 vacuum-residue machinery, and what the resulting
diagnostics mean.

\subsection{One-way injection hook}

Phase~2 implements a vacuum-residue mechanism and exposes a one-parameter
injection hook for a phase-like parameter. In Phase~3 we treat this as an
opportunity to test whether the empirically fitted flavor phase is at least
compatible with the vacuum-side construction.

Operationally, we:
\begin{enumerate}
  \item take the same $\theta$ grid used in the fit;
  \item pass each $\theta$ value through the Phase~2 injection hook, keeping
        all other Phase~2 settings fixed at their baseline values; and
  \item record the resulting residue metric
        $\Delta\rho_{\mathrm{vac}}(\theta)$.
\end{enumerate}
The injection is strictly one-way: the Phase~2 outputs are not fed back into
the flavor fit, and no attempt is made to re-optimise $\theta$ on the basis
of vacuum-side information.

\subsection{Diagnostic curve and interpretation}

The resulting curve $\Delta\rho_{\mathrm{vac}}(\theta)$ is plotted in
\path{phase3/outputs/figures/fig2_delta_rho_vac_vs_theta.pdf}, with metadata
in \path{phase3/outputs/figures/fig2_delta_rho_vac_vs_theta.meta.json}. In
this paper we use the curve purely as a qualitative diagnostic: it shows how
the Phase~2 residue responds as $\theta$ is moved through the corridor defined
by the flavor fit, but we do not claim any quantitative prediction for
cosmological observables.

The only downstream interface that is allowed to act on Phase~3 is the
ledger-facing $\theta$-filter described in Appendix~D, which compares the
Phase~3 admissible region with the existing Phase~0--2 corridor. The injection
described in this section is therefore a ``sanity check'' and a source of
intuition, not a binding constraint.
