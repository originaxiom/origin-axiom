\section{Discussion and limitations}
\label{sec:phase3-limitations}

Phase~3, in its current mechanism-only form, is deliberately modest in
scope. The toy vacuum and non-cancellation floor are designed to test
whether the Origin-Axiom constraint can be implemented in a clean,
diagnostically meaningful way, not to make direct contact with
observed cosmology or particle physics.

\subsection{Scope of the toy vacuum model}

The vacuum ensemble defined in Section~\ref{sec:mechanism-design} is a
controlled but highly idealised construction. The modes have no
spatial structure, no local dynamics, and no connection to a
Hamiltonian or Lagrangian. The global amplitude \(A_0(\theta)\) is an
aggregate diagnostic, not a physical observable.

As a consequence, none of the numerical values reported in
Section~\ref{sec:phase3-results}---including typical scales of
\(A_0(\theta)\), the quantile-based floor \(\epsfloor\), or the
binding fraction \(f_{\mathrm{bind}}\)---should be interpreted as
predictions for the real vacuum. They only demonstrate that a
non-cancellation floor can be imposed without numerical instability,
and that its effect on a simple global observable can be quantified in
a reproducible manner.

\subsection{Choice of floor and \texorpdfstring{$\theta$}{theta} grid}

The present rung treats both the \(\theta\) grid and the floor
selection rule as design choices:

\begin{itemize}
  \item \(\theta\) is scanned uniformly over \([0,2\pi)\) with a fixed
        number of grid points. This choice is convenient for
        diagnostics, but not derived from any underlying symmetry or
        phenomenology.
  \item The non-cancellation floor \(\epsfloor\) is chosen as a simple
        quantile \(Q_{0.25}\bigl(A_0(\theta)\bigr)\) of the
        unconstrained amplitude distribution. This guarantees a
        genuine binding regime (with \(0 < f_{\mathrm{bind}} < 1\)),
        but it is not tied to observed vacuum energy or any other
        external data.
\end{itemize}

In other words, the mechanism demonstrates that a floor \emph{can} be
implemented coherently, not that the specific \(\epsfloor\) or grid
used here is physically distinguished. Future work will need to
explore whether there exist more principled criteria for selecting
both the \(\theta\) domain and the floor, potentially informed by
contact with actual observables.

\subsection{Relation to the Phase~0 contract}

From the perspective of the Phase~0 contract, the present rung
delivers only part of what a full Phase~3 mechanism is expected to
provide:

\begin{itemize}
  \item The toy vacuum and global amplitude are explicitly defined and
        numerically stable.
  \item A genuine binding regime has been demonstrated, with
        diagnostics showing that the non-cancellation floor has a
        quantifiable, global effect on \(A(\theta)\) while leaving a
        non-trivial unconstrained region.
  \item The code paths and outputs required to reproduce these
        diagnostics are bound into the Phase~3 gate.
\end{itemize}

What is deliberately \emph{not} provided at this rung is a
\(\theta\)-filter artifact suitable for ingestion by the Phase~0
ledger. We do not yet claim:

\begin{itemize}
  \item a principled rule for selecting a canonical subset of
        \(\theta\) values from the toy vacuum;
  \item a mapping from the toy vacuum amplitudes to any physical
        observable such as an effective vacuum energy density; or
  \item a corridor-style constraint that meaningfully narrows
        \(\theta\) in combination with earlier phases.
\end{itemize}

These omissions are intentional. The goal of the current Phase~3
mechanism rung is to establish a clean, reproducible baseline on which
more ambitious constructions can be built. Later rungs---and, more
importantly, later phases---will need to connect the non-cancellation
mechanism to external data and to the broader corridor architecture
laid out in Phase~0, or else conclude that the Origin-Axiom framework
is not a productive description of the real vacuum.
