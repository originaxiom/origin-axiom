\section{Limitations and outlook}
\label{sec:phase3-limitations}

The construction presented in this Phase~3 mechanism paper is
deliberately minimal. It is designed to test the internal coherence of
the non-cancellation principle on a controlled toy vacuum, not to make
direct claims about nature. This section collects the main limitations
and outlines how they inform future work.

\subsection{Toy nature of the vacuum ensemble}

The vacuum ensemble used here is a finite collection of complex modes
with phases that depend linearly on a single parameter \(\theta\).
There is no underlying Hamiltonian, no explicit field-theoretic
content, and no attempt to represent known particle species or
interactions. As such, the toy vacuum should be regarded as a
\emph{numerical laboratory} for the non-cancellation floor, not as a
candidate model of the Standard Model vacuum.

Any contact with real-world physics will require:
\begin{itemize}
  \item replacing or extending the toy ensemble with structures that
        have a clear field-theoretic or statistical-mechanical
        interpretation; and
  \item specifying how the global amplitude \(A(\theta)\) relates to
        more conventional observables (energy densities, correlation
        functions, or effective potentials).
\end{itemize}

\subsection{Single-parameter structure and lack of spatial geometry}

In this phase, \(\theta\) is treated as a single global parameter. The
toy vacuum has no explicit spatial degrees of freedom and no notion
of local dynamics. This is sufficient for testing whether a
non-cancellation floor can be implemented in a binding regime, but it
falls short of the structures needed for:
\begin{itemize}
  \item embedding the mechanism in an FRW-like cosmological dynamics;
  \item coupling to local field excitations or matter content; or
  \item studying how the floor interacts with locality and causality
        in the sense of effective field theory.
\end{itemize}

Subsequent work will need to address how the non-cancellation
mechanism can coexist with (or modify) local field-theoretic
descriptions, and whether the global amplitude \(A(\theta)\) can be
given a more invariant geometric or statistical meaning.

\subsection{No selection of \texorpdfstring{\(\theta\)}{theta} or \texorpdfstring{\(\epsfloor\)}{epsfloor}}

The present phase demonstrates that a binding regime exists for a
quantile-based choice of \(\epsfloor\) derived from the distribution
of \(A_0(\theta)\). This is a convenient and reproducible method, but
it is not a derivation from first principles. Similarly, the toy
vacuum does not prefer any particular value of \(\theta\) within
\([0,2\pi)\); all choices are treated symmetrically.

In particular, Phase~3 does not:
\begin{itemize}
  \item single out a distinguished value \(\theta_\star\);
  \item relate \(\epsfloor\) to any measured energy scale; or
  \item derive either quantity from a variational principle, symmetry
        argument, or stability requirement beyond the binding tests
        performed here.
\end{itemize}

Identifying mechanisms that could fix \(\theta\) and \(\epsfloor\) is
one of the central open problems for later phases of the program.

\subsection{No direct contact with empirical data}

Finally, this Phase~3 mechanism paper does not invoke any external
empirical inputs. There is no attempt to match observed vacuum energy,
cosmological expansion histories, or flavor-sector data. The earlier
flavor-calibration experiment is preserved as a separate, archived
artifact in \texttt{experiments/phase3\_flavor\_v1/}, but it is not
part of the canonical Phase~3 mechanism defined here.

Connecting the non-cancellation mechanism to data will require:
\begin{itemize}
  \item embedding the toy vacuum into cosmological models and studying
        how \(A(\theta)\) maps to effective energy densities; and/or
  \item identifying observables in other sectors (such as flavor or
        scattering amplitudes) that could plausibly depend on the same
        phase parameter \(\theta\).
\end{itemize}

\subsection{Outlook}

Within these limitations, the Phase~3 mechanism achieves its primary
goal: it shows that the non-cancellation floor can be implemented on a
toy vacuum in a genuine binding regime, with a quantifiable global
effect on the observable \(A(\theta)\). This provides a concrete,
reproducible mechanism-level foundation for future phases that will:
\begin{itemize}
  \item explore more realistic vacuum constructions with spatial and
        field-theoretic structure;
  \item investigate principles that might select \(\theta\) and
        \(\epsfloor\) (for example, extremal-action or stability
        criteria); and
  \item construct \(\theta\)-filters that tie the mechanism to
        corridor-style constraints and, ultimately, to empirical
        data via the Phase~0 ledger.
\end{itemize}

Those developments lie beyond the scope of the present paper, but the
mechanism and diagnostics established here are intended to serve as
their stable substrate.
