\section{Introduction}
\label{sec:phase3-introduction}

Phase~3 of the \OA{} program, in its current ``mechanism'' incarnation,
has a deliberately narrow role. It is not a general calibration of
\(\theta\) against external data, and it is not a claim to have
derived a canonical value \(\theta_\star\). Instead, Phase~3 serves as
a bridge between the abstract non-cancellation principle articulated
in Phase~0 and the concrete toy implementations constructed in
Phases~1 and~2.

The central question of this phase is structural rather than
phenomenological:
\begin{quote}
  Given a global amplitude observable \(A_0(\theta)\) on a toy vacuum,
  can we implement a strictly positive non-cancellation floor
  \(\epsfloor > 0\) in such a way that the floor is both
  numerically well-behaved and \emph{dynamically non-trivial}?
\end{quote}
More colloquially: is there a regime in which the floor meaningfully
changes the behavior of the toy vacuum, rather than being either a
purely formal constraint or an invisible perturbation?

To answer this, Phase~3 introduces:
\begin{itemize}
  \item a deterministic toy vacuum ensemble, defined as a collection
        of complex modes with phases that depend linearly on a global
        phase parameter \(\theta\);
  \item a global observable \(A_0(\theta)\) given by the modulus of
        the ensemble average, which serves as a stand-in for a vacuum
        ``amplitude'' in the sense of the Origin Axiom; and
  \item a floor-enforced amplitude \(A(\theta) = \max(A_0(\theta),
        \epsfloor)\), together with diagnostics that quantify how the
        floor modifies the distribution of \(A_0(\theta)\) over
        \(\theta \in [0, 2\pi)\).
\end{itemize}

The present paper is intentionally modest in scope. It does \emph{not}
attempt to identify \(\theta\) with any specific physical phase, nor
does it tie \(A(\theta)\) directly to observed vacuum energy or other
data. Instead, it aims to:
\begin{enumerate}
  \item define a clean, reproducible toy vacuum mechanism that exposes
        both unconstrained and floor-enforced observables;
  \item demonstrate the existence of a \emph{binding regime} in which
        the floor is active on a non-zero fraction of the \(\theta\)
        grid while leaving regions where the unconstrained dynamics
        are still visible; and
  \item quantify the global impact of the floor on \(A(\theta)\) via
        simple diagnostics such as mean shifts, \(L^2\) distances, and
        binding fractions.
\end{enumerate}

This mechanism-focused Phase~3 replaces the earlier
flavor-calibration experiment, which is preserved in
\texttt{experiments/phase3\_flavor\_v1/} as a fully reproducible
negative result. The archived experiment remains a valid exploratory
add-on, but it is no longer the canonical definition of Phase~3 in
the sense of the Phase~0 contracts.

\subsection*{Claims and non-claims at the Phase 3 mechanism level}

Within this mechanism-focused scope, Phase~3 claims:
\begin{itemize}
  \item that there exists a deterministic toy vacuum ensemble and
        global amplitude \(A_0(\theta)\) for which a strictly positive
        floor \(\epsfloor\) can be enforced in a numerically stable
        way; and
  \item that, for the baseline configuration studied here, one can
        choose \(\epsfloor\) so that the toy vacuum enters a genuine
        binding regime with a demonstrable global shift in the
        amplitude distribution.
\end{itemize}

Phase~3 explicitly does \emph{not} claim:
\begin{itemize}
  \item that the toy vacuum is a faithful model of the real universe;
  \item that \(\theta\) has been identified with any specific
        physical parameter;
  \item that a unique, canonically derived value \(\theta_\star\) has
        been found; or
  \item that the mechanism constructed here suffices, on its own, to
        explain observed vacuum energy or other empirical data.
\end{itemize}

Later phases and future extensions are responsible for connecting this
mechanism to more realistic field-theoretic models, to cosmological
dynamics, and to empirical constraints. The role of Phase~3 is to
ensure that the non-cancellation principle can be implemented in a
way that is both reproducible and diagnostically meaningful, providing
a solid mechanism-level foundation for those later steps.
