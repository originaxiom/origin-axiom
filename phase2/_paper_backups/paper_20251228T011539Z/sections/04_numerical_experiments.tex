% ============================================================
% Origin Axiom — Phase 2
% Section 4: Numerical Experiments
% ============================================================

\section{Numerical Experiments}
\label{sec:numerical_experiments}

This section reports the results of controlled numerical experiments performed on the finite mode-sum model defined in Sec.~\ref{sec:mode_sum_model}.
All figures presented here are generated automatically via the Phase~2 workflow and are reproducible from the accompanying configuration and run logs.

\subsection{Experimental Protocol}

Each experiment proceeds as follows:
\begin{enumerate}
\item A set of $N$ modes is initialized with frequencies bounded by a cutoff $\Lambda$.
\item Phases are assigned to place the ensemble near destructive interference.
\item The unconstrained amplitude $A_N$ is computed.
\item The Origin Axiom constraint $|A_N| \ge \eps$ is enforced.
\item The residual amplitude and derived observables are recorded.
\end{enumerate}

For each sweep, only one control parameter is varied while the others are held fixed.
This isolates scaling behavior and prevents confounding effects.

\subsection{Figure A: Mode-Sum Residual Convergence}

Figure~A displays the magnitude of the unconstrained mode sum $|A_N|$ as a function of the number of modes $N$.
As $N$ increases, the ensemble approaches near-cancellation, with residual fluctuations decreasing systematically.

This figure establishes the baseline behavior prior to enforcing the global constraint.
It demonstrates that, absent the Origin Axiom, the collective amplitude would continue to decrease toward zero as additional modes are added.

\subsection{Figure B: Scaling with Residual Bound $\eps$}

Figure~B shows the effective residual amplitude as a function of the imposed bound $\eps$, with $N$ and $\Lambda$ held fixed.
The observed relationship follows
\begin{equation}
A_{\mathrm{res}} \;\sim\; \eps ,
\end{equation}
over multiple orders of magnitude.

This confirms that, once the unconstrained amplitude falls below $\eps$, the residual is controlled entirely by the bound itself.
No additional amplification or suppression effects are observed.

\subsection{Figure C: Scaling with Cutoff $\Lambda$}

Figure~C examines the dependence of the residual amplitude on the ultraviolet cutoff $\Lambda$, with $N$ and $\eps$ fixed.
Across the tested range, the residual amplitude remains stable and exhibits no growth with increasing cutoff.

This behavior indicates that the constrained residual is insensitive to ultraviolet extension of the spectrum once near-cancellation is achieved.
The result contrasts with naive expectations from unconstrained mode summation, where increasing $\Lambda$ typically increases total energy.

\subsection{Figure D: Scaling with Number of Modes $N$}

Figure~D explores scaling with respect to the number of modes $N$, holding $\Lambda$ and $\eps$ constant.
After an initial transient regime at small $N$, the residual amplitude saturates and becomes independent of further increases in mode count.

This saturation confirms that the residual is not a finite-size artifact and persists in the large-$N$ limit of the finite ensemble.

\subsection{Summary of Numerical Findings}

Taken together, Figs.~A--D establish three key numerical facts:
\begin{itemize}
\item near-cancellation naturally emerges in structured finite ensembles,
\item the Origin Axiom enforces a stable residual amplitude,
\item the residual is insensitive to both ultraviolet cutoff and mode count.
\end{itemize}

These findings set the stage for interpreting the residual amplitude as an effective vacuum energy scale in a cosmological context, which we address in the next section.

% ============================================================
% End of 04_numerical_experiments.tex
% ============================================================
