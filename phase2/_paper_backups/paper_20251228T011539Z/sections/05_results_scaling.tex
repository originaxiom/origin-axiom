\section{Results: Residual and Scaling Tests}

\subsection{Residual with and without constraint}
\begin{figure}[t]
  \centering
  \includegraphics[width=0.92\linewidth]{../outputs/figures/figA_mode_sum_residual.pdf}
  \caption{\textbf{(Draft)} Mode-sum residual amplitude in code units before and after applying the global cancellation constraint.}
  \label{fig:figA}
\end{figure}

\subsection{Scaling with \texorpdfstring{$\eps$}{epsilon}}
\begin{figure}[t]
  \centering
  \includegraphics[width=0.92\linewidth]{../outputs/figures/figB_scaling_epsilon.pdf}
  \caption{\textbf{(Draft)} Constrained residual amplitude versus \$\eps\$ across a parameter sweep.}
  \label{fig:figB}
\end{figure}

\subsection{Scaling with cutoff}
\begin{figure}[t]
  \centering
  \includegraphics[width=0.92\linewidth]{../outputs/figures/figC_scaling_cutoff.pdf}
  \caption{\textbf{(Draft)} Constrained residual amplitude versus cutoff scale for fixed \$\eps\$ and mode budget.}
  \label{fig:figC}
\end{figure}

\subsection{Scaling with number of modes}
\begin{figure}[t]
  \centering
  \includegraphics[width=0.92\linewidth]{../outputs/figures/figD_scaling_modes.pdf}
  \caption{\textbf{(Draft)} Constrained residual amplitude versus number of included modes.}
  \label{fig:figD}
\end{figure}
