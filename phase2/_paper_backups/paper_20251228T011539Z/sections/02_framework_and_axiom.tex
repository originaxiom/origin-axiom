% ============================================================
% Origin Axiom — Phase 2
% Section 2: Framework and Axiom
% ============================================================

\section{Framework and Global Constraint}
\label{sec:framework}

\subsection{Motivation and Scope}

The Origin Axiom (\OA{}) is introduced as a global consistency condition acting on collective mode sums, rather than as a modification of local quantum field dynamics.
Its purpose is to regulate the net amplitude arising from large ensembles of zero-point modes when destructive interference is nearly exact.
Unlike conventional renormalization prescriptions, the axiom does not remove divergences term by term, nor does it assume cancellations enforced by symmetries such as supersymmetry.
Instead, it constrains the \emph{total} contribution of many modes once interference effects are taken into account.

Throughout this phase we adopt a deliberately minimal setting.
We consider finite ensembles of modes with prescribed frequencies and phases, interpreted as a regulated proxy for vacuum fluctuations.
No assumptions are made regarding the microscopic origin of these modes, their statistics, or their embedding in a specific quantum field theory.
The objective is to isolate and test the scaling behavior induced purely by the global constraint.

\subsection{Constrained Mode Sum}

We model the vacuum contribution as a sum over $N$ modes,
\begin{equation}
A \;\equiv\; \sum_{k=1}^{N} a_k \, ,
\end{equation}
where each mode contribution $a_k$ is taken to be of the form
\begin{equation}
a_k \;=\; \omega_k \, e^{i \phi_k} .
\end{equation}
Here $\omega_k > 0$ denotes a characteristic frequency scale, while $\phi_k$ represents an effective phase.
The precise physical interpretation of $\phi_k$ is left open; it may encode geometric, topological, or dynamical information beyond the scope of the present work.

In the absence of correlations, random phases would generically lead to $|A| \sim \sqrt{N}$.
Conversely, exact phase pairing could yield $A = 0$ identically.
The regime of interest lies between these extremes: configurations in which destructive interference is nearly exact but not perfect.

\subsection{Statement of the Origin Axiom}

The \OA{} asserts that the collective amplitude $A$ is subject to a global lower bound,
\begin{equation}
\label{eq:origin_axiom}
|A| \;\ge\; \eps ,
\end{equation}
where $\eps > 0$ is a small but finite parameter.
This bound is not imposed on individual modes, nor does it depend on $N$ or the ultraviolet cutoff.
Rather, it restricts the \emph{final} result of the collective sum.

Operationally, the axiom may be viewed as excluding configurations in which destructive interference would cancel the vacuum contribution exactly.
Importantly, Eq.~\eqref{eq:origin_axiom} does not specify \emph{how} the bound is enforced dynamically.
It functions as a global admissibility condition on the space of allowed configurations.

\subsection{Residual Energy Interpretation}

We define the residual amplitude
\begin{equation}
A_{\mathrm{res}} \;\equiv\; \max\!\bigl(|A|,\, \eps\bigr),
\end{equation}
and associate an effective vacuum energy density with the squared residual,
\begin{equation}
\rho_{\mathrm{vac}}^{\mathrm{eff}} \;\propto\; A_{\mathrm{res}}^{\,2}.
\end{equation}
The proportionality constant depends on normalization conventions and volume factors, which are treated explicitly in the numerical implementation.
For the purposes of scaling analysis, only the dependence on $N$, cutoff, and $\eps$ is relevant.

This identification is motivated by the observation that energy densities in quantum field theory are quadratic in field amplitudes.
However, we emphasize that $\rho_{\mathrm{vac}}^{\mathrm{eff}}$ should be regarded as an emergent, coarse-grained quantity rather than a microscopic vacuum expectation value.

\subsection{Nonlocal Character and Limitations}

The \OA{} is intrinsically nonlocal in the sense that it constrains a global sum over modes.
It does not arise from a local operator insertion, nor can it be implemented as a modification of the Lagrangian density at a point.
As such, it lies outside the standard framework of effective field theory.

This nonlocality is not assumed to violate causality or locality of observables.
Rather, the axiom is interpreted as a global consistency condition, potentially analogous to selection rules or superselection constraints.
Determining whether such a condition can emerge from a deeper theory remains an open problem.

Finally, we stress that the axiom is introduced here as a hypothesis to be tested.
Phase~2 is concerned solely with its internal consistency, scaling behavior, and phenomenological viability.
Questions of microscopic origin and fundamental justification are deferred to future work.

% ============================================================
% End of 02_framework_and_axiom.tex
% ============================================================
