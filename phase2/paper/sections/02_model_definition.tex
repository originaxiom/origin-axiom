% paper/sections/02_model_definition.tex

\section{Model definition and implementation}
\label{sec:framework}
\label{sec:model_definition}

This section defines the Phase~II testbeds and, crucially, the exact implementation of the Origin Axiom constraint used throughout.
The intent is reproducibility: every quantity appearing in the claims of Secs.~\ref{sec:claim_c21}--\ref{sec:claim_c23} is defined here.

\subsection{Core object: a complex scalar lattice field}
We work with a complex scalar field $\phi(t,\mathbf{x})\in\mathbb{C}$ defined on a cubic periodic lattice of size $N^3$ with lattice spacing set to unity in code units.
We denote the lattice volume by $V=N^3$.
The baseline (unconstrained) dynamics are a standard discretized scalar field evolution; the details of the integrator and potential terms are fixed by the reference implementation shipped with the Phase~II engine.
The key point for Phase~II is not the specific microphysical Lagrangian, but that the baseline evolution admits near-complete destructive interference of the global mode in generic initializations.

We define the global complex amplitude as the spatial mean (zero mode)
\begin{equation}
A(t) \;\equiv\; \frac{1}{V}\sum_{\mathbf{x}} \phi(t,\mathbf{x}) \,.
\label{eq:A_def}
\end{equation}
In the absence of any constraint, $A(t)$ can drift arbitrarily close to $0$ due to cancellation across lattice sites and phases.

\subsection{Origin Axiom constraint: hard floor on the global amplitude}
The Origin Axiom is implemented as a hard inequality constraint
\begin{equation}
|A(t)| \;\ge\; \varepsilon \,,
\qquad \varepsilon>0 \text{ fixed.}
\label{eq:floor_constraint}
\end{equation}
At each integration step, after the baseline update produces a tentative field $\phi_{\text{trial}}$, we compute
\begin{equation}
A_{\text{trial}} \equiv \frac{1}{V}\sum_{\mathbf{x}} \phi_{\text{trial}}(t,\mathbf{x}) \,.
\end{equation}
If $|A_{\text{trial}}|\ge\varepsilon$, we accept $\phi_{\text{trial}}$.
If $|A_{\text{trial}}|<\varepsilon$, we apply a \emph{minimal uniform correction} that shifts only the $k=0$ mode:
\begin{equation}
\phi(t,\mathbf{x}) \;=\; \phi_{\text{trial}}(t,\mathbf{x}) \;+\; \Delta \,,
\label{eq:uniform_shift}
\end{equation}
where $\Delta\in\mathbb{C}$ is chosen such that the corrected global amplitude satisfies $|A|=\varepsilon$ while preserving the phase direction of $A_{\text{trial}}$ when possible.
A concrete choice (used in the reference implementation) is
\begin{equation}
\Delta \;\equiv\; \left(\varepsilon - |A_{\text{trial}}|\right)\,e^{i\,\arg(A_{\text{trial}})} \,,
\qquad \text{for } A_{\text{trial}}\neq 0,
\label{eq:delta_choice}
\end{equation}
and an arbitrary fixed phase direction when $A_{\text{trial}}=0$ (this edge case is measure-zero in floating-point dynamics but is defined for completeness).
Because $\Delta$ is spatially uniform, Eq.~\eqref{eq:uniform_shift} modifies only the zero mode and leaves all nonzero-$k$ structure unchanged.

We emphasize that this is an \emph{algorithmic} enforcement of the axiom.
Phase~II does not claim this is derived from a local action; rather, it is a controlled implementation whose stability and consequences we test.

\subsection{Diagnostics: residual, energy, and enforcement statistics}
We record the following diagnostics for each run:

\paragraph{Amplitude floor diagnostics.}
\begin{itemize}
  \item $|A(t)|$ over time, and its minimum value over a run.
  \item The number of steps for which the correction~\eqref{eq:uniform_shift} is applied (``constraint hits'').
  \item The magnitude $|\Delta|$ of each applied correction and its time distribution.
\end{itemize}

\paragraph{Energy diagnostics.}
We measure total energy $E(t)$ using the same discrete Hamiltonian/energy functional as the baseline evolution (kinetic + gradient + potential terms as implemented).
Because the constraint introduces a uniform shift, it can inject (or remove) a small amount of energy relative to the unconstrained baseline; this is part of the mechanism and is measured rather than assumed away.
We report both $E(t)$ and the difference between constrained and unconstrained runs with matched initial conditions.

\paragraph{Residual energy proxy.}
For the vacuum module we define a residual energy shift
\begin{equation}
\Delta E(\theta) \;\equiv\; E_{\text{constrained}}(\theta) \;-\; E_{\text{free}}(\theta)\,,
\label{eq:deltaE_def}
\end{equation}
evaluated after transients have decayed (late-time mean or end-of-run value, depending on the run protocol).
The dependence on $\theta$ is defined below.

\subsection{Phase parameter $\theta$ and the scan protocol}
Phase~II introduces a single phase control parameter $\theta$ that enters the toy vacuum module through a $\theta$-dependent mapping (e.g.\ an effective mass scale or coupling in the scalar sector), fixed by configuration.
We treat $\theta$ in two ways:
\begin{itemize}
  \item \emph{Scan mode:} $\theta$ is scanned over a fixed interval (typically $[0,2\pi]$) to obtain $\Delta E(\theta)$ and assess sensitivity.
  \item \emph{Anchor mode:} a fixed $\theta^\star$ is supplied as an external input motivated by a separate flavor-inspired procedure, and is used to generate the fiducial pipeline outputs carried into FRW.
\end{itemize}
The paper treats $\theta^\star$ as an input, not an axiom prediction.

\subsection{Toy FRW mapping}
To test cosmological viability (Claim~C2.3), we map the vacuum residual into an effective constant term in a Friedmann--Robertson--Walker background.
We define an effective vacuum energy density proxy $\rho_\Lambda(\theta)$ from the lattice residual (details depend on normalization; the reference implementation uses a fixed conversion in code units).
We then evolve the scale factor $a(t)$ under a standard FRW equation in normalized units,
\begin{equation}
H^2(t) \;\equiv\; \left(\frac{\dot a}{a}\right)^2 \;=\;
\frac{\Omega_r}{a^4(t)} \;+\; \frac{\Omega_m}{a^3(t)} \;+\; \Omega_\Lambda(\theta)\,,
\label{eq:frw}
\end{equation}
where $\Omega_\Lambda(\theta)$ is proportional to $\rho_\Lambda(\theta)$ in the chosen normalization.
We compare trajectories with $\Omega_\Lambda(\theta^\star)$ to a matched reference trajectory (e.g.\ $\Omega_\Lambda=0$ or another baseline), and quantify the relative deviation in $a(t)$ over the integration interval.

\subsection{Paired-run discipline and controls}
For each reported claim, we adopt a paired-run discipline:
the constrained run is always compared against a baseline run with identical initialization and numerical settings but without enforcing~\eqref{eq:floor_constraint}.
This isolates the effect of the axiom enforcement from the underlying dynamics and from stochastic initialization variance.

\subsection{Where the claim evidence lives}
For clarity:
\begin{itemize}
  \item Claim~C2.1 uses a representative constrained-vs-free vacuum run to demonstrate existence of a stable residual (Fig.~A).
  \item Claim~C2.2 uses systematic sweeps in $\varepsilon$ and discretization/UV controls to demonstrate robustness and suppression (Figs.~B--D).
  \item Claim~C2.3 uses the FRW integration driven by the fiducial residual (Fig.~E).
\end{itemize}
Precise artifact provenance (scripts, configs, run IDs, and figure build rules) is documented in Sec.~\ref{sec:reproducibility} and Appendix~\ref{app:run_manifest}.