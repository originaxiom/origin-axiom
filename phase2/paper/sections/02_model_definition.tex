% === P2-S4b.1 STRUCTURE CHECKLIST (keep; delete only after Phase 2 lock) ===
% This section must define ONLY:
% (i) the vacuum mode-sum / discretization objects and parameters,
% (ii) the global floor constraint (where/how applied),
% (iii) the residual scalar/observable used throughout Claims C2.1–C2.3,
% (iv) the FRW mapping as a *consistency test*, not a prediction,
% (v) a single sentence pointing to provenance: outputs/figures + Appendix run manifest.
% No new physics claims; no interpretive leaps.
% === END CHECKLIST ===

% paper/sections/02_model_definition.tex

\section{Model definition and implementation}
\label{sec:framework}

This section defines the Phase~II numerical testbed and the exact algorithmic form of the Origin Axiom constraint used throughout.
The intent is reproducibility: every quantity referenced by Claims~C2.1--C2.3 is defined here, while the remaining microphysical and numerical choices (integrator, potential, and update schedule) are treated as fixed by the Phase~II reference implementation and configuration.

\subsection{Core object: a complex scalar lattice field}
We work with a complex scalar field $\phi(t,\mathbf{x})\in\mathbb{C}$ defined on a cubic periodic lattice of size $N^3$ with lattice spacing set to unity in code units.
We denote the lattice volume by $V=N^3$.
The baseline (unconstrained) evolution is a discretized scalar-field update rule (time integrator and potential terms fixed by the Phase~II reference implementation).
In Phase~II we do not interpret this choice as a derived effective description of the Standard Model vacuum; it is used as a controlled testbed in which a global (zero-mode) diagnostic can, in the unconstrained baseline, approach $0$ through cancellation.

We define the global complex diagnostic amplitude as the spatial mean (zero mode)
\begin{equation}
A(t) \;\equiv\; \frac{1}{V}\sum_{\mathbf{x}} \phi(t,\mathbf{x}) \,.
\label{eq:A_def}
\end{equation}
In the absence of any constraint, $A(t)$ can approach $0$ due to cancellation across lattice sites and phases.

\subsection{Origin Axiom constraint: hard floor on the global amplitude}
The Origin Axiom is implemented as a hard inequality constraint on the diagnostic amplitude:
\begin{equation}
|A(t)| \;\ge\; \varepsilon \,,
\qquad \varepsilon>0 \text{ fixed.}
\label{eq:floor_constraint}
\end{equation}
At each integration step, after the baseline update produces a tentative field $\phi_{\text{trial}}$, we compute
\begin{equation}
A_{\text{trial}} \equiv \frac{1}{V}\sum_{\mathbf{x}} \phi_{\text{trial}}(t,\mathbf{x}) \,.
\end{equation}
If $|A_{\text{trial}}|\ge\varepsilon$, we accept $\phi_{\text{trial}}$.
If $|A_{\text{trial}}|<\varepsilon$, we apply a spatially uniform correction that shifts only the $k=0$ mode:
\begin{equation}
\phi(t,\mathbf{x}) \;=\; \phi_{\text{trial}}(t,\mathbf{x}) \;+\; \Delta \,,
\label{eq:uniform_shift}
\end{equation}
where $\Delta\in\mathbb{C}$ is chosen so that the corrected amplitude satisfies $|A(t)|=\varepsilon$.
Because the correction is uniform in $\mathbf{x}$, Eq.~\eqref{eq:uniform_shift} modifies only the zero mode and leaves all nonzero-$k$ structure unchanged.

A concrete choice (used in the reference implementation) is:
\begin{equation}
\Delta \;\equiv\; \left(\varepsilon - |A_{\text{trial}}|\right)\,u(A_{\text{trial}}),
\qquad
u(z)\equiv
\begin{cases}
z/|z|, & z\neq 0,\\
u_0,   & z=0,
\end{cases}
\label{eq:delta_choice}
\end{equation}
where $u_0$ is a fixed unit complex number (a fixed phase convention) used only for the $A_{\text{trial}}=0$ edge case.
With this definition, the corrected diagnostic satisfies
\begin{equation}
A(t)=A_{\text{trial}}+\Delta
\quad\Rightarrow\quad
|A(t)|=\varepsilon
\end{equation}
whenever the constraint is active.

We emphasize that Phase~II treats this as an \emph{algorithmic} enforcement rule defining a constrained evolution map.
No claim is made that Eq.~\eqref{eq:floor_constraint} follows from a local action or a known symmetry; Phase~II evaluates stability and consequences of the specified constrained implementation.

\subsection{Diagnostics: residual, energy, and enforcement statistics}
We record the following diagnostics for each run.

\paragraph{Amplitude-floor diagnostics.}
\begin{itemize}
  \item The time series $|A(t)|$ and the minimum value attained over a run.
  \item The number of time steps on which the correction~\eqref{eq:uniform_shift} is applied (``constraint hits'').
  \item The magnitude $|\Delta|$ of each applied correction and its time distribution.
\end{itemize}

\paragraph{Energy diagnostics.}
We measure total energy $E(t)$ using the same discrete energy functional as the baseline evolution (kinetic + gradient + potential terms as implemented).
Because the constraint introduces a uniform shift, it can inject or remove energy relative to the unconstrained baseline; Phase~II measures this difference rather than assuming conservation under the constrained map.
We report both $E(t)$ and differences between constrained and unconstrained runs with matched initial conditions.

\paragraph{Residual energy proxy.}
For the vacuum module we define a residual energy shift
\begin{equation}
\Delta E(\theta) \;\equiv\; E_{\text{constrained}}(\theta) \;-\; E_{\text{free}}(\theta)\,,
\label{eq:deltaE_def}
\end{equation}
evaluated after transients have decayed (late-time mean or end-of-run value, depending on the run protocol, fixed by configuration).
This is an operational diagnostic of the effect of enforcing~\eqref{eq:floor_constraint} in the specified implementation; it is not assumed to be a conserved physical vacuum energy in the QFT sense.

\subsection{Phase parameter $\theta$ and the scan protocol}
Phase~II introduces a single phase control parameter $\theta$ that enters the vacuum testbed through a $\theta$-dependent configuration mapping (e.g.\ through an effective mass scale or coupling in the scalar sector, as defined by the Phase~II configuration files).
We use $\theta$ in two modes:
\begin{itemize}
  \item \emph{Scan mode:} $\theta$ is scanned over a fixed interval (typically $[0,2\pi]$) to obtain $\Delta E(\theta)$ and assess sensitivity.
  \item \emph{Anchor mode:} a fixed $\theta^\star$ is supplied as an external input motivated by a separate procedure outside Phase~II, and is used for a fiducial pipeline run carried into the FRW module.
\end{itemize}
Phase~II treats $\theta^\star$ as an input, not an axiom prediction.

\subsection{Toy FRW mapping (consistency test)}
To test cosmological viability (Claim~C2.3), we map the vacuum residual into an effective constant contribution in a Friedmann--Robertson--Walker background.
In Phase~II this mapping is defined operationally in code units: we define a proxy vacuum density $\rho_\Lambda(\theta)$ from the residual diagnostic using a fixed normalization rule (held constant across Phase~II runs for auditability), and we define $\Omega_\Lambda(\theta)$ proportional to that proxy in the FRW integrator's normalization.

We then evolve the scale factor $a(t)$ under a standard flat FRW equation in normalized units,
\begin{equation}
H^2(t) \;\equiv\; \left(\frac{\dot a}{a}\right)^2 \;=\;
\frac{\Omega_r}{a^4(t)} \;+\; \frac{\Omega_m}{a^3(t)} \;+\; \Omega_\Lambda(\theta)\,,
\label{eq:frw}
\end{equation}
where $(\Omega_r,\Omega_m)$ are fixed reference parameters in the FRW module.
We compare trajectories with $\Omega_\Lambda(\theta^\star)$ to a matched reference trajectory (e.g.\ $\Omega_\Lambda=0$ or a fixed baseline) over the same integration interval and numerical settings.
This FRW step is treated strictly as an end-to-end \emph{consistency} embedding, not as a physical prediction or an EFT matching.

\subsection{Paired-run discipline and controls}
For each reported claim, we adopt a paired-run discipline:
the constrained run is always compared against a baseline run with identical initialization and numerical settings but without enforcing~\eqref{eq:floor_constraint}.
This isolates the effect of the axiom enforcement from the underlying baseline dynamics and from stochastic initialization variance.

\subsection{Where the claim evidence lives}
For clarity:
\begin{itemize}
  \item Claim~C2.1 uses a representative constrained-vs-free vacuum run to demonstrate existence of a stable residual (Fig.~A).
  \item Claim~C2.2 uses systematic sweeps in $\varepsilon$ and discretization/UV controls to demonstrate robustness and suppression (Figs.~B--D).
  \item Claim~C2.3 uses the FRW integration driven by the fiducial residual (Fig.~E).
\end{itemize}
Precise artifact provenance (scripts, configs, run IDs, and figure build rules) is documented in Sec.~\ref{sec:reproducibility} and Appendix~\ref{app:run_manifest}.