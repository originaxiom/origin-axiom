% paper/sections/04_claim_C2_2_robustness.tex

\section{Claim C2.2: Robustness and suppression}
\label{sec:claim_c22}

\paragraph{Claim (C2.2).}
The Phase~II residual induced by enforcing $|A|\ge\varepsilon$ remains numerically well-behaved under systematic sweeps in (i) the floor scale $\varepsilon$ and (ii) discretization/UV controls.
In the explored regime, the induced effects are structured and percent-level rather than runaway.
Evidence is provided by three parameter sweeps summarized in Figs.~\ref{fig:scaling_epsilon}--\ref{fig:scaling_modes}.

\subsection{Sweep design}
We run families of paired constrained-vs-free simulations as in Sec.~\ref{sec:claim_c21}, varying one control at a time while holding all other configuration settings fixed:
\begin{itemize}
  \item \textbf{$\varepsilon$-sweep:} vary the non-cancellation floor $\varepsilon$ over a specified range and measure the residual diagnostic (Fig.~\ref{fig:scaling_epsilon}).
  \item \textbf{UV/cutoff control:} vary the effective ultraviolet control parameter used by the implementation (e.g.\ spectral cutoff / smoothing / resolution control, as defined by the Phase~II engine) and measure stability of the residual (Fig.~\ref{fig:scaling_cutoff}).
  \item \textbf{Mode-count / discretization control:} vary the discretization degree-of-freedom proxy used by the scan (e.g.\ number of modes retained, grid resolution proxy, or an equivalent implementation-defined dial) and measure residual behavior (Figs.~\ref{fig:scaling_cutoff}--\ref{fig:scaling_modes}).
\end{itemize}
For each family, the reported quantity is the late-time residual energy proxy $\Delta E$ (Eq.~\eqref{eq:deltaE_def}) or its implementation-defined equivalent, computed from matched constrained and free runs.

\subsection{Results: controlled scaling and stable behavior}

\begin{figure}[t]
  \centering
  \includegraphics[width=0.92\linewidth]{figB_scaling_epsilon.pdf}
  \caption{
  \textbf{$\varepsilon$-sweep (Claim C2.2).}
  Residual diagnostic versus the enforced floor scale $\varepsilon$.
  The residual varies smoothly with $\varepsilon$ and does not exhibit runaway behavior in the explored range.
  }
  \label{fig:scaling_epsilon}
\end{figure}

Figure~\ref{fig:scaling_epsilon} shows the residual diagnostic as $\varepsilon$ is varied.
The dependence is smooth and monotonic in the explored range, consistent with the interpretation that the floor sets the scale of the enforced non-cancellation remainder.
Importantly, the dynamics remain stable across the sweep: the constrained runs do not exhibit numerical blow-up, and the enforcement statistics remain bounded.

\begin{figure}[t]
  \centering
  \includegraphics[width=0.92\linewidth]{figC_scaling_cutoff.pdf}
  \caption{
  \textbf{Discretization/UV control sweep (Claim C2.2).}
  Residual diagnostic as a function of the engine's ultraviolet/discretization control parameter (implementation-defined).
  The residual remains well-behaved and changes in a structured manner, indicating that the observed remainder is not a fragile artifact of a single cutoff choice.
  }
  \label{fig:scaling_cutoff}
\end{figure}

Figure~\ref{fig:scaling_cutoff} shows the residual diagnostic under a sweep of the ultraviolet/discretization control.
While the residual value shifts as expected when changing UV structure, the behavior remains stable and does not exhibit pathological sensitivity.
This supports the interpretation that the mechanism is not merely a numerical coincidence at a special cutoff.

\begin{figure}[t]
  \centering
  \includegraphics[width=0.92\linewidth]{figD_scaling_modes.pdf}
  \caption{
  \textbf{Mode/discretization sweep (Claim C2.2).}
  Residual diagnostic as a function of a mode-count / degrees-of-freedom proxy (implementation-defined).
  Across the explored settings, the induced effect remains controlled and does not grow without bound with increasing degrees of freedom.
  }
  \label{fig:scaling_modes}
\end{figure}

Figure~\ref{fig:scaling_modes} reports the residual diagnostic against a mode-count or degrees-of-freedom proxy.
Across the tested range, the residual remains controlled and does not diverge as the effective number of degrees of freedom changes.
Within Phase~II scope, this is the key robustness point: the mechanism survives discretization changes without producing explosive scaling.

\subsection{Interpretation and quantitative ``nectar''}
The Phase~II robustness sweeps establish two practically important facts:

\paragraph{(i) The constraint behaves as a tunable floor, not a destabilizing force.}
Varying $\varepsilon$ shifts the residual in a predictable, smooth way (Fig.~\ref{fig:scaling_epsilon}), consistent with the axiom functioning as a controlled non-cancellation floor rather than as an uncontrolled injection of energy.

\paragraph{(ii) The observed residual is not a single-cutoff artifact.}
Changes in the engine's UV/discretization controls alter the residual only modestly and without instabilities (Figs.~\ref{fig:scaling_cutoff}--\ref{fig:scaling_modes}).
In the explored regime, the induced effect remains small (percent-level modulation in the relevant scan summaries), supporting the claim that the pipeline is stable and reproducible rather than parameter-explosive.

These results do not constitute a continuum-limit proof.
They \emph{do} demonstrate that the Phase~II engine produces a robust, auditable residual across a nontrivial range of numerical controls, which is the necessary condition for proceeding to the FRW viability test (Claim~C2.3).