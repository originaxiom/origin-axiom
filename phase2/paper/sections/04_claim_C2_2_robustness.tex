% === P2-S4b CLAIM STRUCTURE CHECKLIST (keep; delete only after Phase 2 lock) ===
% File: phase2/paper/sections/04_claim_C2_2_robustness.tex
% Must contain: (i) \paragraph{Claim (C2.2).} with a single-sentence claim,
% (ii) explicit figure pointer(s) (e.g. Fig.~A / Figs.~B--D / Fig.~E),
% (iii) explicit pointer to Appendix run manifest (Appendix~\ref{app:run_manifest}),
% (iv) explicit non-claims boundary sentence.
% No new physics claims; structure/clarity only.
% === END CHECKLIST ===

% paper/sections/04_claim_C2_2_robustness.tex

\section{Claim C2.2: Robustness under numerical controls}
\label{sec:claim_c22}

\paragraph{Claim (C2.2).}
Across systematic sweeps in (i) the floor scale $\varepsilon$ and (ii) discretization/UV-control parameters in the Phase~II engine, the residual proxy induced by enforcing $|A(t)|\ge\varepsilon$ remains numerically well-behaved and varies smoothly over the explored regime.
Evidence is provided by the sweep summaries in Figs.~\ref{fig:scaling_epsilon}--\ref{fig:scaling_modes}; exact run provenance is given by Appendix~\ref{app:run_manifest}.

\subsection{Sweep design}
We run families of paired constrained-vs-free simulations as in Sec.~\ref{sec:claim_c21}, varying one control at a time while holding all other configuration settings fixed.
At each sweep point, the constrained run and its free baseline share identical initialization and numerical settings, differing only by enforcement of the floor constraint.
We consider:
\begin{itemize}
  \item \textbf{$\varepsilon$-sweep:} vary the non-cancellation floor $\varepsilon$ and measure the residual proxy (Fig.~\ref{fig:scaling_epsilon}).
  \item \textbf{UV/discretization control sweep:} vary the implementation-defined ultraviolet/discretization control parameter (e.g.\ cutoff/smoothing/resolution dial, as defined by the Phase~II engine) and measure the residual proxy (Fig.~\ref{fig:scaling_cutoff}).
  \item \textbf{Mode-count / degrees-of-freedom sweep:} vary the implementation-defined degrees-of-freedom proxy (e.g.\ number of modes retained or grid-resolution proxy) and measure the residual proxy (Fig.~\ref{fig:scaling_modes}).
\end{itemize}
For each family, the reported quantity is the residual energy proxy $\Delta E$ (Eq.~\eqref{eq:deltaE_def}) or its implementation-defined equivalent, computed from matched constrained and free runs using the Phase~II run protocol (late-time average or end-of-run value as specified in the run artifacts referenced in Appendix~\ref{app:run_manifest}).

\subsection{Results: controlled variation and numerical stability}

\begin{figure}[t]
  \centering
  \includegraphics[width=0.92\linewidth]{figB_scaling_epsilon.pdf}
  \caption{
  \textbf{$\varepsilon$-sweep (Claim C2.2).}
  Residual proxy versus the enforced floor scale $\varepsilon$.
  The proxy varies smoothly with $\varepsilon$ and does not exhibit runaway behavior within the explored range.
  }
  \label{fig:scaling_epsilon}
\end{figure}

Figure~\ref{fig:scaling_epsilon} shows the residual proxy as $\varepsilon$ is varied.
The dependence is smooth over the explored range, consistent with $\varepsilon$ setting the operational scale of the enforced non-cancellation remainder in this implementation.
Across the sweep, runs complete without numerical blow-up and enforcement statistics remain bounded, providing an operational notion of ``numerically well-behaved'' for Phase~II.

\begin{figure}[t]
  \centering
  \includegraphics[width=0.92\linewidth]{figC_scaling_cutoff.pdf}
  \caption{
  \textbf{UV/discretization control sweep (Claim C2.2).}
  Residual proxy as a function of an implementation-defined ultraviolet/discretization control parameter.
  The proxy remains bounded and changes in a structured manner across the explored settings.
  }
  \label{fig:scaling_cutoff}
\end{figure}

Figure~\ref{fig:scaling_cutoff} reports the residual proxy under the UV/discretization control sweep.
While the proxy value changes as the UV/discretization dial is varied, the behavior remains bounded and the runs remain stable within the explored regime.
This indicates that the observed residual proxy is not confined to a single special numerical setting.

\begin{figure}[t]
  \centering
  \includegraphics[width=0.92\linewidth]{figD_scaling_modes.pdf}
  \caption{
  \textbf{Mode/discretization sweep (Claim C2.2).}
  Residual proxy as a function of an implementation-defined mode-count / degrees-of-freedom proxy.
  Across the explored settings, the proxy remains controlled and does not exhibit unstable growth.
  }
  \label{fig:scaling_modes}
\end{figure}

Figure~\ref{fig:scaling_modes} reports the residual proxy against a mode-count or degrees-of-freedom proxy.
Across the tested range, the proxy remains controlled and does not diverge as the effective number of degrees of freedom is varied.
Within Phase~II scope, this is the key robustness point: the constrained mechanism survives nontrivial discretization changes without producing unstable scaling in the reported proxy.

\subsection{Interpretation within Phase~II scope}
The Phase~II sweeps establish two practically important facts:

\paragraph{(i) The constraint behaves as a tunable floor, not a destabilizing intervention.}
Varying $\varepsilon$ shifts the residual proxy smoothly (Fig.~\ref{fig:scaling_epsilon}), consistent with the axiom functioning as a controlled non-cancellation floor rather than an uncontrolled numerical perturbation.

\paragraph{(ii) The residual proxy persists across multiple numerical controls.}
Changes in the engine's UV/discretization dials and degrees-of-freedom proxies modify the residual proxy but do not trigger numerical pathologies within the explored regime (Figs.~\ref{fig:scaling_cutoff}--\ref{fig:scaling_modes}).
This is the minimum robustness criterion required before attempting an end-to-end embedding (Claim~C2.3).

\noindent\textbf{Non-claims boundary (C2.2).}
These sweeps do \emph{not} establish a continuum limit, renormalization-group invariance, or universality across discretization schemes; they do \emph{not} claim validity beyond the explored parameter ranges; and they do \emph{not} by themselves justify interpreting the residual proxy as a physical vacuum energy.
They only support the bounded statement that the constrained implementation remains stable and produces a reproducible, smoothly varying residual proxy under the stated numerical controls.