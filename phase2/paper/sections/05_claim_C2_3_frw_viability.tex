% === P2-S4b CLAIM STRUCTURE CHECKLIST (keep; delete only after Phase 2 lock) ===
% File: phase2/paper/sections/05_claim_C2_3_frw_viability.tex
% Must contain: (i) \paragraph{Claim (C2.3).} with a single-sentence claim,
% (ii) explicit figure pointer(s) (e.g. Fig.~A / Figs.~B--D / Fig.~E),
% (iii) explicit pointer to Appendix run manifest (Appendix~\ref{app:run_manifest}),
% (iv) explicit non-claims boundary sentence.
% No new physics claims; structure/clarity only.
% === END CHECKLIST ===

% paper/sections/05_claim_C2_3_frw_viability.tex

\section{Claim C2.3: FRW viability}
\label{sec:cosmology}
\label{sec:claim_c23}

\paragraph{Claim (C2.3).}
Mapping the Phase~II residual into an effective constant contribution $\Omega_\Lambda(\theta)$ yields a smooth, modest perturbation to Friedmann--Robertson--Walker evolution in the explored regime, demonstrating end-to-end consistency of the toy pipeline.
Evidence is provided by the paired FRW trajectory comparison in Fig.~\ref{fig:frw_comparison}.

\subsection{From residual to effective $\Omega_\Lambda(\theta)$}
The vacuum module produces a residual diagnostic $\Delta E(\theta)$ (Eq.~\eqref{eq:deltaE_def}) or an equivalent implementation-defined residual measure.
To drive a cosmological background test, we map this residual into an effective vacuum density proxy $\rho_\Lambda(\theta)$ using a fixed conversion in code units (held constant across all Phase~II runs for auditability).
We then define
\begin{equation}
\Omega_\Lambda(\theta) \;\propto\; \rho_\Lambda(\theta)\,,
\label{eq:omega_lambda_def}
\end{equation}
with the proportionality set by the normalization of the FRW integrator used in the Phase~II engine.

We emphasize the scope: the mapping is \emph{not} claimed to be a physically derived EFT matching; it is a controlled toy embedding that tests whether the residual can be consistently interpreted as a smooth background term without generating instabilities or pathological expansion histories.

\subsection{FRW integration protocol}
We integrate the FRW system in the form
\begin{equation}
H^2(a) \;=\; \frac{\Omega_r}{a^4} + \frac{\Omega_m}{a^3} + \Omega_\Lambda(\theta)\,,
\label{eq:frw_again}
\end{equation}
with fixed $(\Omega_r,\Omega_m)$ chosen as a reference background in code units.
We compare two trajectories:
\begin{enumerate}
  \item \textbf{Reference:} a baseline trajectory with $\Omega_\Lambda$ set to the reference value used by the Phase~II engine (often $0$ in the toy setup, or a fixed baseline).
  \item \textbf{Axiom-driven:} a trajectory with $\Omega_\Lambda(\theta^\star)$ set by the residual produced at the anchored phase $\theta^\star$.
\end{enumerate}
Both trajectories are integrated over the same time/scale-factor interval with identical numerical settings.

\subsection{Result: modest, smooth deviation in $a(t)$}
\begin{figure}[t]
  \centering
  \includegraphics[width=0.92\linewidth]{figE_frw_comparison.pdf}
  \caption{
  \textbf{Claim C2.3 evidence: FRW viability under axiom-driven $\Omega_\Lambda(\theta^\star)$.}
  Comparison of FRW scale factor evolution $a(t)$ (or $a$ versus integration time/step, as implemented) for a reference trajectory and an axiom-driven trajectory using $\Omega_\Lambda(\theta^\star)$ derived from the Phase~II residual.
  The deviation remains smooth and modest in the explored regime, indicating that the residual can be embedded as an effective background term without instability.
  }
  \label{fig:frw_comparison}
\end{figure}

Figure~\ref{fig:frw_comparison} shows that introducing $\Omega_\Lambda(\theta^\star)$ derived from the Phase~II residual produces a small but coherent perturbation of the FRW trajectory.
The effect is \emph{modest} (order-percent deviation in the representative Phase~II configuration) and does not induce runaway expansion, oscillatory pathology, or numerical stiffness in the integrator.
Within Phase~II scope, this establishes the minimal viability criterion: the axiom-driven residual can be carried into a cosmological background module without breaking the pipeline.

\subsection{Interpretation and boundaries}
Claim~C2.3 should be read as an end-to-end \emph{consistency} demonstration, not as a quantitative cosmological prediction.
Specifically:
\begin{itemize}
  \item We do not claim the Phase~II normalization matches the observed cosmological constant.
  \item We do not claim a derived relation between $\varepsilon$ and physical vacuum energy.
  \item We do claim that, under a fixed and transparent mapping, the residual behaves like a smooth effective $\Lambda$-term in a toy FRW integrator.
\end{itemize}

This closes the Phase~II loop: the axiom generates a nonzero residual (C2.1), the residual is robust and suppressed under sweeps (C2.2), and the residual can be consistently embedded into a background expansion module without instability (C2.3).