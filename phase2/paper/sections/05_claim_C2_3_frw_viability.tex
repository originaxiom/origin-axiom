% === P2-S4b CLAIM STRUCTURE CHECKLIST (keep; delete only after Phase 2 lock) ===
% File: phase2/paper/sections/05_claim_C2_3_frw_viability.tex
% Must contain: (i) \paragraph{Claim (C2.3).} with a single-sentence claim,
% (ii) explicit figure pointer(s) (e.g. Fig.~A / Figs.~B--D / Fig.~E),
% (iii) explicit pointer to Appendix run manifest (Appendix~\ref{app:run_manifest}),
% (iv) explicit non-claims boundary sentence.
% No new physics claims; structure/clarity only.
% === END CHECKLIST ===

% paper/sections/05_claim_C2_3_frw_viability.tex

\section{Claim C2.3: FRW consistency under a fixed mapping}
\label{sec:cosmology}
\label{sec:claim_c23}

\paragraph{Claim (C2.3).}
Under a fixed and explicitly stated mapping from the Phase~II residual proxy into an effective constant contribution $\Omega_\Lambda(\theta)$, the configured toy FRW background module produces smooth, numerically stable trajectories in the explored regime, demonstrating end-to-end pipeline consistency.
Evidence is provided by the FRW comparison in Fig.~\ref{fig:frw_comparison}; exact run provenance is given by Appendix~\ref{app:run_manifest}.

\subsection{From residual proxy to an effective constant term}
The vacuum module produces a residual proxy $\Delta E(\theta)$ (Eq.~\eqref{eq:deltaE_def}) or an implementation-defined equivalent recorded in the Phase~II run artifacts.
To drive a cosmological background test, we map this residual into an effective vacuum-density proxy $\rho_\Lambda(\theta)$ using a fixed conversion in code units (held constant across all Phase~II runs for auditability).
We then define
\begin{equation}
\Omega_\Lambda(\theta) \;\propto\; \rho_\Lambda(\theta)\,,
\label{eq:omega_lambda_def}
\end{equation}
with the proportionality fixed by the normalization conventions of the FRW integrator used in the Phase~II engine (as recorded in the configuration referenced by Appendix~\ref{app:run_manifest}).

\noindent\textbf{Scope note.}
This mapping is \emph{not} claimed to be an EFT matching or a physically derived relation between $\varepsilon$ and a physical vacuum density; it is a transparent operational embedding whose purpose is to test whether the residual proxy can be carried as a smooth constant term without numerical failure of the background module.

\subsection{FRW integration protocol}
We integrate a flat FRW system in normalized units,
\begin{equation}
H^2(a) \;=\; \frac{\Omega_r}{a^4} + \frac{\Omega_m}{a^3} + \Omega_\Lambda(\theta)\,,
\label{eq:frw_again}
\end{equation}
with fixed background parameters $(\Omega_r,\Omega_m)$ chosen by configuration.
The Phase~II integrator evolves the system according to its configured independent variable (time or scale factor) and records the corresponding trajectory outputs; Fig.~\ref{fig:frw_comparison} plots the staged trajectory product produced by the Phase~II build (see Appendix~\ref{app:run_manifest} for the exact run and script entry point).
We compare two trajectories integrated over the same interval with identical numerical settings:
\begin{enumerate}
  \item \textbf{Reference:} a baseline trajectory with $\Omega_\Lambda$ set to the reference value used by the Phase~II engine (often $0$ in the toy setup, or a fixed baseline).
  \item \textbf{Axiom-driven:} a trajectory with $\Omega_\Lambda(\theta^\star)$ set by the mapped residual at the anchored phase $\theta^\star$.
\end{enumerate}
Here $\theta^\star$ is treated as an externally supplied anchor (Phase~II does not derive it); its role is to test coherent propagation of a single phase input through the pipeline.

\subsection{Result: smooth and stable trajectories under the mapped term}
\begin{figure}[t]
  \centering
  \includegraphics[width=0.92\linewidth]{figE_frw_comparison.pdf}
  \caption{
  \textbf{Claim C2.3 evidence: FRW consistency under an axiom-driven constant term.}
  Comparison of the FRW trajectory output produced by the Phase~II engine for a reference run and an axiom-driven run using $\Omega_\Lambda(\theta^\star)$ obtained from the Phase~II residual proxy under the stated fixed mapping.
  In the explored regime, the trajectories remain smooth and numerically stable (runs complete without integrator failure), indicating that the residual proxy can be embedded as an effective constant contribution without numerical breakdown of the toy background module.
  }
  \label{fig:frw_comparison}
\end{figure}

Figure~\ref{fig:frw_comparison} shows that introducing $\Omega_\Lambda(\theta^\star)$ derived from the Phase~II residual proxy changes the FRW trajectory relative to the reference run.
Within the explored regime, the evolution remains smooth and stable: runs complete without integrator failure attributable to the mapped term, and the output trajectory exhibits no discontinuities or stiffness-driven breakdown under the configured settings.
Within Phase~II scope, this establishes the minimal end-to-end viability criterion: the axiom-driven residual proxy can be carried into the background module under a fixed mapping without breaking the computational chain.

\subsection{Interpretation boundaries}
Claim~C2.3 is a pipeline \emph{consistency} test, not a cosmological prediction.
Specifically:
\begin{itemize}
  \item We do not claim the Phase~II normalization matches the observed cosmological constant.
  \item We do not claim a derived physical relation between $\varepsilon$ and $\rho_\Lambda$.
  \item We do claim that, under a fixed and auditable mapping, the residual proxy functions as a smooth constant-term input to the configured toy FRW integrator in the explored regime.
\end{itemize}

\noindent\textbf{Non-claims boundary (C2.3).}
This result does \emph{not} validate the mapping as physically correct, does \emph{not} fit cosmological datasets, and does \emph{not} establish late-time acceleration in the real universe.
It supports only the bounded statement that the Phase~II residual proxy can be embedded into the configured FRW background module as a constant contribution without inducing numerical failure in the explored regime, with full provenance given by Appendix~\ref{app:run_manifest}.