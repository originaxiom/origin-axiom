% === P2-S4b CLAIM STRUCTURE CHECKLIST (keep; delete only after Phase 2 lock) ===
% File: phase2/paper/sections/03_claim_C2_1_existence.tex
% Must contain: (i) \paragraph{Claim (C2.1).} with a single-sentence claim,
% (ii) explicit figure pointer(s),
% (iii) explicit pointer to Appendix run manifest (Appendix~\ref{app:run_manifest}),
% (iv) explicit non-claims boundary sentence.
% No new physics claims; structure/clarity only.
% === END CHECKLIST ===

% paper/sections/03_claim_C2_1_existence.tex

\section{Claim C2.1: Existence under constraint}
\label{sec:claim_c21}

\paragraph{Claim (C2.1).}
In the Phase~II lattice vacuum testbed, enforcing the Origin Axiom floor $|A(t)|\ge\varepsilon$ yields a stable, nonzero \emph{implementation-defined residual diagnostic} relative to the matched unconstrained baseline, while remaining numerically well-behaved.
Evidence is provided by a paired constrained vs.\ free run with identical initialization, summarized in Fig.~\ref{fig:mode_sum_residual}; exact provenance (run\_id(s), configuration, and scripts) is given in Appendix~\ref{app:run_manifest}.

\subsection{Experimental protocol}
We run two simulations with matched numerical settings and identical initialization (same lattice size, time step, integration length, and random seed where applicable):
\begin{enumerate}
  \item \textbf{Free baseline:} standard evolution with no global constraint.
  \item \textbf{Constrained:} identical evolution with the hard floor $|A(t)|\ge\varepsilon$ enforced at each step via the uniform correction in Eq.~\eqref{eq:uniform_shift}.
\end{enumerate}
We record the global diagnostic amplitude magnitude $|A(t)|$ over time, along with enforcement statistics (number of constraint hits and distribution of correction magnitudes).
All reported curves correspond to the tagged run artifacts referenced in Appendix~\ref{app:run_manifest}.

\subsection{Result: nonzero floor-sustained diagnostic and active enforcement}
Figure~\ref{fig:mode_sum_residual} shows the time series of the global amplitude magnitude $|A(t)|$ for the matched pair.
In the free baseline, $|A(t)|$ approaches near-zero values through cancellation (within floating-point tolerance in the representative configuration).
In the constrained run, $|A(t)|$ is prevented from falling below $\varepsilon$ and instead remains near the floor, with bounded fluctuations.

\begin{figure}[t]
  \centering
  \includegraphics[width=0.92\linewidth]{figA_mode_sum_residual.pdf}
  \caption{
  \textbf{Claim C2.1 evidence: existence under the non-cancellation floor.}
  Global diagnostic amplitude magnitude $|A(t)|$ in matched free vs.\ constrained vacuum runs.
  The free baseline reaches near-zero amplitude via destructive interference, while the constrained run remains at the imposed floor $|A|\approx\varepsilon$, demonstrating a stable nonzero diagnostic under enforcement.
  }
  \label{fig:mode_sum_residual}
\end{figure}

The nonzero diagnostic in the constrained run is actively maintained by the enforcement rule:
the floor is violated by trial updates repeatedly, and the correction~\eqref{eq:uniform_shift} is applied many times across the run (as recorded by the constraint-hit counters).
This verifies that the observed persistence of $|A(t)|\gtrsim\varepsilon$ is produced by enforcing~\eqref{eq:floor_constraint} in the specified implementation, rather than being an incidental feature of the unconstrained baseline or a plotting artifact.

\subsection{Interpretation within Phase~II scope}
Claim~C2.1 establishes the Phase~II starting point: the global floor constraint can be enforced stably in a concrete lattice testbed, and it yields a persistent nonzero remainder in the diagnostic amplitude relative to a matched unconstrained baseline.

\noindent\textbf{Non-claims boundary (C2.1).}
This result does \emph{not} identify $\varepsilon$ with a physical constant, does \emph{not} establish a continuum-limit theorem or universality statement, and does \emph{not} assert a Standard-Model embedding; it is a statement about the demonstrated behavior of the Phase~II constrained update rule under paired-run comparison.

Claim~C2.2 extends this by stress-testing the residual diagnostics under systematic parameter variations, and Claim~C2.3 tests whether the Phase~II residual can be embedded into a toy FRW background module under a fixed, transparent mapping without instability.