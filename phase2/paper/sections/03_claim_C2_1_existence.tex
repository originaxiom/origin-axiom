% === P2-S4b CLAIM STRUCTURE CHECKLIST (keep; delete only after Phase 2 lock) ===
% File: phase2/paper/sections/03_claim_C2_1_existence.tex
% Must contain: (i) \paragraph{Claim (C2.1).} with a single-sentence claim,
% (ii) explicit figure pointer(s) (e.g. Fig.~A / Figs.~B--D / Fig.~E),
% (iii) explicit pointer to Appendix run manifest (Appendix~\ref{app:run_manifest}),
% (iv) explicit non-claims boundary sentence.
% No new physics claims; structure/clarity only.
% === END CHECKLIST ===

% paper/sections/03_claim_C2_1_existence.tex

\section{Claim C2.1: Existence under constraint}
\label{sec:modesum}
\label{sec:claim_c21}

\paragraph{Claim (C2.1).}
In the Phase~II lattice vacuum testbed, enforcing the Origin Axiom floor $|A|\ge\varepsilon$ produces a stable nonzero residual consistent with the intended ``no perfect cancellation'' principle, while remaining numerically well-behaved.
Evidence is provided by a paired constrained vs.\ free run with identical initialization, summarized in Fig.~\ref{fig:mode_sum_residual}.

\subsection{Experimental protocol}
We run two simulations with matched numerical settings and identical initialization (same lattice size, time step, integration length, and random seed where applicable):
\begin{enumerate}
  \item \textbf{Free baseline:} standard evolution with no global constraint.
  \item \textbf{Constrained:} identical evolution with the hard floor $|A(t)|\ge\varepsilon$ enforced at each step via the uniform correction in Eq.~\eqref{eq:uniform_shift}.
\end{enumerate}
We record the global amplitude magnitude $|A(t)|$ over time, along with enforcement statistics (number of constraint hits and distribution of correction magnitudes).

\subsection{Result: the residual is nonzero, stable, and actively enforced}
Figure~\ref{fig:mode_sum_residual} shows the time series of the global amplitude magnitude $|A(t)|$ for the matched pair.
In the free baseline, $|A(t)|$ relaxes toward $0$ (within floating-point tolerance) as cancellations accumulate.
In the constrained run, $|A(t)|$ is prevented from falling below $\varepsilon$ and instead saturates near the floor, with small bounded fluctuations.

\begin{figure}[t]
  \centering
  \includegraphics[width=0.92\linewidth]{figA_mode_sum_residual.pdf}
  \caption{
  \textbf{Claim C2.1 evidence: existence under the non-cancellation floor.}
  Global amplitude magnitude $|A(t)|$ in matched free vs.\ constrained vacuum runs.
  The free baseline approaches near-zero amplitude via destructive interference, while the constrained run saturates at the imposed floor $|A|\approx\varepsilon$, demonstrating a stable nonzero residual.
  }
  \label{fig:mode_sum_residual}
\end{figure}

Crucially, the axiom enforcement is not decorative.
In the constrained run, the floor is hit repeatedly, and the correction~\eqref{eq:uniform_shift} is applied a large number of times across the run (order $10^5$ constraint hits in the representative Phase~II configuration).
This confirms that the observed nonzero residual is causally attributable to enforcing~\eqref{eq:floor_constraint} rather than to a numerical artifact of the unconstrained dynamics.

\subsection{Interpretation within Phase~II scope}
Claim~C2.1 establishes the Phase~II starting point:
the axiom can be enforced in a concrete lattice setting, and it produces exactly what it is supposed to produce---a persistent nonzero remainder when the unconstrained system would cancel toward $0$.
This result does \emph{not} claim any physical identification of $\varepsilon$ with a fundamental constant; it only demonstrates that the axiom can be implemented as a stable constraint and that its effect is measurable and auditable.

Claim~C2.2 extends this by stress-testing the residual under systematic parameter variations, and Claim~C2.3 tests whether the residual can be coherently mapped into a toy FRW background without instability.