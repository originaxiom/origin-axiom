% ============================================================
% Origin Axiom — Phase 2
% Section 3: Finite Mode-Sum Model
% ============================================================

\section{Finite Mode-Sum Model}
\label{sec:mode_sum_model}

\subsection{Ensemble Definition}

To test the consequences of the Origin Axiom in a controlled setting, we introduce a finite ensemble of $N$ modes.
Each mode is characterized by a frequency $\omega_k$ and a phase $\phi_k$, and contributes a complex amplitude
\begin{equation}
a_k = \omega_k e^{i \phi_k} \, .
\end{equation}

The collective amplitude is given by
\begin{equation}
A_N \;=\; \sum_{k=1}^{N} a_k \, .
\end{equation}
All numerical experiments in this work operate on such finite sums.
No continuum limit is assumed, and $N$ is treated as an explicit control parameter.

\subsection{Frequency Spectrum and Cutoff}

Frequencies are drawn from a monotonic spectrum bounded by an ultraviolet cutoff $\Lambda$.
Specifically, we define
\begin{equation}
\omega_k = f(k;\Lambda,N),
\end{equation}
where $f$ is a deterministic mapping that assigns increasing frequency scales as the mode index increases.
In practice, we adopt a linear ordering in $k$ normalized such that
\begin{equation}
\omega_k \in (0,\Lambda] .
\end{equation}

The precise functional form of $f$ is not essential for the scaling results reported here.
What matters is that the ensemble contains a dense collection of modes up to the cutoff scale and that $\Lambda$ can be varied independently of $N$.

\subsection{Phase Structure}

Phases $\phi_k$ encode relative interference between modes.
Rather than sampling phases uniformly at random, we consider structured phase configurations designed to probe near-cancellation regimes.

Specifically, phases are assigned such that the unconstrained sum $A_N$ would approach zero as $N$ increases, up to small residual fluctuations.
This construction ensures that any surviving amplitude arises from the global constraint rather than accidental phase misalignment.

The detailed phase-generation algorithm is described in the numerical implementation.
For the present discussion, it suffices to note that the ensemble is tuned to lie near the boundary of exact destructive interference.

\subsection{Application of the Global Constraint}

Given the unconstrained sum $A_N$, the Origin Axiom imposes the bound
\begin{equation}
|A_N| \;\ge\; \eps .
\end{equation}
Operationally, this is implemented by defining the residual amplitude
\begin{equation}
A_{\mathrm{res}}(N,\Lambda,\eps)
\;\equiv\;
\max\!\bigl(|A_N|,\eps\bigr).
\end{equation}

This prescription does not alter individual mode contributions.
Instead, it selects the minimal admissible collective amplitude consistent with the axiom whenever near-perfect cancellation would otherwise occur.

\subsection{Effective Vacuum Energy Proxy}

To connect the constrained amplitude to an effective vacuum energy scale, we define
\begin{equation}
\rho_{\mathrm{eff}} \;\equiv\; C \, A_{\mathrm{res}}^{\,2},
\end{equation}
where $C$ is a normalization constant incorporating volume factors and dimensional conventions.
Since all results in Phase~2 focus on relative scaling behavior, $C$ is fixed once and held constant across all experiments.

The dependence of $\rho_{\mathrm{eff}}$ on $N$, $\Lambda$, and $\eps$ constitutes the primary observable of interest in the following sections.

\subsection{Control Parameters and Sweeps}

The model admits three independent control parameters:
\begin{itemize}
\item the number of modes $N$,
\item the ultraviolet cutoff $\Lambda$,
\item the residual bound $\eps$.
\end{itemize}

Phase~2 systematically explores scaling with respect to each parameter by holding the others fixed.
These sweeps form the basis of Figs.~B--D and allow us to disentangle finite-size effects from genuine cutoff sensitivity.

\subsection{Interpretational Remarks}

The finite mode-sum model should be understood as a regulated proxy rather than a literal description of vacuum microphysics.
Its purpose is to test whether a global lower bound on interference-induced cancellation can yield a small but stable residual energy scale.

Crucially, no renormalization counterterms, symmetry assumptions, or fine-tuned cancellations are introduced.
All suppression arises solely from collective interference subject to the Origin Axiom constraint.

% ============================================================
% End of 03_mode_sum_model.tex
% ============================================================
