% paper/sections/06_reproducibility_and_provenance.tex

\section{Reproducibility and provenance}
\label{sec:reproducibility}

Phase~II is designed to be reproducible from a clean checkout: every figure in this paper is generated from version-controlled code, explicit configuration, and logged run artifacts.
This section specifies (i) where the authoritative artifacts live, (ii) how figures are built, and (iii) how a third party can reproduce the full PDF.

\subsection{Repository layout and authoritative artifacts}
The Phase~II deliverable is structured as:
\begin{itemize}
  \item \texttt{paper/}: the \LaTeX\ source of the manuscript.
  \item \texttt{scripts/}: the executables that generate intermediate data products and figures.
  \item \texttt{config/}: YAML configuration files specifying all numerical parameters used in the runs.
  \item \texttt{outputs/}: generated data and figures (treated as build artifacts).
  \item \texttt{Snakefile}: the build graph used to regenerate figures and compile the PDF.
\end{itemize}
Figures included in the manuscript are copied or symlinked into \texttt{paper/figures/} for arXiv packaging, with provenance pointers to the original build products in \texttt{outputs/figures/}.

\subsection{Figure-to-run mapping and the run manifest}
Each figure in the main text corresponds to a tagged run or sweep whose provenance is recorded in a run manifest:
\begin{itemize}
  \item Fig.~A: representative paired free vs.\ constrained vacuum run (Claim~C2.1).
  \item Figs.~B--D: systematic sweeps in $\varepsilon$ and discretization/UV controls (Claim~C2.2).
  \item Fig.~E: FRW trajectory comparison driven by the anchored residual (Claim~C2.3).
\end{itemize}
The authoritative mapping from figure filenames to run identifiers, configuration hashes, and script entry points is provided in Appendix~\ref{app:run_manifest}.
This is the primary audit anchor: a reader can trace every plotted curve to a specific run signature.

\subsection{Build system and exact reproduction}
The reference build uses a deterministic build graph (Snakemake) and \texttt{latexmk} for compilation.
From the Phase~II root directory, the following commands rebuild all figures and compile the manuscript:
\begin{verbatim}
snakemake -j 1 paper
\end{verbatim}
This target (i) executes the scripts needed to generate the canonical figures in \texttt{outputs/figures/}, (ii) stages the required figures for arXiv submission under \texttt{paper/figures/}, and (iii) compiles \texttt{paper/main.tex} to \texttt{paper/main.pdf}.
If the repository is cloned on a new machine, the environment requirements (Python version, required packages, and \LaTeX\ toolchain) are specified alongside the Phase~II setup documentation (see repository \texttt{README} and/or environment files).

\subsection{Determinism, seeds, and paired-run discipline}
Where randomized initialization is used, seeds are explicitly set via configuration and recorded in run outputs.
For all main-claim evidence, we employ a paired-run discipline: constrained and free baselines share identical initialization and numerical settings, differing only by the enforcement of the axiom floor.
This isolates the causal impact of the axiom implementation from stochastic variance and from unrelated numerical settings.

\subsection{Data availability}
All data products necessary to reproduce the plots are generated by the build graph and are either:
\begin{itemize}
  \item stored under \texttt{outputs/} as build artifacts, or
  \item regenerable from code + configuration using the commands above.
\end{itemize}
The manuscript does not rely on external proprietary datasets.