% paper/sections/08_conclusion.tex

\section{Conclusion}
\label{sec:conclusion}

Phase~II of the Origin Axiom program was designed as a bounded, auditable test:
if one enforces a strict global non-cancellation floor in a minimal vacuum testbed, does a stable residual emerge; is it robust under controlled numerical variations; and can it be carried through a transparent toy embedding into a cosmological background module without instability?
Within the explicit scope boundaries stated in Sec.~\ref{sec:discussion}, the answer is yes.

We organized the paper around three claims and corresponding evidence:

\begin{itemize}
  \item \textbf{Claim C2.1 (Existence under constraint).}
  Enforcing the hard floor $|A|\ge\varepsilon$ in the Phase~II lattice vacuum testbed yields a persistent nonzero residual diagnostic relative to the matched unconstrained baseline, without numerical pathology (Fig.~\ref{fig:mode_sum_residual}).

  \item \textbf{Claim C2.2 (Robustness under controls).}
  Across systematic sweeps in the floor scale $\varepsilon$ and discretization/UV-control parameters, the induced residual remains well-behaved and exhibits smooth, non-explosive dependence in the explored regime (Figs.~\ref{fig:scaling_epsilon}--\ref{fig:scaling_modes}).

  \item \textbf{Claim C2.3 (End-to-end consistency in a toy FRW module).}
  Under a fixed and explicitly stated mapping from the residual diagnostic to an effective constant contribution $\Omega_\Lambda(\theta)$, FRW trajectories show modest, smooth deviations without pathological behavior, demonstrating pipeline consistency rather than a cosmological prediction (Fig.~\ref{fig:frw_comparison}).
\end{itemize}

A secondary Phase~II objective was to test whether the pipeline can coherently carry a single phase input without instability.
Treating $\theta^\star$ as an externally specified anchor (not derived in Phase~II), we propagated it through the vacuum and FRW modules and observed that residual diagnostics can carry a small, structured phase dependence while remaining numerically stable.
In Phase~II this phase dependence is reported as an operational feature of the implemented pipeline, not as a physical identification claim.

The core contribution of Phase~II is therefore methodological and reproducibility-focused:
a constrained non-cancellation mechanism is specified algorithmically (Sec.~\ref{sec:framework}),
demonstrated to be stable in a controlled setting,
stress-tested under meaningful numerical controls,
and documented with end-to-end provenance so each figure can be traced to tagged runs and regenerated from a clean checkout.
Reproduction instructions and repository structure are stated in Sec.~\ref{sec:reproducibility}, and per-figure run identifiers are indexed in Appendix~\ref{app:run_manifest}.

Finally, we reiterate the boundaries.
Phase~II does not derive the axiom from a local action, does not claim quantitative agreement with the observed cosmological constant, and does not establish a continuum-limit theorem or uniqueness of $\theta^\star$.
What it provides is a concrete, checkable baseline: a fully audited toy pipeline in which ``no perfect cancellation'' is imposed as a global constraint and its consequences can be measured, reproduced, and scrutinized.
This establishes a stable foundation for subsequent phases to address universality, physical embedding, and stronger interpretive targets.