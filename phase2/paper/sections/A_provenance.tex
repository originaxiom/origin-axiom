% phase2/paper/sections/A_provenance.tex

\section{Computational provenance}
\label{app:provenance}

This appendix specifies the audit trail for every numerical result shown in the Phase~II paper.
Its purpose is narrow and concrete: enable an independent reader to (i) map each figure to the exact run artifacts that produced it,
(ii) identify the precise code state and configuration used, and (iii) regenerate the outputs from a clean repository checkout.

\subsection{Audit path: figure $\rightarrow$ run\_id $\rightarrow$ run folder}
The canonical traceability path is:
\begin{enumerate}
  \item \textbf{Figure in the PDF} (e.g.\ Fig.~A--E),
  \item \textbf{Figure sidecar} staged under \texttt{phase2/paper/figures/} with suffix \texttt{.run\_id.txt},
  \item \textbf{Run directory} under \texttt{outputs/runs/<run\_id>/},
  \item \textbf{Machine-readable run metadata} \texttt{outputs/runs/<run\_id>/meta.json},
  \item \textbf{Code state and parameters} recorded in \texttt{meta.json} (including the Git commit hash and resolved configuration).
\end{enumerate}
For multi-run sweeps, the figure sidecar records a list of \texttt{run\_id}s.
The mapping between figures, run identifiers, and claims is summarized in Appendix~\ref{app:run_manifest}, which serves as the paper's primary audit index.

\subsection{Run identifiers and artifact structure}
Each Phase~II numerical execution is assigned a unique \texttt{run\_id} and writes its outputs under:
\begin{verbatim}
outputs/runs/<run_id>/
\end{verbatim}
A run directory contains, at minimum:
\begin{itemize}
  \item \texttt{meta.json}: provenance metadata (code state, configuration, and execution identifiers);
  \item run logs (stdout/stderr or equivalent);
  \item raw numerical outputs and/or intermediate data products used to build figures.
\end{itemize}

\paragraph{Provenance fields recorded in \texttt{meta.json}.}
For auditability, the run metadata records (directly or via explicit pointers):
\begin{itemize}
  \item the Git commit hash identifying the exact code state used for the run;
  \item the resolved configuration parameters (including $\varepsilon$, discretization/UV controls, and any scan ranges);
  \item the \texttt{run\_id} and an execution timestamp;
  \item any random seeds used (if randomized initialization is enabled);
  \item the entry point or script name used to generate the run.
\end{itemize}

\subsection{Figure sidecars and direct traceability}
Every figure included in the manuscript has an associated sidecar file:
\begin{verbatim}
phase2/paper/figures/<figure_name>.run_id.txt
\end{verbatim}
This sidecar records the \texttt{run\_id} (or list of \texttt{run\_id}s) used to generate that figure.
It therefore provides a direct, file-level link from the staged manuscript figure to the authoritative run artifacts under \texttt{outputs/runs/}.
Canonical figure files are generated under \texttt{outputs/figures/} and then staged for submission under \texttt{phase2/paper/figures/}.

\subsection{Regeneration via the build graph}
All figures and the final PDF are generated via a deterministic build graph implemented using Snakemake.
The canonical target:
\begin{verbatim}
snakemake -j 1 paper
\end{verbatim}
(i) regenerates numerical outputs under \texttt{outputs/},
(ii) rebuilds canonical figures under \texttt{outputs/figures/},
(iii) stages submission figures under \texttt{phase2/paper/figures/},
and (iv) compiles \texttt{phase2/paper/main.tex} via \texttt{latexmk}.
No figure appearing in the paper is generated manually or edited outside this build process.

\subsection{Paired-run discipline}
For all main-claim evidence (Claims~C2.1--C2.3), results are obtained using paired runs:
a constrained run enforcing the Origin Axiom floor is compared against an unconstrained baseline with identical initialization, numerical parameters, and (where applicable) random seed.
This isolates the causal effect of the enforcement rule from stochastic variation and unrelated numerical settings.

\subsection{Scope of reproducibility}
The guarantees described here apply strictly to the Phase~II implementation and to the claims as stated in the paper.
They ensure that the numerical evidence presented is transparent, auditable, and reproducible.
They do not imply a continuum-limit result, a unique physical interpretation of the residual, or a derived connection to observed cosmological parameters.