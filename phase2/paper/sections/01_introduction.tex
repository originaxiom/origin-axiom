% ============================================================
% Origin Axiom — Phase 2
% Section 1: Introduction
% ============================================================

\section{Introduction}
\label{sec:introduction}

The observed small but nonzero value of the cosmological constant remains one of the most persistent open problems in fundamental physics.
Na\"ively, quantum field theory predicts a vacuum energy density set by the ultraviolet cutoff,
\begin{equation}
\rho_{\mathrm{vac}} \sim \sum_{\kmode} \tfrac{1}{2}\hbar \omega_{\kmode},
\end{equation}
which exceeds the observed dark energy density by many orders of magnitude.
Despite decades of progress, no universally accepted mechanism explains why these contributions nearly cancel, nor why a small positive residual appears to survive at late times.

The standard $\Lambda$CDM model incorporates this residual phenomenologically through a cosmological constant $\Lambda$, providing an excellent fit to cosmological data.
However, within conventional quantum field theory, $\Lambda$ remains radiatively unstable and highly sensitive to ultraviolet physics.
This tension motivates the exploration of mechanisms that regulate vacuum energy through principles that are not purely local or perturbative.

In this work we investigate such a mechanism, referred to as the \OA{}, which enforces a global constraint on coherent mode sums.
Rather than modifying individual quantum fields or introducing new particles, the axiom restricts the collective amplitude arising from large ensembles of modes.
When destructive interference is nearly exact, the axiom prevents the total amplitude from vanishing identically, leaving a residual bounded from below by a small parameter $\eps$.
Crucially, this residual is not imposed by hand but emerges dynamically from the constrained sum itself.

The goal of Phase~2 is to demonstrate that this mechanism can reproduce a vacuum energy scale compatible with cosmological observations in a setting that approximates realistic quantum field behavior.
We focus on three core questions:
(i)~whether the residual produced by constrained mode cancellation is robust under changes of cutoff, volume, and mode count;
(ii)~whether the scaling behavior remains controlled and free of pathological divergences;
and (iii)~whether the resulting residual, when interpreted as an effective vacuum energy, leads to a cosmological expansion consistent with late-time acceleration.

To address these questions, we construct a finite but extensible model of zero-point mode sums subject to a global cancellation constraint.
The model is implemented numerically on discrete lattices with tunable parameters, allowing systematic sweeps over the number of modes, ultraviolet cutoff, and constraint strength.
This approach does not assume supersymmetry, specific particle content, or exact boson–fermion pairing.
Instead, it isolates the effect of constrained interference itself.

We then couple the resulting residual vacuum energy to a flat Friedmann–Robertson–Walker (\FRW) cosmology.
Interpreting the constrained residual as an effective cosmological constant $\Omega_{\Lambda,\mathrm{eff}}$, we compute its impact on the expansion rate and deceleration parameter.
This step bridges the microscopic cancellation mechanism with macroscopic cosmological dynamics, without invoking additional assumptions about early-universe physics.

Several important limitations are emphasized.
We do not claim to identify the microscopic origin of the interference phase responsible for incomplete cancellation, nor do we connect the mechanism to specific Standard Model sectors.
Furthermore, the \OA{} should be viewed as a global consistency condition rather than a modification of local quantum field dynamics.
Its ultimate embedding in a fundamental theory—possibly involving new symmetries or super-selection rules—remains an open question.

Within these bounds, the results presented here suggest that a small, positive cosmological constant can emerge naturally from constrained vacuum cancellation.
If confirmed, this would reframe the cosmological constant problem not as a failure of quantum field theory, but as an indication that global consistency conditions play an essential role in regulating vacuum energy.

This paper is organized as follows.
In \Sec{sec:framework}, we introduce the constrained mode-sum framework and define the \OA{} precisely.
\Sec{sec:modesum} presents the numerical implementation and residual scaling results.
In \Sec{sec:cosmology}, we examine the cosmological implications of the residual vacuum energy in an \FRW{} background.
We conclude in \Sec{sec:discussion} with a discussion of implications, limitations, and directions for future work.

% ============================================================
% End of 01_introduction.tex
% ============================================================