% ============================================================
% Origin Axiom — Phase 2
% Section 1: Introduction
% ============================================================

\section{Introduction}
\label{sec:introduction}

The small but nonzero late-time vacuum component inferred in cosmology is often framed as a naturalness problem.
In conventional quantum-field-theory reasoning, vacuum contributions exhibit strong ultraviolet sensitivity and are generically large, whereas the effective residual inferred on cosmological scales is many orders of magnitude smaller.
At the level of schematic mode-sum intuition one may write
\begin{equation}
\rho_{\mathrm{vac}} \sim \sum_{\kmode} \tfrac{1}{2}\hbar \omega_{\kmode},
\end{equation}
which motivates the question of how large contributions could nearly cancel and why any nonzero remainder would persist.
For standard reviews and broader context, see Refs.~\cite{weinberg1989ccp,carroll2001cc}.

The $\Lambda$CDM model parameterizes late-time acceleration with a constant term and fits a wide range of cosmological observations extremely well~\cite{planck2018params}.
From the perspective of effective field theory, however, the smallness and apparent stability of such a term under radiative corrections is commonly regarded as conceptually tensioned with generic ultraviolet sensitivity~\cite{weinberg1989ccp,carroll2001cc}.
This motivates exploring whether \emph{global} constraints on collective cancellation can, at least in controlled testbeds, enforce an irreducible residual without relying on detailed particle content, exact boson--fermion pairing, or fine-tuned counterterms.

This paper studies one such constraint mechanism under the name \OA{}.
The defining ingredient in Phase~II is operational and explicitly bounded: we impose a hard \emph{non-cancellation floor} on a global complex diagnostic amplitude $A(t)$ constructed from a minimal vacuum-like numerical model.
In the Phase~II implementation, the axiom takes the form
\begin{equation}
|A(t)| \ge \varepsilon \,, \qquad \varepsilon>0 \text{ fixed,}
\end{equation}
enforced at each integration step by a spatially uniform correction applied only when a trial update would violate the floor.
The constraint acts exclusively on the global (zero-mode) diagnostic and, by construction, does not alter nonzero-mode structure beyond what the baseline dynamics already produce.
Phase~II evaluates the consequences of this constraint \emph{in the specified model and code}, with emphasis on auditability, numerical stability, and bounded interpretability rather than on first-principles derivation.

Accordingly, the Phase~II objective is not to derive the axiom from a local action, nor to claim a quantitative prediction for the observed cosmological constant.
Instead, we ask three tightly scoped questions aligned with the manuscript claims:
(i)~does enforcing the floor produce a stable nonzero residual diagnostic relative to an unconstrained baseline (Claim~C2.1);
(ii)~is that residual robust under systematic sweeps in $\varepsilon$ and numerical controls that probe discretization and ultraviolet sensitivity (Claim~C2.2);
and (iii)~can the resulting residual be carried through an end-to-end pipeline into a toy Friedmann--Robertson--Walker (FRW) background as a smooth constant-term contribution without generating instabilities or pathological evolution (Claim~C2.3).

To answer these questions we define a finite, extensible lattice-field testbed with tunable parameters and run controlled paired experiments and sweeps.
The protocol is intentionally agnostic about supersymmetry, Standard Model particle content, and microscopic cancellation mechanisms.
The focus is isolated and diagnostic: determine whether a global non-cancellation constraint yields a reproducible, well-behaved residual across nontrivial numerical variations in a minimal setting.

For the end-to-end consistency test, we map the Phase~II residual into a constant-term contribution in a flat FRW module and compare trajectories with and without the mapped term.
This embedding is explicitly a \emph{toy} consistency check: it is not a derived effective-field-theory matching procedure and is not used to fit cosmological datasets.
Its purpose is to verify that the pipeline can interpret the Phase~II residual as a smooth background term without breaking numerical or conceptual bookkeeping; for standard FRW background treatments see, e.g., Refs.~\cite{dodelson2003modern,ryden2017intro}.

Several limitations are enforced throughout and stated explicitly later (Sec.~\ref{sec:discussion}).
We do not claim a first-principles origin for the constraint, we do not infer a unique phase anchor from Phase~II alone, and we do not claim quantitative agreement with the observed cosmological constant.
All results are bounded to the demonstrated behavior of the specified implementation under stated parameter ranges.
Reproducibility is treated as a first-class requirement: figures correspond to tagged runs with recorded configurations, run identifiers, and an explicit run manifest.

This paper is organized in claims-first form.
Section~\ref{sec:framework} defines the model, the floor constraint, the residual diagnostic, and the toy FRW mapping.
Sections~\ref{sec:claim_c21}--\ref{sec:claim_c23} present Claims~C2.1--C2.3 and their evidence.
Section~\ref{sec:reproducibility} documents provenance and reproduction instructions, and Section~\ref{sec:discussion} states scope boundaries and limitations.

% ============================================================
% End of 01_introduction.tex
% ============================================================