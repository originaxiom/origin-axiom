% paper/appendices/A_notation_glossary.tex

\section{Notation and glossary}
\label{app:notation}

This appendix defines notation and terms used throughout Phase~0 and inherited by subsequent phases.
Where later phases introduce specialized variants, they must either reuse these symbols consistently or explicitly declare deviations.

\subsection{Core symbols}

\begin{itemize}
  \item $\theta \in [0,2\pi)$: universal phase-like control parameter (defined modulo $2\pi$).
  \item $\theta^\star$: a \emph{fiducial representative} value of $\theta$ used for benchmark runs or illustrative plots; not a derived constant unless corridor collapse is demonstrated.
  \item $\Theta \subset [0,2\pi)$: an admissible set (corridor) of $\theta$ values; may be a union of intervals or a discrete set.
  \item $\Theta_0$: initial prior corridor.
  \item $\Theta_{i}$: corridor after applying filters from phases $1,\dots,i-1$.
  \item $\mathcal{T}_i(\theta)$: Phase-$i$ test suite (explicit pass/fail criteria) evaluated at $\theta$.
  \item $A$: a phase-specified global complex amplitude used to define ``cancellation'' and the OA floor.
  \item $A_{\mathrm{raw}}$: the amplitude produced by the unconstrained (constraint-off) evolution.
  \item $A_{\mathrm{con}}$: the amplitude after enforcement of the OA floor.
  \item $\varepsilon>0$: the strictly positive floor in the Origin Axiom constraint.
  \item $R$: a phase-specified scalar residual diagnostic associated with non-cancellation / remainder.
  \item $R_{\mathrm{raw}}$: residual from the unconstrained run.
  \item $R_{\mathrm{con}}$: residual from the constrained run.
  \item $\Delta R$: difference between constrained and unconstrained residual (definition must be stated; additive or multiplicative).
\end{itemize}

\subsection{Key terms (glossary)}

\paragraph{Cancellation.}
The approach of the chosen global amplitude toward zero magnitude, $|A|\to 0$, within a specified model/phase definition.

\paragraph{Non-cancellation (Origin Axiom).}
The postulate that $|A|$ is bounded below by a strictly positive floor $\varepsilon$ (Eq.~the non-cancellation identity).

\paragraph{Implementation.}
A concrete algorithm or rule that enforces the postulate within a chosen model, including how corrections are applied and how minimality is defined.

\paragraph{Minimal intervention (MI).}
An implementation hypothesis stating that when enforcement is required, the smallest correction (according to a declared norm or cost) is applied to satisfy $|A|\ge\varepsilon$.

\paragraph{Non-binding regime.}
A run/regime in which the raw evolution never violates the floor: $|A_{\mathrm{raw}}(t)|>\varepsilon$ for all times/steps.
In this regime the constraint should be non-invasive.

\paragraph{Binding regime.}
A run/regime in which the raw evolution would violate the floor at least once: $\exists t$ such that $|A_{\mathrm{raw}}(t)|<\varepsilon$.
In this regime enforcement must activate, and constrained and unconstrained runs may differ.

\paragraph{Binding certificate.}
A logged artifact demonstrating that enforcement occurred (e.g.\ a boolean flag plus a quantitative measure of hits) and that the constrained trajectory differs from the constraint-off ablation in a diagnostically relevant way.

\paragraph{Ablation.}
A controlled comparison where a single factor is changed (here: constraint OFF vs ON), holding all other settings fixed (seed, configuration, numerics).
Required for causal attribution.

\paragraph{Test suite.}
A set of explicit criteria used to define admissibility of $\theta$ values in a phase, including binding evidence (when required), stability, robustness, and transfer viability.

\paragraph{Admissible set / corridor.}
The subset of $\theta$ values that pass a phase's test suite; represented as a union of intervals on $[0,2\pi)$ or as a discrete set of points.

\paragraph{Corridor method.}
The protocol that advances the program by intersecting admissible sets across phases:
\[
\Theta_{i+1} = \{\theta\in\Theta_i:\mathcal{T}_i(\theta)\ \mathrm{passes}\}.
\]

\paragraph{Ledger.}
A deterministic mechanism that ingests per-phase \texttt{theta\_filter} artifacts and produces a versioned corridor history and dashboard, enabling audit-grade tracking of corridor narrowing over time.

\paragraph{Phase locking.}
A phase is ``locked'' only when its claims are explicitly stated, tied to regenerable artifacts, supported by required ablations and binding certificates, and accompanied by full provenance and ledger-compatible corridor outputs (Sec.~the Phase 0 main text).