% paper/appendices/B_phase_theta_filter_schema.tex

\section{Phase \texorpdfstring{$\theta$}{theta}-filter artifact schema (ledger interface)}
\label{app:schema}

This appendix defines the minimal machine-readable interface by which each phase reports its admissible $\theta$ set to the Phase~0 ledger.
The goal is to make corridor evolution deterministic and auditable.

\subsection{Design goals}
The \texttt{theta\_filter} artifact must:
\begin{itemize}
  \item be sufficient to reconstruct the phase's admissible set $\Theta_{i+1}$;
  \item record the phase's declared test suite and the pass/fail outcome for each evaluated $\theta$;
  \item contain provenance linking each evaluated $\theta$ (or interval) to reproducible run identifiers;
  \item be stable under minor refactors (schema versioned; backwards compatible when possible).
\end{itemize}

\subsection{Minimal required fields}
Each phase must emit a JSON file named
\begin{equation}
\texttt{phase\_XX\_theta\_filter.json},
\end{equation}
where \texttt{XX} is a zero-padded phase number.

The minimal schema is:

\begin{verbatim}
{
  "schema_version": "1.0",
  "phase": 2,
  "subphase": "optional-string",
  "theta_domain": [0.0, 6.283185307179586],

  "theta_prior": {
    "type": "intervals",
    "intervals": [[2.18, 5.54]]
  },

  "theta_grid": [2.18, 2.20, 2.22, ...],

  "tests": ["binds", "stable", "robust", "transfer_viable"],

  "pass": [true, false, true, ...],

  "fail_reasons": [
    [],
    ["binds=false", "stable=false"],
    []
  ],

  "provenance": {
    "git_commit": "abcdef123456",
    "config_hash": "sha256:...",
    "environment": "pip-freeze-or-env-hash",
    "run_ids": {
      "theta=2.18": "outputs/runs/<run_id>/",
      "theta=2.20": "outputs/runs/<run_id>/"
    }
  }
}
\end{verbatim}

\subsection{Optional fields (recommended)}
The following fields are optional but recommended for diagnostics and ranking:

\paragraph{Scores.}
A phase may provide one or more score arrays aligned with \texttt{theta\_grid}:
\begin{verbatim}
"score": {
  "intervention_cost": [...],
  "stability_margin": [...],
  "transfer_penalty": [...],
  "total": [...]
}
\end{verbatim}
Scores should be documented in the phase paper and interpreted cautiously.
Scores do not replace pass/fail; they rank within the admissible set.

\paragraph{Interval representation.}
If a phase analytically characterizes admissibility as intervals rather than by grid scan, it may omit \texttt{theta\_grid} and instead provide:
\begin{verbatim}
"theta_pass": {
  "type": "intervals",
  "intervals": [[a1, b1], [a2, b2]]
}
\end{verbatim}
The ledger will accept either representation.
If both are present, \texttt{theta\_pass} takes precedence.

\paragraph{Binding metrics.}
When OA impact is claimed, phases should include binding metrics per $\theta$:
\begin{verbatim}
"binding": {
  "epsilon": 1e-6,
  "hit_fraction": [...],
  "n_hits": [...],
  "min_raw_minus_eps": [...]
}
\end{verbatim}

\subsection{Ledger interpretation rules}
To ensure determinism, the ledger applies the following rules:

\begin{enumerate}
  \item The admissible set $\Theta_{i+1}$ is reconstructed from either:
  \begin{itemize}
    \item \texttt{theta\_pass.intervals} (if present), or
    \item the subset of \texttt{theta\_grid} entries where \texttt{pass=true}, grouped into contiguous intervals using a declared grid spacing tolerance.
  \end{itemize}
  \item Corridors are represented internally as unions of intervals on $[0,2\pi)$.
        Any wrap-around interval is represented as two intervals.
  \item The new corridor is computed by intersection:
  \[
    \Theta \leftarrow \Theta \cap \Theta_{i+1}.
  \]
  \item If the intersection is empty, the ledger must emit an error state and record which phase artifact caused the empty intersection.
\end{enumerate}

\subsection{Schema evolution}
Future schema versions must:
\begin{itemize}
  \item bump \texttt{schema\_version},
  \item preserve required fields or provide an explicit compatibility layer in the ledger,
  \item document changes in Phase~0 and in the ledger README.
\end{itemize}