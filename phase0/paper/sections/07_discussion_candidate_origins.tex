% paper/sections/07_discussion_candidate_origins.tex

\section{Discussion: candidate origins and interpretations of non-cancellation}
\label{sec:discussion_origins}

Phase~0 treats the Origin Axiom as a postulate tested by consequences.
Nevertheless, it is useful to catalogue plausible ways in which non-cancellation-like behavior could arise in established scientific language.
This section is intentionally framed as \emph{candidate interpretations} rather than as claims; it is a map of mechanistic hypotheses to be evaluated in later phases.

\subsection{Constraint as a consistency condition}
\label{sec:consistency_condition}

Many physical principles are best understood as consistency constraints: requirements that exclude pathological states rather than constructive dynamical laws.
Examples include positivity of probabilities, unitarity constraints, anomaly cancellation conditions, and stability bounds.
In this spirit, the non-cancellation floor may be interpreted as a consistency requirement on an aggregate quantity $A$:
the theory does not permit trajectories that enter an arbitrarily small neighborhood of $A=0$.

Under this interpretation, the primary scientific tasks are:
(i) to identify a suitable invariant definition of $A$ in realistic settings,
(ii) to demonstrate that enforcing the constraint does not introduce contradictions,
and (iii) to determine whether the constraint yields nontrivial predictions.

\subsection{Selection principle: minimal intervention and extremality}
\label{sec:selection_principle}

If enforcement is framed by a minimal-intervention rule (Sec.~\ref{sec:min_intervention}), one may reinterpret the program as exploring an extremal selection principle:
among all evolutions compatible with the model, the realized evolution minimizes a cost functional associated with approaching perfect cancellation.

Concretely, one might define an intervention cost
\begin{equation}
C(\theta) = \int dt\ \mathcal{L}_{\mathrm{int}}\!\left(A_{\mathrm{raw}}(t;\theta), \varepsilon\right),
\end{equation}
with $\mathcal{L}_{\mathrm{int}}$ penalizing excursions below $\varepsilon$.
Later phases could examine whether the corridor method effectively selects $\theta$ values that minimize such costs while satisfying stability and transfer constraints.
Phase~0 does not commit to this interpretation; it is presented as a viable formalization route.

\subsection{Topological and holonomy-like interpretations}
\label{sec:topology}

Phase-like parameters that survive across domains are often associated with holonomy, winding, or Berry-phase structures.
In such settings, certain values can become preferred or quantized due to topological consistency conditions.
A non-cancellation constraint could plausibly arise as a statement that a global phase defect (or holonomy) cannot be gauged away and therefore enforces an irreducible remainder.

This class of interpretations is attractive because it offers a route to:
(i) invariance under local redefinitions,
(ii) robustness to perturbations, and
(iii) discrete candidate values for $\theta$ through quantization.
Whether the Origin Axiom can be embedded in such a structure is a Phase~III-level question.

\subsection{Information-theoretic perspectives}
\label{sec:info}

Another broad interpretation treats cancellation as erasure.
In information-theoretic language, perfect global cancellation would correspond to a lossless compression of structure into null output.
A non-cancellation postulate can be viewed as forbidding complete erasure of structured information in the system's global bookkeeping.

This perspective is speculative but potentially productive: it suggests defining $A$ as an information-carrying functional and interpreting $\varepsilon$ as a lower bound on retained distinguishability.
Phase~0 does not assert such a mapping; it notes it as a conceptual lane that may later yield invariance principles or scaling expectations.

\subsection{Caution: avoiding ``anything can be enforced''}
\label{sec:anything_enforced}

A central scientific concern is that an arbitrary floor $\varepsilon$ can be imposed on almost any quantity, making the program vacuous.
The corridor method and the reproducibility gates exist to prevent this failure mode.
To remain nontrivial, later phases must show:
\begin{itemize}
  \item binding regimes where OA enforcement demonstrably changes outcomes,
  \item robustness and scaling properties that restrict $\varepsilon$ and implementation choices,
  \item and independent constraints that meaningfully narrow $\theta$ rather than leaving it free.
\end{itemize}
If these conditions are not met, the correct conclusion is not that the axiom is ``true,'' but that the current definitions do not produce discriminating science.