\section{Non-normative note: candidate origins (hypotheses, not claims)}
\label{sec:candidate_origins}

This section is \emph{non-normative}. It exists to record candidate research directions without allowing them to function as evidence or implied conclusions. No statement here is a project claim. No later phase may cite this section as evidence.

\subsection{Purpose}
A governance phase benefits from stating what it refuses to do. Phase 0 does not argue for a specific physical mechanism behind non-cancellation constraints. Instead, it records a small set of plausible origin routes that later phases may treat as hypotheses to test under the Phase 0 contracts.

\subsection{Candidate routes (hypotheses)}
\begin{enumerate}
  \item \textbf{Symmetry / selection route.} A constraint emerges as an effective selection rule from a deeper symmetry principle, appearing as a hard floor or forbidden cancellation condition in a low-energy description.
  \item \textbf{Coarse-graining / renormalization route.} Apparent cancellation at one scale fails to survive consistent coarse-graining across scales, leaving a controlled residue after integrating out degrees of freedom.
  \item \textbf{Global consistency / holonomy route.} A global phase-consistency constraint restricts admissible configurations; locally canceling contributions cannot be made globally consistent, producing an irreducible remainder.
  \item \textbf{Information-theoretic route.} Exact cancellation is prohibited by an encoding constraint (finite specification cost, irreducible precision, or computability limits), producing a controlled residue that behaves as an effective offset.
\end{enumerate}

\subsection{Governance rule}
Future phases may explore any route above only if:
(i) statements are written as bounded claims in the phase claims ledger,
(ii) evidence is provided via canonical artifacts with run provenance,
and (iii) falsifiers are stated explicitly.
If a later phase cannot supply evidence, the route remains speculation and must be labeled as such.
