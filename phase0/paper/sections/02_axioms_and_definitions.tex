% paper/sections/02_axioms_and_definitions.tex

\section{Axioms and definitions}
\label{sec:axioms_defs}

This section fixes the minimal objects and terms that all subsequent phases must use consistently.
Phase~0 is intentionally conservative: we define only what we must, and we label where later phases may refine or replace definitions.

\subsection{Cancellation and the choice of a global amplitude}
\label{sec:global_amplitude}

By \emph{global cancellation} we mean the possibility that a specified aggregate complex quantity $A$ constructed from the instantaneous state of a system can approach zero in magnitude,
\begin{equation}
|A| \to 0.
\end{equation}
The program does not assume a unique universal definition of $A$ across all domains.
Instead, each phase must:
\begin{enumerate}
  \item define $A$ explicitly in terms of the phase's model variables, and
  \item justify why that $A$ is a relevant bookkeeping object for that phase.
\end{enumerate}

\paragraph{Invariance requirement.}
To avoid purely representational artifacts, a phase must state what transformations leave its $A$ invariant (or covariant) and why the constraint is meaningful under those transformations.
Examples include invariance under basis changes, rephasing symmetries, coordinate transformations, or coarse-graining procedures.
If a phase cannot provide an invariance statement, it must explicitly label its $A$ as a \emph{toy diagnostic} rather than a physically invariant observable.

\subsection{The Origin Axiom (non-cancellation floor)}
\label{sec:origin_axiom}

\paragraph{Postulate (OA).}
Given a specified global complex amplitude $A$ for a given phase/model, the Origin Axiom asserts the existence of a strictly positive floor $\varepsilon>0$ such that
\begin{equation}
|A| \ge \varepsilon.
\label{eq:OA_floor}
\end{equation}
In Phase~0 we treat Eq.~\eqref{eq:OA_floor} as a postulate; no derivation from an action, symmetry principle, or microphysical mechanism is claimed.

\paragraph{Interpretation.}
Equation~\eqref{eq:OA_floor} is a ``no-perfect-nothing'' constraint: it forbids $A$ from entering an arbitrarily small neighborhood of the origin in the complex plane.
The scientific question is not whether such a rule is aesthetically appealing, but whether implementing it produces controlled, falsifiable consequences in a chain of models.

\subsection{Implementation: the minimal intervention rule}
\label{sec:min_intervention}

A postulate becomes testable only when implemented.
We therefore define a minimal implementation hypothesis.

\paragraph{Implementation hypothesis (MI).}
When the raw evolution of the model would yield $|A_{\mathrm{raw}}|<\varepsilon$, the constraint is enforced by applying the smallest correction (according to a phase-specified norm) needed to satisfy Eq.~\eqref{eq:OA_floor}.
We denote the enforced amplitude by $A_{\mathrm{con}}$.

The exact definition of ``smallest'' is implementation-dependent and must be specified in each phase.
Common choices include:
\begin{itemize}
  \item radial projection in the complex plane: $A_{\mathrm{con}} = \varepsilon\,A_{\mathrm{raw}}/|A_{\mathrm{raw}}|$ (when $A_{\mathrm{raw}}\neq 0$),
  \item additive correction with minimal norm in a parameterized subspace,
  \item constrained optimization over model degrees of freedom.
\end{itemize}
Regardless of choice, the implementation must log sufficient diagnostics to audit whether and how enforcement occurred.

\subsection{Binding vs non-binding regimes}
\label{sec:binding}

We define two regimes:

\paragraph{Non-binding regime.}
A run is non-binding if the raw evolution satisfies $|A_{\mathrm{raw}}(t)|>\varepsilon$ for the entire run.
In this regime the implementation must be \emph{non-invasive}: constrained and unconstrained runs should coincide up to numerical tolerance.

\paragraph{Binding regime.}
A run is binding if there exists at least one time (or step) at which $|A_{\mathrm{raw}}|<\varepsilon$ and the enforcement rule is applied.
Binding is therefore an objective event that can be certified by logged artifacts.

\paragraph{Binding certificate (required).}
Any phase that claims the axiom has a causal effect must provide, for at least one representative configuration, a binding certificate containing:
\begin{itemize}
  \item \texttt{constraint\_applied = true} (or equivalent),
  \item a quantitative measure of intervention (\texttt{n\_hits}, \texttt{hit\_fraction}, or \texttt{min\_raw\_minus\_eps}),
  \item a paired ablation showing the divergence between constrained and unconstrained trajectories for a diagnostic of interest.
\end{itemize}

\subsection{Residual, remainder, and diagnostic outputs}
\label{sec:residual_def}

When enforcement occurs, the program focuses on what cannot be cancelled: the \emph{remainder}.
We use ``residual'' to denote a scalar diagnostic extracted from the system that is intended to track that remainder.

Each phase must define:
\begin{itemize}
  \item the raw residual $R_{\mathrm{raw}}$,
  \item the constrained residual $R_{\mathrm{con}}$ (under OA enforcement),
  \item and the difference $\Delta R = R_{\mathrm{con}} - R_{\mathrm{raw}}$ (or a ratio) used to quantify axiom impact.
\end{itemize}
The residual is not assumed to be a physical observable in Phase~0; it is an operational diagnostic whose meaning and invariance must be argued phase-by-phase.

\subsection{The universal phase parameter and its domain}
\label{sec:theta_def}

We define a universal phase-like control parameter
\begin{equation}
\theta \in [0,2\pi),
\end{equation}
with the understanding that phases differing by integer multiples of $2\pi$ are equivalent in any dependence that enters only through $e^{i\theta}$ or other $2\pi$-periodic functions.

We distinguish:
\begin{itemize}
  \item \textbf{Fiducial representative} $\theta^\star$: a convenient point chosen for illustration or for a single-run benchmark within an admissible corridor.
  \item \textbf{Admissible corridor} $\Theta$: a set of $\theta$ values that pass a phase's explicit test suite.
\end{itemize}
Phase~0 prohibits treating $\theta^\star$ as a derived constant unless the corridor-intersection protocol collapses $\Theta$ toward a unique value (or a small discrete set) under independent constraints.

\subsection{Ablation principle (causal isolation)}
\label{sec:ablation_def}

Whenever a phase claims that OA enforcement causes an effect, it must include a paired ablation:
\begin{itemize}
  \item Run A (control): constraint OFF (or $\varepsilon=0$ / enforcement disabled),
  \item Run B (treatment): constraint ON with specified $\varepsilon>0$,
\end{itemize}
with identical initialization, seed, and numerical parameters.
Only then can differences be attributed to the axiom implementation rather than to confounders.

\subsection{Phase locking criteria (definition)}
\label{sec:locking_def}

A phase is considered \emph{locked} only when:
\begin{enumerate}
  \item its claims are explicitly stated;
  \item every claim is tied to regenerable artifacts (figures/tables) and run manifests;
  \item binding certificates are provided wherever causal axiom impact is asserted;
  \item non-claims and limitations are explicitly listed;
  \item provenance (git commit, config hash, environment) is recorded for all headline runs.
\end{enumerate}
Phases that do not satisfy these conditions may be explored, but they may not be used as authoritative inputs for later phases.