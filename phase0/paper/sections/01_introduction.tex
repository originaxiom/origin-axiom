% paper/sections/01_introduction.tex

\section{Introduction}
\label{sec:introduction}

\subsection{Euler cancellation as a reference ideal}
A useful reference point for what we mean by \emph{cancellation} is Euler's identity,
\begin{equation}
e^{i\pi} + 1 = 0,
\label{eq:euler}
\end{equation}
which expresses a precise and global destructive balance between two terms.
As a mathematical statement it is exact; as a metaphor it captures a broader notion of \emph{perfect cancellation}:
a structured combination of contributions can sum to identically nothing.
The Origin Axiom program begins by asking a constrained question:
\emph{what changes if global perfect cancellation is forbidden as an attainable outcome of the universe's bookkeeping?}

We emphasize that this is not an argument from aesthetics.
The point of beginning from Eq.~\eqref{eq:euler} is to make the null state concrete: if a theory permits exact global cancellation, then it admits a sharply defined ``nothing'' benchmark.
The program explores the alternative posture: treat ``nothing'' as a boundary that cannot be crossed, and study the remainder forced by that boundary.

\subsection{From philosophical premise to operational postulate}
To move from intuition to science, the program adopts a minimal operational postulate.
We define a specified global complex amplitude $A$ constructed from the state of a model system (the precise construction is phase-dependent and must be stated explicitly).
The Origin Axiom is then expressed as the constraint
\begin{equation}
|A| \ge \varepsilon, \qquad \varepsilon>0,
\label{eq:floor_intro}
\end{equation}
where $\varepsilon$ is a strictly positive floor.
The intent is not to claim that Eq.~\eqref{eq:floor_intro} is already derived from a known fundamental action.
Rather, Eq.~\eqref{eq:floor_intro} is treated as a \emph{postulate to be tested by consequences}.

A crucial distinction follows immediately:
a constraint can exist in code yet be irrelevant in a given regime.
We therefore separate \emph{non-binding} regimes (the raw evolution satisfies $|A|>\varepsilon$ and the constraint never activates) from \emph{binding} regimes (the raw evolution would dip below $\varepsilon$, requiring intervention).
This distinction is mandatory for intellectual honesty: any claim that ``the axiom matters'' must be accompanied by binding-regime evidence and ablations against a constraint-off baseline.

\subsection{Why we introduce a universal phase parameter}
The program's second organizing motif is the exploration of whether a single phase-like control parameter can coherently thread multiple domains.
We denote this parameter by $\theta\in[0,2\pi)$.
The role of $\theta$ is not to serve as a numerological constant, but to act as a candidate \emph{carrier} of a universal ``twist'' that could modulate:
(i) flavor-sector structure (mixings and phases),
(ii) the behavior of vacuum-like residuals under non-cancellation enforcement, and
(iii) the stability and character of downstream cosmological embeddings.

This immediately raises a methodological hazard:
if we fix a specific number $\theta^\star$ too early, later successes can appear to be engineered around a chosen value.
Phase~0 therefore formalizes a corridor-based protocol in which $\theta^\star$ is treated, at most, as a \emph{fiducial representative} within a larger admissible set.

\subsection{Corridors instead of imposed constants}
The key idea is to treat each phase as a filter on $\theta$.
We begin from an initial prior corridor $\Theta_0\subset[0,2\pi)$ (possibly broad).
Each phase $i$ defines explicit pass/fail tests $\mathcal{T}_i(\theta)$ that depend on reproducible artifacts (binding certificates, stability bounds, scaling checks, and transfer viability).
The phase outputs an admissible set
\begin{equation}
\Theta_{i+1} = \{\theta\in\Theta_i : \mathcal{T}_i(\theta)\ \mathrm{passes}\},
\label{eq:corridor_update_intro}
\end{equation}
possibly disconnected, and the program advances by intersecting admissible sets across independent layers.
In this framing, a single $\theta^\star$ is \emph{earned} only if the intersection of independent constraints collapses the corridor toward a unique value (or a small discrete set), rather than being assumed at the outset.

\subsection{What Phase~0 contributes}
Phase~0 locks the epistemic and reproducibility scaffolding required for later phases to be meaningful:
\begin{itemize}
  \item clear definitions of postulate versus implementation versus model versus consequence;
  \item a binding/non-binding regime distinction and the requirement of ablation evidence for causal claims;
  \item the corridor method for propagating admissible $\theta$ sets across phases;
  \item phase-by-phase contracts specifying what each phase is permitted to claim and what it must not claim;
  \item a minimal deterministic ledger mechanism that aggregates standardized per-phase $\theta$-filter artifacts into a versioned corridor history and dashboard.
\end{itemize}
This paper therefore serves as the series' ``contract layer'': it does not attempt unification or parameter prediction.
It establishes the rules by which subsequent claims will be stated, tested, audited, and either locked or rejected.