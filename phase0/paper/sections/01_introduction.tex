\section{Introduction}
\label{sec:introduction}

This repository is organized as a governed, multi-phase research program. The role of Phase 0 is to define the governance layer: how claims are written, what counts as evidence, how phases are scoped, how reproducibility is enforced, and how failure is recorded.

\paragraph{What Phase 0 is.}
Phase 0 is a constitution for the project. It specifies:
(i) a claim taxonomy and required claim structure,
(ii) an evidence contract based on canonical artifacts and file pointers,
(iii) phase scope contracts (including explicit non-claims boundaries),
(iv) a reproducibility contract (deterministic pipelines and run provenance),
(v) explicit failure modes and falsifiers,
and (vi) corridor bookkeeping for global filters (notably $\theta$-corridors).

\paragraph{What Phase 0 is not.}
Phase 0 does not make physics claims. It does not assert a specific value of $\theta^\star$, does not claim a mechanism for vacuum energy, does not infer cosmology, and does not treat any narrative motivation as evidence. Any physical statements belong to later phases and are valid only if they satisfy the contracts defined here.

\paragraph{Why governance is required.}
Without governance, research repositories drift in three predictable ways:
claims drift (text diverges from figures), artifact drift (results live in ad-hoc folders or unreproducible runs), and reproducibility drift (code or environment changes alter outputs without a clear audit trail). Phase 0 is designed to prevent these failure modes by making the project auditable end-to-end.

\paragraph{Project rule: no dark progress.}
Work is only considered ``real'' when it is logged and reproducible. Each change that affects claims, figures, or phase papers must be recorded in the progress log(s) and must be recoverable from versioned inputs (code, configs, and referenced artifacts). This is not bureaucracy; it is the minimal condition for defensible scientific communication.

\paragraph{How to read this document.}
Section~\ref{sec:axioms_definitions} defines the governance vocabulary. Section~\ref{sec:corridor_method} defines corridor artifacts and their append-only history. Section~\ref{sec:phase_contracts} defines what phases may claim and how claims must be structured. Section~\ref{sec:repro_contract} defines reproducibility requirements. Section~\ref{sec:falsifiability} defines failure modes and falsifiers. Later phases are required to satisfy these contracts as a condition of being considered ``locked''.
