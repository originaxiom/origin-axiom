\section{Phase contracts: scope, claims, and evidence binding}
\label{sec:phase_contracts}

A phase is a governed unit of work. This section defines what a phase must declare, what it is allowed to claim, and how claims bind to evidence. These contracts are mandatory: statements that violate a phase contract are invalid by definition.

\subsection{Scope contract (required)}
Every phase must provide a scope contract with:
\begin{enumerate}
  \item \textbf{Scope:} what the phase is intended to establish (and at what claim strength).
  \item \textbf{Non-claims:} a list of statements the phase explicitly does not assert.
  \item \textbf{Primary artifacts:} the canonical outputs used as evidence (figures, tables, summaries, run bundles).
\end{enumerate}

\paragraph{Binding rule.}
A phase may only assert claims that are consistent with its declared scope and non-claims. If a text statement exceeds scope or contradicts a non-claim boundary, it must be removed or moved to a later phase.

\subsection{Claim taxonomy (governance types)}
This project uses a small taxonomy of claim types. A phase must label the type of each claim, because type determines what evidence is required.

\begin{description}
  \item[Existence claim] Asserts that a defined effect or quantity is non-zero / present under stated conditions.
  \item[Robustness claim] Asserts that an effect persists under controlled variations (parameter sweeps, numerical settings).
  \item[Bounded viability claim] Asserts consistency with a comparator or constraint \emph{within a limited test}, without claiming a full physical explanation.
  \item[Mechanism claim] Asserts that a specific causal mechanism is responsible (high burden; typically deferred).
  \item[Prediction claim] Asserts a falsifiable quantitative prediction for external comparison (highest burden; typically deferred).
\end{description}

Phases must not label claims more strongly than their evidence supports.

\subsection{Required claim structure}
Every claim must appear in a claims ledger and must have the following fields.

\paragraph{Claim ID.}
A stable identifier of the form \texttt{P\{phase\}-C\{number\}} (e.g., \texttt{P2-C03}). IDs are stable across edits.

\paragraph{Claim statement (one sentence).}
A single sentence that states exactly what is asserted, including the conditions under which it holds.

\paragraph{Evidence pointers (file paths).}
Explicit repository paths to canonical artifacts supporting the claim (figures, tables, data summaries). Narrative discussion does not count as evidence.

\paragraph{Non-claim boundary.}
A bullet list of statements that the claim does \emph{not} imply. This prevents accidental over-interpretation.

\paragraph{Falsifiers / failure conditions.}
A bullet list of conditions under which the claim should be rejected (numerical instability, violation under parameter sweep, comparator failure, irreproducibility, etc.).

\paragraph{Provenance pointers.}
Where applicable, claims must reference run IDs and configuration identifiers sufficient to regenerate the evidence artifacts.

\subsection{Evidence binding rule}
A claim is valid only if its evidence artifacts can be located and reproduced:
\begin{enumerate}
  \item the artifact exists at the referenced path (or is generated deterministically from versioned inputs);
  \item the artifact is generated by a declared pipeline or script under a declared configuration;
  \item provenance (run ID, config snapshot, code version) is recorded to reproduce it.
\end{enumerate}

If any of these conditions fails, the claim must be marked as unproven (or removed) until evidence is repaired.

\subsection{Phase 0 compliance checklist (reviewer-facing)}
\label{sec:p0_checklist}

A phase is considered \emph{locked} only if it satisfies the following checklist:

\begin{enumerate}
  \item \textbf{Scope contract present:} scope + non-claims + primary artifacts are declared.
  \item \textbf{Claims ledger complete:} every claim has an ID, one-sentence statement, evidence pointers, non-claim boundary, falsifiers.
  \item \textbf{Evidence is canonical:} evidence artifacts are versioned or reproducible from versioned inputs.
  \item \textbf{Reproducibility path exists:} scripts/workflow + config + provenance are sufficient to regenerate evidence.
  \item \textbf{Assumptions/approximations explicit:} documented and linked to sensitivity/limitations.
  \item \textbf{Failure modes explicit:} falsifiers defined; failures are logged, not hidden.
  \item \textbf{Corridor governance obeyed:} any narrowing is recorded as a filter artifact and appended to corridor history with provenance.
\end{enumerate}

Phase 0 satisfies this checklist by construction. Later phases must satisfy it as a condition for publication-ready claims.
