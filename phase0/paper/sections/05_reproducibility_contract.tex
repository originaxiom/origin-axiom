% paper/sections/05_reproducibility_contract.tex

\section{Reproducibility contract}
\label{sec:repro_contract}

Reproducibility is treated here not as an aspirational virtue but as a formal requirement for phase locking.
This section defines the minimum artifact, logging, and provenance standards that each phase must meet.
Any result that does not satisfy these standards may be explored, but it may not be used as an authoritative input for later phases.

\subsection{Artifact-first claims}
\label{sec:artifact_first}

Every technical claim in the series must be reducible to a finite set of regenerable artifacts:
figures, tables, and machine-readable summaries produced by logged runs.
A claim is considered \emph{valid} only if it is accompanied by:
\begin{itemize}
  \item the artifact identifier (figure/table label),
  \item the generating run identifier (run directory or run ID),
  \item the configuration snapshot used by the run,
  \item and the exact code revision (git commit hash).
\end{itemize}
Statements that are not tied to artifacts must be explicitly labeled as motivation, conjecture, or future work.

\subsection{Run manifests}
\label{sec:run_manifests}

Each phase must provide a run manifest mapping every headline figure/table to its provenance.
At minimum, a manifest entry includes:
\begin{itemize}
  \item the script or module entrypoint used to generate the artifact,
  \item the configuration file(s) and the resolved parameter snapshot,
  \item the run directory containing raw outputs and summaries,
  \item the canonical staged artifact included in the paper build,
  \item hashes/signatures sufficient to detect drift (config hash, artifact hash).
\end{itemize}
The manifest may be presented as an appendix table or as a machine-readable file referenced by the paper.

\subsection{Ablation discipline}
\label{sec:ablation_contract}

Causal attribution requires ablation.
Any claim of the form ``OA enforcement produces effect $X$'' must be supported by paired runs:
\begin{enumerate}
  \item \textbf{Control (OFF):} enforcement disabled (or $\varepsilon=0$), producing $(A_{\mathrm{raw}}, R_{\mathrm{raw}})$.
  \item \textbf{Treatment (ON):} enforcement enabled with specified $\varepsilon>0$, producing $(A_{\mathrm{con}}, R_{\mathrm{con}})$.
\end{enumerate}
All other parameters and random seeds must be identical.
If seeds are not applicable, the phase must explain what replaces stochastic reproducibility (e.g.\ deterministic integrators with fixed tolerances).

\subsection{Binding certificates: the non-negotiable gate}
\label{sec:binding_gate}

Because the series centers on non-cancellation enforcement, any phase that claims OA relevance must include binding evidence.
A \emph{binding certificate} consists of:
\begin{itemize}
  \item an explicit \texttt{constraint\_applied} flag (true/false),
  \item a quantitative measure of binding intensity (e.g.\ \texttt{n\_hits}, \texttt{hit\_fraction}, or $\min_t(|A_{\mathrm{raw}}(t)|-\varepsilon)$),
  \item at least one overlay diagnostic comparing constraint-off vs constraint-on behavior in the binding regime.
\end{itemize}
Phases may include non-binding results as sanity checks, but non-binding results cannot be used to support causal claims about OA impact.

\subsection{Logging requirements}
\label{sec:logging_requirements}

Each run that contributes to a headline artifact must log:

\paragraph{Configuration.}
A complete configuration snapshot and a fully resolved parameter file (including defaults expanded and derived values recorded).

\paragraph{Provenance.}
\begin{itemize}
  \item git commit hash,
  \item dirty state indicator (whether uncommitted changes were present),
  \item run timestamp in UTC,
  \item platform information (OS, CPU/GPU if relevant).
\end{itemize}

\paragraph{Numerics.}
\begin{itemize}
  \item step sizes, solver tolerances, discretization parameters,
  \item random seeds and RNG algorithms when applicable,
  \item run length (steps/time), and any termination criteria.
\end{itemize}

\paragraph{Environment.}
A package/environment snapshot sufficient to recreate the runtime (e.g.\ \texttt{pip freeze} or equivalent).

\paragraph{Summaries.}
A machine-readable \texttt{summary.json} containing the scalar quantities used by captions and tables, and pointers to raw arrays when needed.

\subsection{Deterministic phase ledger artifacts}
\label{sec:ledger_contract}

To ensure the corridor method is not merely narrative, each phase must emit a standardized \texttt{theta\_filter} artifact.
At minimum, the artifact includes:
\begin{itemize}
  \item the prior corridor used by the phase,
  \item the evaluated $\theta$ grid or interval representation,
  \item pass/fail outcomes for each $\theta$ under the declared test suite,
  \item optional scores ranking admissible $\theta$ values,
  \item full provenance linking each evaluated $\theta$ to run identifiers.
\end{itemize}
The Phase~0 ledger consumes these artifacts in order and deterministically reproduces the corridor history.
A phase is not lockable unless its \texttt{theta\_filter} artifact can be processed by the ledger without manual intervention.

\subsection{Versioning and drift control}
\label{sec:drift_control}

A common failure mode in long projects is silent drift: figures are regenerated under different assumptions without traceability.
The series therefore adopts the following drift controls:
\begin{itemize}
  \item headline artifacts are staged in canonical locations with signature/hash files;
  \item paper builds are reproducible from a clean checkout at the referenced commit;
  \item any change to configuration defaults or model definitions that could affect results must trigger regeneration of affected artifacts and an explicit changelog entry.
\end{itemize}
These controls are treated as part of the scientific method of the program, not as auxiliary engineering.