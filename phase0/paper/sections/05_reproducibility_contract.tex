\section{Reproducibility contract: deterministic pipelines and run provenance}
\label{sec:repro_contract}

A claim is only as strong as its reproducibility. This section defines the minimum reproducibility requirements for any phase that produces evidence artifacts. These requirements are binding: if an artifact cannot be regenerated under the stated procedure, the corresponding claim must be marked as unproven until reproducibility is restored.

\subsection{Principles}

\paragraph{Determinism over convenience.}
Evidence artifacts must be generated by deterministic scripts or workflows using versioned inputs and explicit parameters. Interactive or manual steps are allowed only if they are fully specified and reproducible (and should be avoided for canonical artifacts).

\paragraph{One-click (or one-command) regeneration.}
Each phase must provide a single documented command (or short sequence) that regenerates all canonical artifacts for that phase from versioned inputs.

\paragraph{No dark progress.}
Evidence is not ``real'' until it is reproducible and logged. Cached outputs, screenshots, and ad-hoc runs are not acceptable as canonical evidence.

\subsection{Canonical artifacts vs. regenerable outputs}

\paragraph{Canonical artifacts.}
Canonical artifacts are the evidence objects referenced by claims (e.g., figures included in phase papers, summary tables, corridor/filter JSON artifacts). Canonical artifacts must have stable paths and must be traceable to a run provenance record.

\paragraph{Regenerable outputs.}
Large intermediate files, caches, and raw simulation outputs may be excluded from version control if they can be regenerated deterministically. If excluded, the phase must specify how regeneration occurs and how equivalence is checked (hashes, summary statistics, or figure reproduction).

\subsection{Run provenance (minimum required)}

Each phase must define a \emph{run provenance record} sufficient to regenerate its canonical artifacts. At minimum, provenance must include:

\begin{itemize}
  \item \textbf{Run ID:} a unique identifier for the execution that produced the artifacts.
  \item \textbf{Code version:} commit hash (or tag) identifying the exact repository state.
  \item \textbf{Configuration snapshot:} the full parameter/config file(s) used (YAML/JSON), stored or copied into the run bundle.
  \item \textbf{Environment snapshot:} package versions and runtime metadata (e.g., \texttt{pip freeze}, OS/Python version).
  \item \textbf{Command/workflow:} the exact command(s) used (or workflow target name), so the run can be repeated.
  \item \textbf{Outputs index:} a list of generated canonical artifacts and their paths.
\end{itemize}

\subsection{Run bundle (recommended structure)}

A \emph{run bundle} is a directory that packages provenance and outputs for a run. A recommended minimal structure is:

\begin{quote}\ttfamily
outputs/runs/<run\_id>/\\
\ \ \ config\_snapshot.yaml\\
\ \ \ env\_snapshot.txt\\
\ \ \ command.txt\\
\ \ \ outputs\_index.json\\
\ \ \ logs/\\
\ \ \ data/ (optional, if not too large)\\
\ \ \ figures/ (canonical figures for this run)
\end{quote}

Phases may extend this structure, but must preserve the minimum provenance fields.

\subsection{Workflow contract}

\paragraph{Workflow engines (example).}
Workflow engines may be used to enforce deterministic artifact regeneration (e.g., Snakemake), but the workflow specification and invocation must be treated as part of the evidence contract.\cite{snakemake2012,snakemake2018}


\paragraph{Pipeline runner.}
If a phase uses a workflow system (e.g., Snakemake), the workflow specification is part of the evidence contract. The phase must document how to invoke the workflow and how the workflow maps to canonical artifacts.

\paragraph{Parameter centralization.}
Phases should centralize parameters into a small number of config files (preferably one). Parameters must not be silently embedded only in code without being surfaced in a config snapshot.

\paragraph{Deterministic randomness.}
If stochastic components exist, phases must fix seeds and record them in the configuration snapshot. If deterministic operation is not possible, the phase must define a reproducibility criterion (e.g., statistical envelope, fixed summary metrics).

\subsection{Reproducibility failure handling}

If a canonical artifact cannot be regenerated:
\begin{enumerate}
  \item record the failure in the phase progress log with a minimal error summary;
  \item mark any dependent claims as ``evidence broken'' until repaired;
  \item repair reproducibility by restoring missing inputs, fixing the pipeline, or updating provenance artifacts;
  \item record the repair (and any changes to outputs) explicitly in the log.
\end{enumerate}

This ensures the repository remains auditable even when failures occur.
