\section{Corridor method: filters, corridors, and append-only history}
\label{sec:corridor_method}

This project introduces \emph{corridor governance} to prevent uncontrolled parameter drift. The corridor method treats global restrictions (notably $\theta$-ranges) as first-class, machine-readable artifacts with append-only provenance.

\subsection{Motivation (governance)}
In multi-step research programs, a key parameter may be implicitly narrowed over time by narrative convention, selective figure choices, or untracked tuning. Corridor governance prevents this: any narrowing must be explicit, justified by evidence, and recorded with provenance.

\subsection{Objects}

\paragraph{Filter artifact.}
A \emph{filter} is a structured, machine-readable restriction applied by a phase. Filters are saved as JSON and validated against a schema. A filter may restrict one or more variables (e.g., $\theta$ intervals), and may include a justification pointer.

\paragraph{Corridor artifact.}
A \emph{corridor} is the authoritative, current restriction for a global variable maintained at the project level (e.g., the current $\theta$ corridor). The corridor is also a schema-validated JSON artifact.

\paragraph{Corridor history.}
The \emph{corridor history} is an append-only log of corridor updates. The history is the source of truth for ``how we got here'' and must be sufficient to audit every narrowing step.

\subsection{Required fields and schemas}

All filter and corridor artifacts must be schema-validated. At minimum, a corridor update entry must record:

\begin{itemize}
  \item \textbf{Target:} which variable is being restricted (e.g., \texttt{theta}).
  \item \textbf{Intervals:} the allowed range(s) after the update (possibly disjoint).
  \item \textbf{Phase provenance:} which phase produced the update.
  \item \textbf{Evidence pointers:} canonical artifact paths that justify the update (figures, tables, summaries).
  \item \textbf{Run provenance:} run ID(s) for the generating pipeline execution.
  \item \textbf{Code provenance:} commit hash (or equivalent version identifier) for the code state.
  \item \textbf{Timestamp:} when the update was made.
  \item \textbf{Rationale:} a short statement of the rule used for narrowing.
\end{itemize}

Schemas define the exact field names and types. A corridor update that fails schema validation is invalid.

\subsection{Update rules (what is allowed)}

\paragraph{Rule 1: No silent narrowing.}
A phase may not narrow a corridor by narrative convention. If a paper or claim relies on a restricted range, that restriction must exist as a filter artifact and must be recorded in the corridor history.

\paragraph{Rule 2: Narrowing requires evidence.}
A corridor may be narrowed only if the narrowing rule is supported by a canonical artifact that:
(i) is reproducible under the phase's pipeline, and
(ii) directly implements the stated narrowing criterion.

Examples of valid narrowing criteria include:
\begin{itemize}
  \item a parameter sweep showing instability outside a range (numerical falsifier),
  \item a fit objective producing a bounded confidence region (statistical constraint),
  \item an explicit pass/fail test (viability filter) applied uniformly across the domain.
\end{itemize}

\paragraph{Rule 3: Narrowing must preserve auditability.}
For any corridor update, an independent reader must be able to:
\begin{enumerate}
  \item locate the exact evidence artifacts cited,
  \item locate the run bundle(s) that generated them,
  \item reproduce the artifacts from versioned inputs,
  \item confirm that the narrowing rule matches what the artifacts show.
\end{enumerate}

\paragraph{Rule 4: Append-only history.}
The corridor history is append-only. Past corridor states are never overwritten. Reversals are allowed (widening is allowed) but must also be recorded as explicit history entries with provenance.

\subsection{Artifacts in this repository}

The repository maintains, at minimum:
\begin{itemize}
  \item \textbf{Schemas:} JSON schemas for filters and corridors.
  \item \textbf{Current corridor:} the authoritative current corridor JSON (e.g., the current $\theta$ corridor).
  \item \textbf{History log:} an append-only JSONL file recording each update entry.
  \item \textbf{Dashboard:} a human-readable summary pointing to the current corridor and the latest update entry.
\end{itemize}

These objects exist to make phase-to-phase narrowing explicit and defensible. Later phases must treat corridor artifacts as binding constraints when a corridor exists.

\subsection{Relationship to falsifiability}
Corridor governance links governance to falsifiability: a narrowing is a \emph{testable commitment}. If later work shows that a narrowed corridor fails a claim's own falsifiers, the corridor update can be reversed (widened) via an explicit history entry. This preserves scientific honesty while maintaining auditability.
