% paper/sections/03_corridor_method.tex

\section{The corridor method}
\label{sec:corridor_method}

The corridor method is the program's mechanism for avoiding premature fixation on a single numerical phase value and for converting multi-domain consistency into a quantitative narrowing of admissible parameter space.
It is a disciplined alternative to ``choose $\theta^\star$ and build around it'': each phase contributes a filter, and the surviving set is the intersection of independent filters.

\subsection{Admissible sets and phase filters}
\label{sec:admissible_sets}

Let $\Theta_0 \subset [0,2\pi)$ denote an initial prior corridor for the universal phase parameter $\theta$.
Phase~0 permits $\Theta_0$ to be broad (including $\Theta_0=[0,2\pi)$) provided later phases define tests that are sufficiently $\theta$-informative to narrow it.

Each phase $i$ defines:
\begin{itemize}
  \item a model family and an implementation of OA enforcement;
  \item a test suite $\mathcal{T}_i(\theta)$ composed of explicit pass/fail criteria;
  \item a scan or evaluation strategy over $\theta \in \Theta_i$ (grid scan, adaptive scan, or analytic characterization).
\end{itemize}
The phase outputs an admissible set
\begin{equation}
\Theta_{i+1} \;=\; \{\theta \in \Theta_i : \mathcal{T}_i(\theta)\ \mathrm{passes}\}.
\label{eq:corridor_update}
\end{equation}
The corridor $\Theta_{i+1}$ may be a single interval, a union of intervals, or a discrete set of isolated points.

\subsection{What belongs in a test suite}
\label{sec:test_suite}

A test suite $\mathcal{T}_i$ should be:
\emph{explicit}, \emph{artifact-backed}, and, when possible, \emph{independent} of tests used in other phases.
Typical elements include:

\paragraph{Binding test (if the phase claims OA impact).}
There must exist a regime within the phase's explored parameter space where enforcement activates (binding certificate),
and the constrained run differs from the constraint-off ablation.

\paragraph{Non-invasiveness test.}
In non-binding regimes, constrained and unconstrained runs must coincide up to numerical tolerance.

\paragraph{Stability test.}
The dynamics must remain bounded (no runaway) and numerically stable under specified sweeps (step size, resolution, cutoff).

\paragraph{Robustness test.}
Headline behavior must persist under a defined set of controlled perturbations (e.g., small parameter changes, discretization changes).

\paragraph{Transfer viability test.}
If the phase exports a derived object to a downstream module (e.g.\ mapping a residual into an FRW term), the exported object must keep the downstream solver stable and within a predeclared envelope.

\subsection{Corridor narrowing and the role of thresholds}
\label{sec:narrowing}

Corridors narrow only if tests are $\theta$-informative.
Phase~0 highlights three common failure modes:
\begin{enumerate}
  \item \textbf{Weak dependence:} the diagnostic is nearly flat in $\theta$, so $\mathcal{T}_i$ does not discriminate.
  \item \textbf{Always non-binding:} the floor never activates for any $\theta$ in the explored regime, so OA impact is not tested.
  \item \textbf{Always binding:} enforcement activates for all $\theta$ in the explored regime and saturates outputs, erasing $\theta$-structure.
\end{enumerate}

Threshold-driven tests are particularly valuable because they can create sharp boundaries in $\theta$-space.
For example, if a binding condition is expressed as $R_{\mathrm{raw}}(\theta) < \varepsilon$, then the set of $\theta$ values for which binding occurs is naturally separated from those for which it does not, producing a ``kink'' structure.
Stacking multiple such thresholds across independent layers is one realistic route by which an initially broad corridor can collapse to a narrow interval or a discrete set.

\subsection{Fiducial representatives and avoiding numerology}
\label{sec:fiducial}

The corridor method permits the use of a fiducial representative $\theta^\star$ for:
\begin{itemize}
  \item illustrative plots,
  \item benchmarking runtime and numerical stability,
  \item pinning a single run for detailed diagnostics.
\end{itemize}
However, $\theta^\star$ must be labeled as \emph{fiducial} unless and until the corridor-intersection protocol collapses the admissible set to a unique value (or a small discrete set) under independent constraints.
This labeling rule is the program's primary defense against accidental numerology.

\subsection{From corridors to a point: when is a unique $\theta^\star$ earned?}
\label{sec:collapse}

A unique $\theta^\star$ is earned only if:
\begin{itemize}
  \item the admissible corridor narrows monotonically (or nearly so) as independent constraints are added, and
  \item the final intersection $\Theta_{\mathrm{final}}$ is either a single sufficiently small interval or a discrete set with a single element after accounting for symmetries (e.g.\ $\theta \sim \theta + 2\pi$ and other model-specific equivalences).
\end{itemize}
Phase~0 does not assume that such collapse must occur.
If the corridor remains broad, the program interprets this as information: either $\theta$ is not uniquely fixed by the considered constraints, or the current phases do not supply sufficiently independent $\theta$-informative tests.

\subsection{The Phase~0 ledger mechanism (deterministic corridor tracking)}
\label{sec:ledger}

To prevent drift and to make corridor evolution auditable, Phase~0 introduces a minimal deterministic ledger mechanism.
Each phase emits a standardized \texttt{theta\_filter} artifact (JSON) encoding:
\begin{itemize}
  \item the prior corridor used by the phase,
  \item the $\theta$ values evaluated (grid or intervals),
  \item pass/fail per $\theta$ under $\mathcal{T}_i$,
  \item optional scores ranking admissible $\theta$ values,
  \item and full provenance (git commit, config hash, run identifiers).
\end{itemize}
The ledger consumes the ordered list of phase artifacts and produces:
\begin{itemize}
  \item the current corridor $\Theta$ as a union of intervals on $[0,2\pi)$,
  \item an append-only corridor history log,
  \item and a human-readable dashboard summarizing which phases narrowed the corridor and why.
\end{itemize}
The ledger is strictly functional: given the same phase artifacts, it deterministically reproduces the same corridor history.