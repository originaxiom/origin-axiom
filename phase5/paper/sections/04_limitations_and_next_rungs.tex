\section{Limitations and Next Rungs}
\label{sec:limitations}

\subsection{Limitations}

The current Phase 5 implementation is intentionally narrow:
\begin{itemize}[leftmargin=*]
  \item It does not yet integrate into the unified paper build pipeline
        as a canonical Phase 5 PDF artifact.
  \item It does not present numerical results, plots, or summary
        statistics beyond the raw interface diagnostics.
  \item It does not define new claims, hypotheses, or falsifiable
        predictions on top of Phases 3 and 4.
  \item It does not ship with external datasets; the FRW distance file,
        if used, is explicitly optional and may be absent from a clean
        clone of the repository.
\end{itemize}

These limitations are consistent with the role of this document as a
skeleton: it is preferable to under-claim and then move carefully to
stronger rungs.

\subsection{Next Rungs (Conceptual)}

Possible next rungs for Phase 5 include:
\begin{itemize}[leftmargin=*]
  \item wiring a Phase 5 paper build into a dedicated gate script and, if
        appropriate, into the unified build driver;
  \item adding a small set of well-motivated summary tables that pull
        together Phase 3 and Phase 4 diagnostics without duplicating
        their internal derivations;
  \item defining a minimal, explicit set of ``Phase 5 claims'' that are
        purely about program structure and reproducibility (for example,
        that a particular version of the repository exposes a complete
        and self-consistent contract between phases);
  \item extending the interface to support multiple mapping families or
        vacuum constructions while keeping the contracts explicit and
        versioned.
\end{itemize}

Each of these steps should be introduced via its own rung, with clear
diffs and explicit documentation of what changed and why.
