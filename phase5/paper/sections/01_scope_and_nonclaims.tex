\section{Scope and Non-Claims}
\label{sec:scope}

\subsection{Scope}

The scope of Phase 5, at this skeleton stage, is purely programmatic:
\begin{itemize}[leftmargin=*]
  \item It reads a configuration file that specifies which Phase 3 and
        Phase 4 outputs are required and where they live in the
        repository.
  \item It runs an interface script that verifies that these assets
        exist, reports their sizes, and records this information in a
        machine-readable summary.
  \item It treats the Phase 3 and Phase 4 outputs as \emph{locked
        inputs}: Phase 5 may reason about them, but does not modify or
        regenerate them.
\end{itemize}

Concretely, the current implementation is centered on:
\begin{itemize}[leftmargin=*]
  \item the configuration file
        \texttt{phase5/config/phase5\_inputs\_v1.json}, and
  \item the summary written by
        \texttt{phase5/src/phase5/phase5\_interface\_v1.py}, namely
        \texttt{phase5/outputs/tables/phase5\_interface\_v1\_summary.json}.
\end{itemize}

\subsection{Non-Claims}

At this rung, Phase 5 explicitly \emph{does not}:
\begin{itemize}[leftmargin=*]
  \item assert any new physical explanation of the cosmological
        constant, vacuum structure, or field content;
  \item introduce new parameter fits or change the numerical values
        produced by Phases 3 and 4;
  \item draw quantitative cosmological conclusions from external data
        sources;
  \item make any statements about observational viability beyond what is
        already locked into the upstream phases.
\end{itemize}

All interpretive, explanatory, or phenomenological claims remain
anchored in the earlier phases. Phase 5 is an interface and program
layer, not a new theoretical rung.
