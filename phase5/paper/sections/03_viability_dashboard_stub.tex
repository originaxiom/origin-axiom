\section{Interface-Level Viability Dashboard (Rung 2 Skeleton)}
\label{sec:phase5-viability-dashboard}

At this stage, the ``viability dashboard'' implemented by Phase 5 is
deliberately minimal and purely programmatic. Its role is not to define
new physical metrics or scores, but to present a compact, machine- and
human-readable view of the Phase 3/4 artifacts that the interface sees
as inputs.

The script
\begin{center}
  \texttt{phase5/src/phase5/make\_interface\_dashboard\_v1.py}
\end{center}
takes as input the interface summary JSON produced by Rung~0,
\begin{center}
  \texttt{phase5/outputs/tables/phase5\_interface\_v1\_summary.json},
\end{center}
and emits a flattened CSV
\begin{center}
  \texttt{phase5/outputs/tables/phase5\_interface\_dashboard\_v1\_summary.csv}.
\end{center}

Each row of this CSV corresponds to a single entry in the interface
configuration (for example, a Phase 3 table or a Phase 4 FRW diagnostic
file). The columns are purely structural:
\begin{itemize}[leftmargin=*]
  \item \textbf{section}: a coarse label such as \texttt{phase3},
        \texttt{phase4}, or \texttt{external};
  \item \textbf{key}: the short identifier used in the configuration
        (e.g.\ \texttt{mech\_baseline\_diagnostics},
        \texttt{frw\_shape\_probe});
  \item \textbf{relpath}: the repository-relative path, when the entry
        corresponds to an on-disk artifact;
  \item \textbf{exists}: a boolean indicating whether the file is
        present on disk;
  \item \textbf{size\_bytes}: the file size in bytes (when the file
        exists);
  \item \textbf{is\_path}: a flag distinguishing path-like entries from
        purely descriptive notes;
  \item \textbf{note}: an optional text annotation for non-path entries
        (for example, to document that a dataset is external and
        optional).
\end{itemize}

In this Rung 2 skeleton, the dashboard is intentionally limited to
reporting these structural facts. It does \emph{not}:
\begin{itemize}[leftmargin=*]
  \item introduce any new derived quantities or scores,
  \item aggregate diagnostics into a single ``viability'' number, or
  \item perform comparisons against external cosmological datasets.
\end{itemize}

For convenience, the column structure can be summarized schematically as
follows:
\begin{center}
\begin{tabular}{lllllll}
\hline
section & key & relpath & exists & size\_bytes & is\_path & note \\
\hline
\multicolumn{7}{c}{\dots rows populated by
\texttt{make\_interface\_dashboard\_v1.py} \dots} \\
\hline
\end{tabular}
\end{center}

Future rungs of Phase 5 may refine this dashboard by:
\begin{itemize}[leftmargin=*]
  \item introducing explicit, phase-aligned summary metrics derived from
        existing diagnostics;
  \item adding small LaTeX tables or figures that present these metrics
        in a compact way; and
  \item wiring selected summary artifacts into the unified program-level
        dashboard discussed in the planning documents.
\end{itemize}
Any such extensions must remain honest about what is computed where and
must not silently re-fit or reinterpret the upstream phases.

In the current Rung~2 configuration, the dashboard also includes a row
for the Phase~4 external FRW diagnostics JSON. This row mirrors the
interface contract: it reports the presence, size, and status of
\texttt{phase4\_F1\_frw\_external\_diagnostics.json} as seen by
\texttt{phase5\_interface\_v1}. No additional aggregation or scoring is
performed in Phase~5; the entry exists solely so that collaborators can
see at a glance whether an external FRW distance dataset has been
attached and diagnosed upstream.

In addition, Rung~4 introduces a simple CSV sanity table,
\texttt{phase5\_rung4\_sanity\_table\_v1.csv}, which is generated from
the interface summary. Each row records the section, key, relative
path, existence flag, and file size for a Phase~3, Phase~4, or external
entry, with optional enrichment for the external FRW diagnostics
(\texttt{frw\_external\_diagnostics}) drawn from
\texttt{phase4\_F1\_frw\_external\_diagnostics.json}. This table is
purely program-level: it does not introduce any new scores or
likelihoods, but it makes the interface state easy to inspect and
compare across repositories and commits.
