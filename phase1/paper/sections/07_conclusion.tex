\section{Conclusion and Phase II Compatibility}
\label{sec:conclusion}

Phase~1 introduced a minimal \OA{} in the form of a global non-cancellation constraint $|A|>\epsfloor$. Using explicit, reproducible numerical artifacts, we demonstrated that: (i) finite cancellation ensembles generically exhibit strong residual suppression with system size in the absence of fine-tuning, (ii) imposing an amplitude floor yields a persistent nonzero baseline, (iii) the floor can be enforced within a dynamical lattice toy model without inducing obvious numerical instabilities, and (iv) the constrained mean residual saturates near $\epsfloor$ while the unconstrained residual continues to decrease with increasing system size.

These results establish the internal coherence of the axiom as a structural principle: exact cancellation is nongeneric in finite interference systems, and an explicit non-cancellation rule produces a stable residual scale under coarse variations of model parameters.

\paragraph{Phase~II requirements.}
To move beyond plausibility and toward physical relevance, Phase~2 must address several logically independent extensions of the present work:
\begin{itemize}
  \item a principled origin, selection rule, or dynamical mechanism for the scale $\epsfloor$,
  \item a physically interpretable mapping between residual amplitude and vacuum energy density,
  \item compatibility with locality and Lorentz structure, or a controlled and explicitly motivated violation thereof,
  \item regulator dependence, continuum behavior, and robustness under model generalization.
\end{itemize}

Phase~1 therefore serves as an auditable starting point: it cleanly separates axiom from implementation, demonstrates the phenomenon to be explained, and provides a reproducible existence proof without presupposing the outcome of subsequent theoretical development.