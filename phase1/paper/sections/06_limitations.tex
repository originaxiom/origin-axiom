\section{Limitations}

\subsection*{Relation to existing global-constraint ideas (context, not equivalence)}

The \OA{} is not presented as equivalent to established proposals, but it is conceptually adjacent to several frameworks in which nonlocal structure plays a role in the cosmological constant problem.

\paragraph{Unimodular gravity.}
In unimodular approaches, the cosmological constant appears as an integration constant or global degree of freedom rather than a parameter fixed directly by local vacuum energy contributions. Reviews emphasize both the appeal and the limitations of what is, and is not, resolved by this reformulation. \cite{Jirousek2023}

\paragraph{Vacuum energy sequestering.}
Sequestering models implement explicit global constraints that prevent matter-sector vacuum energy contributions from gravitating in the usual way, using nonlocal ingredients at the level of the action or equations of motion. \cite{KaloperPadilla2014,KaloperPadilla2014Framework}

\paragraph{Residual and fluctuation arguments in causal set theory.}
Causal set approaches have suggested that a small effective cosmological constant may arise as a residual effect associated with discreteness, with fluctuations scaling inversely with a large counting parameter. \cite{Sorkin2007,Amed2013} Phase~1 draws no dynamical or structural equivalence with these models; the relevance is limited to the shared logical possibility that small residuals can emerge from global counting or summation effects.

Phase~1 does not claim equivalence to any of the above frameworks. The narrower point is that the introduction of nonlocal constraints is not without precedent in serious theoretical work, and thus constitutes a legitimate hypothesis class subject to consistency and predictive scrutiny rather than dismissal on formal grounds alone.

\subsection*{Scope limitations of Phase~1}

Phase~1 is intentionally restricted in scope and makes several simplifying assumptions that delimit its interpretive reach.

\paragraph{Nonlocal constructions by design.}
Both the phasor ensemble and the lattice toy model operate on global observables constructed as sums or means over microscopic degrees of freedom. No notion of locality, causal propagation, or relativistic structure is imposed. Phase~1 therefore does not address how residual non-cancellation might arise from, or coexist with, strictly local field dynamics.

\paragraph{Toy-model nature.}
All numerical constructions in Phase~1 are schematic. The phasor ensemble abstracts interference without geometry, while the lattice model introduces geometry without physically motivated interactions. These models are intended solely as controlled existence proofs.

\paragraph{Absence of gravity and spacetime dynamics.}
No gravitational degrees of freedom are included, and no assumptions are made about spacetime curvature, expansion, or metric dynamics. Phase~1 does not constitute a cosmological model.

\paragraph{No distinguished angles or internal structure.}
The twist parameter $\theta$ functions only as a generic misalignment variable. No preferred value is derived or assumed, and no additional internal phases or hidden degrees of freedom are introduced.

\paragraph{No Standard Model embedding.}
Phase~1 does not incorporate Standard Model fields, symmetries, or interactions. Any connection between the residual non-cancellation scale and particle physics is explicitly outside the present scope.

Taken together, these limitations imply that Phase~1 should be read strictly as an existence and robustness study. It demonstrates that finite interference systems generically exhibit persistent residuals under minimal assumptions, without yet explaining their physical origin or ultimate interpretation.