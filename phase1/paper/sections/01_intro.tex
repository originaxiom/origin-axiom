\section{Introduction}

The cosmological constant problem and related questions concerning vacuum energy continue to motivate attempts to understand why large microscopic contributions fail to gravitate at macroscopic scales. Within standard local quantum field theory, cancellations can be arranged through symmetry or tuning, but generic radiative corrections tend to destabilize small values unless protected by additional structure. A persistent difficulty is therefore structural rather than numerical: why should the net vacuum contribution vanish exactly, rather than merely become small?

This Phase~1 work isolates a minimal hypothesis that directly targets this structural tension. We postulate a global non-cancellation constraint that forbids perfect destructive interference of a global complex amplitude. The proposal is intentionally stripped of phenomenological commitments: we avoid parameter extraction, special-number selection (such as $\varphi$ or $\varphi^\varphi$), flavor-physics inputs, or claims of uniqueness. The objective is narrower and more foundational: to state the principle cleanly and to demonstrate, in reproducible toy settings, that it yields a stable nonzero residual amplitude in regimes where unconstrained cancellation would otherwise drive the residual toward zero.

\paragraph{Scope and non-claims.}
This paper does not claim a solution to the cosmological constant problem, nor a derivation of the scale $\epsfloor$, nor an embedding within the Standard Model or a realistic cosmological framework. Instead, it provides a proof-of-concept that a global non-cancellation principle can be implemented consistently in controlled models and that such an implementation produces a robust residual under coarse variations of system size, initial conditions, and cancellation strength.

\paragraph{Structure.}
Section~\ref{sec:axiom} states the minimal axiom and clarifies its meaning, assumptions, and limitations. Section~\ref{sec:toy} introduces finite-dimensional phasor ensembles as a baseline model of interference and cancellation. Sections~\ref{sec:methods} and \ref{sec:results} present a reproducible lattice existence proof together with a scaling analysis. Section~\ref{sec:limitations} summarizes limitations of Phase~1 and delineates the requirements for Phase~2.