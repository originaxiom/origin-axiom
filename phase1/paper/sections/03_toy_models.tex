\section{Toy Models: Finite Phasor Ensembles}
\label{sec:toy}

We begin with a finite-dimensional model of cancellation: sums of complex unit phasors.
Let
\begin{equation}
  S \;=\; \sum_{j=1}^{N} e^{i\alpha_j},
\end{equation}
with phases $\alpha_j$ drawn from a distribution on $[0,2\pi)$. In typical random ensembles,
$|S|$ exhibits partial cancellation; a standard expectation is that $|S|$ scales sublinearly with $N$
(e.g.\ random-walk behavior).

\subsection{A controlled ``twist'' parameter}

To represent a tunable departure from perfect anti-alignment, we introduce a parameter $\theta$ that
shifts a fraction of phases by a fixed offset. This produces ensembles in which cancellation remains
strong but is not finely tuned to exact anti-alignment.

\subsection{Incommensurate twists as a non-fine-tuning device (Phase I)}
\label{sec:irrational_twist}

A common concern is whether near-cancellation might still produce exact zeros at special times or system sizes.
In finite systems with rational commensurabilities, exact recurrences can occur.
A minimal way to suppress such recurrences is to avoid commensurability in relative phases,
for example by choosing $\theta/2\pi \notin \mathbb{Q}$.

Phase I uses this only as a \emph{genericity device}: the point is not that a particular irrational
is selected, but that exact recurrences become non-generic once commensurability is avoided.
Accordingly, Phase I remains $\theta$-agnostic: it treats ``irrational'' as a class of choices, not a
distinguished value. Any later claim that a specific irrational is selected must be justified by an
independent, non-numerological principle and is explicitly deferred.

\subsection{Non-cancellation floor}

Phase I implements the axiom at the level of the observable residual by considering
\begin{equation}
  |S| \mapsto \max(|S|,\epsfloor).
\end{equation}
This is not presented as a microphysical law or dynamical equation but as a minimal operational
encoding of \eqref{eq:axiom} for a toy observable.

\subsection{A baseline cancellation lemma: random phasor sums}
\label{sec:phasor_lemma}

To quantify typical cancellation without fine-tuning, consider i.i.d.\ phases
$\alpha_j \sim \mathrm{Unif}(0,2\pi)$ and define
$S=\sum_{j=1}^N e^{i\alpha_j}$.
Write $S = X+iY$ with
$X=\sum \cos\alpha_j$ and $Y=\sum \sin\alpha_j$.
Then $\mathbb{E}[X]=\mathbb{E}[Y]=0$ and
$\mathrm{Var}(X)=\mathrm{Var}(Y)=N/2$.
For large $N$, $(X,Y)$ is approximately a two-dimensional Gaussian with isotropic variance $N/2$,
so $R=|S|=\sqrt{X^2+Y^2}$ is approximately Rayleigh distributed with scale
$\sigma=\sqrt{N/2}$, giving
\begin{equation}
\mathbb{E}[|S|]\;\approx\;\sigma\sqrt{\frac{\pi}{2}} \;=\;\frac{\sqrt{\pi N}}{2}.
\label{eq:rayleigh_mean}
\end{equation}
Therefore the \emph{intensive} mean amplitude scales as
\begin{equation}
\mathbb{E}\!\left[\frac{|S|}{N}\right] \sim \mathcal{O}\!\left(\frac{1}{\sqrt{N}}\right).
\label{eq:one_over_sqrtN}
\end{equation}

Equations~\eqref{eq:rayleigh_mean}--\eqref{eq:one_over_sqrtN} provide a clean baseline:
even without any axiom, typical destructive interference produces a residual that shrinks with system size.
The \OA{} does not dispute this behavior; it posits that, beyond this generic cancellation,
\emph{exact} cancellation is forbidden by a global rule.

\subsection{Artifact: Fig A}

Figure~\ref{fig:phasor} shows the ensemble-averaged residual as a function of the twist $\theta$,
along with the floored variant. The qualitative result is that once the floor is applied,
the residual cannot drift below $\epsfloor$ even in regimes where unconstrained cancellation
would produce smaller values.

\begin{figure}[t]
  \centering
  \includegraphics[width=\linewidth]{../outputs/figures/figA_phasor_toy.pdf}
  \caption{Phase I Fig A: finite phasor ensemble residual vs.\ twist $\theta$ (raw vs.\ floored).
  The floored curve enforces $\max(|S|,\epsfloor)$, producing a persistent nonzero baseline.}
  \label{fig:phasor}
\end{figure}