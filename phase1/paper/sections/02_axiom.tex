\section{Axiom}
\label{sec:axiom}

\subsection{Minimal statement}

Let a toy system admit a global complex amplitude $A \in \mathbb{C}$ constructed as a sum or mean
over microscopic degrees of freedom (e.g.\ a collective phasor, order parameter, or amplitude-like
functional of the state). The \OA{} posits:

\begin{equation}
  \boxed{|A| > \epsfloor \quad \text{for some fixed } \epsfloor>0.}
  \label{eq:axiom}
\end{equation}

Equation~\eqref{eq:axiom} is an \emph{axiom} in Phase I: it is not derived from a local Lagrangian,
equation of motion, or variational principle. It acts as a global constraint that forbids exact
destructive cancellation of $A$.

\subsection{Interpretation}

\paragraph{Global / nonlocal character.}
The constraint is imposed on a global functional of the state, not on local field values or
pointwise operators. This is intentional: Phase I treats \eqref{eq:axiom} as a conceptual candidate
mechanism that could arise from topology, boundary conditions, quantum-gravitational selection, or
other intrinsically nonlocal structures. No such microscopic origin is assumed or derived here.

\paragraph{What is fixed vs.\ free.}
The axiom introduces a positive parameter $\epsfloor$.
In Phase I, $\epsfloor$ is treated as a free constant specifying the non-cancellation floor.
Its numerical value is not predicted at this stage. A central goal of Phase II will be to
relate or derive $\epsfloor$ from deeper structure and to connect it to a physical energy-density
scale, if possible.

\paragraph{Relation to cancellation intuition.}
If microscopic degrees of freedom contribute with varying phases, ordinary destructive
interference can make $|A|$ parametrically small. The axiom forbids the limit $|A|\to 0$.
In this restricted sense, the axiom addresses only the structural question
``why not exactly zero?'' It does not, by itself, determine the magnitude of the residual.

\subsection{Relation to Weinberg-type no-go arguments}
\label{sec:weinberg}

A well-known obstruction to dynamical ``adjustment mechanisms'' for the cosmological constant is
that, within broadly standard local effective field theory assumptions, fields introduced to
cancel vacuum energy generically either fail, destabilize the theory, or reintroduce fine-tuning
once radiative corrections are included. Weinberg’s classic review summarizes this logic and why
naive self-adjustment is difficult in local QFT coupled to gravity~\cite{Weinberg1989}.

Phase I does not claim to evade these obstructions within ordinary local EFT. Instead, it
\emph{changes the starting premise}: the \OA{} is imposed as a \emph{global constraint} on an
amplitude-like functional of the state, not as a local field equation derived from a standard
Lagrangian. In this sense, Phase I is explicit about what is being relinquished relative to
Weinberg’s assumptions: strict locality and standard Wilsonian radiative arguments are not assumed
at the level of the axiom itself.

This is not offered as a resolution, only as a transparent statement of logical scope.
If a correct theory contains global or nonlocal selection rules (for example arising from topology,
boundary conditions, or quantum-gravitational constraints), then Weinberg’s specific premises do
not apply directly. Phase II must either (i) supply a consistent derivation of such a global rule
in a UV-complete framework or (ii) demonstrate that an effective description can remain predictive
despite the presence of nonlocal constraints.

\subsection{Phase I claims governed by the axiom}

Phase I is restricted to the following auditable claims:

\begin{itemize}
  \item \textbf{C1:} Equation~\eqref{eq:axiom} is postulated as an axiom, not derived.
  \item \textbf{C2:} In finite phasor ensembles, unconstrained destructive interference generically yields small but nonzero residuals.
  \item \textbf{C3:} Enforcing \eqref{eq:axiom} in a lattice toy model is dynamically stable (existence proof).
  \item \textbf{C4:} In unconstrained systems, the mean residual decreases with system size, while the constrained mean saturates at $\epsfloor$.
  \item \textbf{C5:} Constraint enforcement does not induce order-one pathologies in the toy energy diagnostic within the explored parameter range.
\end{itemize}

All claims are linked to explicit, reproducible numerical artifacts presented in
Section~\ref{sec:results}.