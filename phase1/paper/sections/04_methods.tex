\section{Methods: Lattice Existence Proof and Scaling Test}
\label{sec:methods}

Phase I uses a lattice toy model as an \emph{existence proof} that a global amplitude floor can be enforced
stably in an explicit dynamical system. The lattice model is not claimed to represent a realistic QFT vacuum;
it serves as a controlled numerical sandbox for testing the logical consistency of the axiom.

\subsection{Global amplitude observable}

Let $\phi(x)$ be a complex scalar degree of freedom on a cubic lattice with $N_{\mathrm{sites}}$ sites.
We define the global amplitude
\begin{equation}
  A_{\mathrm{sum}} = \sum_{x}\phi(x),
\end{equation}
and the intensive mean amplitude
\begin{equation}
  \Amean = \frac{A_{\mathrm{sum}}}{N_{\mathrm{sites}}}.
\end{equation}

Phase I applies the non-cancellation floor to the \emph{mean} amplitude:
\begin{equation}
  |\Amean| \ge \epsfloor.
\end{equation}
Operationally, the implementation enforces an upstream constraint on $|A_{\mathrm{sum}}|$ with
$\epsfloor^{(\mathrm{sum})} = \epsfloor \, N_{\mathrm{sites}}$, ensuring correct intensive scaling
as system size varies.

\subsection{A minimal illustrative mapping from amplitude floor to vacuum energy}
\label{sec:toy_vacuum_map}

Phase I distinguishes two logically separate questions:
(i) why not exactly zero (structural non-cancellation), and
(ii) what sets the observed magnitude of vacuum energy.
Phase I addresses (i) only; the discussion below is included solely to make dimensional requirements explicit.

Let $\mu$ be a characteristic energy scale governing how a residual amplitude might contribute to an
effective energy density. A generic dimensional estimate is
\begin{equation}
\rho_{\mathrm{vac}} \sim \mu^4 \, f(|\Amean|),
\end{equation}
where $f$ is dimensionless. The simplest analytic choice is $f(|\Amean|)=|\Amean|^2$, or any monotone
function with a nonzero floor. Under this illustrative mapping, the \OA{} implies a strictly positive bound
\begin{equation}
\rho_{\mathrm{vac}} \gtrsim \mu^4 \, \epsfloor^2.
\label{eq:rho_floor}
\end{equation}

The observed dark energy density is commonly expressed as
$\rho_{\mathrm{DE}} \simeq (2.3~\mathrm{meV})^4$~\cite{Hardy2020}.
Equation~\eqref{eq:rho_floor} highlights the scale discipline required:
if $\mu$ is very large, $\epsfloor$ must be correspondingly small; if $\mu$ is itself meV-like,
$\epsfloor$ could be order unity. Phase I selects neither $\mu$ nor $\epsfloor$.
No claim is made that this toy mapping captures the true microphysical origin of vacuum energy
or its gravitational backreaction.

\subsection{Determinism and execution traceability}

All runs are deterministic given a fixed seed, configuration, and code revision.
Each run emits:
\begin{itemize}
  \item raw time series (compressed),
  \item a machine-readable summary (YAML),
  \item provenance metadata (git hash, parameters, seed, environment snapshot).
\end{itemize}

\subsection{Claims-to-artifacts audit map}

Table~\ref{tab:claims} maps each Phase I claim to a concrete output artifact.

\begin{table}[t]
\centering
\begin{tabular}{@{}lll@{}}
\toprule
Claim & Description & Artifact(s) \\ \midrule
C1 & Axiom postulated ($|A|>\epsfloor$) & Sec.~\ref{sec:axiom} \\
C2 & Phasor residual and floor behavior & Fig.~\ref{fig:phasor} (figA\_phasor\_toy.pdf) \\
C3 & Lattice enforcement existence proof & Fig.~\ref{fig:latticeAmp} + run meta.json \\
C4 & Scaling: unconstrained $\to 0$, constrained $\to \epsfloor$ & Fig.~\ref{fig:scaling} + scaling\_summary.yaml \\
C5 & No order-one pathology in toy energy diagnostic & Stored $E(t)$ in NPZ outputs \\ \bottomrule
\end{tabular}
\caption{Phase I claims-to-artifacts map. Each artifact is generated via the Snakemake pipeline.}
\label{tab:claims}
\end{table}