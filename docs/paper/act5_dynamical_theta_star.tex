\subsection{R5: Dynamical tolerance of the $\theta_\star$ bridge}
\label{subsec:theta_star_dynamical_tolerance}

With the effective vacuum map in hand (R2--R3), we can now ask how
sensitive the late--time acceleration is to small deformations of the
microcavity phase $\theta_\star$. Concretely, we define
\begin{equation}
  \Omega_\Lambda(\theta_\star)
  \equiv
  k_{\rm scale}\,\Delta E(\theta_\star)
  \,,
\end{equation}
where $k_{\rm scale}$ is fixed by the fiducial calibration
$\Omega_\Lambda(\theta_{\rm fid}) = 0.7$ with
$\theta_{\rm fid} = 3.63~{\rm rad}$ and
$\Delta E(\theta_\star)$ is taken from the 1D microcavity scan
of Sec.~\ref{subsec:microcavity_scan}.

We then scan $\theta_\star$ over the same ``cosmologically allowed band''
\begin{equation}
  \theta_{\rm band} \in [2.18,\,5.54]~{\rm rad}
\end{equation}
and evaluate $\Omega_\Lambda(\theta_\star)$ at 41 equally--spaced
samples. The resulting profile is shown in
Fig.~\ref{fig:effective_vacuum_band_scan}. A broad plateau is visible:
for
\begin{equation}
  \theta_\star \in [2.516,\,3.692]~{\rm rad}
\end{equation}
we find
\begin{equation}
  \bigl|\Omega_\Lambda(\theta_\star) - 0.7\bigr| \le 0.05
  \quad\Rightarrow\quad
  \Omega_\Lambda(\theta_\star) \in [0.65,\,0.75] \,,
\end{equation}
i.e.\ nine grid points in our scan lie within a conservative
$\pm 0.05$ tolerance band around the observational target
$\Omega_\Lambda \simeq 0.7$.

Within this window the local slope remains modest. Near the left and
right edges we find
\begin{equation}
  \frac{{\rm d}\Omega_\Lambda}{{\rm d}\theta_\star}
  \sim \mathcal{O}(0.2\text{--}0.4)\;{\rm rad}^{-1},
\end{equation}
while around the mid--window point
$\theta_{\rm mid} \simeq 3.10~{\rm rad}$ the profile is extremely
flat,
\begin{equation}
  \left.\frac{{\rm d}\Omega_\Lambda}{{\rm d}\theta_\star}\right|_{\rm mid}
  \approx 2.3\times 10^{-2}\;{\rm rad}^{-1}.
\end{equation}
A naive linear estimate based on this plateau slope would allow
$\mathcal{O}(1)$--rad excursions in $\theta_\star$ before leaving the
$\pm 0.05$ band. In practice, the true tolerance is limited by the
edges of the window itself,
\begin{equation}
  \Delta\theta_\star^{\rm (band)}
  \simeq \frac{3.692 - 2.516}{2}
  \approx 0.59~{\rm rad},
\end{equation}
which still corresponds to order--unity fractional changes in
$\theta_\star$.

We summarise R5 as follows:
\begin{quote}
  \textbf{R5 (Dynamical robustness).} Once the microcavity parameters
  are fixed by the fiducial calibration (R2--R3), there exists a
  broad plateau in $\theta_\star$ of width $\sim 1.2~{\rm rad}$ on
  which $\Omega_\Lambda(\theta_\star)$ remains within
  $\pm 0.05$ of $0.7$. Small to moderate drifts in the microscopic
  phase $\theta_\star$ therefore do not catastrophically destabilise
  the late--time acceleration; the $\theta_\star$--bridge from
  microphysics to FRW is dynamically tolerant.
\end{quote}
In other words, once the non--cancelling microcavity is tuned onto the
observed $\Omega_\Lambda$ slice, the ensuing cosmic acceleration is
robust under $\mathcal{O}(0.1\text{--}0.6)$--rad changes in the
underlying phase.


\subsection{R5: Effective vacuum patch ensemble}

The band scan in Fig.~\ref{fig:effective_vacuum_band_scan} shows that there is
an extended interval in $\theta_\star$ within the flavour--side prior band
$\theta_\star \in [2.18, 5.54]~\mathrm{rad}$ where the microcavity--backed
effective vacuum delivers $\Omega_\Lambda(\theta_\star)$ close to the observed
value $\Omega_\Lambda \simeq 0.7$. To make this more concrete, we now treat this
interval as a pool of candidate ``vacuum patches'' with slightly different
$\theta_\star$, and ask what fraction of patches would look like our Universe,
in the narrow sense of matching the late--time dark energy fraction.

Operationally, we draw $N_\mathrm{patch}$ values of $\theta_\star$ uniformly
within the band $[2.18, 5.54]~\mathrm{rad}$, evaluate the corresponding
$\Omega_\Lambda(\theta_\star)$ using the same effective vacuum scaling as in
Sec.~\ref{subsec:frw-from-effective-vacuum}, and classify a patch as
``good'' whenever
\begin{equation}
  \bigl|\Omega_\Lambda(\theta_\star) - 0.7\bigr| \le 0.05.
\end{equation}
Figure~\ref{fig:effective_vacuum_patch_ensemble} shows a representative
realisation with $N_\mathrm{patch} = 1000$. The horizontal dashed line marks
the target value $\Omega_\Lambda = 0.7$, and the dotted lines indicate the
$\pm 0.05$ tolerance band. Patches falling inside the band are highlighted.

In this toy ensemble, roughly $f_\mathrm{good} \simeq 0.21$ of all patches
land in the observational window. Equivalently, if $\theta_\star$ were
distributed roughly uniformly over the prior band defined by the flavour
constraints, then a Universe with $\Omega_\Lambda$ in the observed range would
not be an exponentially rare accident: it would be a ${\cal O}(10^{-1})$
probability outcome inside this coarse prior. At the present level, this is
only an internal consistency check for the $\theta_\star$--driven picture; in
later work, the toy ensemble can be replaced by a more realistic measure over
$\theta_\star$ determined by dynamical evolution in the flavour sector.

\begin{figure}[t]
  \centering
  \includegraphics[width=0.7\textwidth]{figures/effective_vacuum_patch_ensemble}
  \caption{%
    Effective vacuum patch ensemble (R5). Each point corresponds to a toy
    vacuum patch with some $\theta_\star$ drawn uniformly from the flavour--side
    prior band. The vertical coordinate is the corresponding
    $\Omega_\Lambda(\theta_\star)$ obtained from the microcavity--backed
    effective vacuum scaling. The dashed line marks
    $\Omega_\Lambda = 0.7$; dotted lines indicate the $\pm 0.05$ tolerance
    adopted here. Highlighted points fall inside the observational window.
    In the illustrated realisation with $N_\mathrm{patch} = 1000$, roughly
    $21\%$ of patches lie in this window.
  }
  \label{fig:effective_vacuum_patch_ensemble}
\end{figure}


\subsubsection*{R5: Effective vacuum patch ensemble}

% (your explanatory paragraph here)

\begin{figure}[t]
  \centering
  \includegraphics[width=0.7\textwidth]{figures/effective_vacuum_patch_ensemble}
  \caption{%
    Histogram of $\Omega_\Lambda$ values obtained by sampling $10^3$
    $\theta_\star$ patches uniformly from the effective band
    $[2.18, 5.54]\,\mathrm{rad}$. The shaded region shows the observational
    window $\Omega_\Lambda = 0.70 \pm 0.05$. About $21.5\%$ of patches fall
    in this window, illustrating that the observed value is not an isolated
    fine–tuned point in this toy landscape.
  }
  \label{fig:effective-vacuum-patch-ensemble}
\end{figure}


\subsection{Act V summary: from microcavity to an effective cosmological constant}
\label{sec:act5-summary}

In this act we closed the loop between a microscopic, discretised model
of vacuum energy and an effective cosmological constant at FRW scales.
Starting from the one-dimensional microcavity scan
$\Delta E(\theta_\star)$ over a full $0$--$2\pi$ band, we identified a
fiducial angle $\theta_{\star,\text{fid}} \simeq 3.63~\mathrm{rad}$ at
which the vacuum energy difference $\Delta E_{\rm fid}$ is negative and
of order $10^{-3}$ in the toy units of the lattice.  By demanding that
this point reproduces an observationally motivated value of the dark
energy density, $\Omega_\Lambda(\theta_{\star,\text{fid}}) = 0.7$, we
fixed a single proportionality constant $k_{\rm scale}$ that maps the
dimensionless microcavity energy shifts into an effective
$\Omega_\Lambda(\theta_\star)$.

With this prescription in place, we constructed an effective vacuum
model in which each choice of $\theta_\star$ inside a ``good'' band
$[2.18, 5.54]~\mathrm{rad}$ is associated to a different cosmological
constant.  Solving the Friedmann equations for a flat, matter-plus-vacuum
universe shows that the fiducial point $(\Omega_m,\Omega_\Lambda) =
(0.3, 0.7)$ reproduces the expected accelerated expansion relative to a
purely matter-dominated history, and yields distance--redshift relations
consistent with a standard $\Lambda$CDM-like background at the level of
this toy model.

The band scan reveals that, within the admissible $\theta_\star$ range,
$\Omega_\Lambda(\theta_\star)$ varies between $0$ and $\simeq 0.78$,
with a finite interval of angles for which $\Omega_\Lambda$ lies close
to $0.7$.  By sampling $N_{\rm patch} = 1000$ random $\theta_\star$
values uniformly inside the band, we built an effective ``patch
ensemble'' and found that the resulting distribution of
$\Omega_\Lambda$ has mean $\langle \Omega_\Lambda \rangle \simeq 0.45$,
standard deviation $\sigma_{\Omega_\Lambda} \simeq 0.31$, and median
$\tilde{\Omega}_\Lambda \simeq 0.57$.  Approximately $21.5\%$ of the
patches fall into the observationally inspired window
$\Omega_\Lambda = 0.70 \pm 0.05$, and the central $68\%$ interval of
the ensemble extends up to $\Omega_\Lambda \simeq 0.76$.  In other
words, once the microcavity band is accepted as the relevant dynamical
domain, a patch with $\Omega_\Lambda \approx 0.7$ is statistically
typical rather than exponentially fine-tuned.

The construction here is deliberately minimalistic and remains a toy
model in several respects: the microcavity is one-dimensional, the
mapping from $\Delta E$ to $\Omega_\Lambda$ is linear and fixed by a
single fiducial point, and we have not yet coupled the effective vacuum
to realistic matter content beyond a dust-like component.  Nevertheless,
Act~V shows that the $\theta_\star$ degree of freedom can serve as a
dynamical label for vacuum energy in an ensemble of patches, and that
moderately typical values of $\theta_\star$ can reproduce a
$\Lambda$-like component of the right order of magnitude without
invoking a $10^{-120}$-level tuning by hand.  This is the bridge we
need between the underlying axiom and a macroscopic, FRW-level
description of accelerated expansion in the toy universe.


\paragraph{Patch ensemble statistics.}
To make the microcavity--backed effective vacuum more concrete, we draw an
ensemble of $N_{\rm patches} = 1000$ independent patches with
$\theta_\star$ sampled uniformly from the allowed band
$[2.18, 5.54]\,\mathrm{rad}$ and map each draw to an effective
vacuum fraction $\Omega_\Lambda(\theta_\star)$ using the calibrated
microcavity scaling.
In this ensemble we find
$\Omega_\Lambda \in [0.000, 0.775]$, with mean
$\langle \Omega_\Lambda \rangle \simeq 0.445$,
standard deviation $\sigma_{\Omega_\Lambda} \simeq 0.307$,
and median $\mathrm{med}(\Omega_\Lambda) \simeq 0.573$.
Focusing on the observationally motivated target window
$\Omega_\Lambda = 0.70 \pm 0.05$, we obtain
$215/1000 \approx 21.5\%$ of patches in the wide window,
$80/1000 \approx 8.0\%$ in the tighter
$\Omega_\Lambda = 0.70 \pm 0.02$ band,
and $39/1000 \approx 3.9\%$ in the most restrictive
$\Omega_\Lambda = 0.70 \pm 0.01$ window.
Thus, within this toy ensemble the ``$\Lambda$-like'' patches are not
exponentially rare in $\theta_\star$-space; rather, they occupy a
non-negligible fraction of the available band, reinforcing the idea
that a microcavity-induced effective vacuum can naturally populate
$\Omega_\Lambda \sim 0.7$ without fine-tuning.


\subsection{R6: Random-walk residence of $\theta_\star$ in the $\Omega_\Lambda$ band}
\label{sec:theta_star_random_walk_residence}

In this rung we treat $\theta_\star$ as a toy dynamical variable that executes
a one-dimensional random walk within the microcavity-supported band
$[\theta_{\min}, \theta_{\max}] = [2.18, 5.54]~\mathrm{rad}$.
The microscopic input is exactly the same as in the previous rungs:
we use the microcavity scan
$\Delta E(\theta_\star)$ from the file
\texttt{data/processed/theta\_star\_microcavity\_scan\_full\_2pi.npz}
and the calibrated scaling
$k_{\mathrm{scale}}$ from
\texttt{data/processed/theta\_star\_microcavity\_core\_summary.json},
so that
\begin{equation}
  \Omega_\Lambda(\theta_\star)
  = \mathrm{clip}\!\left(
      k_{\mathrm{scale}}\,\Delta E(\theta_\star),
      0,\;0.999
    \right),
\end{equation}
with $k_{\mathrm{scale}} \simeq -131.4$ chosen such that
$\Omega_\Lambda(\theta_{\star,\mathrm{fid}}) \simeq 0.7$
at $\theta_{\star,\mathrm{fid}} = 3.63~\mathrm{rad}$.

We then evolve $N_{\mathrm{traj}} = 200$ independent trajectories of
$\theta_\star(t)$ as a simple Gaussian random walk with reflecting
boundaries at $\theta_{\min}$ and $\theta_{\max}$ and a typical step size
$\Delta\theta \simeq 0.02~\mathrm{rad}$ per timestep.
For each trajectory we record the induced $\Omega_\Lambda(t)$ and measure
the fraction of trajectory-time spent in three observational windows,
\begin{equation}
  \bigl|\Omega_\Lambda - 0.7\bigr| \leq \{0.05,\,0.02,\,0.01\}.
\end{equation}
The resulting residence fractions are approximately
$20.5\%$, $7.4\%$, and $3.7\%$ respectively, i.e.\ the random-walk motion
in $\theta_\star$ spends a non-negligible fraction of its time in a band
that is observationally indistinguishable from the fiducial
$\Lambda$CDM value.

Figure~\ref{fig:theta_star_random_walk_residence} shows the resulting
distribution of $\Omega_\Lambda$ values sampled along all trajectories,
together with the broad observational window
$\bigl|\Omega_\Lambda - 0.7\bigr| \leq 0.05$.
This figure is not an attempt at a physical equation of motion for
$\theta_\star$, but rather a toy demonstration that once the microcavity
scaling is fixed, a simple stochastic dynamics in $\theta_\star$ naturally
generates residence-time statistics in the observed $\Omega_\Lambda$ band
at the $\sim 10\%$ level.

\begin{figure}[t]
  \centering
  \includegraphics[width=0.7\textwidth]{figures/theta_star_random_walk_residence}
  \caption{
    Residence-time distribution of $\Omega_\Lambda$ for the random-walk
    ensemble of $\theta_\star$ trajectories (R6).
    We histogram all sampled $\Omega_\Lambda(t)$ values from
    $N_{\mathrm{traj}} = 200$ random walks in the band
    $\theta_\star \in [2.18, 5.54]~\mathrm{rad}$, using the microcavity-backed
    mapping $\Omega_\Lambda(\theta_\star)$ fixed in previous rungs.
    The vertical dashed line marks the fiducial value
    $\Omega_\Lambda = 0.7$ and the shaded band indicates the window
    $\bigl|\Omega_\Lambda - 0.7\bigr| \leq 0.05$.
    In this toy model, roughly $20\%$ of the total trajectory-time lies
    within this observational window.
  }
  \label{fig:theta_star_random_walk_residence}
\end{figure}




