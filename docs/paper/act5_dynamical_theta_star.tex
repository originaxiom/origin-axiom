\section{Act V: Dynamical $\theta_\star$ bridge and effective vacuum}
\label{sec:act5-dynamical-theta-star}

In this act we close the loop between the flavour--sector determination of the
master phase $\theta_\star$, a simple $\theta_\star$--sensitive microcavity toy,
and an effective cosmological constant at FRW scales.  The goal is not to
derive $\Lambda$ from first principles in a realistic field theory, but to show
that a single phase parameter $\theta_\star$ can consistently label families of
vacua whose effective dark--energy fraction $\Omega_\Lambda(\theta_\star)$
naturally populates the phenomenologically relevant regime
$\Omega_\Lambda \sim 0.7$ without hand--building this value into the prior.

Throughout this act we work with the Act~II flavour--based summary of
$\theta_\star$ exported into this repo as
\texttt{config/theta\_star\_config.json}.  In particular,
\begin{equation}
  \theta_{\star,\mathrm{fid}} \simeq 3.63~\mathrm{rad}, \qquad
  \theta_{\star,\mathrm{band}} \simeq [2.18, 5.54]~\mathrm{rad},
\end{equation}
where $\theta_{\star,\mathrm{fid}}$ is a convenient fiducial value and
$\theta_{\star,\mathrm{band}}$ is the working prior band inherited from the
joint CKM+PMNS fits of Act~II.

We first recall the 1D microcavity construction and the mapping from the
dimensionless vacuum shift $\Delta E(\theta_\star)$ to an effective
$\Omega_\Lambda(\theta_\star)$ (R1--R3).  We then quantify the dynamical
tolerance of this mapping (R4), the statistical weight of
$\Lambda$CDM--like patches in a simple ensemble (R5), a toy random--walk
residence test (R6), and finally the induced prior on $\Omega_\Lambda$ from the
original $\theta_\star$ prior (R7).


\subsection{R1: 1D $\theta_\star$ microcavity toy}
\label{subsec:microcavity_scan}

We start from a minimal 1D scalar lattice with open boundaries, mass parameter
$m_0$ and nearest--neighbour coupling $c$.  A central ``microcavity'' region
occupying a fixed fraction of the sites is assigned a $\theta_\star$--dependent
mass profile
\begin{equation}
  m^2(x;\theta_\star) \;=\;
  \begin{cases}
    m_0^2, & x \notin \text{cavity}, \\[3pt]
    m_0^2 \bigl[1 + \alpha_{\rm mass}\cos(\theta_\star - \theta_0)\bigr]^2,
      & x \in \text{cavity},
  \end{cases}
\end{equation}
with $\alpha_{\rm mass}$ a dimensionless modulation depth and $\theta_0$ a
reference phase.  The single--field Hamiltonian is
\begin{equation}
  H(\theta_\star) = \mathrm{diag}\bigl(m^2(x;\theta_\star) + 2c\bigr)
  - c\,(\text{nearest--neighbour}),
\end{equation}
and we define the vacuum shift as
\begin{equation}
  \Delta E(\theta_\star)
  \;\equiv\;
  E_{0,\mathrm{cavity}}(\theta_\star) - E_{0,\mathrm{uniform}},
\end{equation}
where $E_{0,\mathrm{uniform}}$ is the ground--state energy of the same chain
with a spatially uniform mass $m_0$.

The script \texttt{src/scan\_1d\_theta\_star\_microcavity\_full\_band.py}
samples $\theta_\star$ over one full $0$--$2\pi$ cycle and stores
\begin{equation}
  \bigl\{\theta_\star,\,\Delta E(\theta_\star)\bigr\}
\end{equation}
in \texttt{data/processed/theta\_star\_microcavity\_scan\_full\_2pi.npz}.
For the parameter choices used here, the global minimum of the vacuum shift
lies at
\begin{equation}
  \theta_{\star,\mathrm{min}} \simeq \pi, \qquad
  \Delta E_{\mathrm{min}} \simeq -5.9\times 10^{-3},
\end{equation}
and the Act~II band $[2.18, 5.54]~\mathrm{rad}$ fully contains this minimum.
At the fiducial flavour value
$\theta_{\star,\mathrm{fid}} = 3.63~\mathrm{rad}$ the scan yields
\begin{equation}
  \Delta E_{\mathrm{fid}}
  \equiv \Delta E(\theta_{\star,\mathrm{fid}})
  \simeq -5.33\times 10^{-3}.
\end{equation}

For later convenience these summary numbers (fiducial angle, band, and
$\Delta E$ at the fiducial point) are stored in
\texttt{data/processed/theta\_star\_microcavity\_core\_summary.json}.


\subsection{R2: Effective vacuum scaling from $\Delta E(\theta_\star)$}
\label{subsec:effective-vacuum-scaling}

The microcavity scan provides a dimensionless vacuum shift
$\Delta E(\theta_\star)$ as a function of the phase.  To connect this to
cosmology we introduce a simple effective mapping
\begin{equation}
  \Omega_\Lambda(\theta_\star)
  \;=\;
  \mathrm{clip}\!\left(k_{\mathrm{scale}}\,\Delta E(\theta_\star),
    0,\;0.999\right),
\end{equation}
where $k_{\mathrm{scale}}$ is a single proportionality constant and the
clipping keeps $\Omega_\Lambda$ in the physically sensible interval
$[0,1)$.  The constant $k_{\mathrm{scale}}$ is fixed by the requirement that
the fiducial flavour phase reproduce an observationally motivated dark--energy
fraction,
\begin{equation}
  \Omega_\Lambda(\theta_{\star,\mathrm{fid}}) = 0.7,
\end{equation}
which gives
\begin{equation}
  k_{\mathrm{scale}} \;=\;
  \frac{\Omega_\Lambda(\theta_{\star,\mathrm{fid}})}{\Delta E_{\mathrm{fid}}}
  \;\simeq\; -1.31\times 10^{2}.
\end{equation}
With this choice, the effective matter fraction follows from flatness,
\begin{equation}
  \Omega_m(\theta_\star) = 1 - \Omega_\Lambda(\theta_\star),
\end{equation}
and at the fiducial phase we obtain
\begin{equation}
  \Omega_\Lambda(\theta_{\star,\mathrm{fid}}) \simeq 0.7, \qquad
  \Omega_m(\theta_{\star,\mathrm{fid}}) \simeq 0.3.
\end{equation}

In practice we implement this mapping in a small helper
\texttt{EffectiveVacuumModel} (\texttt{src/vacuum\_effective.py}), which
reads the Act~II $\theta_\star$ configuration and the microcavity scan and
provides an interpolating function $\Omega_\Lambda(\theta_\star)$ together
with the calibrated $k_{\mathrm{scale}}$ and fiducial point.


\subsection{R3: FRW toy universe from effective vacuum}
\label{subsec:frw-from-effective-vacuum}

Given a flat FRW model with matter and vacuum,
\begin{equation}
  H^2(a) = H_0^2 \bigl[ \Omega_m a^{-3} + \Omega_\Lambda \bigr],
\end{equation}
we can treat $\Omega_m$ and $\Omega_\Lambda$ as effective parameters supplied
by the microcavity--backed mapping $\Omega_\Lambda(\theta_\star)$ at the
fiducial phase.  The script
\texttt{src/run\_frw\_from\_effective\_vacuum.py} does exactly this: it builds
an \texttt{EffectiveVacuumModel}, evaluates
$\Omega_{\Lambda,\mathrm{fid}} = \Omega_\Lambda(\theta_{\star,\mathrm{fid}})$
and $\Omega_{m,\mathrm{fid}} = 1-\Omega_{\Lambda,\mathrm{fid}}$, and compares
two FRW histories:

\begin{itemize}
\item a matter--only universe with $(\Omega_m,\Omega_\Lambda) = (1,0)$;
\item an effective--vacuum universe with
  $(\Omega_m,\Omega_\Lambda) \simeq (0.3,0.7)$ obtained from the microcavity
  scaling at $\theta_{\star,\mathrm{fid}}$.
\end{itemize}

Working in units where $H_0 = 1$ and starting at $a_{\mathrm{init}} = 10^{-3}$,
the matter--only case follows the familiar decelerating power law
$a(t)\propto t^{2/3}$ and reaches $a(t_{\mathrm{final}})\simeq 3.8$ over a
few Hubble times.  The effective--vacuum case undergoes accelerated expansion
and reaches $a(t_{\mathrm{final}})\simeq 3\times 10^{1}$ over the same time
interval.  The corresponding $a(t)$ and Hubble--diagram style
$d_L(z)$ curves are stored in
\texttt{data/processed/frw\_from\_effective\_vacuum.npz} and visualised in
\texttt{figures/frw\_from\_effective\_vacuum\_a\_of\_t.pdf} and
\texttt{figures/frw\_effective\_hubble\_diagram.png}.

At this stage $\Omega_\Lambda$ is still effectively a one--point calibration:
we use the fiducial $\theta_{\star,\mathrm{fid}}$ to pin down
$k_{\mathrm{scale}}$ and then verify that the resulting FRW history behaves
like a standard $(\Omega_m,\Omega_\Lambda)\simeq(0.3,0.7)$ cosmology in terms
of acceleration and Hubble diagram shape.


\subsection{R4: Dynamical tolerance of the $\theta_\star$ bridge}
\label{subsec:theta_star_dynamical_tolerance}

With the effective vacuum map in hand, we can now ask how sensitive the
late--time acceleration is to small deformations of the microcavity phase
$\theta_\star$.  Concretely, we define
\begin{equation}
  \Omega_\Lambda(\theta_\star)
  \equiv
  k_{\rm scale}\,\Delta E(\theta_\star)
  \,,
\end{equation}
where $k_{\rm scale}$ is fixed by the fiducial calibration
$\Omega_\Lambda(\theta_{\rm fid}) = 0.7$ with
$\theta_{\rm fid} = 3.63~{\rm rad}$ and
$\Delta E(\theta_\star)$ is taken from the 1D microcavity scan
of Sec.~\ref{subsec:microcavity_scan}.

We then scan $\theta_\star$ over the same ``cosmologically allowed band''
\begin{equation}
  \theta_{\rm band} \in [2.18,\,5.54]~{\rm rad}
\end{equation}
and evaluate $\Omega_\Lambda(\theta_\star)$ at 41 equally--spaced
samples. The resulting profile is shown in
Fig.~\ref{fig:effective_vacuum_band_scan}. A broad plateau is visible:
for
\begin{equation}
  \theta_\star \in [2.516,\,3.692]~{\rm rad}
\end{equation}
we find
\begin{equation}
  \bigl|\Omega_\Lambda(\theta_\star) - 0.7\bigr| \le 0.05
  \quad\Rightarrow\quad
  \Omega_\Lambda(\theta_\star) \in [0.65,\,0.75] \,,
\end{equation}
i.e.\ nine grid points in our scan lie within a conservative
$\pm 0.05$ tolerance band around the observational target
$\Omega_\Lambda \simeq 0.7$.

Within this window the local slope remains modest. Near the left and
right edges we find
\begin{equation}
  \frac{{\rm d}\Omega_\Lambda}{{\rm d}\theta_\star}
  \sim \mathcal{O}(0.2\text{--}0.4)\;{\rm rad}^{-1},
\end{equation}
while around the mid--window point
$\theta_{\rm mid} \simeq 3.10~{\rm rad}$ the profile is extremely
flat,
\begin{equation}
  \left.\frac{{\rm d}\Omega_\Lambda}{{\rm d}\theta_\star}\right|_{\rm mid}
  \approx 2.3\times 10^{-2}\;{\rm rad}^{-1}.
\end{equation}
A naive linear estimate based on this plateau slope would allow
$\mathcal{O}(1)$--rad excursions in $\theta_\star$ before leaving the
$\pm 0.05$ band. In practice, the true tolerance is limited by the
edges of the window itself,
\begin{equation}
  \Delta\theta_\star^{\rm (band)}
  \simeq \frac{3.692 - 2.516}{2}
  \approx 0.59~{\rm rad},
\end{equation}
which still corresponds to order--unity fractional changes in
$\theta_\star$.

We summarise R4 as follows:
\begin{quote}
  \textbf{R4 (Dynamical robustness).} Once the microcavity parameters
  are fixed by the fiducial calibration (R1--R3), there exists a
  broad plateau in $\theta_\star$ of width $\sim 1.2~{\rm rad}$ on
  which $\Omega_\Lambda(\theta_\star)$ remains within
  $\pm 0.05$ of $0.7$. Small to moderate drifts in the microscopic
  phase $\theta_\star$ therefore do not catastrophically destabilise
  the late--time acceleration; the $\theta_\star$--bridge from
  microphysics to FRW is dynamically tolerant.
\end{quote}
In other words, once the non--cancelling microcavity is tuned onto the
observed $\Omega_\Lambda$ slice, the ensuing cosmic acceleration is
robust under $\mathcal{O}(0.1\text{--}0.6)$--rad changes in the
underlying phase.


\subsection{R5: Effective vacuum patch ensemble}
\label{subsec:effective-vacuum-patch-ensemble}

The band scan in Fig.~\ref{fig:effective_vacuum_band_scan} shows that there is
an extended interval in $\theta_\star$ within the flavour--side prior band
$\theta_\star \in [2.18, 5.54]~\mathrm{rad}$ where the microcavity--backed
effective vacuum delivers $\Omega_\Lambda(\theta_\star)$ close to the observed
value $\Omega_\Lambda \simeq 0.7$. To make this more concrete, we now treat this
interval as a pool of candidate ``vacuum patches'' with slightly different
$\theta_\star$, and ask what fraction of patches would look like our Universe,
in the narrow sense of matching the late--time dark energy fraction.

Operationally, we draw $N_\mathrm{patch} = 1000$ values of $\theta_\star$
uniformly within the band $[2.18, 5.54]~\mathrm{rad}$, evaluate the
corresponding $\Omega_\Lambda(\theta_\star)$ using the same effective vacuum
scaling as in Sec.~\ref{subsec:frw-from-effective-vacuum}, and classify a patch
as ``good'' whenever
\begin{equation}
  \bigl|\Omega_\Lambda(\theta_\star) - 0.7\bigr| \le 0.05.
\end{equation}
Figure~\ref{fig:effective_vacuum_patch_ensemble} shows a representative
realisation with $N_\mathrm{patch} = 1000$. The horizontal dashed line marks
the target value $\Omega_\Lambda = 0.7$, and the dotted lines indicate the
$\pm 0.05$ tolerance band. Patches falling inside the band are highlighted.

In this toy ensemble, roughly $f_\mathrm{good} \simeq 0.215$ of all patches
land in the observational window. Equivalently, if $\theta_\star$ were
distributed roughly uniformly over the prior band defined by the flavour
constraints, then a Universe with $\Omega_\Lambda$ in the observed range would
not be an exponentially rare accident: it would be a ${\cal O}(10^{-1})$
probability outcome inside this coarse prior.

\paragraph{Patch ensemble statistics.}
In the same ensemble we find
$\Omega_\Lambda \in [0.000, 0.775]$, with mean
$\langle \Omega_\Lambda \rangle \simeq 0.445$,
standard deviation $\sigma_{\Omega_\Lambda} \simeq 0.307$,
and median $\mathrm{med}(\Omega_\Lambda) \simeq 0.573$.
Focusing on the observationally motivated target window
$\Omega_\Lambda = 0.70 \pm 0.05$, we obtain
$215/1000 \approx 21.5\%$ of patches in this window,
$80/1000 \approx 8.0\%$ in the tighter
$\Omega_\Lambda = 0.70 \pm 0.02$ band,
and $39/1000 \approx 3.9\%$ in the most restrictive
$\Omega_\Lambda = 0.70 \pm 0.01$ window.
Thus, within this toy ensemble the ``$\Lambda$-like'' patches are not
exponentially rare in $\theta_\star$-space; rather, they occupy a
non-negligible fraction of the available band.

\begin{figure}[t]
  \centering
  \includegraphics[width=0.7\textwidth]{figures/effective_vacuum_patch_ensemble}
  \caption{%
    Effective vacuum patch ensemble (R5). Each point corresponds to a toy
    vacuum patch with some $\theta_\star$ drawn uniformly from the flavour--side
    prior band. The vertical coordinate is the corresponding
    $\Omega_\Lambda(\theta_\star)$ obtained from the microcavity--backed
    effective vacuum scaling. The dashed line marks
    $\Omega_\Lambda = 0.7$; dotted lines indicate the $\pm 0.05$ tolerance
    adopted here. Highlighted points fall inside the observational window.
    In the illustrated realisation with $N_\mathrm{patch} = 1000$, roughly
    $21.5\%$ of patches lie in this window.
  }
  \label{fig:effective_vacuum_patch_ensemble}
\end{figure}


\subsection{R6: Random-walk residence of $\theta_\star$ in the $\Omega_\Lambda$ band}
\label{sec:theta_star_random_walk_residence}

In this rung we treat $\theta_\star$ as a toy dynamical variable that executes
a one-dimensional random walk within the microcavity-supported band
$[\theta_{\min}, \theta_{\max}] = [2.18, 5.54]~\mathrm{rad}$.
The microscopic input is exactly the same as in the previous rungs:
we use the microcavity scan
$\Delta E(\theta_\star)$ from the file
\texttt{data/processed/theta\_star\_microcavity\_scan\_full\_2pi.npz}
and the calibrated scaling
$k_{\mathrm{scale}}$ from
\texttt{data/processed/theta\_star\_microcavity\_core\_summary.json},
so that
\begin{equation}
  \Omega_\Lambda(\theta_\star)
  = \mathrm{clip}\!\left(
      k_{\mathrm{scale}}\,\Delta E(\theta_\star),
      0,\;0.999
    \right),
\end{equation}
with $k_{\mathrm{scale}} \simeq -131.4$ chosen such that
$\Omega_\Lambda(\theta_{\star,\mathrm{fid}}) \simeq 0.7$
at $\theta_{\star,\mathrm{fid}} = 3.63~\mathrm{rad}$.

We then evolve $N_{\mathrm{traj}} = 200$ independent trajectories of
$\theta_\star(t)$ as a simple Gaussian random walk with reflecting
boundaries at $\theta_{\min}$ and $\theta_{\max}$ and a typical step size
$\Delta\theta \simeq 0.02~\mathrm{rad}$ per timestep.
For each trajectory we record the induced $\Omega_\Lambda(t)$ and measure
the fraction of trajectory-time spent in three observational windows,
\begin{equation}
  \bigl|\Omega_\Lambda - 0.7\bigr| \leq \{0.05,\,0.02,\,0.01\}.
\end{equation}
The resulting residence fractions are approximately
$20.5\%$, $7.4\%$, and $3.7\%$ respectively, i.e.\ the random-walk motion
in $\theta_\star$ spends a non-negligible fraction of its time in a band
that is observationally indistinguishable from the fiducial
$\Lambda$CDM value.

Figure~\ref{fig:theta_star_random_walk_residence} shows the resulting
distribution of $\Omega_\Lambda$ values sampled along all trajectories,
together with the broad observational window
$\bigl|\Omega_\Lambda - 0.7\bigr| \leq 0.05$.
This figure is not an attempt at a physical equation of motion for
$\theta_\star$, but rather a toy demonstration that once the microcavity
scaling is fixed, a simple stochastic dynamics in $\theta_\star$ naturally
generates residence-time statistics in the observed $\Omega_\Lambda$ band
at the $\sim 10\%$ level.

\begin{figure}[t]
  \centering
  \includegraphics[width=0.7\textwidth]{figures/theta_star_random_walk_residence}
  \caption{
    Residence-time distribution of $\Omega_\Lambda$ for the random-walk
    ensemble of $\theta_\star$ trajectories (R6).
    We histogram all sampled $\Omega_\Lambda(t)$ values from
    $N_{\mathrm{traj}} = 200$ random walks in the band
    $\theta_\star \in [2.18, 5.54]~\mathrm{rad}$, using the microcavity-backed
    mapping $\Omega_\Lambda(\theta_\star)$ fixed in previous rungs.
    The vertical dashed line marks the fiducial value
    $\Omega_\Lambda = 0.7$ and the shaded band indicates the window
    $\bigl|\Omega_\Lambda - 0.7\bigr| \leq 0.05$.
    In this toy model, roughly $20\%$ of the total trajectory-time lies
    within this observational window.
  }
  \label{fig:theta_star_random_walk_residence}
\end{figure}


\subsection{R7: Induced $\theta_\star$ prior in $\Omega_\Lambda$}
\label{sec:theta-star-prior-vs-effective}

In earlier steps we constructed a one–dimensional prior over $\theta_\star$,
based on a small set of representative lattice vacua in the defected 1D chain
and stored in
\texttt{data/processed/theta\_star\_prior\_1d\_vacuum\_samples.npz}.
Here we ask what that prior implies for the effective dark–energy density,
once the microcavity calibration is taken seriously.

We start from the discrete prior samples
$\{\theta_{\star,i}\}$ and map each point through the microcavity–backed
effective vacuum model
\begin{equation}
  \Omega_\Lambda(\theta_\star)
  \;=\;
  \mathrm{clip}\!\left(
    k_{\rm scale}\,\Delta E(\theta_\star),\,0,\,0.999
  \right),
\end{equation}
where $k_{\rm scale}$ is fixed by the requirement
$\Omega_\Lambda(\theta_\star^{\rm fid}) = 0.7$ at
$\theta_\star^{\rm fid} = 3.63~\mathrm{rad}$.
The resulting histogram of $\Omega_\Lambda$ values defines an
``induced'' prior for the cosmological constant, shown in
Fig.~\ref{fig:theta_star_prior_vs_effective_vacuum}.

In this toy setup the induced $\Omega_\Lambda$ prior is broad:
it places substantial weight at values below the observed
$\Omega_\Lambda \simeq 0.7$, but comfortably reaches and slightly
exceeds the target range.  In other words, a simple prior over
$\theta_\star$ in the microcavity band does not narrowly fine–tune
$\Omega_\Lambda$, yet still assigns non–negligible probability
to a $\Lambda$CDM–like universe.

\begin{figure}[t]
  \centering
  \includegraphics[width=0.65\textwidth]{figures/theta_star_prior_vs_effective_vacuum}
  \caption{
    Induced prior for the dark–energy density parameter
    $\Omega_\Lambda$ obtained by mapping the discrete
    $\theta_\star$ prior through the microcavity–calibrated
    effective vacuum model.
    The vertical line marks the target value
    $\Omega_\Lambda \simeq 0.7$.
  }
  \label{fig:theta_star_prior_vs_effective_vacuum}
\end{figure}


\subsection{Act V summary: from microcavity to an effective cosmological constant}
\label{sec:act5-summary}

In this act we closed the loop between a microscopic, discretised model
of vacuum energy and an effective cosmological constant at FRW scales.
Starting from the one-dimensional microcavity scan
$\Delta E(\theta_\star)$ over a full $0$--$2\pi$ band, we identified a
fiducial angle $\theta_{\star,\text{fid}} \simeq 3.63~\mathrm{rad}$ at
which the vacuum energy difference $\Delta E_{\rm fid}$ is negative and
of order $10^{-3}$ in the toy units of the lattice.  By demanding that
this point reproduces an observationally motivated value of the dark
energy density, $\Omega_\Lambda(\theta_{\star,\text{fid}}) = 0.7$, we
fixed a single proportionality constant $k_{\rm scale}$ that maps the
dimensionless microcavity energy shifts into an effective
$\Omega_\Lambda(\theta_\star)$.

With this prescription in place, we constructed an effective vacuum
model in which each choice of $\theta_\star$ inside a ``good'' band
$[2.18, 5.54]~\mathrm{rad}$ is associated to a different cosmological
constant.  Solving the Friedmann equations for a flat, matter-plus-vacuum
universe shows that the fiducial point $(\Omega_m,\Omega_\Lambda) =
(0.3, 0.7)$ reproduces the expected accelerated expansion relative to a
purely matter-dominated history, and yields distance--redshift relations
consistent with a standard $\Lambda$CDM-like background at the level of
this toy model.

The band scan reveals that, within the admissible $\theta_\star$ range,
$\Omega_\Lambda(\theta_\star)$ varies between $0$ and $\simeq 0.78$,
with a finite interval of angles for which $\Omega_\Lambda$ lies close
to $0.7$.  By sampling $N_{\rm patch} = 1000$ random $\theta_\star$
values uniformly inside the band, we built an effective ``patch
ensemble'' and found that the resulting distribution of
$\Omega_\Lambda$ has mean $\langle \Omega_\Lambda \rangle \simeq 0.45$,
standard deviation $\sigma_{\Omega_\Lambda} \simeq 0.31$, and median
$\tilde{\Omega}_\Lambda \simeq 0.57$.  Approximately $21.5\%$ of the
patches fall into the observationally inspired window
$\Omega_\Lambda = 0.70 \pm 0.05$, and the central $68\%$ interval of
the ensemble extends up to $\Omega_\Lambda \simeq 0.76$.  In other
words, once the microcavity band is accepted as the relevant dynamical
domain, a patch with $\Omega_\Lambda \approx 0.7$ is statistically
typical rather than exponentially fine-tuned.

We then promoted $\theta_\star$ to a toy dynamical variable undergoing a
random walk within the same band.  Using the microcavity-backed
$\Omega_\Lambda(\theta_\star)$, we showed that such a random walk spends
$\mathcal{O}(10\%)$ of its trajectory-time in the broad observational
window $\Omega_\Lambda = 0.70 \pm 0.05$, and a few percent in tighter
$\pm 0.02$ and $\pm 0.01$ bands.  Finally, by propagating the original
one-dimensional $\theta_\star$ prior through the same effective vacuum
model, we obtained a broad induced prior over $\Omega_\Lambda$ that
covers the observed value without building in any preference for it.

The construction here is deliberately minimalistic and remains a toy
model in several respects: the microcavity is one-dimensional, the
mapping from $\Delta E$ to $\Omega_\Lambda$ is linear and fixed by a
single fiducial point, and we have not yet coupled the effective vacuum
to realistic matter content beyond a dust-like component.  Nevertheless,
Act~V shows that the $\theta_\star$ degree of freedom can serve as a
dynamical label for vacuum energy in an ensemble of patches, and that
moderately typical values of $\theta_\star$ can reproduce a
$\Lambda$-like component of the right order of magnitude without
invoking a $10^{-120}$-level tuning by hand.  This is the bridge we
need between the underlying axiom and a macroscopic, FRW-level
description of accelerated expansion in the toy universe.