\subsection{R5: Dynamical tolerance of the $\theta_\star$ bridge}
\label{subsec:theta_star_dynamical_tolerance}

With the effective vacuum map in hand (R2--R3), we can now ask how
sensitive the late--time acceleration is to small deformations of the
microcavity phase $\theta_\star$. Concretely, we define
\begin{equation}
  \Omega_\Lambda(\theta_\star)
  \equiv
  k_{\rm scale}\,\Delta E(\theta_\star)
  \,,
\end{equation}
where $k_{\rm scale}$ is fixed by the fiducial calibration
$\Omega_\Lambda(\theta_{\rm fid}) = 0.7$ with
$\theta_{\rm fid} = 3.63~{\rm rad}$ and
$\Delta E(\theta_\star)$ is taken from the 1D microcavity scan
of Sec.~\ref{subsec:microcavity_scan}.

We then scan $\theta_\star$ over the same ``cosmologically allowed band''
\begin{equation}
  \theta_{\rm band} \in [2.18,\,5.54]~{\rm rad}
\end{equation}
and evaluate $\Omega_\Lambda(\theta_\star)$ at 41 equally--spaced
samples. The resulting profile is shown in
Fig.~\ref{fig:effective_vacuum_band_scan}. A broad plateau is visible:
for
\begin{equation}
  \theta_\star \in [2.516,\,3.692]~{\rm rad}
\end{equation}
we find
\begin{equation}
  \bigl|\Omega_\Lambda(\theta_\star) - 0.7\bigr| \le 0.05
  \quad\Rightarrow\quad
  \Omega_\Lambda(\theta_\star) \in [0.65,\,0.75] \,,
\end{equation}
i.e.\ nine grid points in our scan lie within a conservative
$\pm 0.05$ tolerance band around the observational target
$\Omega_\Lambda \simeq 0.7$.

Within this window the local slope remains modest. Near the left and
right edges we find
\begin{equation}
  \frac{{\rm d}\Omega_\Lambda}{{\rm d}\theta_\star}
  \sim \mathcal{O}(0.2\text{--}0.4)\;{\rm rad}^{-1},
\end{equation}
while around the mid--window point
$\theta_{\rm mid} \simeq 3.10~{\rm rad}$ the profile is extremely
flat,
\begin{equation}
  \left.\frac{{\rm d}\Omega_\Lambda}{{\rm d}\theta_\star}\right|_{\rm mid}
  \approx 2.3\times 10^{-2}\;{\rm rad}^{-1}.
\end{equation}
A naive linear estimate based on this plateau slope would allow
$\mathcal{O}(1)$--rad excursions in $\theta_\star$ before leaving the
$\pm 0.05$ band. In practice, the true tolerance is limited by the
edges of the window itself,
\begin{equation}
  \Delta\theta_\star^{\rm (band)}
  \simeq \frac{3.692 - 2.516}{2}
  \approx 0.59~{\rm rad},
\end{equation}
which still corresponds to order--unity fractional changes in
$\theta_\star$.

We summarise R5 as follows:
\begin{quote}
  \textbf{R5 (Dynamical robustness).} Once the microcavity parameters
  are fixed by the fiducial calibration (R2--R3), there exists a
  broad plateau in $\theta_\star$ of width $\sim 1.2~{\rm rad}$ on
  which $\Omega_\Lambda(\theta_\star)$ remains within
  $\pm 0.05$ of $0.7$. Small to moderate drifts in the microscopic
  phase $\theta_\star$ therefore do not catastrophically destabilise
  the late--time acceleration; the $\theta_\star$--bridge from
  microphysics to FRW is dynamically tolerant.
\end{quote}
In other words, once the non--cancelling microcavity is tuned onto the
observed $\Omega_\Lambda$ slice, the ensuing cosmic acceleration is
robust under $\mathcal{O}(0.1\text{--}0.6)$--rad changes in the
underlying phase.