% act1_origin_axiom.tex
% Act I: Motivation and conceptual backbone of the Origin Axiom program

\section{Introduction: vacuum, cancellation, and the Origin Axiom}
\label{sec:act1-intro}

The observed Universe appears to be dominated, at late times, by a nearly
constant vacuum-like contribution to the stress--energy tensor, usually
parameterised as a cosmological constant $\Lambda$ in general relativity.
In the simplest $\Lambda$CDM fits to cosmological data, this component
represents roughly $70\%$ of the current energy budget of the Universe,
with the remaining $30\%$ in matter (baryonic and dark) and a negligible
fraction in radiation.\footnote{Numerical values quoted in this paper are
illustrative and are not intended as a precision fit.}

From the point of view of quantum field theory (QFT), however, the vacuum
energy is not a separately tunable parameter but an emergent consequence of
the spectrum of fluctuations and the way their contributions ``add up.''
If one naively sums zero-point energies of known fields up to a high-energy
cutoff, the resulting vacuum energy density is many orders of magnitude
larger than the value inferred from cosmology. The fact that the observed
vacuum energy is small, yet apparently nonzero, is the essence of the
\emph{vacuum problem} or \emph{cosmological constant problem}.

In the standard effective-field-theory view, one therefore imagines that
different contributions to the vacuum energy---from different fields,
sectors, and symmetry-breaking events---nearly cancel, leaving only a tiny
residual. But the known contributions to the vacuum energy are not
dynamically required to cancel; they are simply summed in the Einstein
equations. This raises a conceptual question: is the vacuum energy just
whatever happens to be left over after a sequence of unrelated
contributions, or is there a deeper organising principle that constrains
these contributions \emph{never} to perfectly cancel?

The \emph{Origin Axiom} programme, of which this paper is Act~II--III, starts
from the latter possibility. Instead of assuming that vacuum contributions
are independent and free to cancel arbitrarily, we investigate toy models
in which the vacuum is constrained by a global non-cancelling principle.
This principle acts on coarse-grained amplitudes of the vacuum state, and
it can be implemented in simple lattice and continuum simulations.

\subsection{The non-cancelling principle}
\label{subsec:non-cancelling-principle}

The core conceptual ingredient is a constraint that forbids exact
cancellation of certain global amplitudes. In its simplest toy form, we
consider a complex amplitude $A(t)$ associated with a coarse-grained degree
of freedom of the vacuum. Left unconstrained, the dynamics of $A(t)$ in a
toy system might resemble a random walk in the complex plane, with its
magnitude $|A(t)|$ fluctuating around zero. The \emph{non-cancelling
principle} introduces a rule that prevents $|A(t)|$ from falling below a
prescribed lower bound, which we denote schematically as
\begin{equation}
  |A(t)| \ge \epsilon > 0.
\end{equation}
In practice, this is implemented as a corrective step in the time
integration of the toy system: whenever $|A(t)|$ attempts to cross below
the threshold $\epsilon$, the evolution is modified so that the amplitude
is ``clipped'' back onto the boundary $|A| = \epsilon$ instead of passing
through the origin.

We emphasise that this clipping procedure is purely phenomenological in
the current work: it is not derived from an underlying microscopic
Hamiltonian. Its purpose is more modest and conceptual. It allows us to
explore, in controlled numerical experiments, the consequences of imposing
a non-cancelling rule on coarse-grained vacuum amplitudes. In the toy
models studied in later sections, the constraint has two robust effects:
\begin{enumerate}
  \item it enforces a persistent, non-vanishing vacuum amplitude even when
  the unconstrained dynamics would drive $A(t)$ through the origin, and
  \item it introduces an effective energy cost associated with maintaining
  this nonzero amplitude, which can be interpreted as a vacuum-like
  contribution to the stress--energy tensor at large scales.
\end{enumerate}

The goal of the Origin Axiom programme is \emph{not} to claim that the
Universe literally implements the particular clipping algorithm used in
our simulations. Rather, it is to identify structural features that would
be shared by any theory in which the vacuum is globally constrained to
avoid perfect cancellation. These structural features can then be used to
build effective models that bridge between microstructure and cosmological
observables.

\subsection{A single phase \texorpdfstring{$\thetastar$}{theta*} as a bridge}
\label{subsec:theta-star-bridge}

One of the striking empirical facts about the Standard Model is that the
structure of flavour mixing, encoded in the PMNS matrix for leptons and the
CKM matrix for quarks, is highly nontrivial yet remarkably compact. Both
matrices can be parametrised in terms of three mixing angles and a single
Dirac CP-violating phase, plus additional Majorana phases if neutrinos are
Majorana particles. In this work we explore the hypothesis that a single
underlying phase, which we denote $\thetastar$, may organise both
\emph{flavour} structure and aspects of vacuum microstructure.

In Act~II, we construct phenomenological ans\"atze in which the leptonic
mixing angles and CP phase depend on $\thetastar$ through smooth
modulations around their experimentally favoured values. By scanning over
$\thetastar$ and nuisance parameters, and fitting to global neutrino
oscillation data, we obtain a posterior band for $\thetastar$ and a
fiducial value $\thetastar_{\rm fid}$ that summarises the region of phase
space compatible with current flavour data. We then promote this band and
fiducial value to a \emph{working prior} on $\thetastar$ for subsequent
acts of the programme.

In Act~III, we introduce toy models of vacuum microstructure in which the
vacuum energy depends on $\thetastar$ via an effective microcavity
mechanism. The details of the microcavity construction are deliberately
simplified; the important point is that the same $\thetastar$ that
organises flavour is used to modulate the vacuum energy in a way that can
be translated into an effective cosmological constant $\Lambda_{\rm eff}$.
We then study a Friedmann--Robertson--Walker (FRW) toy universe in which
the late-time acceleration is driven by $\Lambda_{\rm eff}(\thetastar)$,
and we compare the resulting age, deceleration parameter, and Hubble
diagram to those of standard $\Lambda$CDM.

Thus $\thetastar$ plays the role of a \emph{bridge parameter}: it is
constrained from one side by flavour data (Act~II) and used on the other
side to define vacuum properties in toy cosmologies (Act~III). Our aim is
not to derive $\thetastar$ from first principles, but to demonstrate that a
single phase can consistently carry information between these two domains
in a way that is numerically and conceptually transparent.

\subsection{Scope and limitations}
\label{subsec:scope-limitations}

The present paper is explicitly framed as Act~II--III of a longer
programme. It does not claim to solve the cosmological constant problem,
nor to provide a fully realistic model of vacuum microstructure. Instead,
its contributions are deliberately modest and methodological:
\begin{itemize}
  \item We define and implement a global non-cancelling constraint on
  vacuum amplitudes in simple lattice-based toy models, and we document
  its effect on the evolution of amplitudes and energies.
  \item We construct a family of $\thetastar$-dependent flavour ans\"atze,
  use them to fit neutrino and quark mixing data, and extract a posterior
  band and fiducial value for $\thetastar$.
  \item We couple this $\thetastar$ prior to a toy microcavity model of
  vacuum energy and build an effective-vacuum description that can be
  plugged into a FRW cosmology.
  \item We compute a small set of cosmology-style observables---age,
  deceleration parameter, and a Hubble-diagram-style distance relation---to
  illustrate how the effective vacuum behaves relative to a matter-only
  universe and a standard $\Lambda$CDM benchmark.
\end{itemize}

At every stage we make strong simplifying assumptions. The toy lattice
models are not derived from a realistic field content; the microcavity
construction is not tied to any specific microphysical mechanism; the
mapping from dimensionless energy shifts $\Delta E(\thetastar)$ to
cosmological energy densities is defined by an explicit scaling rather than
a derivation. These choices are intentional: they allow us to isolate the
\emph{structure} of the non-cancelling principle and the $\thetastar$
bridge without over-committing to details that would require a full
quantum-gravity framework.

The rest of the paper is organised as follows. In Act~II
(Section~\ref{sec:act2-theta-star}) we describe the construction of the
$\thetastar$ flavour ans\"atze and the resulting posterior band on
$\thetastar$. In Act~II$\,\tfrac{1}{2}$
(Section~\ref{sec:act2-vacuum-toys}) we present the non-cancelling toy
models of vacuum microstructure, including the microcavity construction and
its $\thetastar$-dependence. In Act~III
(Section~\ref{sec:act3-effective-vacuum}) we build the effective-vacuum
interface, couple it to an FRW toy universe, and compare selected
cosmological observables to standard benchmarks. We conclude with a
discussion of limitations and future directions in
Section~\ref{sec:discussion}.
