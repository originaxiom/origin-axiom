% ACT III: FRW observable-style checks
\section{Observable-style checks: matter-only vs effective vacuum}
\label{sec:frw-observables}

As a first ``Act~III'' observable-style cross-check we compare two flat
Friedmann--Robertson--Walker cosmologies in a deliberately simplified setting:
(i) a matter-only universe with $(\Omega_m, \Omega_\Lambda) = (1, 0)$ and
(ii) an ``effective vacuum'' universe with $(\Omega_m, \Omega_\Lambda)
= (0.3, 0.7)$, where the vacuum fraction $\Omega_\Lambda$ is inherited
from the Act~II $\theta_\star$ prior (through the microcavity-based effective
vacuum mapping discussed earlier).

For a flat universe containing only pressureless matter and a cosmological
constant, the dimensionless Hubble function is
\begin{equation}
  E(a) \equiv \frac{H(a)}{H_0}
  = \sqrt{\frac{\Omega_m}{a^3} + \Omega_\Lambda}\,,
\end{equation}
where $a$ is the scale factor normalised to $a_0 = 1$. The age in units of
the Hubble time $H_0^{-1}$ is then
\begin{equation}
  t_0 H_0 = \int_{a_{\rm min}}^{1} \frac{{\rm d}a}{a\,E(a)}\,,
\end{equation}
with $a_{\rm min} \sim 10^{-4}$ in our toy integration.  The deceleration
parameter today is
\begin{equation}
  q_0 \equiv -\frac{\ddot a a}{\dot a^2}\Big|_{a=1}
  = \frac{1}{2}\,\Omega_m - \Omega_\Lambda\,.
\end{equation}

Using $H_0 \simeq 70~{\rm km\,s^{-1}\,Mpc^{-1}}$ (Hubble time
$t_H \simeq 14~{\rm Gyr}$) we find
\begin{align}
  t_0 H_0 &\simeq 0.67\,, &
  t_0 &\simeq 9.3~{\rm Gyr}\,, &
  q_0 &\simeq +0.5
  && \text{(matter-only)} \,, \\[0.3em]
  t_0 H_0 &\simeq 0.96\,, &
  t_0 &\simeq 13.5~{\rm Gyr}\,, &
  q_0 &\simeq -0.55
  && \text{(effective vacuum)} \,.
\end{align}

The matter-only universe is always decelerating ($q_0 > 0$) and
is significantly younger than a $\Lambda$CDM-like universe.  By contrast,
the effective-vacuum cosmology yields a negative deceleration parameter
($q_0 < 0$) and an age scale $\sim 13.5~{\rm Gyr}$, qualitatively in line
with the observed late-time Universe.

We emphasise that this test does not yet \emph{predict} the value of the
cosmological constant.  Instead, it makes explicit that once the
$\theta_\star$-backed effective vacuum fraction $\Omega_\Lambda$ is fixed
to a $\Lambda$CDM-like value, the resulting FRW histories naturally live
in the correct qualitative regime (accelerated expansion with a plausible
age) for further, more detailed comparisons to data.
