\subsection{Effective vacuum band and the $\theta_\star$--$\Omega_\Lambda$ map}
\label{subsec:effective_vacuum_band}

In the previous steps we established two ingredients:

\begin{enumerate}
  \item A microscopic \emph{microcavity} model that assigns to each
    twist angle $\theta_\star$ a vacuum--energy difference
    $\Delta E(\theta_\star)$ between a ``uniform'' and a
    ``cavity'' configuration. Numerically, this is encoded in the
    full $2\pi$ scan
    \verb|data/processed/theta_star_microcavity_scan_full_2pi.npz|,
    which provides arrays
    \(\{\theta_{\rm grid}, \Delta E(\theta_{\rm grid})\}\)
    together with the underlying ground--state energies
    \(E_{0,\mathrm{uniform}}\) and \(E_{0,\mathrm{cavity}}\).
  \item A cosmological effective description in which a single
    scalar parameter $\theta_\star$ controls the dark--energy density
    via an effective scaling relation
    \begin{equation}
      \Omega_\Lambda(\theta_\star) \;=\;
      k_{\rm scale}\,\Delta E(\theta_\star)\,,
    \end{equation}
    where $k_{\rm scale}$ is determined by matching to a fiducial
    ``observed'' value of $\Omega_\Lambda$ at some preferred
    $\theta_\star$.
\end{enumerate}

In this subsection we combine these two ingredients to construct an
\emph{effective vacuum band}:
a continuous interval of $\theta_\star$ values that are simultaneously
compatible with the microcavity spectrum and with an observationally
allowed range of $\Omega_\Lambda$.

\paragraph{Fiducial point and scaling.}

On the ``flavor side'' of the theory we have argued for a specific
fiducial twist angle,
\begin{equation}
  \theta_{\rm fid} \equiv \theta_\star^{\rm (fid)} \approx 3.63~\mathrm{rad},
\end{equation}
which we adopt as the working reference point in the microcavity model.
From the full $2\pi$ scan we extract the nearest grid point
\(\theta_{\rm grid} \approx 3.6325~\mathrm{rad}\)
and find the corresponding vacuum--energy shift
\begin{equation}
  \Delta E_{\rm fid}
  \;\equiv\;
  \Delta E(\theta_\star^{\rm (fid)})
  \;\approx\;
  -5.33\times 10^{-3}\,.
\end{equation}

On the cosmology side we impose a fiducial dark--energy density
\begin{equation}
  \Omega_\Lambda^{\rm (fid)} \equiv \Omega_\Lambda(\theta_\star^{\rm (fid)})
  \;=\; 0.7\,,
\end{equation}
with a corresponding matter fraction
\(\Omega_m^{\rm (fid)} = 1 - \Omega_\Lambda^{\rm (fid)} = 0.3\),
as used in the FRW integrations of Sec.~\ref{subsec:frw_from_effective}.
Matching the microscopic and macroscopic descriptions then fixes the
single scaling parameter
\begin{equation}
  k_{\rm scale}
  \;=\;
  \frac{\Omega_\Lambda^{\rm (fid)}}{\Delta E_{\rm fid}}
  \;\approx\;
  -1.31\times 10^{2}\,.
\end{equation}
Numerically, this calibration step is recorded in
\verb|data/processed/theta_star_microcavity_core_summary.json|,
which stores
\(\theta_{\rm fid}\), \(\Delta E_{\rm fid}\),
\(\Omega_\Lambda^{\rm (fid)} = 0.7\), and the derived
\(k_{\rm scale}\).

\paragraph{Defining the $\theta_\star$ band.}

The $\theta_\star$ prior emerging from the one--dimensional vacuum
construction and flavor--sector considerations is not a single point
but a \emph{band}
\begin{equation}
  \theta_\star \in [\theta_{\min}, \theta_{\max}] \;=\;
  [2.18, 5.54]~\mathrm{rad}\,.
\end{equation}
This interval contains the fiducial value
\(\theta_{\rm fid} \approx 3.63\) and is chosen such that:

\begin{itemize}
  \item it lies within the range where the microcavity scan is
    numerically reliable;
  \item it respects the prior constraints from the $1$D twisted vacuum
    energy (no pathological instabilities, consistent sign of
    $\Delta E$);
  \item it encompasses the flavor--motivated twist while excluding
    obviously unphysical tails of the microcavity spectrum.
\end{itemize}

Within this band, we define a discrete sampling of
\(N_{\rm band} = 41\) equally spaced values,
\(\{\theta_{\star,i}\}_{i=1}^{N_{\rm band}}\),
and compute for each point:
\begin{equation}
  \Omega_\Lambda(\theta_{\star,i})
  \;=\;
  k_{\rm scale}\,\Delta E(\theta_{\star,i})\,,
\end{equation}
where \(\Delta E(\theta_{\star,i})\) is obtained by interpolating the
full microcavity scan.  This procedure is implemented in
\verb|src/run_effective_vacuum_band_scan.py| and the resulting arrays
\(\{\theta_{\star,i}, \Delta E_i, \Omega_{\Lambda,i}\}\) are stored in
\verb|data/processed/effective_vacuum_band_scan.npz|.

\paragraph{Structure of the effective vacuum band.}

The output of the band scan can be summarized as follows:

\begin{itemize}
  \item The sampled band in $\theta_\star$ spans
    \(\theta_\star \in [2.18, 5.54]~\mathrm{rad}\).
  \item The corresponding dark--energy fraction varies within
    \begin{equation}
      \Omega_\Lambda(\theta_\star) \in [0,\, 0.775]\;,
    \end{equation}
    where the lower end of the band approaches an effectively
    vanishing dark--energy contribution.
  \item Around the fiducial value \(\Omega_\Lambda = 0.7\) there exists
    a finite ``$\Lambda$--window'' in $\theta_\star$:
    if we require
    \(|\Omega_\Lambda(\theta_\star) - 0.7| \le 0.05\), we find
    \emph{nine} sample points in the band that satisfy this criterion.
    A representative subset is
    \begin{align}
      \theta_\star &\approx 2.516~\mathrm{rad}, &
      \Omega_\Lambda(\theta_\star) &\approx 0.651\,, \\
      \theta_\star &\approx 2.600~\mathrm{rad}, &
      \Omega_\Lambda(\theta_\star) &\approx 0.683\,, \\
      \theta_\star &\approx 2.684~\mathrm{rad}, &
      \Omega_\Lambda(\theta_\star) &\approx 0.710\,, \\
      \theta_\star &\approx 2.768~\mathrm{rad}, &
      \Omega_\Lambda(\theta_\star) &\approx 0.732\,, \\
      \theta_\star &\approx 2.852~\mathrm{rad}, &
      \Omega_\Lambda(\theta_\star) &\approx 0.750\,.
    \end{align}
\end{itemize}

Several points are conceptually important:

\begin{enumerate}
  \item The microcavity model does \emph{not} single out a unique,
    infinitely fine--tuned $\theta_\star$ that yields
    $\Omega_\Lambda \simeq 0.7$.  Instead, it produces a
    \emph{continuous band} of twist angles for which the effective
    dark--energy fraction lies in an observationally acceptable range.
  \item The fiducial choice \(\theta_\star^{\rm (fid)} \approx 3.63\)
    used in the FRW histories of Sec.~\ref{subsec:frw_from_effective}
    sits naturally inside a broader stability window: small
    perturbations of $\theta_\star$ do not immediately destroy the
    effective $\Lambda$ behaviour.
  \item The upper edge of the band, where
    \(\Omega_\Lambda(\theta_\star) \approx 0.775\), provides a
    natural limit beyond which the microcavity--driven effective
    vacuum would overproduce dark energy relative to the fiducial
    cosmological target.
\end{enumerate}

From an ``axiom to atom'' perspective, this effective vacuum band is the
first concrete realization of the idea that a single microscopic twist
parameter $\theta_\star$ can generate both (i) a nontrivial spectrum of
vacuum energies in a controlled microscopic model and (ii) an
observationally viable range of macroscopic dark--energy densities,
once a single matching condition at a fiducial point is imposed.
In the next subsection we will use this band structure as the starting
point for connecting $\theta_\star$ to flavor mixing and, eventually,
to laboratory observables.

\begin{figure}[t]
    \centering
    \includegraphics[width=0.7\textwidth]{figures/effective_vacuum_band_scan}
    \caption{%
      Effective-vacuum prediction $\Omega_\Lambda(\theta_\star)$ as a function of
      $\theta_\star$ across the band $[2.18, 5.54]$\,rad. The solid curve shows
      the microcavity-backed $\Omega_\Lambda(\theta_\star)$ and the dashed line
      marks the target $\Omega_{\Lambda,\rm fid} = 0.70$.  The region where
      $|\Omega_\Lambda(\theta_\star) - 0.70| \lesssim 0.05$ provides an allowed
      band of $\theta_\star$ values consistent with late-time acceleration.
    }
    \label{fig:effective_vacuum_band_scan}
\end{figure}