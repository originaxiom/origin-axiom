% main_origin_axiom.tex
% Top-level paper file for the Origin Axiom / theta* programme.

\documentclass[11pt,a4paper]{article}

\usepackage[margin=2.5cm]{geometry}
\usepackage{amsmath,amssymb,amsfonts}
\usepackage{bm}
\usepackage{graphicx}
\usepackage{hyperref}
\usepackage[numbers]{natbib}

% Basic macros
\newcommand{\thetastar}{\theta^\star}
\newcommand{\OmegaLambda}{\Omega_{\Lambda}}
\newcommand{\Omegam}{\Omega_{\mathrm{m}}}
\newcommand{\Hzero}{H_{0}}

\begin{document}

\title{Origin Axiom Act II--III:\\
A $\thetastar$ prior from flavour and a toy effective vacuum for cosmology}

\author{D.\ Mehmetaj and collaborators}
\date{\today}

\maketitle

\begin{abstract}
We explore a toy implementation of the \emph{Origin Axiom}: a programme in
which the vacuum is constrained by a global non-cancelling principle and a
single phase $\thetastar$ simultaneously organises flavour mixing and an
effective vacuum energy. In Act~II we build phenomenological
$\thetastar$--dependent ans\"atze for the leptonic and quark mixing
matrices, fit them to current flavour data, and extract a posterior band
and fiducial value for $\thetastar$. This band is encoded as a simple
configuration file that can be consumed by other codes. In Act~III we
construct lattice toy models of vacuum microstructure, including a
$\thetastar$--dependent microcavity, and we use them to define an
effective-vacuum contribution to a Friedmann--Robertson--Walker (FRW) toy
universe. We compare the resulting age, deceleration parameter, and
Hubble-diagram-style distance relation to standard matter-only and
$\Lambda$CDM benchmarks. The purpose of this paper is methodological: to
demonstrate that a single phase prior derived from flavour can be wired, in
a fully reproducible way, into simple vacuum and cosmology models, and to
make explicit the scaling assumptions required to connect toy microphysics
to observed vacuum energy scales.
\end{abstract}

\tableofcontents

%------------------------------------------------------------
% Act I: Motivation and conceptual backbone
%------------------------------------------------------------
% act1_origin_axiom.tex
% Act I: Motivation and conceptual backbone of the Origin Axiom program

\section{Introduction: vacuum, cancellation, and the Origin Axiom}
\label{sec:act1-intro}

The observed Universe appears to be dominated, at late times, by a nearly
constant vacuum-like contribution to the stress--energy tensor, usually
parameterised as a cosmological constant $\Lambda$ in general relativity.
In the simplest $\Lambda$CDM fits to cosmological data, this component
represents roughly $70\%$ of the current energy budget of the Universe,
with the remaining $30\%$ in matter (baryonic and dark) and a negligible
fraction in radiation.\footnote{Numerical values quoted in this paper are
illustrative and are not intended as a precision fit.}

From the point of view of quantum field theory (QFT), however, the vacuum
energy is not a separately tunable parameter but an emergent consequence of
the spectrum of fluctuations and the way their contributions ``add up.''
If one naively sums zero-point energies of known fields up to a high-energy
cutoff, the resulting vacuum energy density is many orders of magnitude
larger than the value inferred from cosmology. The fact that the observed
vacuum energy is small, yet apparently nonzero, is the essence of the
\emph{vacuum problem} or \emph{cosmological constant problem}.

In the standard effective-field-theory view, one therefore imagines that
different contributions to the vacuum energy---from different fields,
sectors, and symmetry-breaking events---nearly cancel, leaving only a tiny
residual. But the known contributions to the vacuum energy are not
dynamically required to cancel; they are simply summed in the Einstein
equations. This raises a conceptual question: is the vacuum energy just
whatever happens to be left over after a sequence of unrelated
contributions, or is there a deeper organising principle that constrains
these contributions \emph{never} to perfectly cancel?

The \emph{Origin Axiom} programme, of which this paper is Act~II--III, starts
from the latter possibility. Instead of assuming that vacuum contributions
are independent and free to cancel arbitrarily, we investigate toy models
in which the vacuum is constrained by a global non-cancelling principle.
This principle acts on coarse-grained amplitudes of the vacuum state, and
it can be implemented in simple lattice and continuum simulations.

\subsection{The non-cancelling principle}
\label{subsec:non-cancelling-principle}

The core conceptual ingredient is a constraint that forbids exact
cancellation of certain global amplitudes. In its simplest toy form, we
consider a complex amplitude $A(t)$ associated with a coarse-grained degree
of freedom of the vacuum. Left unconstrained, the dynamics of $A(t)$ in a
toy system might resemble a random walk in the complex plane, with its
magnitude $|A(t)|$ fluctuating around zero. The \emph{non-cancelling
principle} introduces a rule that prevents $|A(t)|$ from falling below a
prescribed lower bound, which we denote schematically as
\begin{equation}
  |A(t)| \ge \epsilon > 0.
\end{equation}
In practice, this is implemented as a corrective step in the time
integration of the toy system: whenever $|A(t)|$ attempts to cross below
the threshold $\epsilon$, the evolution is modified so that the amplitude
is ``clipped'' back onto the boundary $|A| = \epsilon$ instead of passing
through the origin.

We emphasise that this clipping procedure is purely phenomenological in
the current work: it is not derived from an underlying microscopic
Hamiltonian. Its purpose is more modest and conceptual. It allows us to
explore, in controlled numerical experiments, the consequences of imposing
a non-cancelling rule on coarse-grained vacuum amplitudes. In the toy
models studied in later sections, the constraint has two robust effects:
\begin{enumerate}
  \item it enforces a persistent, non-vanishing vacuum amplitude even when
  the unconstrained dynamics would drive $A(t)$ through the origin, and
  \item it introduces an effective energy cost associated with maintaining
  this nonzero amplitude, which can be interpreted as a vacuum-like
  contribution to the stress--energy tensor at large scales.
\end{enumerate}

The goal of the Origin Axiom programme is \emph{not} to claim that the
Universe literally implements the particular clipping algorithm used in
our simulations. Rather, it is to identify structural features that would
be shared by any theory in which the vacuum is globally constrained to
avoid perfect cancellation. These structural features can then be used to
build effective models that bridge between microstructure and cosmological
observables.

\subsection{A single phase \texorpdfstring{$\thetastar$}{theta*} as a bridge}
\label{subsec:theta-star-bridge}

One of the striking empirical facts about the Standard Model is that the
structure of flavour mixing, encoded in the PMNS matrix for leptons and the
CKM matrix for quarks, is highly nontrivial yet remarkably compact. Both
matrices can be parametrised in terms of three mixing angles and a single
Dirac CP-violating phase, plus additional Majorana phases if neutrinos are
Majorana particles. In this work we explore the hypothesis that a single
underlying phase, which we denote $\thetastar$, may organise both
\emph{flavour} structure and aspects of vacuum microstructure.

In Act~II, we construct phenomenological ans\"atze in which the leptonic
mixing angles and CP phase depend on $\thetastar$ through smooth
modulations around their experimentally favoured values. By scanning over
$\thetastar$ and nuisance parameters, and fitting to global neutrino
oscillation data, we obtain a posterior band for $\thetastar$ and a
fiducial value $\thetastar_{\rm fid}$ that summarises the region of phase
space compatible with current flavour data. We then promote this band and
fiducial value to a \emph{working prior} on $\thetastar$ for subsequent
acts of the programme.

In Act~III, we introduce toy models of vacuum microstructure in which the
vacuum energy depends on $\thetastar$ via an effective microcavity
mechanism. The details of the microcavity construction are deliberately
simplified; the important point is that the same $\thetastar$ that
organises flavour is used to modulate the vacuum energy in a way that can
be translated into an effective cosmological constant $\Lambda_{\rm eff}$.
We then study a Friedmann--Robertson--Walker (FRW) toy universe in which
the late-time acceleration is driven by $\Lambda_{\rm eff}(\thetastar)$,
and we compare the resulting age, deceleration parameter, and Hubble
diagram to those of standard $\Lambda$CDM.

Thus $\thetastar$ plays the role of a \emph{bridge parameter}: it is
constrained from one side by flavour data (Act~II) and used on the other
side to define vacuum properties in toy cosmologies (Act~III). Our aim is
not to derive $\thetastar$ from first principles, but to demonstrate that a
single phase can consistently carry information between these two domains
in a way that is numerically and conceptually transparent.

\subsection{Scope and limitations}
\label{subsec:scope-limitations}

The present paper is explicitly framed as Act~II--III of a longer
programme. It does not claim to solve the cosmological constant problem,
nor to provide a fully realistic model of vacuum microstructure. Instead,
its contributions are deliberately modest and methodological:
\begin{itemize}
  \item We define and implement a global non-cancelling constraint on
  vacuum amplitudes in simple lattice-based toy models, and we document
  its effect on the evolution of amplitudes and energies.
  \item We construct a family of $\thetastar$-dependent flavour ans\"atze,
  use them to fit neutrino and quark mixing data, and extract a posterior
  band and fiducial value for $\thetastar$.
  \item We couple this $\thetastar$ prior to a toy microcavity model of
  vacuum energy and build an effective-vacuum description that can be
  plugged into a FRW cosmology.
  \item We compute a small set of cosmology-style observables---age,
  deceleration parameter, and a Hubble-diagram-style distance relation---to
  illustrate how the effective vacuum behaves relative to a matter-only
  universe and a standard $\Lambda$CDM benchmark.
\end{itemize}

At every stage we make strong simplifying assumptions. The toy lattice
models are not derived from a realistic field content; the microcavity
construction is not tied to any specific microphysical mechanism; the
mapping from dimensionless energy shifts $\Delta E(\thetastar)$ to
cosmological energy densities is defined by an explicit scaling rather than
a derivation. These choices are intentional: they allow us to isolate the
\emph{structure} of the non-cancelling principle and the $\thetastar$
bridge without over-committing to details that would require a full
quantum-gravity framework.

The rest of the paper is organised as follows. In Act~II
(Section~\ref{sec:act2-theta-star}) we describe the construction of the
$\thetastar$ flavour ans\"atze and the resulting posterior band on
$\thetastar$. In Act~II$\,\tfrac{1}{2}$
(Section~\ref{sec:act2-vacuum-toys}) we present the non-cancelling toy
models of vacuum microstructure, including the microcavity construction and
its $\thetastar$-dependence. In Act~III
(Section~\ref{sec:act3-effective-vacuum}) we build the effective-vacuum
interface, couple it to an FRW toy universe, and compare selected
cosmological observables to standard benchmarks. We conclude with a
discussion of limitations and future directions in
Section~\ref{sec:discussion}.

% --- Act II: flavour → theta-star prior → vacuum toys
\input{act2_theta_star_section}
% ACT II / early ACT III: FRW toy universe with vacuum
\subsection{FRW toy universe: matter vs vacuum domination}
\label{subsec:frw_vacuum_demo}

To connect the microscopic non--cancelling rule to a familiar cosmological observable, we embed a simple vacuum component into a spatially flat Friedmann--Robertson--Walker (FRW) background and study how it modifies the expansion history.

We work in dimensionless units with present--day Hubble parameter $H_0 = 1$, and consider a scale factor $a(t)$ evolving under
\begin{equation}
  \left(\frac{\dot a}{a}\right)^2
  = H_0^2 \left[ \Omega_m a^{-3} + \Omega_\Lambda \right],
\end{equation}
with $\Omega_m$ the matter density parameter and $\Omega_\Lambda$ an effective vacuum component.  In this toy setup we treat $\Omega_\Lambda$ as a free parameter, postponing its derivation from the microcavity vacuum shift $\Delta E(\theta_\star)$ to later sections.

We numerically integrate the Friedmann equation starting from a small initial scale factor $a_{\rm init} \ll 1$ up to $t \sim \mathrm{few}\,H_0^{-1}$ for three benchmark cosmologies:
\begin{enumerate}
  \item a matter--only universe, $(\Omega_m, \Omega_\Lambda) = (1.0, 0.0)$;
  \item a mixed case with a modest vacuum component, $(\Omega_m, \Omega_\Lambda) = (0.7, 0.3)$;
  \item a vacuum--dominated case, $(\Omega_m, \Omega_\Lambda) = (0.3, 0.7)$.
\end{enumerate}
The corresponding scale--factor histories are shown in Fig.~\ref{fig:frw_vacuum_demo}.  At early times all three curves track each other closely, reflecting the fact that the $a^{-3}$ matter term dominates the right-hand side of the Friedmann equation.  As the universe expands and the matter density dilutes, the constant vacuum term takes over in the mixed and vacuum--dominated cases, leading to accelerated expansion and a rapidly growing $a(t)$.

\begin{figure}
  \centering
  % The plotting script currently saves a PNG figure at
  % data/processed/figures/frw_vacuum_demo_a_of_t.png.
  % For the paper we will generate a PDF version and place it under figures/.
  \includegraphics[width=0.75\textwidth]{figures/frw_vacuum_demo_a_of_t}
  \caption{%
    FRW toy universe with matter and an effective vacuum component.
    We show the scale factor $a(t)$ for three benchmark cosmologies:
    matter--only $(\Omega_m, \Omega_\Lambda) = (1.0, 0.0)$,
    mixed $(0.7, 0.3)$, and vacuum--dominated $(0.3, 0.7)$.
    All runs start from the same small initial scale factor, but the
    vacuum--dominated case rapidly transitions into accelerated
    expansion.  In the current stage of the project $\Omega_\Lambda$
    is a free knob; in later Acts it will be tied to the microscopic
    vacuum shift $\Delta E(\theta_\star)$ induced by the
    non--cancelling rule.
  }
  \label{fig:frw_vacuum_demo}
\end{figure}

This FRW toy model provides the ``top--down'' view of the vacuum:
starting from an assumed effective $\Omega_\Lambda$ we see how the
global expansion responds.  The bottom--up view comes from the
1D microcavity and lattice models of Sec.~\ref{sec:microcavity},
which compute a small but nonzero vacuum energy shift
$\Delta E(\theta_\star)$ consistent with the Act II $\theta_\star$
prior.  The long--term goal is to connect these two descriptions via
\begin{equation}
  \Delta E(\theta_\star)
  \;\longrightarrow\;
  \rho_\Lambda^{\rm eff}
  \;\longrightarrow\;
  \Omega_\Lambda
  \;\longrightarrow\;
  a(t),
\end{equation}
closing the loop between microscopic non--cancelling dynamics and
coarse--grained cosmological expansion.

\input{act2_frw_from_microcavity}
\input{act2_lambda_scaling_check}

% --- Act III: effective vacuum + FRW observables
\input{act3_effective_vacuum_interface}
% ACT III: FRW observable-style checks
\section{Observable-style checks: matter-only vs effective vacuum}
\label{sec:frw-observables}

As a first ``Act~III'' observable-style cross-check we compare two flat
Friedmann--Robertson--Walker cosmologies in a deliberately simplified setting:
(i) a matter-only universe with $(\Omega_m, \Omega_\Lambda) = (1, 0)$ and
(ii) an ``effective vacuum'' universe with $(\Omega_m, \Omega_\Lambda)
= (0.3, 0.7)$, where the vacuum fraction $\Omega_\Lambda$ is inherited
from the Act~II $\theta_\star$ prior (through the microcavity-based effective
vacuum mapping discussed earlier).

For a flat universe containing only pressureless matter and a cosmological
constant, the dimensionless Hubble function is
\begin{equation}
  E(a) \equiv \frac{H(a)}{H_0}
  = \sqrt{\frac{\Omega_m}{a^3} + \Omega_\Lambda}\,,
\end{equation}
where $a$ is the scale factor normalised to $a_0 = 1$. The age in units of
the Hubble time $H_0^{-1}$ is then
\begin{equation}
  t_0 H_0 = \int_{a_{\rm min}}^{1} \frac{{\rm d}a}{a\,E(a)}\,,
\end{equation}
with $a_{\rm min} \sim 10^{-4}$ in our toy integration.  The deceleration
parameter today is
\begin{equation}
  q_0 \equiv -\frac{\ddot a a}{\dot a^2}\Big|_{a=1}
  = \frac{1}{2}\,\Omega_m - \Omega_\Lambda\,.
\end{equation}

Using $H_0 \simeq 70~{\rm km\,s^{-1}\,Mpc^{-1}}$ (Hubble time
$t_H \simeq 14~{\rm Gyr}$) we find
\begin{align}
  t_0 H_0 &\simeq 0.67\,, &
  t_0 &\simeq 9.3~{\rm Gyr}\,, &
  q_0 &\simeq +0.5
  && \text{(matter-only)} \,, \\[0.3em]
  t_0 H_0 &\simeq 0.96\,, &
  t_0 &\simeq 13.5~{\rm Gyr}\,, &
  q_0 &\simeq -0.55
  && \text{(effective vacuum)} \,.
\end{align}

The matter-only universe is always decelerating ($q_0 > 0$) and
is significantly younger than a $\Lambda$CDM-like universe.  By contrast,
the effective-vacuum cosmology yields a negative deceleration parameter
($q_0 < 0$) and an age scale $\sim 13.5~{\rm Gyr}$, qualitatively in line
with the observed late-time Universe.

We emphasise that this test does not yet \emph{predict} the value of the
cosmological constant.  Instead, it makes explicit that once the
$\theta_\star$-backed effective vacuum fraction $\Omega_\Lambda$ is fixed
to a $\Lambda$CDM-like value, the resulting FRW histories naturally live
in the correct qualitative regime (accelerated expansion with a plausible
age) for further, more detailed comparisons to data.

\subsection{Hubble-style distance comparison}
\label{subsec:actiii_hubble_diagram}

In order to make a first contact with observable expansion data, we constructed a simple Hubble-style distance comparison between
\begin{enumerate}
  \item a matter-only flat FRW universe with \(\Omega_m = 1\) and \(\Omega_\Lambda = 0\), and
  \item an ``effective vacuum'' FRW universe in which \(\Omega_\Lambda\) is fixed by the Act~II \(\theta_\star\) prior and the 1D microcavity scan (Sec.~\ref{sec:act3_effective_vacuum}).
\end{enumerate}
In both cases we assume spatial flatness and neglect radiation.

We work in units where \(c = 1\) and measure distances in units of \(c/H_0\).  The dimensionless Hubble factor for a flat matter+Lambda model is
\begin{equation}
  E(z) \equiv \frac{H(z)}{H_0}
  = \sqrt{\Omega_m (1+z)^3 + \Omega_\Lambda} \, ,
\end{equation}
and the dimensionless comoving distance to redshift \(z\) is
\begin{equation}
  \chi(z) = \int_0^z \frac{dz'}{E(z')} \, .
\end{equation}
The luminosity distance in these units is then
\begin{equation}
  d_L(z) = (1+z)\,\chi(z) \, .
\end{equation}

The script \texttt{scripts/compare\_hubble\_diagram.py} evaluates these expressions using a simple trapezoidal rule on a uniform redshift grid up to \(z_{\max} \simeq 2\).  The matter-only case is computed with
\begin{equation}
  (\Omega_m, \Omega_\Lambda) = (1, 0) \, ,
\end{equation}
while the effective-vacuum case uses the fiducial parameters exported from Act~II and the microcavity analysis,
\begin{equation}
  (\Omega_m^{\rm eff}, \Omega_\Lambda^{\rm eff}) \simeq (0.3, 0.7) \, ,
\end{equation}
as encoded in the shared configuration file \texttt{config/theta\_star\_config.json}.

Figure~\ref{fig:actiii_hubble_diagram} shows the resulting luminosity distance \(d_L(z)\) as a function of redshift for both cosmologies.  For a given \(z\), the effective-vacuum curve yields a \emph{larger} luminosity distance than the matter-only curve.  In Hubble-diagram language, this means that standard candles (e.g.\ Type~Ia supernovae) would appear fainter at fixed redshift in the effective-vacuum cosmology than in a matter-only universe---the classic signature of late-time cosmic acceleration.

We emphasize that this comparison is intentionally simple: it does not attempt precision fits to real supernova data.  Its purpose is to demonstrate that once \(\Omega_\Lambda\) is fixed by the \(\theta_\star\)--microcavity pipeline, the resulting FRW background behaves qualitatively like the observed accelerating universe when viewed through a distance--redshift relation.

\begin{figure}
  \centering
  \includegraphics[width=0.7\textwidth]{figures/frw_effective_hubble_diagram}
  \caption{Toy Hubble-style luminosity distance comparison between a matter-only flat FRW universe and the effective-vacuum cosmology supported by the Act~II \(\theta_\star\) prior.  Distances are shown in units of \(c/H_0\).  The larger \(d_L(z)\) at fixed \(z\) in the effective-vacuum case corresponds to the familiar dimming of standard candles in an accelerating universe.}
  \label{fig:actiii_hubble_diagram}
\end{figure}

\subsection{Effective vacuum band and the $\theta_\star$--$\Omega_\Lambda$ map}
\label{subsec:effective_vacuum_band}

In the previous steps we established two ingredients:

\begin{enumerate}
  \item A microscopic \emph{microcavity} model that assigns to each
    twist angle $\theta_\star$ a vacuum--energy difference
    $\Delta E(\theta_\star)$ between a ``uniform'' and a
    ``cavity'' configuration. Numerically, this is encoded in the
    full $2\pi$ scan
    \verb|data/processed/theta_star_microcavity_scan_full_2pi.npz|,
    which provides arrays
    \(\{\theta_{\rm grid}, \Delta E(\theta_{\rm grid})\}\)
    together with the underlying ground--state energies
    \(E_{0,\mathrm{uniform}}\) and \(E_{0,\mathrm{cavity}}\).
  \item A cosmological effective description in which a single
    scalar parameter $\theta_\star$ controls the dark--energy density
    via an effective scaling relation
    \begin{equation}
      \Omega_\Lambda(\theta_\star) \;=\;
      k_{\rm scale}\,\Delta E(\theta_\star)\,,
    \end{equation}
    where $k_{\rm scale}$ is determined by matching to a fiducial
    ``observed'' value of $\Omega_\Lambda$ at some preferred
    $\theta_\star$.
\end{enumerate}

In this subsection we combine these two ingredients to construct an
\emph{effective vacuum band}:
a continuous interval of $\theta_\star$ values that are simultaneously
compatible with the microcavity spectrum and with an observationally
allowed range of $\Omega_\Lambda$.

\paragraph{Fiducial point and scaling.}

On the ``flavor side'' of the theory we have argued for a specific
fiducial twist angle,
\begin{equation}
  \theta_{\rm fid} \equiv \theta_\star^{\rm (fid)} \approx 3.63~\mathrm{rad},
\end{equation}
which we adopt as the working reference point in the microcavity model.
From the full $2\pi$ scan we extract the nearest grid point
\(\theta_{\rm grid} \approx 3.6325~\mathrm{rad}\)
and find the corresponding vacuum--energy shift
\begin{equation}
  \Delta E_{\rm fid}
  \;\equiv\;
  \Delta E(\theta_\star^{\rm (fid)})
  \;\approx\;
  -5.33\times 10^{-3}\,.
\end{equation}

On the cosmology side we impose a fiducial dark--energy density
\begin{equation}
  \Omega_\Lambda^{\rm (fid)} \equiv \Omega_\Lambda(\theta_\star^{\rm (fid)})
  \;=\; 0.7\,,
\end{equation}
with a corresponding matter fraction
\(\Omega_m^{\rm (fid)} = 1 - \Omega_\Lambda^{\rm (fid)} = 0.3\),
as used in the FRW integrations of Sec.~\ref{subsec:frw_from_effective}.
Matching the microscopic and macroscopic descriptions then fixes the
single scaling parameter
\begin{equation}
  k_{\rm scale}
  \;=\;
  \frac{\Omega_\Lambda^{\rm (fid)}}{\Delta E_{\rm fid}}
  \;\approx\;
  -1.31\times 10^{2}\,.
\end{equation}
Numerically, this calibration step is recorded in
\verb|data/processed/theta_star_microcavity_core_summary.json|,
which stores
\(\theta_{\rm fid}\), \(\Delta E_{\rm fid}\),
\(\Omega_\Lambda^{\rm (fid)} = 0.7\), and the derived
\(k_{\rm scale}\).

\paragraph{Defining the $\theta_\star$ band.}

The $\theta_\star$ prior emerging from the one--dimensional vacuum
construction and flavor--sector considerations is not a single point
but a \emph{band}
\begin{equation}
  \theta_\star \in [\theta_{\min}, \theta_{\max}] \;=\;
  [2.18, 5.54]~\mathrm{rad}\,.
\end{equation}
This interval contains the fiducial value
\(\theta_{\rm fid} \approx 3.63\) and is chosen such that:

\begin{itemize}
  \item it lies within the range where the microcavity scan is
    numerically reliable;
  \item it respects the prior constraints from the $1$D twisted vacuum
    energy (no pathological instabilities, consistent sign of
    $\Delta E$);
  \item it encompasses the flavor--motivated twist while excluding
    obviously unphysical tails of the microcavity spectrum.
\end{itemize}

Within this band, we define a discrete sampling of
\(N_{\rm band} = 41\) equally spaced values,
\(\{\theta_{\star,i}\}_{i=1}^{N_{\rm band}}\),
and compute for each point:
\begin{equation}
  \Omega_\Lambda(\theta_{\star,i})
  \;=\;
  k_{\rm scale}\,\Delta E(\theta_{\star,i})\,,
\end{equation}
where \(\Delta E(\theta_{\star,i})\) is obtained by interpolating the
full microcavity scan.  This procedure is implemented in
\verb|src/run_effective_vacuum_band_scan.py| and the resulting arrays
\(\{\theta_{\star,i}, \Delta E_i, \Omega_{\Lambda,i}\}\) are stored in
\verb|data/processed/effective_vacuum_band_scan.npz|.

\paragraph{Structure of the effective vacuum band.}

The output of the band scan can be summarized as follows:

\begin{itemize}
  \item The sampled band in $\theta_\star$ spans
    \(\theta_\star \in [2.18, 5.54]~\mathrm{rad}\).
  \item The corresponding dark--energy fraction varies within
    \begin{equation}
      \Omega_\Lambda(\theta_\star) \in [0,\, 0.775]\;,
    \end{equation}
    where the lower end of the band approaches an effectively
    vanishing dark--energy contribution.
  \item Around the fiducial value \(\Omega_\Lambda = 0.7\) there exists
    a finite ``$\Lambda$--window'' in $\theta_\star$:
    if we require
    \(|\Omega_\Lambda(\theta_\star) - 0.7| \le 0.05\), we find
    \emph{nine} sample points in the band that satisfy this criterion.
    A representative subset is
    \begin{align}
      \theta_\star &\approx 2.516~\mathrm{rad}, &
      \Omega_\Lambda(\theta_\star) &\approx 0.651\,, \\
      \theta_\star &\approx 2.600~\mathrm{rad}, &
      \Omega_\Lambda(\theta_\star) &\approx 0.683\,, \\
      \theta_\star &\approx 2.684~\mathrm{rad}, &
      \Omega_\Lambda(\theta_\star) &\approx 0.710\,, \\
      \theta_\star &\approx 2.768~\mathrm{rad}, &
      \Omega_\Lambda(\theta_\star) &\approx 0.732\,, \\
      \theta_\star &\approx 2.852~\mathrm{rad}, &
      \Omega_\Lambda(\theta_\star) &\approx 0.750\,.
    \end{align}
\end{itemize}

Several points are conceptually important:

\begin{enumerate}
  \item The microcavity model does \emph{not} single out a unique,
    infinitely fine--tuned $\theta_\star$ that yields
    $\Omega_\Lambda \simeq 0.7$.  Instead, it produces a
    \emph{continuous band} of twist angles for which the effective
    dark--energy fraction lies in an observationally acceptable range.
  \item The fiducial choice \(\theta_\star^{\rm (fid)} \approx 3.63\)
    used in the FRW histories of Sec.~\ref{subsec:frw_from_effective}
    sits naturally inside a broader stability window: small
    perturbations of $\theta_\star$ do not immediately destroy the
    effective $\Lambda$ behaviour.
  \item The upper edge of the band, where
    \(\Omega_\Lambda(\theta_\star) \approx 0.775\), provides a
    natural limit beyond which the microcavity--driven effective
    vacuum would overproduce dark energy relative to the fiducial
    cosmological target.
\end{enumerate}

From an ``axiom to atom'' perspective, this effective vacuum band is the
first concrete realization of the idea that a single microscopic twist
parameter $\theta_\star$ can generate both (i) a nontrivial spectrum of
vacuum energies in a controlled microscopic model and (ii) an
observationally viable range of macroscopic dark--energy densities,
once a single matching condition at a fiducial point is imposed.
In the next subsection we will use this band structure as the starting
point for connecting $\theta_\star$ to flavor mixing and, eventually,
to laboratory observables.

\begin{figure}[t]
    \centering
    \includegraphics[width=0.7\textwidth]{figures/effective_vacuum_band_scan}
    \caption{%
      Effective-vacuum prediction $\Omega_\Lambda(\theta_\star)$ as a function of
      $\theta_\star$ across the band $[2.18, 5.54]$\,rad. The solid curve shows
      the microcavity-backed $\Omega_\Lambda(\theta_\star)$ and the dashed line
      marks the target $\Omega_{\Lambda,\rm fid} = 0.70$.  The region where
      $|\Omega_\Lambda(\theta_\star) - 0.70| \lesssim 0.05$ provides an allowed
      band of $\theta_\star$ values consistent with late-time acceleration.
    }
    \label{fig:effective_vacuum_band_scan}
\end{figure}

% --- Act IV: microstructure upgrades and robustness
\section{Act IV: Microstructure upgrades and robustness of the microcavity bridge}
\label{sec:act4-microstructure}

In this Act we move beyond the minimal one-dimensional scalar microcavity used
in Act~III and explore two complementary directions:
(i) robustness of the existing $\Delta E(\theta_\star)$ bridge under variations
of the toy microcavity parameters, and (ii) upgraded microstructure models that
are still computationally cheap but closer in spirit to realistic vacuum
microstructure.

\subsection{Robustness of the 1D microcavity \texorpdfstring{$\Delta E(\theta_\star)$}{
% TODO: Describe the parameter sweep (mass contrast, cavity width, lattice size),
% and the main qualitative conclusions. Point to scripts and data products.% the summary statistics (position/value of 

\subsection{Upgraded microstructure toy models}

% TODO: Introduce at least one upgra% TODO: Introduce at least one upgra% TODO: Inr % TODO: Introduce at least one upgra% TODO: Isl% TODO: Introduce at least one upgra% TODO: Introduce at least one upgra% TODO: Itt% TODO: Introduce at least ones f% TODO: Introduce at least one upgra% TODO: Introduce at least one upgra% TODn % TODO: Introduce at least one upgra% TODO: Introdum the baseline cavity to the upgraded microstructure, and
% how this feeds into the FRW-effective-vacuum story.


%------------------------------------------------------------
% Discussion and outlook
%------------------------------------------------------------
\section{Discussion and outlook}
\label{sec:discussion}

We close by summarising the main structural ingredients of the Origin Axiom
programme and outlining concrete next steps. First, we have shown that a
single phase $\thetastar$ can be used as a bridge parameter between
flavour physics and toy models of vacuum microstructure. The $\thetastar$
posterior extracted from flavour fits is encoded in a compact configuration
file and subsequently consumed by independent scripts that implement vacuum
and FRW toy models. This explicit wiring is a small but important step
towards a reproducible ``end-to-end'' story, from flavour data to
cosmological observables, within a controlled toy setting.

Second, we have seen that the non-cancelling principle---implemented here
as a simple clipping rule on coarse-grained amplitudes---naturally
generates persistent vacuum-like contributions whose energy scale can be
matched to the observed cosmological constant by an explicit scaling
assumption. In the present work that scaling is imposed rather than
derived, and the microcavity model is deliberately minimal. Nonetheless,
the resulting effective-vacuum cosmology reproduces the qualitative
features of an accelerating universe, with an age and deceleration
parameter broadly compatible with standard $\Lambda$CDM.

Finally, the limitations of the present toy models suggest a clear roadmap
for future Acts. A more realistic implementation would require:
(i) deriving a non-cancelling constraint from an underlying field theory
or quantum-information-theoretic principle, (ii) constructing microcavity
or defect models tied to specific particle content and interactions,
(iii) integrating the flavour and vacuum sectors into a single dynamical
framework, and (iv) confronting the resulting effective cosmology with
precision data beyond the simple observables considered here. These steps
will be the focus of future work in the Origin Axiom programme.

\nocite{origin_axiom_inprep}
\bibliographystyle{unsrtnat}
\bibliography{origin_axiom}

\end{document}
