\section{Act VI: Observable \texorpdfstring{$\theta_\star$}{theta*} corridor and $\Lambda$CDM consistency}
\label{sec:act6-theta-star-observable-corridor}

In Act~V we constructed a minimal bridge from the flavour-informed master phase $\theta_\star$ to an effective cosmological constant in a flat FRW background. The key ingredients were a one-dimensional microcavity toy that produces a $\theta_\star$-dependent vacuum-energy shift $\Delta E(\theta_\star)$, and a single phenomenological scaling that promotes this shift into an effective dark-energy fraction $\Omega_\Lambda(\theta_\star)$. Once the scaling is fixed at a fiducial phase $\theta_{\star,\mathrm{fid}} \simeq 3.63~\mathrm{rad}$, drawn from the Act~II quark+lepton fits, each value of $\theta_\star$ inside the flavour prior band defines a complete matter-plus-vacuum FRW model. Act~VI asks a sharper question: within that band, which $\theta_\star$ values produce cosmologies that are broadly consistent with a standard $\Lambda$CDM-like universe, not only at the level of the background expansion, but also in terms of linear structure growth?

Throughout this act we treat the microcavity-backed FRW construction of Act~V as a fixed mapping
\begin{equation}
  \theta_\star
  \;\longrightarrow\;
  \bigl[\Omega_m(\theta_\star),\Omega_\Lambda(\theta_\star)\bigr]
  \;\longrightarrow\;
  \bigl\{t_0(\theta_\star),q_0(\theta_\star),d_L(z;\theta_\star),D(a;\theta_\star)\bigr\},
\end{equation}
and use it as an internal surrogate for a family of $\Lambda$CDM-like cosmologies. The goal is not to perform a precision fit to real data, but to identify a qualitatively acceptable ``observable corridor'' in $\theta_\star$ where background and growth observables sit within a few per cent of an internal $(\Omega_m,\Omega_\Lambda)=(0.3,0.7)$ reference model.

\subsection{R22: Background comparison to a $\Lambda$CDM backbone}
\label{subsec:r22-background-vs-lcdm}

R22 compares the effective-vacuum FRW band directly to a simple $\Lambda$CDM backbone. Operationally, we start from the precomputed scan
\begin{center}
  \texttt{data/processed/effective\_vacuum\_theta\_frw\_scan.npz},
\end{center}
which tabulates, for $N_\theta = 41$ samples across the Act~II prior band
\begin{equation}
  \theta_\star \in [2.18,5.54]~\mathrm{rad},
\end{equation}
the corresponding effective-vacuum parameters
\begin{equation}
  \bigl(\Omega_m(\theta_\star),\Omega_\Lambda(\theta_\star)\bigr),
\end{equation}
the age of the universe $t_0(\theta_\star)$ in gigayears, the deceleration parameter $q_0(\theta_\star)$, and a few toy luminosity distances $d_L(z;\theta_\star)/(c/H_0)$ at redshifts $z=\{0.3,0.5,1.0\}$, all evaluated for $H_0 = 70~\mathrm{km\,s^{-1}\,Mpc^{-1}}$.

As a reference we construct an internal $\Lambda$CDM backbone with fixed parameters
\begin{equation}
  (\Omega_m,\Omega_\Lambda)_{\mathrm{ref}} = (0.3,0.7)
\end{equation}
and integrate the corresponding FRW background to obtain
\begin{equation}
  t_{0,\mathrm{ref}},\qquad
  q_{0,\mathrm{ref}},\qquad
  d_{L,\mathrm{ref}}(z)/(c/H_0)\quad\text{for }z=\{0.3,0.5,1.0\}.
\end{equation}
For each $\theta_\star$ sample we then form simple fractional residuals
\begin{equation}
  \frac{\Delta t_0}{t_{0,\mathrm{ref}}}
  \;\equiv\;
  \frac{t_0(\theta_\star) - t_{0,\mathrm{ref}}}{t_{0,\mathrm{ref}}},
  \qquad
  \frac{\Delta d_L(z)}{d_{L,\mathrm{ref}}(z)}
  \;\equiv\;
  \frac{d_L(z;\theta_\star) - d_{L,\mathrm{ref}}(z)}{d_{L,\mathrm{ref}}(z)},
\end{equation}
as a coarse measure of how far each slice sits from the backbone.

Over the full $\theta_\star$ band the residuals span a wide range. The effective vacuum fraction varies between $\Omega_\Lambda(\theta_\star)\in[0,0.78]$, the age ranges from $t_0\simeq 9.3$ to $14.6~\mathrm{Gyr}$, and the deceleration parameter moves from $q_0\simeq 0.5$ (decelerating, matter-dominated) to $q_0\simeq -0.66$ (strong vacuum domination). The corresponding fractional age differences $|\Delta t_0|/t_{0,\mathrm{ref}}$ can reach $\sim 30\%$, and the distance differences at $z\sim 1$ can reach the $\sim 20$--$25\%$ level. This behaviour is expected: we are scanning across an entire band that includes both nearly matter-only and strongly vacuum-dominated slices.

\subsection{R23: Observable $\theta_\star$ corridor in background space}
\label{subsec:r23-observable-corridor-background}

In R23 we restrict attention to those $\theta_\star$ values whose effective-vacuum cosmologies look broadly compatible with late-time observations at the level of background expansion alone. We implement this by imposing three simple cuts on the FRW scan:
\begin{equation}
  0.60 \le \Omega_\Lambda(\theta_\star) \le 0.80,\qquad
  12.0~\mathrm{Gyr} \le t_0(\theta_\star) \le 15.0~\mathrm{Gyr},\qquad
  q_0(\theta_\star) < 0,
\end{equation}
corresponding to a vacuum fraction in a generous $\Lambda$CDM-like band, a plausible age for $H_0\simeq 70~\mathrm{km\,s^{-1}\,Mpc^{-1}}$, and a currently accelerating expansion.

Applied to the 41-point FRW scan, these cuts select 18 samples, defining an ``observable corridor''
\begin{equation}
  \theta_{\star,\mathrm{corridor}}
  \;\simeq\;
  [2.43,3.86]~\mathrm{rad},
\end{equation}
with associated ranges
\begin{equation}
  \Omega_\Lambda(\theta_\star)\in[0.61,0.78],\qquad
  t_0(\theta_\star)\in[12.5,14.6]~\mathrm{Gyr},\qquad
  q_0(\theta_\star)\in[-0.66,-0.41].
\end{equation}
Within this corridor, the fractional age residuals shrink to
\begin{equation}
  \left|\frac{\Delta t_0}{t_{0,\mathrm{ref}}}\right|
  \lesssim 8\%,
\end{equation}
and the luminosity-distance residuals at $z=\{0.3,0.5,1.0\}$ are at the level of a few per cent, typically $\lesssim 5\%$. A best-matching slice near the original flavour median, at $\theta_{\star,\mathrm{best}}\simeq 3.61~\mathrm{rad}$, yields
\begin{equation}
  (\Omega_m,\Omega_\Lambda) \simeq (0.292,0.708),\qquad
  t_0 \simeq 13.6~\mathrm{Gyr},\qquad
  q_0 \simeq -0.56,
\end{equation}
with distance residuals of order $0.2$--$0.5\%$ over the three redshifts sampled. At the level of smooth background observables, this slice is effectively indistinguishable from the internal $(0.3,0.7)$ backbone.

The full background comparison is summarised in Fig.~\ref{fig:theta-star-lcdm-background-corridor}. The top panels show $\Omega_\Lambda(\theta_\star)$, $t_0(\theta_\star)$, and $q_0(\theta_\star)$ across the Act~II band, with the observable corridor highlighted. The bottom panels show the corresponding fractional residuals in $t_0$ and $d_L(z)$ relative to the $\Lambda$CDM backbone. In this representation the observable corridor appears as a narrow tube around the canonical $(\Omega_m,\Omega_\Lambda)=(0.3,0.7)$ track in the space of age and low-redshift distances.

\begin{figure}[t]
  \centering
  \includegraphics[width=0.75\textwidth]{figures/theta_star_lcdm_background_corridor}
  \caption{%
    Observable $\theta_\star$ corridor in the microcavity-backed effective-vacuum model (R22--R23). Top panels: effective dark-energy fraction $\Omega_\Lambda(\theta_\star)$, cosmic age $t_0(\theta_\star)$, and deceleration parameter $q_0(\theta_\star)$ across the Act~II band, with the observable corridor defined by simple FRW cuts highlighted. Bottom panels: fractional residuals in $t_0$ and luminosity distances at $z=\{0.3,0.5,1.0\}$ relative to an internal $(\Omega_m,\Omega_\Lambda)=(0.3,0.7)$ $\Lambda$CDM backbone. Within the corridor, the effective-vacuum cosmologies differ from the backbone only at the few-per-cent level in these coarse background observables.}
  \label{fig:theta-star-lcdm-background-corridor}
\end{figure}

\subsection{R24: Linear growth and a $\sigma_8$-like consistency check}
\label{subsec:r24-linear-growth}

Background consistency is necessary but not sufficient: a viable $\Lambda$CDM-like model must also produce reasonable structure growth. In R24 we therefore compare the linear growth of matter perturbations in the effective-vacuum band to the same internal backbone.

We start from the growth scan
\begin{center}
  \texttt{data/processed/effective\_vacuum\_theta\_growth\_scan.npz},
\end{center}
which tabulates, for the same set of 41 $\theta_\star$ samples,
the linear growth factor $D(a;\theta_\star)$ evaluated at $a=1$ and normalised to the Einstein--de Sitter value $D_{\mathrm{EdS}}(a=1)$. This defines a dimensionless suppression factor
\begin{equation}
  D_{\mathrm{rel}}(a=1;\theta_\star)
  \;\equiv\;
  \frac{D(a=1;\theta_\star)}{D_{\mathrm{EdS}}(a=1)}.
\end{equation}
Across the full band we find
\begin{equation}
  D_{\mathrm{rel}}(a=1;\theta_\star) \in [0.73,1.00],
\end{equation}
interpolating smoothly between strongly vacuum-dominated and matter-dominated slices.

Restricting to the observable corridor $\theta_{\star,\mathrm{corridor}}\simeq[2.43,3.86]~\mathrm{rad}$ tightens this range to
\begin{equation}
  D_{\mathrm{rel}}(a=1;\theta_\star) \simeq 0.73\text{--}0.83,
\end{equation}
with a mean around $0.77$, i.e.\ a $\sim 20$--$25\%$ suppression of linear growth relative to an Einstein--de Sitter universe, in the ballpark of a canonical $\Lambda$CDM model with $\Omega_\Lambda \sim 0.7$.

To make contact with $\sigma_8$-style constraints, we again single out the best-matching slice $\theta_{\star,\mathrm{best}}$ identified in the background comparison. We then define a purely relative diagnostic
\begin{equation}
  R_{\sigma_8}(\theta_\star)
  \;\equiv\;
  \frac{D_{\mathrm{rel}}(a=1;\theta_\star)}{D_{\mathrm{rel}}(a=1;\theta_{\star,\mathrm{best}})},
\end{equation}
which captures how much the late-time linear growth amplitude at a given $\theta_\star$ differs from that of the internally ``normalised'' slice. Within the observable corridor we find
\begin{equation}
  R_{\sigma_8}(\theta_\star) \in [0.94,1.07],
\end{equation}
with an average very close to unity. In other words, once a single slice in the corridor is used to anchor the overall amplitude (e.g.\ by matching an observed $\sigma_8$), all other $\theta_\star$ values allowed by the background cuts produce late-time growth that differs by at most a few per cent.

\subsection{Act VI summary: a $\theta_\star$ corridor that looks like $\Lambda$CDM}
\label{subsec:act6-summary}

Act~VI layers a set of simple consistency tests on top of the effective-vacuum bridge constructed in Act~V. Starting from a flavour-informed prior band $\theta_\star \in [2.18,5.54]~\mathrm{rad}$ and a microcavity-backed mapping $\theta_\star \mapsto \Omega_\Lambda(\theta_\star)$ calibrated at a single fiducial point, we have:

\begin{itemize}
  \item Constructed a flat matter+vacuum FRW model for each $\theta_\star$ in the band and compared its age and luminosity distances to an internal $(\Omega_m,\Omega_\Lambda)=(0.3,0.7)$ $\Lambda$CDM backbone (R22).
  \item Defined an ``observable corridor'' of phases that yield vacuum fractions, ages, and deceleration parameters in a broad $\Lambda$CDM-like range, and shown that within this corridor the background observables deviate from the backbone only at the few-per-cent level (R23).
  \item Extended the comparison to linear structure growth, finding that the relative suppression of growth at $a=1$ is tightly clustered around a single reference slice, with a $\sigma_8$-like amplitude ratio $R_{\sigma_8}(\theta_\star)$ confined to a $\sim \pm 5\%$ band across the corridor (R24).
\end{itemize}

At this stage the construction remains a toy model: the microcavity is one-dimensional, the mapping from $\Delta E$ to $\Omega_\Lambda$ is linear and fixed by hand at one fiducial point, and we have not yet included realistic field content beyond a dust-like matter component. Nevertheless, the picture that emerges is structurally nontrivial: a single phase $\theta_\star$, originally introduced to organise flavour mixing, can be propagated through a non-cancelling microcavity toy into an effective vacuum component that admits an extended corridor of $\theta_\star$ values whose background expansion and linear growth are both qualitatively consistent with a standard $\Lambda$CDM cosmology.

This does not yet constitute a prediction of the cosmological constant. What it does provide is a concrete, end-to-end pipeline,
\begin{equation}
  \theta_\star \;\longrightarrow\; \Delta E(\theta_\star) \;\longrightarrow\; \Omega_\Lambda(\theta_\star) \;\longrightarrow\; \bigl\{a(t),D(a)\bigr\},
\end{equation}
on which more realistic microstructure models and data-driven constraints can be mounted. In subsequent acts we will turn to the next step in the ``universe $\to$ atom'' bridge: using the same non-cancelling machinery to link the $\theta_\star$-selected vacuum not only to FRW-scale observables, but also to particle-level structures that carry the same phase information.
