\section{Reproducibility notes for Phase~4}
\label{app:phase4-reproducibility}

This appendix records the concrete steps required to reproduce the
Phase~4 diagnostics and the main paper artefact from a fresh clone of
the repository. The emphasis is on (i) identifying the precise scripts
and tables that underpin the claims in the main text and (ii)
clarifying the logical flow from Phase~3 outputs to Phase~4 FRW-facing
summaries.

\subsection*{B.1. Prerequisites}

Phase~4 is designed to sit on top of the Phase~3 mechanism. The
following assumptions are therefore required before running the
Phase~4 scripts:
\begin{itemize}
  \item The repository is checked out at a Phase~4-compatible commit
    and a Python environment satisfying the documented dependencies
    (NumPy, SciPy, Matplotlib, etc.) is available.
  \item The Phase~3 baseline scan has been executed, producing the
    amplitude table and diagnostics JSON
    \texttt{phase3/outputs/tables/mech\_baseline\_scan\_diagnostics.json}
    as in the Phase~3 paper.
  \item The \texttt{phase3\_gate} script (or its Phase~3 equivalent)
    has been run to ensure that the baseline mechanism artefacts are
    present and consistent.
\end{itemize}
In a typical workflow one would run
\begin{verbatim}
  bash scripts/phase3_gate.sh --level A
\end{verbatim}
before entering the Phase~4 stage, so that all Phase~3 tables are
present and up to date.

\subsection*{B.2. Phase~4 mapping and diagnostics scripts}

Phase~4 introduces a single mapping family (F1) and layered
diagnostics built on top of it. The core scripts are:

\paragraph{(1) F1 scalar construction and basic diagnostics.}

\begin{itemize}
  \item \textbf{Script:}
    \texttt{phase4/src/phase4/run\_f1\_sanity.py}
  \item \textbf{Inputs:}
    Phase~3 diagnostics JSON
    \texttt{phase3/outputs/tables/mech\_baseline\_scan\_diagnostics.json}
    (for the amplitude range and quantile-based floor) and the
    corresponding amplitude table.
  \item \textbf{Outputs:}
    \begin{itemize}
      \item \texttt{phase4/outputs/tables/phase4\_F1\_sanity\_curve.csv},
        containing \(\theta\) and \(E_{\mathrm{vac}}(\theta)\) for the
        F1 mapping with \(\alpha = 1\), \(\beta = 4\);
      \item \texttt{phase4/outputs/tables/phase4\_F1\_sanity\_curve\_diagnostics.json},
        recording global extrema, basic moments, and configuration
        metadata.
    \end{itemize}
\end{itemize}

\paragraph{(2) F1 shape diagnostics and toy corridor.}

\begin{itemize}
  \item \textbf{Script:}
    \texttt{phase4/src/phase4/run\_f1\_shape\_diagnostics.py}
  \item \textbf{Inputs:}
    \texttt{phase4\_F1\_sanity\_curve.csv}.
  \item \textbf{Outputs:}
    \begin{itemize}
      \item \texttt{phase4/outputs/tables/phase4\_F1\_shape\_diagnostics.json},
        a JSON summary of the F1 scalar, including a toy
        one-\(\sigma\)-style corridor definition;
      \item \texttt{phase4/outputs/tables/phase4\_F1\_shape\_mask.csv},
        a per-\(\theta\) CSV with Boolean \texttt{in\_toy\_corridor}
        flags.
    \end{itemize}
\end{itemize}

\paragraph{(3) FRW-inspired toy diagnostics.}

\begin{itemize}
  \item \textbf{Script:}
    \texttt{phase4/src/phase4/run\_f1\_frw\_toy\_diagnostics.py}
  \item \textbf{Inputs:}
    \texttt{phase4\_F1\_sanity\_curve.csv} and the chosen toy
    parameters \(\Omega_m\), \(\Omega_r\), a target mean
    \(\langle\Omega_\Lambda\rangle\), and a scale-factor window
    \([a_{\min}, a_{\max}]\).
  \item \textbf{Outputs:}
    \begin{itemize}
      \item \texttt{phase4/outputs/tables/phase4\_F1\_frw\_toy\_diagnostics.json},
        which records global extrema and moments of
        \(E_{\mathrm{vac}}(\theta)\) and the toy
        \(\Omega_\Lambda(\theta)\), along with a simple FRW-sanity
        fraction;
      \item \texttt{phase4/outputs/tables/phase4\_F1\_frw\_toy\_mask.csv},
        a per-\(\theta\) mask with \(\Omega_\Lambda(\theta)\) and a
        Boolean FRW-sanity indicator. In the baseline configuration
        the FRW toy is treated as a non-binding diagnostic only.
    \end{itemize}
\end{itemize}

\paragraph{(4) FRW viability scan.}

\begin{itemize}
  \item \textbf{Script:}
    \texttt{phase4/src/phase4/run\_f1\_frw\_viability.py}
  \item \textbf{Inputs:}
    \texttt{phase4\_F1\_sanity\_curve.csv} and a choice of FRW
    parameters (\(\Omega_m\), \(\Omega_r\), \(H_0\)) as well as an age
    window \([t_{\min}, t_{\max}]\).
  \item \textbf{Outputs:}
    \begin{itemize}
      \item \texttt{phase4/outputs/tables/phase4\_F1\_frw\_viability\_diagnostics.json},
        recording basic statistics of the FRW histories across the
        \(\theta\)-grid, the fraction of points with a matter-dominated
        era, late-time acceleration, smooth \(H^2(a;\theta)\), and an
        age within the specified window;
      \item \texttt{phase4/outputs/tables/phase4\_F1\_frw\_viability\_mask.csv},
        a per-\(\theta\) mask with columns
        \texttt{theta}, \texttt{E\_vac}, \(\Omega_\Lambda(\theta)\),
        the inferred age in Gyr, Boolean flags
        (\texttt{has\_matter\_era}, \texttt{has\_late\_accel},
        \texttt{smooth\_H2}), and a combined \texttt{frw\_viable}
        flag.
    \end{itemize}
\end{itemize}

\paragraph{(5) Corridor extraction on the FRW-viable subset.}

\begin{itemize}
  \item \textbf{Script:}
    \texttt{phase4/src/phase4/run\_f1\_frw\_corridors.py}
  \item \textbf{Inputs:}
    \texttt{phase4\_F1\_frw\_viability\_mask.csv}.
  \item \textbf{Outputs:}
    \begin{itemize}
      \item \texttt{phase4/outputs/tables/phase4\_F1\_frw\_corridors.json},
        a JSON summary of contiguous ``corridors'' of FRW-viable
        \(\theta\)-values, including a principal corridor with its
        \(\theta\)-extent and basic statistics of
        \(\Omega_\Lambda(\theta)\);
      \item \texttt{phase4/outputs/tables/phase4\_F1\_frw\_corridors.csv},
        a per-corridor CSV listing the \(\theta\)-ranges and sizes.
    \end{itemize}
\end{itemize}

\paragraph{(6) \(\Lambda\)CDM-like probe.}

\begin{itemize}
  \item \textbf{Script:}
    \texttt{phase4/src/phase4/run\_f1\_frw\_lcdm\_probe.py}
  \item \textbf{Inputs:}
    \texttt{phase4\_F1\_frw\_viability\_mask.csv} and a broad target
    window for \(\Omega_\Lambda\) (e.g.\ \(0.7 \pm 0.1\)) and the age
    of the Universe (e.g.\ \(13.8 \pm 1.0\)~Gyr), with the same FRW
    parameters as the viability scan.
  \item \textbf{Outputs:}
    \begin{itemize}
      \item \texttt{phase4/outputs/tables/phase4\_F1\_frw\_lcdm\_probe.json},
        which records the fraction of FRW-viable grid points that
        fall into this broad \(\Lambda\)CDM-like window, together with
        the \(\theta\)-extent and the ranges of
        \(\Omega_\Lambda(\theta)\) and age across the selected subset;
      \item \texttt{phase4/outputs/tables/phase4\_F1\_frw\_lcdm\_probe\_mask.csv},
        a per-\(\theta\) mask augmenting the FRW-viability columns
        with a Boolean \texttt{lcdm\_like} flag.
    \end{itemize}
\end{itemize}
In the baseline configuration this probe is explicitly non-binding: it
is used to demonstrate the existence of \(\theta\)-regions with
broadly \(\Lambda\)CDM-like FRW histories, not to select a unique
\(\theta_\star\) or to fit observational data.

\paragraph{(7) Joint shape/FRW probe.}

\begin{itemize}
  \item \textbf{Script:}
    \texttt{phase4/src/phase4/run\_f1\_frw\_shape\_probe.py}
  \item \textbf{Inputs:}
    \texttt{phase4\_F1\_shape\_mask.csv},
    \texttt{phase4\_F1\_frw\_viability\_mask.csv}, and
    \texttt{phase4\_F1\_frw\_lcdm\_probe\_mask.csv}.
  \item \textbf{Outputs:}
    \begin{itemize}
      \item \texttt{phase4/outputs/tables/phase4\_F1\_frw\_shape\_probe.json},
        which reports the fractions and \(\theta\)-ranges associated
        with the toy shape corridor, the FRW-viable subset, the
        \(\Lambda\)CDM-like subset, and their intersections;
      \item \texttt{phase4/outputs/tables/phase4\_F1\_frw\_shape\_probe\_mask.csv},
        a joined mask that carries the shape, FRW-viability, and
        \(\Lambda\)CDM-like flags on a common \(\theta\)-grid.
    \end{itemize}
\end{itemize}
Figure~\ref{fig:phase4-shape-probe} in the main text is derived from
these joined tables.

\paragraph{(8) Data-facing probe (stub).}

\begin{itemize}
  \item \textbf{Script:}
    \texttt{phase4/src/phase4/run\_f1\_frw\_data\_probe.py}
  \item \textbf{Inputs:}
    \texttt{phase4\_F1\_frw\_viability\_mask.csv} and an optional
    external binned-distance data file
    \texttt{phase4/data/external/frw\_distance\_binned.csv}.
  \item \textbf{Outputs:}
    \begin{itemize}
      \item \texttt{phase4/outputs/tables/phase4\_F1\_frw\_data\_probe.json},
        a diagnostics file recording whether external data were found,
        the number of data points, and---if present---summary
        \(\chi^2\) statistics;
      \item \texttt{phase4/outputs/tables/phase4\_F1\_frw\_data\_probe\_mask.csv},
        a per-\(\theta\) mask with a Boolean \texttt{data\_ok} flag.
    \end{itemize}
\end{itemize}
In the baseline repository checkout no external FRW distance data are
bundled, so the script records \texttt{data\_available = false} and
sets \texttt{data\_ok = 0} for all grid points. This is an explicit,
structured negative result rather than an implicit limitation: it
demonstrates how data-level tests will be integrated in later work
without making any present data-driven claims.

\subsection*{B.3. Recommended execution flow}

Putting the pieces together, a minimal end-to-end reproduction of the
Phase~4 artefact proceeds as follows:
\begin{enumerate}
  \item Run the Phase~3 gate at an appropriate level, e.g.
\begin{verbatim}
  bash scripts/phase3_gate.sh --level A
\end{verbatim}
  to ensure the baseline mechanism scan and diagnostics are present.
  \item Execute the Phase~4 scripts in the logical order described
    above, from F1 sanity (\texttt{run\_f1\_sanity.py}) through shape
    diagnostics, FRW toy, FRW viability, FRW corridors, the
    \(\Lambda\)CDM-like probe, the joint shape/FRW probe, and finally
    the optional data-facing stub.
  \item Regenerate the Phase~4 paper and canonical artefact via
\begin{verbatim}
  bash scripts/phase4_gate.sh --level A
\end{verbatim}
    which runs \texttt{latexmk} on \texttt{phase4/paper/main.tex} and
    copies the resulting \texttt{main.pdf} to
    \texttt{phase4/outputs/paper/phase4\_paper.pdf} and
    \texttt{phase4/artifacts/origin-axiom-phase4.pdf}.
\end{enumerate}

The claims made in the main text and in the Phase~4 claims table are
deliberately modest: they refer only to the existence and behaviour of
these diagnostics and corridors \emph{within this pipeline}. No attempt
is made to extract precise cosmological parameter values or to promote
any diagnostic into a binding filter on the origin--axiom mechanism.
