\section*{Appendix B: Reproducibility notes (draft)}
\label{app:phase4_reproducibility}

At this rung, Phase~4 builds on the Phase~3 toy vacuum mechanism and
its non-cancellation floor, introduces the F1 mapping family, and
adds internal diagnostics and a toy, non-binding corridor. This
appendix summarises how to rebuild the Phase~4 artifacts used in the
paper.

\subsection*{Directory structure}

Phase~4 lives in the top-level directory \texttt{phase4/} with the
following relevant subdirectories:

\begin{itemize}
  \item \texttt{phase4/paper/}: LaTeX sources for the Phase~4 paper
        (including this appendix).
  \item \texttt{phase4/src/phase4/}: Python source for the F1 mapping
        family and diagnostics.
  \item \texttt{phase4/outputs/}:
        \begin{itemize}
          \item \texttt{outputs/paper/}: built Phase~4 PDF;
          \item \texttt{outputs/tables/}: CSV/JSON artifacts from F1
                diagnostics.
        \end{itemize}
  \item \texttt{phase4/artifacts/}: canonical Phase~4 PDF artifact
        exported by the gate script.
\end{itemize}

\subsection*{Upstream Phase 3 prerequisites}

Phase~4 assumes that the Phase~3 baseline scan and binding-certificate
artifacts have been generated at least once. In practice this can be
ensured by running, from the repository root:

\begin{verbatim}
bash scripts/phase3_gate.sh --level A
\end{verbatim}

and, if needed, explicitly running the Phase~3 scripts:

\begin{verbatim}
python phase3/src/phase3_mech/run_baseline_scan.py
python phase3/src/phase3_mech/run_binding_certificate.py
\end{verbatim}

These commands populate:

\begin{itemize}
  \item \texttt{phase3/outputs/tables/mech\_baseline\_scan.csv}
  \item \texttt{phase3/outputs/tables/mech\_baseline\_scan\_diagnostics.json}
  \item \texttt{phase3/outputs/tables/mech\_binding\_certificate.csv}
  \item \texttt{phase3/outputs/tables/mech\_binding\_certificate\_diagnostics.json}
\end{itemize}

The F1 mapping family reads
\texttt{mech\_baseline\_scan\_diagnostics.json} to reuse the
quantile-based floor and grid configuration.

\subsection*{Phase 4 F1 mapping and diagnostics}

With the Phase~3 prerequisites in place, the current Phase~4
diagnostics can be rebuilt via:

\begin{verbatim}
python phase4/src/phase4/run_f1_sanity.py
python phase4/src/phase4/run_f1_shape_diagnostics.py
\end{verbatim}

These scripts write:

\begin{itemize}
  \item \texttt{phase4/outputs/tables/phase4\_F1\_sanity\_curve.csv}
        (per-\(\theta\) values of \(E_{\mathrm{vac}}(\theta)\) and
        associated metadata);
  \item \texttt{phase4/outputs/tables/phase4\_F1\_shape\_diagnostics.json}
        (summary statistics and toy corridor definition);
  \item \texttt{phase4/outputs/tables/phase4\_F1\_shape\_mask.csv}
        (per-\(\theta\) Boolean mask indicating membership in the toy
        corridor).
\end{itemize}

At this rung these diagnostics are explicitly non-binding: they do not
define a canonical \(\theta\)-filter, but they are used in the Phase~4
paper to illustrate the behaviour of the F1 mapping.

\subsection*{Building the Phase 4 paper and artifact}

The Phase~4 paper and its canonical PDF artifact can be rebuilt from
the repository root via:

\begin{verbatim}
bash scripts/phase4_gate.sh --level A
\end{verbatim}

This gate script runs a Snakemake workflow in
\texttt{phase4/workflow/Snakefile} that invokes \texttt{latexmk} on
\texttt{phase4/paper/main.tex} and copies the resulting PDF to:

\begin{itemize}
  \item \texttt{phase4/outputs/paper/phase4\_paper.pdf}
  \item \texttt{phase4/artifacts/origin-axiom-phase4.pdf}
\end{itemize}

The gate currently covers the Phase~4 paper only; F1 diagnostics are
rebuilt by the explicit Python commands above.

\subsection*{Planned extensions}

The design note \texttt{phase4/FRW\_TOY\_DESIGN.md} specifies a
minimal FRW-inspired toy module that may be implemented in later
rungs. If such a module is added, this appendix will be extended to
include:

\begin{itemize}
  \item the relevant Python scripts and configuration files;
  \item any additional JSON/CSV outputs; and
  \item updated gate levels if FRW-like diagnostics become part of the
        canonical Phase~4 artifact.
\end{itemize}

Until then, reproducibility for Phase~4 consists of:

\begin{enumerate}
  \item regenerating the Phase~3 baseline and binding-certificate
        artifacts;
  \item regenerating the Phase~4 F1 diagnostics; and
  \item rebuilding the Phase~4 paper and artifact via the Level~A
        gate.
\end{enumerate}
