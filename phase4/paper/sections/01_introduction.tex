\section{Introduction}
\label{sec:phase4-introduction}

Phase~3 of the origin--axiom programme constructed a concrete lattice
mechanism for scanning a dimensionless parameter \(\theta\) and
produced a baseline diagnostic of the resulting amplitude landscape.
In particular, it delivered a reproducible pipeline which, given a
fixed mechanism configuration, yields a numerically stable
\emph{amplitude} \(A(\theta)\), a global floor, and a summary JSON
file with basic moments and quantiles of the scan. Phase~3 is
deliberately agnostic about cosmology: it stops at a well-controlled
\(\theta \mapsto A(\theta)\) map together with a mechanism-facing
diagnostic layer.

Phase~4 takes this Phase~3 output as a starting point and asks a
deliberately modest but physically oriented question: \emph{can we
systematically stress-test the Phase~3 landscape through simple,
Friedmann--Robertson--Walker (FRW) inspired summaries without
over-claiming physical predictions or selecting a unique
\(\theta_\star\)?} The guiding philosophy, inherited from Phase~0, is
to separate (i) the construction of reproducible mechanism- and
mapping-level objects from (ii) any stronger claims about the
structure of the real Universe.

At this rung we therefore introduce a single, explicit mapping family
%
\begin{equation}
  E_{\mathrm{vac}}(\theta)
  \;=\;
  \alpha\, A(\theta)^{\beta},
\end{equation}
%
denoted ``F1'', which turns the Phase~3 amplitude \(A(\theta)\) into a
non-negative scalar \(E_{\mathrm{vac}}(\theta)\) intended to play the
role of a toy vacuum-energy density. We adopt a simple baseline choice
\(\alpha = 1\), \(\beta = 4\) and treat the mapping as a tunable,
mechanism-facing object rather than as a physical law. On top of F1
we build a stack of FRW-inspired diagnostics: a toy FRW sanity check,
a more structured FRW ``viability'' scan, an extraction of contiguous
\(\theta\)-corridors, and a broad \(\Lambda\)CDM-like probe, together
with an explicit hook for future data-level tests.

Crucially, all of these layers are \emph{non-binding}. Phase~4 does
\emph{not} attempt to fit cosmological data, does \emph{not} claim a
preferred value of \(\theta\), and does \emph{not} promote any FRW
diagnostic into a hard filter on the origin--axiom mechanism. Instead,
the goal is to show, in a fully auditable way, that the Phase~3
landscape can be connected to simple cosmology-facing summaries and
that both positive and negative outcomes (including empty or narrow
corridors) are recorded and reasoned about explicitly. Subsequent
phases are free to refine the mapping family, the FRW summaries, and
the data interfaces, but they do so on top of the reproducible
infrastructure established here.

In the concrete baseline implementation reported in this paper,
Phase~4 \emph{does} instantiate a specific mapping family and a stack
of FRW-inspired diagnostics. The F1 mapping takes the Phase~3 scalar
as input and, via a simple normalisation, produces a toy
\(\Omega_\Lambda(\theta)\); the associated scripts implement
FRW-facing sanity, viability, corridor, and \(\Lambda\)CDM-like
probes as explicit, reproducible computations with per-grid masks and
JSON summaries. What this rung does \emph{not} yet provide is a
binding Phase~4 \(\theta\)-filter or any claim of a fit to
observational data. All FRW-facing results are deliberately presented
as non-binding worked examples and structured sanity checks, rather
than as evidence for a preferred \(\theta_\star\).
