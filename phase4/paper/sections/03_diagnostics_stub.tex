\section{Diagnostics and toy corridors (draft)}
\label{sec:phase4_diagnostics}

At this rung Phase~4 focuses on internal diagnostics of the scalar
\(E_{\mathrm{vac}}(\theta) = \alpha A(\theta)^{\beta}\) produced by
the F1 mapping family, together with simple, non-binding
\(\theta\)-corridors. No FRW module is yet implemented; instead we
prepare the ground for later FRW-like tests.

\subsection{Vacuum-curve sanity check}

The first diagnostic is a direct sanity check of the F1 mapping. Using
the Phase~3 baseline configuration and floor recorded in
\texttt{phase3/outputs/tables/mech\_baseline\_scan\_diagnostics.json},
we evaluate \(A(\theta)\) and \(E_{\mathrm{vac}}(\theta)\) on a
uniform grid of \(N_{\theta} = 2048\) points in \([0, 2\pi)\). The
script
\begin{center}
  \texttt{phase4/src/phase4/run\_f1\_sanity.py}
\end{center}
writes the per-grid values to
\begin{center}
  \texttt{phase4/outputs/tables/phase4\_F1\_sanity\_curve.csv},
\end{center}
together with mapping metadata and a summary of basic moments.

For the baseline configuration used here,
\(E_{\mathrm{vac}}(\theta)\) is strictly positive and remains on a
small, controlled scale, reflecting the scale of the underlying
amplitude and the chosen \((\alpha, \beta)\). This establishes that
the F1 mapping is at least numerically well behaved and correctly
wired to the Phase~3 mechanism.

\subsection{Toy shape diagnostics and non-binding corridor}

The second diagnostic probes the \emph{shape} of
\(E_{\mathrm{vac}}(\theta)\). The script
\begin{center}
  \texttt{phase4/src/phase4/run\_f1\_shape\_diagnostics.py}
\end{center}
rebuilds the same F1 curve and computes:

\begin{itemize}
  \item global extrema and moments \((E_{\mathrm{vac},\min},
        E_{\mathrm{vac},\max}, \text{mean}, \text{std})\);
  \item a toy, non-binding \(\theta\)-corridor defined by
        \[
          E_{\mathrm{vac}}(\theta) \le
          E_{\mathrm{vac},\min} + k_{\sigma} \sigma,
          \quad k_{\sigma} = 1;
        \]
  \item the fraction of the grid lying inside this corridor and the
        induced \(\theta\)-range.
\end{itemize}

The resulting summary is written to
\begin{center}
  \texttt{phase4/outputs/tables/phase4\_F1\_shape\_diagnostics.json},
\end{center}
while a per-\(\theta\) mask, indicating membership in the toy
corridor, is written to
\begin{center}
  \texttt{phase4/outputs/tables/phase4\_F1\_shape\_mask.csv}.
\end{center}

This corridor is explicitly labelled as \emph{exploratory and
non-binding}. It does not define a canonical \(\theta_{\star}\) or a
Phase~4 \(\theta\)-filter; it is only a structured way of selecting a
low-\(E_{\mathrm{vac}}\) region that later rungs can reuse when
designing FRW-like toy modules.

\subsection{FRW-like toy diagnostics (design only)}

To keep the Phase~4 narrative aligned with the Phase~0 contract, we
separate the internal diagnostics above from any FRW-like behaviour
tests. A separate design note
\begin{center}
  \texttt{phase4/FRW\_TOY\_DESIGN.md}
\end{center}
specifies a minimal FRW-inspired toy module in which the F1 scalar
acts as a driving term for a dimensionless scale factor
\(a(\tau)\) and Hubble-like quantity \(H(\tau)\).

At the present rung this module is \emph{not} implemented, and no
FRW-style diagnostics enter the claims table. The only purpose of the
design work is to:

\begin{itemize}
  \item define clear, auditable interfaces between
        \(E_{\mathrm{vac}}(\theta)\), the toy corridor mask, and FRW-like
        quantities; and
  \item constrain future work so that any FRW-like diagnostics remain
        simple, reproducible, and explicitly non-claiming unless
        promoted to a Phase~4 \(\theta\)-filter.
\end{itemize}

Subsequent rungs may instantiate this toy module in code or, if it
proves unhelpful, retire it in favour of alternative diagnostics. In
either case, the Phase~4 paper will distinguish binding
\(\theta\)-filters from non-binding exploratory diagnostics in line
with the Phase~0 corridor semantics.
