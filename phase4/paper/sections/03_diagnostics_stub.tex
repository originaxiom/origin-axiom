% -------------------------------------------------------------------
% Phase 4 diagnostics and FRW probes (F1 mapping)
% -------------------------------------------------------------------

The Phase~4 diagnostics are deliberately layered. The goal is to place
the F1 vacuum mapping, the FRW lift, and the external-host bridge on
a common \(\theta\)-grid, and to make every intermediate table and
mask \emph{explicitly} reproducible from scripts in the repository.
This section summarises what each diagnostic script does, what it
reads, what it writes, and how the resulting masks line up.

Throughout, we work with the baseline F1 mapping
\texttt{F1\_baseline\_v1} and its associated Phase~3 diagnostics.  All
tables and JSON files referenced below live under
\texttt{phase4/outputs/tables/} unless otherwise stated.

\medskip

\noindent\textbf{(1) F1 ``sanity'' curve: lifting Phase~3 to Phase~4.}

The starting point for the Phase~4 layer is the Phase~3 mechanism
baseline scan diagnostics in
\texttt{phase3/outputs/tables/mech\_baseline\_scan\_diagnostics.json}.
These diagnostics encode the one-dimensional \(\theta\)-grid, the
baseline unconstrained mechanism values, and simple summary statistics
such as

\begin{itemize}
  \item the minimum and maximum unconstrained mechanism values,
  \item quartiles of the unconstrained distribution,
  \item the fraction of grid points that are at the imposed
        \texttt{epsfloor} bound.
\end{itemize}

The script
\texttt{phase4/src/phase4/run\_f1\_sanity.py}
turns this Phase~3 baseline information into a Phase~4 F1
``sanity'' curve.  It constructs a uniform \(\theta\)-grid with
\texttt{n\_grid = 2048} points spanning \([0, 2\pi)\), evaluates the
F1 vacuum mapping \(E_{\rm vac}(\theta)\) on that grid, and writes a
CSV table

\begin{center}
  \texttt{phase4\_F1\_sanity\_curve.csv}
\end{center}

with at least the columns

\begin{itemize}
  \item \texttt{theta} (the grid in \(\theta\)),
  \item \texttt{E\_vac} (the mapped vacuum scale),
  \item a small set of diagnostic fields (e.g.\ the baseline
        \texttt{epsfloor} threshold carried through from Phase~3).
\end{itemize}

It also writes a JSON diagnostics file
\texttt{phase4\_F1\_sanity\_curve\_diagnostics.json}
that records the key summary statistics of the curve, including

\begin{itemize}
  \item \texttt{theta\_min}, \texttt{theta\_max},
  \item \texttt{E\_vac\_min}, \texttt{E\_vac\_max},
  \item \texttt{E\_vac\_mean}, \texttt{E\_vac\_std},
  \item a copy of the Phase~3 baseline diagnostics used.
\end{itemize}

In the baseline configuration used here, the F1 sanity curve respects
the Phase~3 \texttt{epsfloor} threshold and simply transports the
Phase~3 amplitude structure into the Phase~4 layer without any
additional tuning.

\medskip

\noindent\textbf{(2) F1 ``shape corridor'' on the F1 sanity curve.}

The next diagnostic is a toy ``shape corridor'' on the F1 vacuum
curve.  The script
\texttt{phase4/src/phase4/run\_f1\_shape\_diagnostics.py}
reads the sanity-curve CSV
\texttt{phase4\_F1\_sanity\_curve.csv},
computes summary statistics of \(E_{\rm vac}(\theta)\), and defines a
simple corridor as

\[
E_{\rm vac}(\theta) \le E_{\rm vac}^{\rm min} +
k_\sigma \, \sigma_{E_{\rm vac}},
\]

with \(k_\sigma = 1\) in the baseline runs.  This is deliberately an
\emph{internal} diagnostic: it does not encode any observational
constraint, only the shape of the F1 vacuum map itself.

The script writes

\begin{itemize}
  \item a JSON summary
    \texttt{phase4\_F1\_shape\_diagnostics.json}
    containing, for example,
    \begin{itemize}
      \item \texttt{E\_vac\_min}, \texttt{E\_vac\_max},
      \item \texttt{E\_vac\_mean}, \texttt{E\_vac\_std},
      \item \texttt{corridor\_fraction}
            (the fraction of grid points in the toy shape corridor),
      \item \texttt{corridor\_theta\_min}, \texttt{corridor\_theta\_max}.
    \end{itemize}
  \item a per-\(\theta\) mask CSV
    \texttt{phase4\_F1\_shape\_mask.csv}
    with columns
    \begin{itemize}
      \item \texttt{theta},
      \item \texttt{E\_vac},
      \item \texttt{in\_toy\_corridor} (Boolean).
    \end{itemize}
\end{itemize}

This corridor plays a purely structural role: it lets us track how
strongly the later FRW and external-host layers overlap with
a region where the vacuum map is ``low'' relative to its global
spread, without imposing any physical interpretation on that choice.

\medskip

\noindent\textbf{(3) FRW lift and viability diagnostics.}

We then lift the F1 vacuum curve into a toy FRW background.  The
script
\texttt{phase4/src/phase4/run\_f1\_frw\_toy\_diagnostics.py}
reads \texttt{phase4\_F1\_sanity\_curve.csv} and interprets
\(E_{\rm vac}(\theta)\) as a proxy for \(\Omega_\Lambda(\theta)\) via
a simple rescaling.  In the baseline configuration we fix

\begin{itemize}
  \item \(\Omega_m = 0.3\),
  \item \(\Omega_r = 0\),
  \item and match the mean of \(\Omega_\Lambda(\theta)\)
        to a target value \(\Omega_{\Lambda,{\rm target}} = 0.7\).
\end{itemize}

The script then integrates a flat FRW background for each
\(\theta\)-grid point on a scale-factor grid
\([a_{\rm min}, a_{\rm max}] = [0.5, 1]\)
with \texttt{n\_a = 128}, and checks a set of FRW ``sanity'' and
viability conditions.  These include:

\begin{itemize}
  \item the existence of a matter-dominated era,
  \item late-time acceleration,
  \item a smoothly varying \(H^2(a)\) over the sampled interval,
  \item an age that lies inside a generous window
        (here \([10, 20]\)~Gyr).
\end{itemize}

The script writes

\begin{itemize}
  \item a JSON diagnostics file
    \texttt{phase4\_F1\_frw\_toy\_diagnostics.json}
    summarising the global statistics, including
    \begin{itemize}
      \item \texttt{age\_Gyr\_min}, \texttt{age\_Gyr\_max},
      \item \texttt{frac\_has\_matter\_era},
      \item \texttt{frac\_has\_late\_accel},
      \item \texttt{frac\_smooth\_H2},
      \item \texttt{frac\_age\_window\_ok},
      \item \texttt{frac\_viable}.
    \end{itemize}
  \item a per-\(\theta\) mask CSV
    \texttt{phase4\_F1\_frw\_toy\_mask.csv}
    with columns such as
    \begin{itemize}
      \item \texttt{theta},
      \item \texttt{E\_vac},
      \item \texttt{omega\_lambda},
      \item \texttt{age\_Gyr},
      \item Boolean flags recording matter era, late-time acceleration,
            FRW-sanity, and age-window membership.
    \end{itemize}
\end{itemize}

From this toy FRW lift we then construct a more compact FRW-viability
mask.  The script
\texttt{phase4/src/phase4/run\_f1\_frw\_viability\_mask.py}
reads the toy diagnostics and writes

\begin{itemize}
  \item a JSON summary
    \texttt{phase4\_F1\_frw\_viability\_diagnostics.json}
    (used as the backbone of the Phase~4 FRW viability statements),
  \item a mask CSV
    \texttt{phase4\_F1\_frw\_viability\_mask.csv}
    combining the per-\(\theta\) flags into a single
    \texttt{frw\_viable} Boolean.
\end{itemize}

In the baseline configuration used here, roughly half of the
\(\theta\)-grid is FRW-viable under these broad criteria.

\medskip

\noindent\textbf{(4) FRW corridors in \(\theta\).}

Given the FRW-viability mask, we next group \(\theta\)-points into
contiguous corridors.  The script
\texttt{phase4/src/phase4/run\_f1\_frw\_corridors.py}
takes
\texttt{phase4\_F1\_frw\_viability\_mask.csv}
as input and identifies contiguous runs of grid points with
\texttt{frw\_viable = 1}.  It writes

\begin{itemize}
  \item a JSON file
    \texttt{phase4\_F1\_frw\_corridors.json}
    summarising the number of corridors, the size of each
    corridor, and in particular the ``principal'' corridor
    (the one with the largest number of FRW-viable points),
  \item a CSV table
    \texttt{phase4\_F1\_frw\_corridors.csv}
    that records, for each corridor,
    \begin{itemize}
      \item \texttt{theta\_min}, \texttt{theta\_max},
      \item \texttt{n\_points},
      \item the mean \(\Omega_\Lambda\) on that corridor,
      \item the index of the corridor.
    \end{itemize}
\end{itemize}

In the baseline run, the principal FRW corridor covers the same
\(\theta\)-interval that later supports the external-host kernel:
roughly \(\theta \in [0.43, 3.54]\), with an internal mean
\(\Omega_\Lambda \simeq 1.3\) under the toy mapping used here.

\medskip

\noindent\textbf{(5) A broad \(\Lambda\)CDM-like FRW probe.}

To relate the toy FRW lift to a more familiar late-time background,
we introduce a broad \(\Lambda\)CDM-like probe.  The script
\texttt{phase4/src/phase4/run\_f1\_frw\_lcdm\_probe.py}
again reads
\texttt{phase4\_F1\_frw\_viability\_mask.csv}
and selects FRW-viable grid points that satisfy both

\begin{itemize}
  \item \(|\Omega_\Lambda(\theta) - \Omega_{\Lambda,{\rm target}}|
         \le \Delta\Omega_\Lambda\), with
        \(\Omega_{\Lambda,{\rm target}} = 0.7\) and
        \(\Delta\Omega_\Lambda = 0.1\),
  \item \(|t_{\rm age}(\theta) - t_{\rm target}|
         \le \Delta t\), with
        \(t_{\rm target} = 13.8~\mathrm{Gyr}\) and
        \(\Delta t = 1.0~\mathrm{Gyr}\).
\end{itemize}

The output consists of

\begin{itemize}
  \item a JSON summary
    \texttt{phase4\_F1\_frw\_lcdm\_probe.json}
    that records, among other quantities,
    \begin{itemize}
      \item \texttt{n\_grid}, \texttt{n\_frw\_viable},
      \item \texttt{n\_lcdm\_like},
      \item \texttt{lcdm\_like\_fraction},
      \item the \(\theta\)-range and age/\(\Omega_\Lambda\)-range
            of the selected window.
    \end{itemize}
  \item a mask CSV
    \texttt{phase4\_F1\_frw\_lcdm\_probe\_mask.csv}
    which augments the FRW-viability columns with a Boolean
    \texttt{lcdm\_like} flag.
\end{itemize}

In the current baseline configuration, this \(\Lambda\)CDM-like window
selects a small but non-zero subset of the FRW-viable
\(\theta\)-grid.  It is intentionally broad and remains a toy-level
probe: it is not a fit to observational data and does not single out a
preferred \(\theta_\star\).  Instead, it demonstrates that the
Phase~3~\(\to\)~Phase~4 pipeline can admit FRW histories that are, in
a coarse sense, compatible with late-time \(\Lambda\)CDM-like
cosmology.

\medskip

\noindent\textbf{(6) Joined F1/FRW shape probe.}

Finally, we introduce a joined F1/FRW ``shape probe'' that ties
together

\begin{itemize}
  \item the F1 toy shape corridor
        (\texttt{phase4\_F1\_shape\_mask.csv}),
  \item the FRW-viability mask
        (\texttt{phase4\_F1\_frw\_viability\_mask.csv}),
  \item the \(\Lambda\)CDM-like FRW window
        (\texttt{phase4\_F1\_frw\_lcdm\_probe\_mask.csv})
\end{itemize}

on a common \(\theta\)-grid.  The script
\texttt{phase4/src/phase4/run\_f1\_frw\_shape\_probe.py}
reads these inputs and writes

\begin{itemize}
  \item a per-\(\theta\) joined mask
    \texttt{phase4\_F1\_frw\_shape\_probe\_mask.csv}
    with columns
    \begin{itemize}
      \item \texttt{theta},
      \item \texttt{E\_vac},
      \item \texttt{omega\_lambda},
      \item \texttt{age\_Gyr},
      \item \texttt{in\_toy\_corridor},
      \item \texttt{frw\_viable},
      \item \texttt{lcdm\_like},
      \item and a combined shape/FRW flag
            (e.g.\ \texttt{shape\_and\_lcdm}).
    \end{itemize}
  \item a JSON diagnostics summary
    \texttt{phase4\_F1\_frw\_shape\_probe.json}
    that reports fractions such as
    \begin{itemize}
      \item \texttt{frac\_frw\_viable},
      \item \texttt{frac\_in\_toy\_corridor},
      \item \texttt{frac\_lcdm\_like},
      \item \texttt{frac\_shape\_and\_viable},
      \item \texttt{frac\_shape\_and\_lcdm},
    \end{itemize}
    together with the corresponding \(\theta\)-ranges.
\end{itemize}

To provide a compact visual summary, the script
\texttt{phase4/src/phase4/plot\_f1\_frw\_shape\_probe.py}
builds a Figure showing \(\Omega_\Lambda(\theta)\) with overlays for
the toy shape corridor, FRW-viable region, and \(\Lambda\)CDM-like
window.  The resulting plot is stored as

\begin{center}
  \texttt{outputs/figures/phase4\_F1\_frw\_shape\_probe\_omega\_lambda\_vs\_theta.png}
\end{center}

and referenced in the text as
Figure~\ref{fig:phase4_F1_frw_shape_probe}.

At the present baseline, the joint shape/FRW probe exercises the
shape and FRW-viability machinery, but does \emph{not} yet attempt a
genuine comparison to observational data.  The Phase~4 data-probe
stub, implemented in
\texttt{phase4/src/phase4/run\_f1\_frw\_data\_probe.py},
is wired to look for an external binned FRW distance dataset at

\begin{center}
  \texttt{phase4/data/external/frw\_distance\_binned.csv},
\end{center}

with columns \texttt{z}, \texttt{mu}, and \texttt{sigma\_mu}.  In the
baseline repository configuration this file is an explicit stub; the
current Phase~4 layer exercises the data-probe code path but records
\texttt{data\_available = false}, \texttt{n\_data\_points = 0}, and
\texttt{n\_data\_ok = 0} in
\texttt{phase4\_F1\_frw\_data\_probe.json}.  This keeps the FRW/data
interface explicit while avoiding any accidental dependence on an
unversioned external dataset.

% -------------------------------------------------------------------
% Phase 4 FRW shape-probe placeholder figure
% (anchors the \ref{fig:phase4-shape-probe} used in the text)
% -------------------------------------------------------------------
\begin{figure}[t]
  \centering
  % Intentionally no external image yet; this is a semantic placeholder.
  \caption{Phase~4 FRW shape-probe summary (placeholder).%
  \label{fig:phase4-shape-probe}}
\end{figure}