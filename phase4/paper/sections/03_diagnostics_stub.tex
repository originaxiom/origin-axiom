\section{Diagnostics and toy corridors}
\label{sec:diagnostics}

Given a mapping family from the Phase~3 global amplitude to a scalar
observable, Phase~4 must provide diagnostics that are:

\begin{itemize}
  \item simple enough to be implemented and audited end-to-end;
  \item honest about their physical status (toy vs.~realistic); and
  \item compatible with the Phase~0 corridor / filter semantics.
\end{itemize}

At the present rung we focus on the F1 family, which maps the
floor-enforced global amplitude \(A(\theta)\) from Phase~3 into a
toy vacuum-energy-like scalar
\begin{equation}
  E_{\mathrm{vac}}(\theta)
  \;=\;
  \alpha\, A(\theta)^{\beta},
\end{equation}
with \(\alpha = 1\) and \(\beta = 4\) in the baseline configuration.
We reuse the Phase~3 baseline diagnostics (including the quantile-based
non-cancellation floor) and evaluate \(E_{\mathrm{vac}}(\theta)\) on a
uniform grid of \(N_{\theta} = 2048\) points in \([0, 2\pi)\).

The per-grid values and a simple shape analysis are generated by the
script
\begin{center}
  \texttt{phase4/src/phase4/run\_f1\_shape\_diagnostics.py},
\end{center}
which writes:

\begin{itemize}
  \item a summary diagnostics file
        \texttt{phase4/outputs/tables/phase4\_F1\_shape\_diagnostics.json};
  \item a per-theta mask
        \texttt{phase4/outputs/tables/phase4\_F1\_shape\_mask.csv}.
\end{itemize}

The diagnostics include:

\begin{itemize}
  \item the global minimum and maximum of \(E_{\mathrm{vac}}(\theta)\);
  \item the mean and standard deviation over the grid;
  \item a \emph{toy, non-binding corridor} defined by
        \begin{equation}
          E_{\mathrm{vac}}(\theta)
          \;\le\;
          E_{\mathrm{vac},\min}
          \;+\;
          \sigma_{E},
        \end{equation}
        where \(E_{\mathrm{vac},\min}\) is the global minimum and
        \(\sigma_{E}\) is the standard deviation;
  \item the fraction of grid points inside this toy corridor; and
  \item the corresponding \(\theta\)-interval spanned by the corridor.
\end{itemize}

This construction is explicitly labelled as a \emph{non-binding}
diagnostic: it is not a claim that the resulting interval defines a
physically meaningful \(\theta\)-corridor, let alone a canonical
\(\theta_{\star}\). Its purpose is to:

\begin{itemize}
  \item demonstrate that the F1 mapping yields a numerically sensible
        \(\theta\)-dependence;
  \item provide a reproducible starting point for more structured
        corridor definitions in later rungs; and
  \item illustrate how shape-based criteria on
        \(E_{\mathrm{vac}}(\theta)\) can be turned into filters
        compatible with the Phase~0 ledger semantics, once the
        physical interpretation is better understood.
\end{itemize}

Later rungs will either refine these diagnostics into more physically
motivated corridor conditions, or record structured negative results
if no robust, non-pathological corridors emerge from the tested
mapping families.
