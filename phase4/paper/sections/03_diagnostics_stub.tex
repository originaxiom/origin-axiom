
% ------------------------------------------------------------------
% Phase 4 diagnostics: vacuum curve, corridors, FRW checks, and probes
% ------------------------------------------------------------------

\section{Phase 4 diagnostics: vacuum map, corridors, and FRW-facing probes}
\label{sec:diagnostics}

Given the F1 mapping family defined in Section~\ref{sec:mappings}, the Phase~4 diagnostics are designed to answer a sequence of concrete questions: (i) Is the F1 vacuum curve numerically well behaved on the full $\theta$ grid and consistent with the Phase~3 baseline diagnostics? (ii) Does the vacuum curve admit a simple ``shape corridor'' that can be used as a toy, non-binding reference band in $\theta$? (iii) When the curve is lifted into FRW backgrounds, for what fraction of the grid do we obtain sensible, late-accelerating cosmologies? (iv) Within that FRW-viable set, is there a window of $\theta$ where the background looks $\Lambda$CDM-like in a coarse sense? (v) How does that window overlap Stage~2 external hosts and external data-facing probes? This section records how those questions are implemented in the repository and which tables and masks connect the mapping layer to the rest of the program.

\subsection{Baseline diagnostics and F1 sanity sweep}

The starting point is the Phase~3 baseline diagnostics JSON \texttt{phase3/outputs/tables/mech\_baseline\_scan\_diagnostics.json}, which summarizes the unconstrained amplitude distribution and the numerical floor $\epsilon_{\mathrm{floor}}$ used in Equation~\eqref{eq:phase4-f1-definition}. The Phase~4 script \texttt{phase4/src/phase4/run\_f1\_sanity.py} reads that JSON, constructs the F1 vacuum curve on the Phase~3 grid, and writes: (i) the per--$\theta$ CSV \texttt{phase4\_F1\_sanity\_curve.csv}; and (ii) the JSON diagnostics \texttt{phase4\_F1\_sanity\_curve\_diagnostics.json}. The JSON records, among other quantities, the minimum, maximum, mean, and standard deviation of $E_{\mathrm{vac}}^{\mathrm{F1}}(\theta)$, the grid length $N$, and the imported Phase~3 floor and unconstrained range. These numbers are used throughout the Phase~4 text whenever we summarize the raw F1 vacuum behaviour.

\subsection{Toy shape corridor on the F1 vacuum curve}

As a first, deliberately non-binding diagnostic, Phase~4 defines a toy ``shape corridor'' on the F1 vacuum curve. The goal is to mark a simple band in $\theta$ where the vacuum energy sits close to its global minimum, without claiming any direct physical interpretation. The script \texttt{phase4/src/phase4/run\_f1\_shape\_diagnostics.py} consumes \texttt{phase4\_F1\_sanity\_curve.csv}, computes the global minimum $E_{\mathrm{vac,min}}$ and standard deviation $\sigma$ of the curve, and defines a toy corridor by $E_{\mathrm{vac}}^{\mathrm{F1}}(\theta) \le E_{\mathrm{vac,min}} + k_{\sigma}\,\sigma$, with a fixed $k_{\sigma}$ specified in the script (currently $k_{\sigma}=1$). The script writes: (i) \texttt{phase4\_F1\_shape\_diagnostics.json}, summarizing the global range of $E_{\mathrm{vac}}^{\mathrm{F1}}$ and the fraction of grid points that satisfy the toy corridor condition; and (ii) \texttt{phase4\_F1\_shape\_mask.csv}, a per--$\theta$ CSV that marks which grid points lie inside the corridor. The paper only uses this corridor as a visual and conceptual aid when discussing the shape of the F1 vacuum; it is not used as a hard constraint in any FRW or external-host diagnostic.

\subsection{Toy FRW backgrounds and viability mask}

The next diagnostic lifts the F1 vacuum curve into FRW backgrounds by treating $E_{\mathrm{vac}}^{\mathrm{F1}}(\theta)$ as the source of an effective $\Omega_{\Lambda}(\theta)$ on a fixed matter--radiation background. The script \texttt{phase4/src/phase4/run\_f1\_frw\_toy\_diagnostics.py} reads \texttt{phase4\_F1\_sanity\_curve.csv}, fixes $(\Omega_m,\Omega_r)$ and $H_0$ to a simple flat-$\Lambda$CDM baseline inspired by Planck Collaboration 2018 constraints~\cite{Planck2018}, and constructs $H^2(a;\theta)$ on a grid in the scale factor $a$. It then applies a set of purely internal sanity checks: the presence of a radiation/matter era, a late-accelerating phase, and the absence of obvious numerical pathologies in $H^2(a)$. The outputs are: (i) \texttt{phase4\_F1\_frw\_toy\_diagnostics.json}, which reports the fraction of $\theta$ values that pass the FRW sanity checks and the corresponding age range for the toy Universe; and (ii) \texttt{phase4\_F1\_frw\_toy\_mask.csv}, a per--$\theta$ CSV that marks FRW-viable grid points. From this point on, all FRW-facing diagnostics in Phase~4 are restricted to the FRW-viable subset of the grid.

\subsection{FRW corridors and principal band in $\theta$}

Given the FRW-viable mask, the script \texttt{phase4/src/phase4/run\_f1\_frw\_corridors.py} identifies connected FRW-viable corridors in $\theta$ and highlights a principal band that will be used as the main Phase~4 FRW window. It reads the per--$\theta$ FRW viability information, groups consecutive viable grid points into intervals, and records: (i) \texttt{phase4\_F1\_frw\_corridors.json}, which summarizes the fraction of the grid that is FRW-viable, the number of corridors, and the $\theta$ span of the principal band; and (ii) \texttt{phase4\_F1\_frw\_corridors.csv}, which lists the individual corridor intervals and their basic properties. The principal corridor is the object that is later compared to Stage~2 host kernels and external data-facing windows.

\subsection{Toy FRW viability and $\Lambda$CDM-like probe}

To connect the F1 FRW backgrounds to more familiar cosmologies, Phase~4 introduces a coarse $\Lambda$CDM-like probe. The script \texttt{phase4/src/phase4/run\_f1\_frw\_lcdm\_probe.py} takes the FRW-viable mask as input, fixes target values for $(\Omega_m,\Omega_{\Lambda},H_0)$ and a broad target age window, and asks: for which FRW-viable $\theta$ does the background age and $\Omega_{\Lambda}(\theta)$ fall within those tolerances? The corresponding outputs are: (i) \texttt{phase4\_F1\_frw\_lcdm\_probe.json}, recording the fraction of FRW-viable grid points that are deemed $\Lambda$CDM-like, together with the $\theta$ and age ranges of that set; and (ii) \texttt{phase4\_F1\_frw\_lcdm\_probe\_mask.csv}, a per--$\theta$ CSV marking which FRW-viable points also fall into the $\Lambda$CDM-like window. This probe is intentionally coarse: it is a structured diagnostic, not a fit to any particular dataset. Its role is to identify a window in $\theta$ where the toy FRW background is compatible with a broad, textbook $\Lambda$CDM picture.

\subsection{External data-facing probe}

Finally, Phase~4 provides a hook for direct comparison with an external FRW distance dataset. The script \texttt{phase4/src/phase4/run\_f1\_frw\_data\_probe.py} is wired to read a binned FRW distance table from \texttt{phase4/data/external/frw\_distance\_binned.csv}, compute a simple $\chi^2$ misfit for each FRW-viable $\theta$, and mark grid points whose $\chi^2$ per degree of freedom lies below a configurable threshold. In the current baseline configuration, the data file is a stub and no $\theta$ passes the data-quality mask, but the numerical plumbing and output formats are in place. The outputs are: (i) \texttt{phase4\_F1\_frw\_data\_probe.json}, which reports whether any usable data were found, the number of FRW-viable grid points tested, and the fraction that pass the $\chi^2$ threshold (zero in the present stub); and (ii) \texttt{phase4\_F1\_frw\_data\_probe\_mask.csv}, a per--$\theta$ CSV marking FRW-viable points that are also data-consistent under the chosen threshold. Once a real external dataset is wired in, this probe becomes the natural entry point for Phase~4 to talk directly to observations.

\subsection{Shape-probe summary figure (placeholder)}

For readability, the paper includes a single shape-probe figure that is meant to summarize the overlap of three key masks: FRW viability, the toy shape corridor, and the $\Lambda$CDM-like window. The underlying plot is generated by a Python helper under \texttt{phase4/src/phase4/} that consumes the masks described above and writes per--$\theta$ summary tables; the figure itself is currently represented in the source as a semantic placeholder:
\begin{figure}[t]
  \centering
  % Intentionally no external image yet; this is a semantic placeholder.
  \caption{Phase~4 F1 FRW shape-probe summary (placeholder).}
  \label{fig:phase4_F1_frw_shape_probe}
\end{figure}
The label \texttt{fig:phase4\_F1\_frw\_shape\_probe} is used elsewhere in the Phase~4 text to refer to this summary. Replacing the placeholder with the actual plot only requires regenerating the figure and updating the local \texttt{\textbackslash includegraphics} call, without touching any of the numerical diagnostics described above.