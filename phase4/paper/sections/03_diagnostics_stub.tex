\section{Diagnostics and toy corridors}
\label{sec:phase4_diagnostics}

Phase~4 currently provides three layers of diagnostics built on the
Phase~3 vacuum mechanism and the F1 mapping family. All of them are
explicitly non-binding: they are used for internal checks and
illustration, not as canonical \(\theta\)-filters.

\subsection{F1 sanity curve}

The first layer is a direct sanity check of the F1 mapping. Using the
Phase~3 baseline configuration and the quantile-based floor recorded
in \texttt{phase3/outputs/tables/mech\_baseline\_scan\_diagnostics.json},
we evaluate the floor-enforced amplitude \(A(\theta)\) on a uniform
grid of \(N_\theta = 2048\) points in \([0, 2\pi)\) and construct a
toy vacuum-energy-like scalar
\[
  E_{\mathrm{vac}}(\theta) = \alpha A(\theta)^{\beta},
\]
with \(\alpha = 1\) and \(\beta = 4\) at this rung. The script
\texttt{phase4/src/phase4/run\_f1\_sanity.py} writes the per-grid
values and metadata to
\begin{center}
  \texttt{phase4/outputs/tables/phase4\_F1\_sanity\_curve.csv},
\end{center}
which establishes that \(E_{\mathrm{vac}}(\theta)\) is strictly
positive, numerically well-behaved, and tied cleanly to the Phase~3
baseline diagnostics.

\subsection{F1 shape diagnostics and toy corridor}

The second layer probes the \emph{shape} of \(E_{\mathrm{vac}}(\theta)\)
by defining a toy, non-binding corridor in the space of scalar values.
The script
\texttt{phase4/src/phase4/run\_f1\_shape\_diagnostics.py}
constructs a mask based on
\[
  E_{\mathrm{vac}}(\theta) \le
  E_{\mathrm{vac}, \min} + k_\sigma \sigma,
  \quad k_\sigma = 1,
\]
where \(E_{\mathrm{vac}, \min}\) and \(\sigma\) are the global minimum
and standard deviation of the F1 curve. This is intentionally weak: it
retains a substantial fraction of the \(\theta\)-grid while discarding
the largest excursions.

The script writes a summary diagnostics file
\begin{center}
  \texttt{phase4/outputs/tables/phase4\_F1\_shape\_diagnostics.json}
\end{center}
and a per-grid mask
\begin{center}
  \texttt{phase4/outputs/tables/phase4\_F1\_shape\_mask.csv},
\end{center}
which later rungs and toy modules can reuse. At this stage the mask is
explicitly described as a \emph{toy corridor}: it does not define a
canonical \(\theta_\star\) and is not elevated to a Phase~4 filter.

\subsection{FRW-inspired toy diagnostics}

The third layer introduces a minimal FRW-inspired toy diagnostic that
treats \(E_{\mathrm{vac}}(\theta)\) as a source for a
\emph{dimensionless} vacuum term in a simple Hubble-like quantity.
From the sanity curve we define
\[
  \tilde{E}_{\mathrm{vac}}(\theta) =
  \frac{E_{\mathrm{vac}}(\theta)}{\langle E_{\mathrm{vac}} \rangle}
\]
and rescale it to a toy \(\Omega_\Lambda(\theta)\) with mean
\(\approx 0.7\). Together with fixed, dimensionless matter and
radiation parameters \(\Omega_m = 0.3\) and \(\Omega_r = 0\), we
evaluate
\[
  H^2(a; \theta) =
  \Omega_r a^{-4} + \Omega_m a^{-3} + \Omega_\Lambda(\theta)
\]
on a scale-factor grid \(a \in [0.1, 1]\). No physical units are
assigned and no claim is made that these numbers match observations.

The script
\texttt{phase4/src/phase4/run\_f1\_frw\_toy\_diagnostics.py}
writes:

\begin{itemize}
  \item a JSON summary
    \texttt{phase4/outputs/tables/phase4\_F1\_frw\_toy\_diagnostics.json},
    recording global extrema and moments of \(E_{\mathrm{vac}}\),
    \(\Omega_\Lambda(\theta)\), the chosen \(\Omega\)-parameters, and
    a simple FRW-sanity fraction; and
  \item a per-grid mask
    \texttt{phase4/outputs/tables/phase4\_F1\_frw\_toy\_mask.csv},
    containing \(\theta\), \(\Omega_\Lambda(\theta)\), and a Boolean
    indicator of whether \(H^2(a; \theta)\) stays positive and within
    a bounded variation factor over the scale-factor grid.
\end{itemize}

In the current baseline configuration (with
\(\Omega_m = 0.3\), \(\Omega_r = 0\),
\(\langle \Omega_\Lambda \rangle \approx 0.7\), and a variation bound
of order \(10\) on the ratio of maximal to minimal \(H^2\) across the
scale-factor grid), the FRW-sanity mask happens to be empty:
the corresponding diagnostics report a very small FRW-sanity fraction
(\(\texttt{frac\_sane} \approx 0\)). This is recorded as a
\emph{toy-level negative result} for this particular normalisation and
sanity criterion. It is not used to define a Phase~4 filter and does
not by itself constrain \(\theta\); its role is purely to illustrate
how the F1 scalar can feed into FRW-like checks and to demonstrate how
Phase~4 handles empty-corridor outcomes in a structured way.
