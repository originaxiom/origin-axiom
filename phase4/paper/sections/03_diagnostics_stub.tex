\section{Diagnostics and toy corridors}
\label{sec:phase4_diagnostics}

Phase~4 currently provides three layers of diagnostics built on the
Phase~3 vacuum mechanism and the F1 mapping family. All of them are
explicitly non-binding: they are used for internal checks and
illustration, not as canonical \(\theta\)-filters.

\subsection{F1 sanity curve}

The first layer is a direct sanity check of the F1 mapping. Using the
Phase~3 baseline configuration and the quantile-based floor recorded
in \texttt{phase3/outputs/tables/mech\_baseline\_scan\_diagnostics.json},
we evaluate the floor-enforced amplitude \(A(\theta)\) on a uniform
grid of \(N_\theta = 2048\) points in \([0, 2\pi)\) and construct a
toy vacuum-energy-like scalar
\[
  E_{\mathrm{vac}}(\theta) = \alpha A(\theta)^{\beta},
\]
with \(\alpha = 1\) and \(\beta = 4\) at this rung. The script
\texttt{phase4/src/phase4/run\_f1\_sanity.py} writes the per-grid
values and metadata to
\begin{center}
  \texttt{phase4/outputs/tables/phase4\_F1\_sanity\_curve.csv},
\end{center}
which establishes that \(E_{\mathrm{vac}}(\theta)\) is strictly
positive, numerically well-behaved, and tied cleanly to the Phase~3
baseline diagnostics.

\subsection{F1 shape diagnostics and toy corridor}

The second layer probes the \emph{shape} of \(E_{\mathrm{vac}}(\theta)\)
by defining a toy, non-binding corridor in the space of scalar values.
The script
\texttt{phase4/src/phase4/run\_f1\_shape\_diagnostics.py}
constructs a mask based on
\[
  E_{\mathrm{vac}}(\theta) \le
  E_{\mathrm{vac}, \min} + k_\sigma \sigma,
  \quad k_\sigma = 1,
\]
where \(E_{\mathrm{vac}, \min}\) and \(\sigma\) are the global minimum
and standard deviation of the F1 curve. This is intentionally weak: it
retains a substantial fraction of the \(\theta\)-grid while discarding
the largest excursions.

The script writes a summary diagnostics file
\begin{center}
  \texttt{phase4/outputs/tables/phase4\_F1\_shape\_diagnostics.json}
\end{center}
and a per-grid mask
\begin{center}
  \texttt{phase4/outputs/tables/phase4\_F1\_shape\_mask.csv},
\end{center}
which later rungs and toy modules can reuse. At this stage the mask is
explicitly described as a \emph{toy corridor}: it does not define a
canonical \(\theta_\star\) and is not elevated to a Phase~4 filter.

\subsection{FRW-inspired toy diagnostics}

The third layer introduces a minimal FRW-inspired toy diagnostic that
treats \(E_{\mathrm{vac}}(\theta)\) as a source for a
\emph{dimensionless} vacuum term in a simple Hubble-like quantity.
From the sanity curve we define
\[
  \tilde{E}_{\mathrm{vac}}(\theta) =
  \frac{E_{\mathrm{vac}}(\theta)}{\langle E_{\mathrm{vac}} \rangle}
\]
and rescale it to a toy \(\Omega_\Lambda(\theta)\) with mean
\(\approx 0.7\). Together with fixed, dimensionless matter and
radiation parameters \(\Omega_m = 0.3\) and \(\Omega_r = 0\), we
evaluate
\[
  H^2(a; \theta) =
  \Omega_r a^{-4} + \Omega_m a^{-3} + \Omega_\Lambda(\theta)
\]
on a late-time scale-factor grid \(a \in [0.5, 1]\). No physical units
are assigned and no claim is made that these numbers match
observations.

The script
\texttt{phase4/src/phase4/run\_f1\_frw\_toy\_diagnostics.py}
writes:

\begin{itemize}
  \item a JSON summary
    \texttt{phase4/outputs/tables/phase4\_F1\_frw\_toy\_diagnostics.json},
    recording global extrema and moments of \(E_{\mathrm{vac}}\),
    \(\Omega_\Lambda(\theta)\), the chosen \(\Omega\)-parameters, and
    a simple FRW-sanity fraction; and
  \item a per-grid mask
    \texttt{phase4/outputs/tables/phase4\_F1\_frw\_toy\_mask.csv},
    containing \(\theta\), \(\Omega_\Lambda(\theta)\), and a Boolean
    indicator of whether \(H^2(a; \theta)\) stays positive and within
    a bounded variation factor over the scale-factor grid.
\end{itemize}

Depending on the chosen toy parameters and sanity criterion, the
resulting FRW-sanity fraction can be small or even zero; such
empty-corridor outcomes are treated as informative negative results,
recorded in the diagnostics and logs rather than hidden. In all cases
the FRW-inspired module remains strictly non-binding: it does not fix
a preferred \(\theta_\star\) and does not, by itself, define a Phase~4
\(\theta\)-filter.

\subsection*{FRW-inspired viability mask on F1}

Beyond the FRW-inspired toy sanity check described above, we also
define a simple FRW \emph{viability} diagnostic on the F1 mapping.
Here we keep the same background ansatz
\[
  H^2(a; \theta) = \Omega_r a^{-4} + \Omega_m a^{-3} +
  \Omega_\Lambda(\theta),
\]
with fixed \(\Omega_m = 0.3\), \(\Omega_r = 0\), and a Hubble
parameter \(H_0 = 70~\mathrm{km\,s^{-1}\,Mpc^{-1}}\). For every
\(\theta\) on the F1 grid we:

\begin{itemize}
  \item rescale the F1 scalar \(E_{\mathrm{vac}}(\theta)\) into a
    toy \(\Omega_\Lambda(\theta)\) whose mean matches a target
    \(\langle \Omega_\Lambda \rangle \approx 0.7\);
  \item integrate a standard FRW age integral \(t_0(\theta)\) over a
    scale-factor grid \(a \in [a_{\min}, a_{\max}]\) and require the
    resulting cosmic age to lie in a coarse window
    \(t_0 \in [10, 20]~\mathrm{Gyr}\);
  \item check for a clear matter-dominated epoch and a late-time
    accelerating regime in which the deceleration parameter becomes
    negative; and
  \item enforce a smoothness criterion on \(H^2(a; \theta)\) by
    bounding the ratio of its maximal to minimal value over the
    grid.
\end{itemize}

The script
\texttt{phase4/src/phase4/run\_f1\_frw\_viability.py} evaluates these
conditions over the full F1 grid, writing
\begin{itemize}
  \item a diagnostic summary
    \texttt{phase4/outputs/tables/phase4\_F1\_frw\_viability\_diagnostics.json},
    including the age window, global moments of \(t_0(\theta)\), and
    the fractions of \(\theta\)-points that satisfy each FRW condition;
    and
  \item a per-grid mask
    \texttt{phase4/outputs/tables/phase4\_F1\_frw\_viability\_mask.csv},
    with columns \(\theta\), \(\Omega_\Lambda(\theta)\), and a Boolean
    viability indicator.
\end{itemize}

In the current baseline configuration we find that essentially all
\(\theta\)-points satisfy the age window, matter-era, and smoothness
requirements, while a non-trivial subset (roughly half of the F1
grid) additionally exhibits late-time acceleration. The resulting
viability mask therefore singles out a first, explicitly FRW-flavoured
subset of \(\theta\)-values as \emph{toy-viable} cosmological
histories. This mask is not yet promoted to a canonical
Phase~4 \(\theta\)-filter, but it plays a key role as a concrete,
physics-facing bridge between the Phase~3 mechanism and simple
FRW-like observables.
