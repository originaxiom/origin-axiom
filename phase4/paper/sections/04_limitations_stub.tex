\section{Limitations and scope of Phase~4}
\label{sec:phase4_limitations}

Phase~4 is intentionally modest in scope. Its purpose is to demonstrate
how a simple mapping family and a small stack of diagnostics can be
wired, in a fully reproducible way, on top of the Phase~3 vacuum
mechanism. It is \emph{not} a proposal for a definitive
\(\theta\)-corridor, a preferred \(\theta_\star\), or a calibrated
cosmological model. In this section we make explicit what the current
Phase~4 construction does and does not claim.

\subsection{Structural vs.\ physical content}

The F1 mapping family, the toy corridor, and the FRW-inspired module
are all \emph{structural} devices. They are designed to test whether
the Phase~3 amplitudes can be converted into simple scalar observables
that admit the kind of corridor and probe machinery laid out in
Sec.~\ref{sec:mappings}. The construction deliberately holds fixed
a number of choices that, in a full physical proposal, would need to
be re-opened and justified on their own terms.

In particular, Phase~4:
\begin{itemize}
  \item fixes the Phase~3 baseline vacuum (including the grid,
        \(\varepsilon\)-floor, and diagnostic summary) and does not explore
        alternative vacua or hyperparameters;
  \item fixes the F1 mapping family and its parameters
        (here \(\alpha = 1\), \(\beta = 4\)) rather than treating them as
        fit parameters or scanning over a broader family;
  \item defines a toy shape corridor using a simple
        \(E_{\rm vac}(\theta)\)-based threshold, chosen for readability
        and reproducibility, not for optimality;
  \item translates the vacuum curve into FRW-like quantities with a
        highly simplified background, intended to expose failure modes
        rather than to compete with standard cosmological analyses.
\end{itemize}
None of these structural choices are claimed to be unique or optimal.
They are scaffolding for Phase~4; any future attempt to argue for a
specific \(\theta_\star\) or for phenomenological consequences would
need to revisit and justify them.

\subsection{FRW-facing limitations}

The FRW-facing layers of Phase~4 (toy sanity checks, the viability
scan, corridor construction, and the \(\Lambda\)CDM-like probe) are
deliberately simple. They fix \(\Omega_m\), \(\Omega_r\), and
\(H_0\) by hand, work with a single baseline mapping family, and use
broad box-shaped windows in age and \(\Omega_\Lambda\) rather than a
likelihood-based treatment. Radiation, baryons, curvature, and
observational systematics are all treated in an approximate or
implicit way, if at all.

As such, the FRW module is best read as a structured plausibility
check on the F1 mapping and on the Phase~3 mechanism, not as a
competitive cosmological fit. In particular:
\begin{itemize}
  \item the \(\Lambda\)CDM-like probe is defined by generous age and
        \(\Omega_\Lambda\) windows rather than by a full parameter
        inference against data;
  \item the shape probe and corridor overlap summaries are qualitative
        diagnostics of how the vacuum curve behaves when pushed
        through a simple FRW calculator, not precision constraints;
  \item the FRW data probe is wired into the pipeline but, in the
        baseline repository configuration, operates on a stub external
        dataset and records that no real data have been applied.
\end{itemize}
The numerical fractions and bands reported in the Phase~4 diagnostics
and paper should therefore be read as internally consistent checks
\emph{within this specific scaffold}, not as evidence for or against
any particular cosmological model.

\subsection{External host and flatness limitations}

The Stage~2 external host layer and its flatness-based kernel provide
an additional structural bridge between the Phase~4 FRW module and an
external source of FRW background grids. Even here, however, the
construction remains deliberately conservative.

The external host kernel used in Phase~4:
\begin{itemize}
  \item is defined by a near-flatness criterion on \(\Omega_{\rm tot}\)
        together with broad age windows, rather than by a full joint
        likelihood over background parameters;
  \item is restricted to a small 12-point kernel that matches the
        Phase~4 FRW viable corridor and the toy shape corridor; and
  \item is summarised by simple bands in \(\theta\),
        \(\Omega_\Lambda\), age, and a single mechanism scalar
        \(\mathrm{mech\_baseline\_A0}\), rather than by a higher-
        dimensional posterior.
\end{itemize}
The external host kernel is thus a consistency and alignment device:
it checks that a simple external FRW grid and the internal Phase~4
construction agree on a small region of parameter space. It is not a
claim that the kernel defines an observationally selected
\(\theta\)-interval, nor that it captures all relevant uncertainties
in the external constructions it mirrors.

\subsection{Data-facing limitations and future rungs}

Finally, Phase~4 does not perform any real data assimilation. The FRW
data-probe rung is implemented as a reproducible stub: the code
expects a binned distance dataset in a standard format but, in the
baseline repository, operates on a placeholder file and records that
no external data are present. This is intentional. It keeps the
artifact self-contained, avoids bundling proprietary or versioned
datasets, and prevents accidental over-interpretation of the toy
FRW outputs as data-driven constraints.

Any future phase that seeks to confront data would need to:
\begin{itemize}
  \item specify concrete datasets and preprocessing steps;
  \item define likelihoods and nuisance-parameter treatments for those
        datasets;
  \item revisit the mapping-family assumptions and FRW backgrounds
        used here; and
  \item articulate clear, limited claims (and non-claims) about what a
        given rung does or does not establish.
\end{itemize}
Those tasks lie outside the scope of the present Phase~4 artifact.
Here, the goal is narrower: to show that the Phase~3 vacuum mechanism
can be wired, in a disciplined way, into simple FRW-inspired probes
and into a small external host bridge, while keeping all assumptions
and limitations explicit.
