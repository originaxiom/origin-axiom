\section{Limitations and scope of Phase~4}
\label{sec:phase4_limitations}

Phase~4 is intentionally modest in scope. Its purpose is to demonstrate
how a simple mapping family and a small stack of diagnostics can be
wired, in a fully reproducible way, on top of the Phase~3 vacuum
mechanism. It is \emph{not} a proposal for a definitive
\(\theta\)-corridor, a preferred \(\theta_\star\), or a calibrated
cosmological model. In this section we make explicit what the current
Phase~4 construction does and does not claim.

\subsection{Structural vs.\ physical content}

The F1 mapping family, the toy corridor, and the FRW-inspired module
are all \emph{structural} devices. They are designed to test whether
the Phase~3 amplitudes can be converted into simple scalar observables
in a way that is numerically sane, auditable, and easy to reproduce.

No part of the present Phase~4 implementation is calibrated to
observational data or to a concrete field-theoretic model. The choices
of mapping exponents, normalisations, and diagnostic thresholds are
deliberately simple and should be read as examples of the
\emph{workflow}, not as physically motivated parameter inference.

\subsection{Status of \texorpdfstring{\(\theta\)}{theta}-corridors and \texorpdfstring{\(\theta_\star\)}{theta*}}

The F1 toy corridor is defined by a \(k_\sigma\)-style condition on
\(E_{\mathrm{vac}}(\theta)\): points whose mapped scalar lies within a
band of width \(k_\sigma\) times the empirical standard deviation
around the minimum are marked as ``inside''. This is a purely
diagnostic definition. It is:

\begin{itemize}
  \item \emph{not} justified by an underlying measure on
        \(\theta\)-space;
  \item \emph{not} tied to any observational likelihood; and
  \item \emph{not} used to select a unique \(\theta_\star\).
\end{itemize}

At this rung we do not claim that the corridor has any physical
significance beyond being a compact way of summarising the shape of
\(E_{\mathrm{vac}}(\theta)\). The Phase~4 artifact therefore does not
define a canonical \(\theta_\star\), nor does it promote any specific
subset of \(\theta\)-values as physically preferred.

\subsection{FRW-inspired toys and cosmology}

The FRW-inspired module treats the F1 scalar as a proxy for a
\(\theta\)-dependent vacuum-energy density \(\Omega_\Lambda(\theta)\),
choosing the normalisation so that the mean of
\(\Omega_\Lambda(\theta)\) over the grid is close to a target value
(e.g.\ \(\approx 0.7\)). Together with fixed toy values of
\(\Omega_m\) and \(\Omega_r\), this defines a simple
FRW-like quantity
\[
  H^2(a; \theta)
    = \Omega_r a^{-4}
    + \Omega_m a^{-3}
    + \Omega_\Lambda(\theta),
\]
evaluated on a finite grid of scale factors \(a\).

The diagnostic checks used here (positivity of \(H^2\) on the grid and
a bound on the ratio of maximal to minimal \(H^2\) per \(\theta\))
are deliberately crude. They are not meant to reproduce precision
cosmology, nor to constrain \(\theta\) from data. Their role is to
illustrate how one could connect the Phase~3 mechanism to FRW-like
sanity checks and to record both positive and negative outcomes (such
as empty-corridor cases) in a structured way.

\subsection{Numerical and implementation limitations}

All Phase~4 results are computed on finite grids: a finite
\(N_\theta\)-grid for \(\theta\) and a finite \(N_a\)-grid for the
scale factor in the FRW-inspired module. The diagnostics such as
``corridor fraction'' or ``FRW-sanity fraction'' are therefore
\emph{grid-level} summaries and may change under refinement of these
grids.

The current implementation uses a single mapping family (F1) and a
single baseline configuration inherited from Phase~3. There is no
systematic exploration of alternative mapping families, floors,
normalisations, or FRW toy parameters. The artifact should therefore
be read as documenting \emph{one concrete, reproducible instance} of
the workflow, not as a survey of the available design space.

\subsection{Future extensions and non-claims}

Phase~4 is explicitly positioned as a bridge between the Phase~3
mechanism and future corridor work, not as the end-point of that
programme. In particular:

\begin{itemize}
  \item we do not claim to have derived a physically meaningful
        \(\theta\)-corridor;
  \item we do not claim to have identified a preferred
        \(\theta_\star\); and
  \item we do not claim to have solved any cosmological fine-tuning
        problem.
\end{itemize}

What Phase~4 does claim is that:

\begin{enumerate}
  \item the Phase~3 vacuum mechanism can be cleanly wired into a
        mapping family and scalar diagnostics;
  \item the resulting workflow can be captured in a small, auditable
        code surface with explicit inputs and outputs; and
  \item both positive and negative diagnostic outcomes (including
        empty-corridor cases) can be reported in a way that is aligned
        with the Phase~0 philosophy of structured, reproducible
        progress.
\end{enumerate}

Any future tightening of the corridor notion, introduction of a
candidate \(\theta_\star\), or calibration to observational data would
require additional rungs beyond the present Phase~4 artifact and
would have to make their own assumptions and limitations explicit.

\paragraph{FRW-facing limitations.}
The FRW-facing layers of Phase~4 (toy sanity checks, the viability scan,
corridor construction, and the \(\Lambda\)CDM-like probe) are deliberately
simple. They fix \(\Omega_m\), \(\Omega_r\), and \(H_0\) by hand, impose broad
box-shaped windows in age and \(\Omega_\Lambda\) rather than using
likelihoods, and ignore radiation, baryons, curvature, and observational
systematics. As such, they are best read as structured plausibility checks on
the F1 mapping and on the Phase~3 mechanism, not as competitive cosmological
fits. A full FRW analysis would require a more complete treatment of the
background and of data, and lies beyond the scope of this phase.
