\section{Mapping families: first pass (F1)}
\label{sec:phase4-mappings}

Phase~4 takes as input the Phase~3 global-amplitude mechanism: a toy
vacuum with an unconstrained observable \(A_0(\theta)\), a
non-cancellation floor \(\epsfloor\), and a floor-enforced amplitude
\(A(\theta) = \max(A_0(\theta), \epsfloor)\) defined on a grid
\(\theta \in [0, 2\pi)\). The present paper introduces a first,
explicit mapping family, denoted \textbf{F1}, from this structure to a
toy vacuum-energy-like scalar.

\subsection{F1: direct scalar mapping from \(A(\theta)\)}
\label{sec:phase4-F1}

The F1 family is intentionally simple. For a fixed Phase~3 vacuum
configuration and floor \(\epsfloor\) (taken from the Phase~3 baseline
diagnostics), we define a scalar
\begin{equation}
  E_{\mathrm{vac}}(\theta)
  \;=\;
  \alpha \, A(\theta)^{\beta},
  \label{eq:phase4-F1-mapping}
\end{equation}
where \(\alpha > 0\) and \(\beta > 0\) are explicit, configurable
parameters. At this rung we adopt a conservative default,
\(\alpha = 1\) and \(\beta = 2\), and focus on structural behaviour
rather than numerical normalisation.

Operationally, Phase~4 reuses the Phase~3 baseline configuration
\texttt{baseline\_v1} and the floor \(\epsfloor\) recorded in
\texttt{phase3/outputs/tables/mech\_baseline\_scan\_diagnostics.json}.
We then evaluate \(A(\theta)\) and \(E_{\mathrm{vac}}(\theta)\) on a
uniform grid of \(N_{\theta} = 2048\) points in \([0, 2\pi)\). The
per-grid values and summary diagnostics are written to
\begin{center}
  \texttt{phase4/outputs/tables/phase4\_F1\_sanity\_curve.csv},
\end{center}
together with metadata describing the mapping parameters and the
underlying Phase~3 diagnostics.

At this stage F1 is a \emph{non-binding} mapping family: it does not
yet define a \(\theta\)-corridor or a Phase~4 \(\theta\)-filter.
Instead it serves as a concrete, auditable bridge between the Phase~3
mechanism and simple scalar observables that later rungs can connect
to FRW-like toy modules and corridor construction.
