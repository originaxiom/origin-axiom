\section{Mapping family and scalar \(E_{\mathrm{vac}}(\theta)\)}
\label{sec:phase4-mappings}

Phase~3 furnishes a fixed, mechanism-level map
\(\theta \mapsto A(\theta)\) together with a global floor and a
summary diagnostics file. Concretely, a baseline configuration of the
Phase~3 mechanism produces a table of amplitudes \(A(\theta_j)\) on a
uniform \(\theta\)-grid and writes, among other outputs, a JSON file
%
\begin{equation}
  \texttt{phase3/outputs/tables/mech\_baseline\_scan\_diagnostics.json},
\end{equation}
%
which records the global minimum and maximum, basic moments, and
quantiles of the scan. Phase~3 itself stops at this level: it does not
interpret \(A(\theta)\) as a cosmological quantity.

Phase~4 introduces a simple, explicit mapping family that turns the
Phase~3 amplitude into a non-negative scalar intended to play the role
of a toy vacuum-energy density:
%
\begin{equation}
  E_{\mathrm{vac}}(\theta)
  \;=\;
  \alpha\, A(\theta)^{\beta},
  \qquad
  \alpha > 0,\; \beta > 0.
  \label{eq:phase4-f1-definition}
\end{equation}
%
We refer to this family as ``F1''. The two parameters \(\alpha\) and
\(\beta\) are treated as tunable, mechanism-facing hyperparameters:
they shape the relative weighting of amplitude variations across the
\(\theta\)-grid but do not correspond to any claimed physical
constants.

In this Phase~4 baseline rung we adopt the concrete choice
\begin{equation}
  \alpha = 1,
  \qquad
  \beta = 4,
\end{equation}
motivated by three simple considerations:
\begin{enumerate}
  \item \(A(\theta)\) is already normalised at the mechanism level, so
    it is natural to absorb any overall scale into a separate FRW
    toy-model normalisation rather than into \(\alpha\);
  \item an even power \(\beta\) guarantees that \(E_{\mathrm{vac}}\)
    remains non-negative wherever \(A(\theta)\) is defined;
  \item a quartic choice \(\beta = 4\) accentuates contrasts between
    low- and high-amplitude regions without being numerically extreme
    on the Phase~3 baseline scan.
\end{enumerate}
These points are \emph{pragmatic} rather than physical: they select a
well-behaved default for the present rung while keeping the door open
to future exploration of alternative exponents and normalisations.

Given a fixed Phase~3 amplitude table, the Phase~4 script
\texttt{phase4/src/phase4/run\_f1\_sanity.py} constructs the scalar
curve \(E_{\mathrm{vac}}(\theta)\) according to
Eq.~\eqref{eq:phase4-f1-definition} with \(\alpha = 1\) and
\(\beta = 4\), reusing the global minimum and quantile-based floor
reported in
\texttt{mech\_baseline\_scan\_diagnostics.json}. It writes:
\begin{itemize}
  \item a per-\(\theta\) CSV
    \texttt{phase4/outputs/tables/phase4\_F1\_sanity\_curve.csv}
    containing \(\theta\) and \(E_{\mathrm{vac}}(\theta)\); and
  \item a JSON summary
    \texttt{phase4/outputs/tables/phase4\_F1\_sanity\_curve\_diagnostics.json}
    with global extrema, basic moments, and configuration metadata.
\end{itemize}
This establishes a reproducible, one-dimensional scalar landscape on
top of the Phase~3 mechanism. All subsequent Phase~4 diagnostics---
including the toy corridor, FRW-inspired sanity checks, FRW viability
scan, corridor extraction, and \(\Lambda\)CDM-facing probes---are built
\emph{on this F1 scalar}, and their outputs always reference
\texttt{phase4\_F1\_*} tables rather than the raw Phase~3 amplitudes.

Throughout this rung, F1 is treated as a \emph{diagnostic mapping}
rather than a physical law. In particular, we do not claim that
\(E_{\mathrm{vac}}(\theta)\) is the actual vacuum energy of the real
Universe; instead, we use it as a controlled, reproducible proxy for
asking whether the Phase~3 landscape can coherently feed into
FRW-like summaries without generating obviously pathological behaviour
(e.g.\ negative or wildly oscillatory \(H^2(a;\theta)\)) across the
\(\theta\)-grid. Future work is free to vary \(\alpha\), \(\beta\), or
even the functional form of the mapping, but such variations will sit
on top of the baseline pipeline documented here.

At this stage the F1 family should be read strictly as a structural,
toy choice.  The exponents and normalisation \((\alpha,\beta) = (1,4)\)
are selected for numerical convenience and to accentuate contrast in
the Phase~3 amplitude; they are \emph{not} proposed as a physically
motivated relation between the toy amplitude and any real vacuum
energy density.  Any future, data-facing use of the Phase~4 pipeline
would be expected to revisit and possibly replace this mapping.

% -------------------------------------------------------------------
% F1 mapping definition stub (matches Phase 3 baseline configuration)
% -------------------------------------------------------------------
\begin{equation}
  E_{\mathrm{vac}}^{\mathrm{(F1)}}(\theta)
  =
  E_{\mathrm{vac}}^{\mathrm{baseline}}(\theta;\alpha=1,\beta=4)
  \label{eq:phase4-f1-definition}
\end{equation}


% -------------------------------------------------------------------
% F1 mapping definition anchor (used by \ref{eq:phase4-f1-definition})
% -------------------------------------------------------------------
We collect the F1 mapping into a single symbolic definition for later
reference:
\begin{equation}
  E_{\mathrm{vac}}^{\mathrm{F1}}(\theta)
  \equiv E_{\mathrm{vac}}(\theta; \alpha, \beta)\,,
  \label{eq:phase4-f1-definition}
\end{equation}
where \(\alpha\), \(\beta\), and the baseline diagnostics are fixed by
the Phase~3 vacuum scan; the explicit implementation lives in the
Phase~4 F1 scripts rather than being re-derived here.
