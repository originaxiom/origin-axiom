% ------------------------------------------------------------------
% Phase 4 mappings: F1 baseline vacuum map and FRW lift
% ------------------------------------------------------------------

\section{Phase 4 mappings: F1 baseline vacuum map and FRW lift}
\label{sec:mappings}

Phase~4 takes the Phase~3 baseline mechanism and turns it into a family of vacuum curves and FRW backgrounds that can be probed against toy diagnostics, Stage~2 hosts, and external data. In this section we define the F1 mapping family at a symbolic level and record how it is implemented in the code.

The starting point is the Phase~3 baseline diagnostics JSON \texttt{phase3/outputs/tables/mech\_baseline\_scan\_diagnostics.json}. That file summarizes the baseline mechanism scan on a uniform $\theta$ grid of length $N$ and reports, among other quantities, the unconstrained amplitude range and a reference ``floor'' value $\epsilon_{\mathrm{floor}}$ that is used to avoid numerical pathologies in regions where the mechanism is effectively flat. Phase~4 does not change that baseline; it only wraps it into a more structured mapping family.

\subsection{Symbolic F1 mapping definition}

At a symbolic level, the F1 mapping family is a one-parameter deformation of the Phase~3 baseline amplitude table into a vacuum curve $E_{\mathrm{vac}}(\theta)$ on a fixed grid. We denote by $f_{\mathrm{baseline}}(\theta;\alpha,\beta)$ the rescaled baseline amplitude after applying the F1 hyperparameters $(\alpha,\beta)$, and by $\epsilon_{\mathrm{floor}}$ the numerical floor imported from Phase~3. The Phase~4 F1 vacuum curve is then
\begin{equation}
  E_{\mathrm{vac}}^{\mathrm{F1}}(\theta;\alpha,\beta) = \max\!\Bigl\{\epsilon_{\mathrm{floor}}, f_{\mathrm{baseline}}(\theta;\alpha,\beta)\Bigr\},
  \label{eq:phase4-f1-definition}
\end{equation}
evaluated on the same $N$-point grid used in the Phase~3 diagnostics. In the current baseline configuration, the scripts fix $(\alpha,\beta) = (1,4)$ and treat $\epsilon_{\mathrm{floor}}$ and the grid bounds $[\theta_{\min},\theta_{\max}]$ as imported from the Phase~3 JSON diagnostics.

Equation~\eqref{eq:phase4-f1-definition} is intentionally symbolic: the paper does not attempt to re-derive the precise interpolation or rescaling choices inside $f_{\mathrm{baseline}}$. Those details are locked in the code and documented by the Phase~3 diagnostics JSON; the role of the paper is to show how the resulting $E_{\mathrm{vac}}^{\mathrm{F1}}(\theta)$ behaves once lifted into FRW backgrounds and probed by the diagnostics in Section~\ref{sec:diagnostics}.

\subsection{Repository implementation and sanity curve}

In the repository, the F1 mapping is implemented and exercised by the script \texttt{phase4/src/phase4/run\_f1\_sanity.py}. Running this script reads the Phase~3 baseline diagnostics JSON, constructs the F1 vacuum curve according to Equation~\eqref{eq:phase4-f1-definition}, and writes two outputs under \texttt{phase4/outputs/tables/}:
\begin{itemize}
  \item a CSV file \texttt{phase4\_F1\_sanity\_curve.csv} containing per--$\theta$ columns for $\theta$, $E_{\mathrm{vac}}^{\mathrm{F1}}(\theta)$, and any additional helper quantities used downstream;
  \item a JSON diagnostics file \texttt{phase4\_F1\_sanity\_curve\_diagnostics.json} summarizing basic statistics of the curve (minimum, maximum, mean, and standard deviation), together with the grid length $N$ and the imported Phase~3 floor and unconstrained range.
\end{itemize}

The key point is that Phase~4 never reconstructs the baseline physics from scratch. Instead, it treats the Phase~3 baseline as a frozen numerical object and uses the F1 mapping family to explore how a single, coherent vacuum curve can support FRW backgrounds and external probes.

\subsection{From F1 vacuum map to FRW backgrounds}

The F1 vacuum curve $E_{\mathrm{vac}}^{\mathrm{F1}}(\theta)$ is then lifted into FRW backgrounds by a family of scripts under \texttt{phase4/src/phase4/} that are described in Section~\ref{sec:diagnostics}. For each grid point $\theta_i$ on the F1 curve, those scripts construct an effective Hubble history $H^2(a;\theta_i)$ with fixed $(\Omega_m,\Omega_r)$ and an effective $\Omega_\Lambda(\theta_i)$ determined by $E_{\mathrm{vac}}^{\mathrm{F1}}$. The resulting FRW backgrounds are then subjected to a sequence of diagnostics: sanity checks on $H^2(a)$, toy ``shape corridor'' tests, FRW viability and late-acceleration masks, $\Lambda$CDM-like windows, and external-host and data-facing probes.

All of those steps depend on the F1 mapping only through the numerical curve and diagnostics written by \texttt{run\_f1\_sanity.py}. This separation is deliberate: it allows the mapping layer in Equation~\eqref{eq:phase4-f1-definition} to be swapped or refined in future work without changing the overall FRW and external-host machinery, as long as the same output format and grid structure are respected.
