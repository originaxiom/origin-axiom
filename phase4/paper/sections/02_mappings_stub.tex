% -------------------------------------------------------------------
% Phase 4 mappings stub: F1 family
% -------------------------------------------------------------------
\section{Mappings from Phase 3 to Phase 4}
\label{sec:mappings}
\label{sec:phase4-mappings}

Our goal in Phase~4 is not to promote every possible Phase~3 configuration into a cosmological background, but to define a small, explicit set of ``mapping families'' that lift a Phase~3 vacuum curve $E_{\mathrm{vac}}(\theta)$ into a toy FRW setting. In this paper we restrict attention to a single family, which we call F1, and to a particular baseline choice of its parameters. The intent is to make every step of the promotion chain explicit and reproducible, without yet claiming that the resulting cosmological corridor is unique, optimal, or observationally complete.

We assume the Phase~3 vacuum curve has already been computed on a uniform $\theta$--grid of length $N_{\theta}$, with a baseline JSON diagnostics file \texttt{phase3/outputs/tables/mech\_baseline\_scan\_diagnostics.json}. This file records, among other things, the global minimum and a set of quantiles of the unconstrained vacuum curve. The Phase~4 mapping family F1 uses these diagnostics to define a dimensionless vacuum scale and a simple two-parameter deformation that can be lifted into a spatially flat FRW background.

Concretely, let $E_{\mathrm{vac}}(\theta)$ denote the Phase~3 vacuum curve on the fixed $\theta$--grid. We introduce a dimensionless rescaled curve
\begin{equation}
  \tilde{E}_{\mathrm{vac}}(\theta) \equiv \frac{E_{\mathrm{vac}}(\theta) - E_{\min}}{\sigma_{\mathrm{vac}}},
\end{equation}
where $E_{\min}$ is the global minimum of $E_{\mathrm{vac}}(\theta)$ over the grid and $\sigma_{\mathrm{vac}}$ is a representative spread (e.g. the standard deviation or an inter-quantile range). The precise choice of $\sigma_{\mathrm{vac}}$ is recorded in the Phase~3 baseline diagnostics and carried forward as metadata rather than hard-coded into the mapping.

The F1 family is then defined symbolically by
\begin{equation}
  \Omega_{\Lambda}(\theta; \alpha, \beta) \equiv \alpha\,\tilde{E}_{\mathrm{vac}}(\theta) + \beta,
  \label{eq:phase4-f1-definition}
\end{equation}
with two real parameters $(\alpha, \beta)$ and a fixed matter contribution $\Omega_{\mathrm{m}}$ and radiation contribution $\Omega_{\mathrm{r}}$. In the toy baseline configuration used in this paper we take
\begin{equation}
  \Omega_{\mathrm{m}} = 0.3,\qquad \Omega_{\mathrm{r}} = 0,\qquad \alpha = 1,\qquad \beta = 4,
\end{equation}
so that the mean of $\Omega_{\Lambda}(\theta)$ over the full grid is approximately $0.7$ and the resulting total density parameter $\Omega_{\mathrm{tot}}(\theta) = \Omega_{\mathrm{m}} + \Omega_{\mathrm{r}} + \Omega_{\Lambda}(\theta)$ remains of order unity across the grid. This choice is not unique or claimed to be special; it is simply a concrete baseline that makes the toy FRW viability tests in Section~\ref{sec:diagnostics} well posed.

In the repository, the F1 mapping is implemented and exercised by the script \texttt{phase4/src/phase4/run\_f1\_sanity.py}. This script reads the Phase~3 baseline diagnostics JSON, constructs a uniformly spaced $\theta$--grid of length $N_{\theta} = 2048$, and evaluates the vacuum curve $E_{\mathrm{vac}}(\theta)$ on that grid using the Phase~3 machinery. It then computes $\tilde{E}_{\mathrm{vac}}(\theta)$ and $\Omega_{\Lambda}(\theta)$ according to Eq.~\eqref{eq:phase4-f1-definition}, with the baseline parameter choices recorded in the output metadata.

The main outputs of \texttt{run\_f1\_sanity.py} are:
\begin{itemize}
  \item a CSV file \texttt{phase4/outputs/tables/phase4\_F1\_sanity\_curve.csv} containing per--$\theta$ columns \texttt{theta}, \texttt{E\_vac}, and \texttt{omega\_lambda};
  \item a JSON diagnostics file \texttt{phase4/outputs/tables/phase4\_F1\_sanity\_curve\_diagnostics.json} recording the global minimum, mean, and standard deviation of $E_{\mathrm{vac}}(\theta)$ and $\Omega_{\Lambda}(\theta)$, as well as the baseline Phase~3 quantiles that were used to define $\tilde{E}_{\mathrm{vac}}(\theta)$.
\end{itemize}

For the purposes of Phase~4, the F1 family is intentionally simple. It does not attempt to solve the full inverse problem of mapping an arbitrary vacuum curve into a realistic cosmological parameter space; instead, it provides a reproducible, linearly parametrised bridge between the Phase~3 vacuum structure and a toy FRW background. More elaborate families (e.g. with non-linear deformations, explicit curvature terms, or time-dependent effective dark energy) can be added in future phases without changing the baseline F1 definition used here.