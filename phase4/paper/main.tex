\documentclass[11pt]{article}

\usepackage[a4paper,margin=1in]{geometry}
\usepackage{amsmath,amssymb}
\usepackage{graphicx}
\usepackage{hyperref}
\usepackage{booktabs}
\usepackage{physics}
\usepackage{caption}
\usepackage{subcaption}

\graphicspath{{../outputs/figures/}}

\hypersetup{
  colorlinks=true,
  linkcolor=blue,
  citecolor=blue,
  urlcolor=blue
}

\newcommand{\OA}{\textit{Origin Axiom}}
\newcommand{\epsfloor}{\varepsilon}


\title{Phase 4: Vacuum-to-FRW Consistency and Scale Sanity\\[4pt]
{\large A corridor-style test of the Phase 3 global-amplitude mechanism}}
\author{Origin Axiom Program}
\date{\today}

\begin{document}
\maketitle

\begin{abstract}
Phase~3 of the origin-axiom programme defined a reproducible scalar
vacuum diagnostic \(E_{\mathrm{vac}}(\theta)\) over an abstract
\(\theta\)-grid. Phase~4 tests whether that diagnostic can be connected
to simple Friedmann--Robertson--Walker (FRW) histories in a structurally
reasonable way. We introduce a first mapping family (F1) from
\(E_{\mathrm{vac}}(\theta)\) to a toy \(\Omega_\Lambda(\theta)\), and we
stack several FRW-inspired diagnostics: a sanity check on the shape of
\(\Omega_\Lambda(\theta)\), a viability scan for FRW histories with
matter, radiation and late-time acceleration, an automated extraction of
FRW-viable \(\theta\)-corridors, and a coarse \(\Lambda\)CDM-like probe
centred on \(\Omega_\Lambda \approx 0.7\) and \(t_0 \approx 13.8\,\mathrm{Gyr}\).
For the baseline toy parameters used here, roughly half of the
\(\theta\)-grid yields FRW-viable histories and a small but non-zero
subset falls inside the \(\Lambda\)CDM-like window. We emphasise that
these results are \emph{non-binding}: Phase~4 does not propose a unique
\(\theta_\star\) or an observational fit, but instead provides a
reproducible FRW-facing pipeline and a concrete worked example of how
Phase~3 outputs can be tested against cosmological sanity criteria.
\end{abstract}

\section{Introduction}
\label{sec:introduction}

\subsection{Philosophical and Physical Motivation}
The question ``Why is there something rather than nothing?'' has persisted across philosophy and physics for millennia. Classical physics permits a stable vacuum state of absolute nothingness, yet quantum mechanics reveals pervasive vacuum fluctuations, suggesting a non-zero baseline energy density. This tension between classical stability and quantum non-triviality motivates the Origin Axiom: the universe is structurally forbidden from perfect global cancellation, enforced on complex scalar amplitudes $A(\mathcal{C})$ over the space of all possible configurations $\mathcal{C}$.

Formally, the axiom imposes a strict lower bound on the modulus:
\begin{equation}
|A(\mathcal{C})| > \epsilon > 0,
\label{eq:axiom_bound}
\end{equation}
where $\epsilon$ is a minimal positive floor, derived from quantum gravity considerations (simulated as $\sim 10^{-12}$ in normalized units for computational purposes). This bound prevents $|A| \to 0$, rendering absolute nothingness impossible.

Philosophically, the axiom implies that existence is not contingent but inherent—a structural bias against nothingness that favors complexity and differentiation. Intellectually, it offers a principled alternative to fine-tuning arguments: rather than invoking anthropic selection or multiverses, the axiom posits that the universe must exhibit non-zero structure at all scales, as enforced by the underlying configuration space geometry.

\subsection{The Unifying Phase \texorpdfstring{$\thetastar$}{theta*}}
To anchor the axiom in empirical physics, we introduce a unifying phase $\thetastar$ extracted from Standard Model flavor observables. $\thetastar$ is determined via joint $\chi^2$ minimization fits to the PMNS neutrino mixing matrix and CKM quark mixing matrix, yielding a fiducial value of 3.63 radians with a robust uncertainty band of $[2.18, 5.54]$ radians (detailed in the companion repository \texttt{origin-axiom-theta-star}).

This phase $\thetastar$ serves as a universal bridge between flavor phenomenology and scalar dynamics. Simulations demonstrate that $\thetastar$-dependent effects propagate consistently across scales—from neutrino masses and vacuum energy modulation to cosmological expansion, microstructure stability, baryogenesis, dark matter relics, and quantum gravity-derived parameters—suggesting a deep underlying unity rooted in the Origin Axiom.

\subsection{Unification Roadmap}
We demonstrate the axiom's unifying power through a systematic chain of simulations:

\begin{itemize}
    \item \textbf{Flavor bridge} (Section \ref{sec:flavor}): $\thetastar$-modulated seesaw mechanism reproduces PDG neutrino masses and mixing.
    \item \textbf{Vacuum shift} (Section \ref{sec:vacuum}): Microcavity models reveal $\sim$2.2\% modulation in energy shift $\Delta E(\thetastar)$.
    \item \textbf{Cosmological expansion} (Section \ref{sec:cosmology}): FRW evolution with $\Lambda(\thetastar)$ yields $\sim$1\% acceleration in scale factor growth.
    \item \textbf{Microstructure} (Section \ref{sec:microstructure}): 3D lattices with defects exhibit $\thetastar$-dependent stability.
    \item \textbf{Quantum gravity foundation} (Section \ref{sec:qg_epsilon}): Derivation of $\epsilon(\thetastar)$ from Planck-scale and holographic principles.
    \item \textbf{Standard Model compatibility} (Section \ref{sec:sm_integration}): Integration with electroweak symmetry breaking and gauge coupling modulation.
    \item \textbf{Baryogenesis} (Section \ref{sec:baryogenesis}): $\thetastar$-driven CP asymmetry yielding baryon asymmetry $\eta_B$.
    \item \textbf{Dark matter} (Section \ref{sec:dark_matter}): Defects as scalar candidates with computed relic density $\Omega_{\text{DM}} h^2$.
    \item \textbf{Experimental predictions} (Section \ref{sec:predictions}): $\thetastar$-modulated observables including $\theta_{13}$, CMB $\Delta T/T$, and $H_0$.
    \item \textbf{Synthesis} (Section \ref{sec:synthesis}): Full chain reveals aligned patterns and emergent phenomena.
\end{itemize}

The framework is fully reproducible via open-source code at \url{https://github.com/originaxiom/origin-axiom} and \url{https://github.com/originaxiom/origin-axiom-theta-star}, with git-tagged versions ensuring traceability.

\section{Mapping the Phase 3 vacuum into FRW-like observables}
\label{sec:mappings}

This section will, in later rungs, define explicit mappings from the
Phase 3 floor-enforced global amplitude \(A(\theta)\) (or a derived
residue) into:

\begin{itemize}
  \item one or more toy FRW modules; and/or
  \item simple vacuum-energy-like scalar observables.
\end{itemize}

The design goal is to keep these mappings:
\begin{itemize}
  \item simple and explicit enough to be implemented and audited;
  \item flexible enough to explore several normalisation and coupling
        choices; and
  \item constrained enough that the resulting behaviour can be turned
        into a reproducible \(\theta\)-filter, even if the outcome is
        an empty corridor.
\end{itemize}

At this rung the section serves only as a placeholder documenting the
intended role of Phase 4. No specific mapping is yet fixed or used in
claims.

\section{Diagnostics and toy corridors (draft)}
\label{sec:phase4_diagnostics}

At this rung Phase~4 focuses on internal diagnostics of the scalar
\(E_{\mathrm{vac}}(\theta) = \alpha A(\theta)^{\beta}\) produced by
the F1 mapping family, together with simple, non-binding
\(\theta\)-corridors. No FRW module is yet implemented; instead we
prepare the ground for later FRW-like tests.

\subsection{Vacuum-curve sanity check}

The first diagnostic is a direct sanity check of the F1 mapping. Using
the Phase~3 baseline configuration and floor recorded in
\texttt{phase3/outputs/tables/mech\_baseline\_scan\_diagnostics.json},
we evaluate \(A(\theta)\) and \(E_{\mathrm{vac}}(\theta)\) on a
uniform grid of \(N_{\theta} = 2048\) points in \([0, 2\pi)\). The
script
\begin{center}
  \texttt{phase4/src/phase4/run\_f1\_sanity.py}
\end{center}
writes the per-grid values to
\begin{center}
  \texttt{phase4/outputs/tables/phase4\_F1\_sanity\_curve.csv},
\end{center}
together with mapping metadata and a summary of basic moments.

For the baseline configuration used here,
\(E_{\mathrm{vac}}(\theta)\) is strictly positive and remains on a
small, controlled scale, reflecting the scale of the underlying
amplitude and the chosen \((\alpha, \beta)\). This establishes that
the F1 mapping is at least numerically well behaved and correctly
wired to the Phase~3 mechanism.

\subsection{Toy shape diagnostics and non-binding corridor}

The second diagnostic probes the \emph{shape} of
\(E_{\mathrm{vac}}(\theta)\). The script
\begin{center}
  \texttt{phase4/src/phase4/run\_f1\_shape\_diagnostics.py}
\end{center}
rebuilds the same F1 curve and computes:

\begin{itemize}
  \item global extrema and moments \((E_{\mathrm{vac},\min},
        E_{\mathrm{vac},\max}, \text{mean}, \text{std})\);
  \item a toy, non-binding \(\theta\)-corridor defined by
        \[
          E_{\mathrm{vac}}(\theta) \le
          E_{\mathrm{vac},\min} + k_{\sigma} \sigma,
          \quad k_{\sigma} = 1;
        \]
  \item the fraction of the grid lying inside this corridor and the
        induced \(\theta\)-range.
\end{itemize}

The resulting summary is written to
\begin{center}
  \texttt{phase4/outputs/tables/phase4\_F1\_shape\_diagnostics.json},
\end{center}
while a per-\(\theta\) mask, indicating membership in the toy
corridor, is written to
\begin{center}
  \texttt{phase4/outputs/tables/phase4\_F1\_shape\_mask.csv}.
\end{center}

This corridor is explicitly labelled as \emph{exploratory and
non-binding}. It does not define a canonical \(\theta_{\star}\) or a
Phase~4 \(\theta\)-filter; it is only a structured way of selecting a
low-\(E_{\mathrm{vac}}\) region that later rungs can reuse when
designing FRW-like toy modules.

\subsection{FRW-like toy diagnostics (design only)}

To keep the Phase~4 narrative aligned with the Phase~0 contract, we
separate the internal diagnostics above from any FRW-like behaviour
tests. A separate design note
\begin{center}
  \texttt{phase4/FRW\_TOY\_DESIGN.md}
\end{center}
specifies a minimal FRW-inspired toy module in which the F1 scalar
acts as a driving term for a dimensionless scale factor
\(a(\tau)\) and Hubble-like quantity \(H(\tau)\).

At the present rung this module is \emph{not} implemented, and no
FRW-style diagnostics enter the claims table. The only purpose of the
design work is to:

\begin{itemize}
  \item define clear, auditable interfaces between
        \(E_{\mathrm{vac}}(\theta)\), the toy corridor mask, and FRW-like
        quantities; and
  \item constrain future work so that any FRW-like diagnostics remain
        simple, reproducible, and explicitly non-claiming unless
        promoted to a Phase~4 \(\theta\)-filter.
\end{itemize}

Subsequent rungs may instantiate this toy module in code or, if it
proves unhelpful, retire it in favour of alternative diagnostics. In
either case, the Phase~4 paper will distinguish binding
\(\theta\)-filters from non-binding exploratory diagnostics in line
with the Phase~0 corridor semantics.

\section{Limitations and outlook}
\label{sec:limitations}

Phase 4 is intentionally narrow in scope. Even once the mappings and
diagnostics are implemented, the phase will not claim:
\begin{itemize}
  \item a full derivation of cosmological parameters;
  \item a proof that the Origin Axiom is realised in nature; or
  \item a unique mechanism for connecting vacuum structure to FRW
        dynamics.
\end{itemize}

Instead, the goal is to provide a clean yes-or-no style test for a
specific question:

\begin{quote}
  Can the Phase 3 global-amplitude mechanism support scale-sane
  FRW-like behaviour, in at least one simple mapping family, without
  producing a degenerate or empty \(\theta\)-corridor?
\end{quote}

If the answer is ``no'' for all tested mapping families, Phase 4 will
record this as a structured negative result, signalling that either
the Phase 3 mechanism or the mapping strategy needs revision before
further unification attempts.


\appendix
\section*{Appendix A: Phase 4 claims table (draft)}
\label{app:phase4_claims_table}

Table~\ref{tab:phase4_claims} summarises the intended Phase 4 claims.
At this rung all entries are draft and non-binding.

\begin{table}[h]
  \centering
  \caption{Draft Phase 4 claims. Binding status will be updated once
  the phase is complete and audited.}
  \label{tab:phase4_claims}
  \begin{tabular}{llp{0.55\textwidth}}
    \toprule
    ID & Binding? & Summary \\
    \midrule
    C4.1 & no & Existence of at least one explicit mapping from the
                 Phase 3 global amplitude or residue into an FRW-like
                 or vacuum-energy-like observable with numerically
                 stable behaviour. \\
    C4.2 & no & Existence of a non-empty, non-trivial \(\theta\)-corridor
                 for at least one such mapping. \\
    C4.3 & no & Structured negative result if all tested mappings yield
                 empty or pathological corridors. \\
    \bottomrule
  \end{tabular}
\end{table}

\section{Reproducibility and gate levels}
\label{sec:phase3-reproducibility}

Phase~3 is implemented as a self-contained, reproducible unit inside
the \texttt{origin-axiom} repository. This appendix records the
filesystem layout, the gate script used to regenerate the canonical
artifact, and the minimal commands needed to reproduce the Phase~3
paper and figures.

\subsection*{Filesystem layout}

The Phase~3 tree is organised as follows:
\begin{itemize}
  \item \texttt{phase3/src/} --- source code for the Phase~3
    mechanism and diagnostics;
  \item \texttt{phase3/paper/} --- LaTeX sources for the Phase~3 paper,
    including \texttt{main.tex}, section stubs, and appendices;
  \item \texttt{phase3/workflow/} --- the Snakemake workflow
    driving the paper build, in
    \texttt{phase3/workflow/Snakefile};
  \item \texttt{phase3/outputs/} --- derived outputs produced by
    Phase~3 runs, including figures and the built paper;
  \item \texttt{phase3/artifacts/} --- canonical Phase~3 artifacts,
    including the versioned PDF used as an external reference.
\end{itemize}

At this rung the primary paper artifact is
\texttt{phase3/artifacts/origin-axiom-phase3.pdf},
with the corresponding build product in
\texttt{phase3/outputs/paper/phase3\_paper.pdf}.  The main figure used
in the text is stored as
\texttt{phase3/outputs/figures/fig1\_mech\_binding\_profile.pdf}.

\subsection*{Gate script and build pipeline}

Phase~3 uses a dedicated gate script
\texttt{scripts/phase3\_gate.sh} to orchestrate the build. The
level--A gate regenerates the Phase~3 paper and canonical artifact via
the Snakemake workflow in \texttt{phase3/workflow/Snakefile}.  From
the repository root, the human-facing entry point is:
\begin{quote}
\texttt{bash scripts/phase3\_gate.sh}
\end{quote}
which, at level~A, performs the following high-level steps:
\begin{enumerate}
  \item assembles the list of section and appendix files under
    \texttt{phase3/paper/};
  \item invokes Snakemake on \texttt{phase3/workflow/Snakefile};
  \item runs \texttt{latexmk} on \texttt{phase3/paper/main.tex} to
    produce \texttt{main.pdf};
  \item copies \texttt{main.pdf} to
    \texttt{phase3/outputs/paper/phase3\_paper.pdf} and
    \texttt{phase3/artifacts/origin-axiom-phase3.pdf}.
\end{enumerate}

The Snakemake rule \texttt{build\_phase3\_paper\_pdf} declares
\texttt{main.tex}, the section and appendix files, and
\texttt{Reference.bib} as its inputs, and produces both the
\texttt{phase3\_paper.pdf} output and the canonical
\texttt{origin-axiom-phase3.pdf} artifact.  This makes the LaTeX
dependencies explicit and allows Snakemake to detect when a rebuild is
necessary.

\subsection*{Assumptions and environment}

The Phase~3 paper build assumes:
\begin{itemize}
  \item a reasonably recent \LaTeX{} distribution (e.g.\ TeX Live~2025
    or similar) providing \texttt{latexmk}, \texttt{pdflatex}, and the
    standard packages used in Phase~0--4;
  \item a POSIX shell environment with \texttt{bash} and
    \texttt{make}-like tooling sufficient to run the gate script;
  \item Python and any libraries required by the Phase~3 mechanism
    code, for the production of figures and diagnostic tables.
\end{itemize}

At the present rung, Phase~3 defines a placeholder bibliography file
\texttt{phase3/paper/Reference.bib}, since the paper does not yet
make external-citation claims.  This file is still part of the
declared inputs for the build, so that the pipeline remains stable
when references are introduced at later rungs.

\subsection*{Reproducibility scope}

The Level--A gate guarantees that:
\begin{itemize}
  \item the Phase~3 paper builds successfully from the tracked
    \texttt{phase3/paper} sources and \texttt{Reference.bib};
  \item the canonical artifact
    \texttt{phase3/artifacts/origin-axiom-phase3.pdf} is
    regenerable from these sources via the Snakemake workflow; and
  \item the figures referenced in the text (in particular the
    binding-profile figure) are present under
    \texttt{phase3/outputs/figures/} and can be regenerated from the
    Phase~3 source code.
\end{itemize}

Higher gate levels, if introduced, would be expected to add explicit
checks on numerical outputs, diagnostic tables, and mechanism
parameters.  At this rung, however, the emphasis remains on ensuring
that the narrative and structural content of the Phase~3 paper is
fully reproducible from the repository state.

\subsection*{Auxiliary measure probe}

In addition to the main binding-profile experiment, Phase~3 includes a
non-binding auxiliary script
\texttt{phase3/src/phase3\_mech/measure\_v1.py} that probes the
empirical distribution of the baseline amplitude \(A_0(\theta)\) for a
large number of independently sampled toy ensembles.  This script does
not affect any of the main Phase~3 claims or floor definitions; it is
included solely to make the measure structure of the current toy
configuration explicit.

A typical invocation is:
\begin{verbatim}
  oa && python phase3/src/phase3_mech/measure_v1.py
\end{verbatim}
which writes:
\begin{itemize}
  \item a JSON diagnostics file
    \texttt{phase3/outputs/tables/phase3\_measure\_v1\_stats.json}
    containing basic summary statistics and quantiles of the
    \(A_0\) distribution; and
  \item a histogram CSV
    \texttt{phase3/outputs/tables/phase3\_measure\_v1\_hist.csv}
    with binned counts over \(A_0\).
\end{itemize}
The console output also prints a small selection of quantiles and
fractions below a few illustrative \(\varepsilon\) thresholds.  All of
these numbers are toy-model diagnostics and should be interpreted as
such; they are not promoted to binding corridor constraints or
physical scales at this rung.


\bibliographystyle{unsrt}
\bibliography{Reference}

\end{document}
