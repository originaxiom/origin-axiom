\documentclass[11pt,a4paper]{article}

\usepackage[margin=1in]{geometry}
\usepackage{amsmath,amssymb,amsfonts}
\usepackage{hyperref}
\usepackage{physics}
\usepackage{graphicx}

\title{A Toy Scalar Universe with a Non-Cancelling Constraint\\
on a Discrete Three-Torus}
\author{Drit\"ero Mehmetaj}
\date{\today}

\begin{document}

\maketitle

\begin{abstract}
We construct and numerically explore a minimal ``toy universe'' in which the Origin Axiom---the claim that reality cannot reach a perfectly cancelling global state---is implemented explicitly. The model consists of a complex scalar field on a discrete three-torus with local Klein--Gordon-type dynamics and a global constraint that excludes a neighbourhood of a cancelling configuration of the total field amplitude. We analyse the resulting dynamics, track energy and global amplitude behaviour, and compare the three-dimensional lattice model to a one-dimensional twisted scalar chain in which the vacuum spectrum can be computed analytically. The goal is not to reproduce realistic cosmology, but to provide a controlled testbed for the non-cancelling principle and to identify the conditions under which a global phase twist $\theta_\ast$ becomes physically nontrivial.
\end{abstract}

\section{Introduction}

The Origin Axiom posits that absolute nothingness is not a coherent possibility and that the universe is dynamically forbidden from evolving into a perfectly cancelling global state. A companion paper formulates this axiom and outlines a research program in which the obstruction to global cancellation is encoded as a nontrivial phase twist $\theta_\ast$ on a scalar backbone.

In this work we take a first concrete step: we construct a simple scalar toy universe on a discrete three-torus and implement the non-cancelling principle as a global constraint on the total field amplitude. We then study the resulting dynamics numerically and compare them to those of a one-dimensional twisted scalar model whose vacuum spectrum can be obtained in closed form.

\section{Three-Dimensional Lattice Model}

\subsection{Microstructure: Discrete Three-Torus}

We model space as a finite cubic lattice with periodic boundary conditions,
\begin{equation}
    \mathbf{n} = (n_x,n_y,n_z), \qquad
    n_i \in \{0,\dots,N_i-1\},
\end{equation}
with identifications $n_i \equiv n_i + N_i$. Each site has six nearest neighbours and the geometry is that of a discrete three-torus $T^3$.

This choice provides locality (nearest-neighbour couplings), approximate homogeneity and isotropy (up to lattice artefacts), and a compact space without boundaries, which is convenient when imposing global constraints.

\subsection{Field Content and Local Dynamics}

At each site $\mathbf{n}$ and time $t$ we place a complex scalar field
\begin{equation}
    \Phi_{\mathbf{n}}(t) \in \mathbb{C}, \qquad
    \Phi_{\mathbf{n}} = R_{\mathbf{n}} e^{i\varphi_{\mathbf{n}}}.
\end{equation}
The real and imaginary parts (or amplitude and phase) can be thought of as a ``yin--yang'' pair: two interpenetrating components that flow into each other rather than annihilating.

The local dynamics are given by a discrete nonlinear Klein--Gordon equation,
\begin{equation}
    \frac{d^2 \Phi_{\mathbf{n}}}{dt^2}
    = c^2 \Delta \Phi_{\mathbf{n}}
    - m^2 \Phi_{\mathbf{n}}
    - \lambda |\Phi_{\mathbf{n}}|^2 \Phi_{\mathbf{n}},
\end{equation}
where $\Delta$ is the discrete Laplacian with periodic boundary conditions. This yields finite-speed propagation, interference, and---for suitable parameters---the possibility of localized structures.

In the code, the equation is integrated using a leapfrog scheme, and a discrete energy functional
\begin{equation}
    E = \sum_{\mathbf{n}} \left[
      \frac12 |\dot{\Phi}_{\mathbf{n}}|^2
      + \frac{c^2}{2} \sum_{\mathbf{m}\in\mathcal{N}(\mathbf{n})} |\Phi_{\mathbf{m}} - \Phi_{\mathbf{n}}|^2
      + \frac{m^2}{2} |\Phi_{\mathbf{n}}|^2
      + \frac{\lambda}{4} |\Phi_{\mathbf{n}}|^4
    \right]
\end{equation}
is used as a diagnostic. In the absence of the global constraint, $E$ is approximately conserved for small time steps.

\subsection{Global Constraint from the Origin Axiom}

To implement the Origin Axiom, we introduce the global complex amplitude
\begin{equation}
    A(t) = \sum_{\mathbf{n}} \Phi_{\mathbf{n}}(t),
\end{equation}
and define a forbidden region $\mathcal{F}_{\theta_\ast}$ in the complex $A$-plane corresponding to (approximate) global cancellation. In the simplest implementation studied here, we focus on excluding a neighbourhood of $A \approx 0$; more elaborate choices can encode a specific cancelling phase $A_\ast(\theta_\ast)$.

In numerical experiments, the constraint is realised as a post-update modification. After each free leapfrog step, we check whether the updated $A(t)$ lies within a distance $\epsilon$ of the forbidden region. If it does, we apply a minimal global modification
\begin{equation}
    \Phi_{\mathbf{n}} \;\rightarrow\; \Phi_{\mathbf{n}} + \delta
\end{equation}
with a uniform complex offset $\delta$ chosen such that the new amplitude $A'(t)$ lies on the boundary $\partial\mathcal{F}_{\theta_\ast}$, at $|A' - A_\ast| = \epsilon$. This ``gentle projection'' enforces the non-cancelling condition without violently rescaling the field or introducing large energy artefacts.

\section{Numerical Experiments in Three Dimensions}

We initialize the field with small random complex noise on lattices of moderate size and track:
\begin{itemize}
    \item the magnitude of the global amplitude $|A(t)|$, and
    \item the discrete energy $E(t)$,
\end{itemize}
both with and without the global constraint.

Preliminary simulations with generic random initial conditions show that:
\begin{itemize}
    \item Without the constraint, $E(t)$ remains approximately constant and $|A(t)|$ executes small fluctuations around its initial value, as expected for linear or weakly nonlinear wave-like dynamics.
    \item For small $\epsilon$ and small initial noise, the global amplitude never approaches $A \approx 0$, so the constraint remains effectively inactive: the constrained and unconstrained runs lie almost exactly on top of each other in both $|A(t)|$ and $E(t)$.
\end{itemize}

To probe the regime in which the Origin Axiom becomes nontrivial, we deliberately construct an initial condition that is almost globally cancelling. Starting from a small random complex field on a $16^3$ lattice, we subtract the mean value so that the initial amplitude satisfies $|A(0)| \sim 10^{-16}$. We then evolve this configuration twice with identical parameters ($c=1.0$, $m=0.1$, $\lambda=0$, $dt=0.01$, $N_t=500$ steps):

\begin{itemize}
    \item \textbf{Unconstrained run.} The global amplitude remains extremely small throughout the evolution, with $|A(t)|$ fluctuating at the level of numerical roundoff around zero. The energy $E(t)$ stays in a narrow band around $E \approx 2.46 \times 10^{-2}$.
    \item \textbf{Constrained run with $\epsilon = 0.05$.} The Origin Axiom constraint is applied whenever $|A(t)| < \epsilon$. Starting from the same mean-subtracted initial field, the first projection moves $A(0)$ to $|A(0)| = \epsilon$. Subsequent steps keep $|A(t)|$ pinned at $|A| = 0.05$ to plotting precision. Over the full run of $500$ time steps the constraint fires $501$ times (once at $t=0$ and once per step), yet the energy evolution closely tracks the unconstrained case: the $E(t)$ curves for the two runs lie essentially on top of each other.
\end{itemize}

Figure~\ref{fig:toy_A_E_compare_meanzero} summarizes this comparison. The left panel shows $|A(t)|$ with and without the constraint: the unconstrained universe falls and remains in a nearly cancelling configuration, whereas the constrained universe is prevented from entering the disc $|A| < \epsilon$ and instead occupies a minimal-amplitude orbit. The right panel shows that the total energy is almost insensitive to the constraint at this level of coarse description.

\begin{figure}[t]
  \centering
  \includegraphics[width=0.48\textwidth]{toy_v0_1_compare_Amod_epsilon005_meanzero.png}
  \includegraphics[width=0.48\textwidth]{toy_v0_1_compare_energy_epsilon005_meanzero.png}
  \caption{Three-dimensional scalar toy universe on a $16^3$ lattice with mean-subtracted initial data and $\epsilon = 0.05$. \emph{Left:} magnitude of the global amplitude $|A(t)|$ with no constraint (blue) and with the Origin Axiom constraint (orange). The unconstrained run remains in a nearly cancelling configuration with $|A(t)| \approx 0$, while the constrained run is projected onto the boundary $|A| = \epsilon$ and kept there throughout the evolution. \emph{Right:} total energy $E(t)$ for the same runs. Despite $501$ constraint activations in the constrained case, the energy tracks the unconstrained evolution closely, indicating that the gentle additive projection enforces non-cancellation without dramatically disturbing the local dynamics.}
  \label{fig:toy_A_E_compare_meanzero}
\end{figure}

These experiments demonstrate that the non-cancelling principle can be implemented as a consistent global constraint in a simple lattice field theory. The constraint selects a subset of trajectories in configuration space that avoid strictly cancelling global configurations, while leaving the coarse-grained energetic behaviour almost unchanged.

\section{One-Dimensional Twisted Model}

To complement the three-dimensional lattice simulations with an analytically tractable example, we consider a one-dimensional chain of $N$ sites with twisted boundary condition
\begin{equation}
    \Phi_{n+N} = e^{i\theta_\ast} \Phi_n.
\end{equation}
This leads to mode momenta
\begin{equation}
    k_m(\theta_\ast) = \frac{\theta_\ast + 2\pi m}{N}, \qquad m = 0,\dots,N-1,
\end{equation}
and mode frequencies
\begin{equation}
    \omega_m(\theta_\ast) = \sqrt{m_0^2 + 4c^2 \sin^2\!\big(k_m(\theta_\ast)/2\big)}.
\end{equation}
The vacuum energy is then
\begin{equation}
    E_0(\theta_\ast) = \frac12 \sum_{m=0}^{N-1} \omega_m(\theta_\ast).
\end{equation}

For a finite ring with full translation invariance and no boundaries, the set of allowed $k_m$ values is merely \emph{relabeled} when $\theta_\ast$ is shifted. As a consequence, the total vacuum energy $E_0(\theta_\ast)$ is numerically found to be independent of $\theta_\ast$ in this simple model: the computed $\Delta E_0(\theta_\ast) = E_0(\theta_\ast) - E_0(0)$ stays at the level of numerical noise ($\sim 10^{-13}$ in our scans). This is a useful null result: it shows that a global twist can be physically trivial at the level of integrated quantities in highly symmetric settings, and that boundaries, defects, or distinguished scales are needed for $\theta_\ast$ to have observable impact on the vacuum structure.

\section{Discussion and Outlook}

The models presented here should be viewed as a first ``sandbox'' for the Origin Axiom. The three-dimensional lattice implementation demonstrates that a non-cancelling constraint on a global amplitude can be enforced consistently in a local, wave-like field theory. The one-dimensional twisted chain illustrates that, in highly symmetric settings, a global phase twist may leave certain aggregate quantities (such as the total vacuum energy) invariant.

Together, these examples clarify both the promise and the limitations of the current approach. On the one hand, the scalar lattice framework provides a concrete arena in which to explore the dynamical consequences of non-cancelling existence. On the other hand, the insensitivity of $E_0(\theta_\ast)$ in the simplest one-dimensional model suggests that nontrivial effects of $\theta_\ast$ may only arise in the presence of boundaries, defects, or additional structure that break the naive equivalence of twisted and untwisted sectors.

Future work will extend these models to include:
\begin{itemize}
    \item richer microstructures (e.g.\ open chains, defected lattices, or higher-genus graphs),
    \item more systematic parameter scans and scaling studies,
    \item and a closer comparison with continuum limits and known results from quantum field theory in nontrivial topologies.
\end{itemize}
The ultimate goal is to identify whether there exist choices of microstructure, dynamics, and $\theta_\ast$ for which the non-cancelling scalar framework reproduces key qualitative features of our universe.

\bibliographystyle{unsrt}
\bibliography{origin_axiom_refs}

\end{document}
