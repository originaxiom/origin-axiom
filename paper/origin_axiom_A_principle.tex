\documentclass[11pt,a4paper]{article}

\usepackage[margin=2.5cm]{geometry}
\usepackage{amsmath,amssymb,amsfonts}
\usepackage{bm}
\usepackage{hyperref}
\usepackage{physics}

\title{The Origin Axiom:\\
A Non-Cancelling Principle for Physical Configuration Space}
\author{Drit\"ero Mehmetaj}
\date{\today}

\begin{document}
\maketitle

\begin{abstract}
This paper motivates and formulates the \emph{Origin Axiom}, a proposed
structural constraint on the configuration space of the universe. The basic
intuition is that absolute nothingness---a state with no fields, no degrees
of freedom and no possibility of change---is not a coherent element of
physical reality. More generally, global configurations that cancel in such a
way as to be indistinguishable from nothing should not be dynamically
realised.

We express this idea in terms of global complex amplitudes \(A(C)\) assigned
to configurations \(C\), and propose the Origin Axiom as the requirement that
physically realised configurations avoid a small neighbourhood of a special
value \(A_\ast\). In the simplest case \(A_\ast = 0\), this excludes
states with perfect global cancellation. The axiom is agnostic about the
microscopic details of fields and interactions; it is intended as an extra
structural layer atop otherwise standard dynamics.

The present paper is conceptual and programmatic. We:
(i) analyse the incoherence of absolute nothingness and the ubiquity of
cancellation in physics;
(ii) define a general framework for global amplitudes and the associated
non--cancelling rule;
(iii) discuss consistency requirements with known physics; and
(iv) outline concrete realisations in simple toy models.
Companion work implements the axiom in a minimal scalar toy universe and
studies the behaviour of global non--cancelling constraints in explicit
simulations.
\end{abstract}

\section{Motivation: why absolute nothingness is unstable}

Attempts to imagine ``nothing'' often smuggle in more structure than they
intend: an empty space, a dark void, a blank state. All of these already
assume the existence of a background stage with geometric and causal
properties. By contrast, \emph{absolute nothingness} would have to be a state
without space, time, laws, fields or even potentiality. It is not clear that
such an object can be coherently described.

Even if one could write down a candidate ``nothing'' configuration, treating
it as a member of configuration space immediately promotes it to
\emph{something}---an element that can be compared with others and assigned
probabilities. This suggests a tension: the very act of including absolute
nothingness as a possible state undermines its intended role.

We take a strong stance:

\medskip
\noindent
\textbf{Principle.} \emph{Absolute nothingness is not a coherent member of
physical configuration space; existence in some form is the default.}

\medskip

This principle does not tell us which specific universe is realised, but it
rules out the idea that the universe could ever evolve to or from a
state that is in every sense ``no universe at all''. The Origin Axiom builds
on this by sharpening what is meant by ``nothingness'' in terms of global
cancellations.

\section{Cancellation systems and global neutrality}

Many familiar physical systems exhibit cancellations:
\begin{itemize}
  \item electric charge neutrality in macroscopic matter,
  \item destructive interference of waves,
  \item cancellation of positive and negative contributions in path integrals,
  \item gravitationally bound systems with zero total momentum.
\end{itemize}
These are examples of what we call \emph{cancellation systems}: arrangements
in which local degrees of freedom are nontrivial, but some global quantity
(e.g.\ total charge, total field amplitude) vanishes.

In conventional treatments, global neutrality is often benign. A neutral atom
is not ``nothing''; it still has nonzero energy, structure and dynamics.
However, certain theoretical constructions---especially in quantum field
theory and cosmology---lean heavily on the idea that vacuum contributions
might cancel ``exactly enough'' to leave no trace. For example, naive
estimates of vacuum energy can exceed observed values by many orders of
magnitude, and one sometimes appeals to cancellations between sectors.

The Origin Axiom does \emph{not} deny ordinary neutrality. Instead, it singles
out a stricter notion of \emph{global cancellation} in terms of a chosen
complex amplitude \(A(C)\). When \(A(C)\) vanishes exactly, or lies in a
sufficiently small neighbourhood of a distinguished point \(A_\ast\), the
configuration is regarded as suspect: it is too close to the forbidden
``nothing'' state.

The key hypothesis is that there exists at least one such global amplitude
whose near--vanishing is structurally forbidden, in the same sense that some
systems forbid certain topological configurations.

\section{Global amplitudes and configuration space}

Let \(\mathcal{C}\) denote the configuration space of some class of physical
systems, for instance the set of field configurations on a Cauchy surface.
A \emph{global amplitude} is a complex--valued functional
\begin{equation}
  A : \mathcal{C} \to \mathbb{C}
\end{equation}
with the following properties:
\begin{itemize}
  \item \textbf{Linearity or quasi--linearity.} For many examples, \(A\) is
  linear in the fields, e.g.\ an integral or sum of a local function.
  More general quasi--linear forms are also possible.
  \item \textbf{Sensitivity to cancellation.} When local degrees of freedom
  arrange themselves into symmetric or antisymmetric patterns, the resulting
  contributions to \(A\) can cancel.
  \item \textbf{Globality.} \(A\) is not a local density; it depends on the
  configuration as a whole and cannot be inferred from any finite patch.
\end{itemize}

The simplest example, used in our toy models, is the volume sum of a complex
scalar field:
\begin{equation}
  A(C) = \sum_{\mathbf{n}} \Phi_{\mathbf{n}},
\end{equation}
which vanishes when the field configuration is perfectly balanced between
positive and negative contributions.

In general, we allow for a family of possible amplitudes, some of which may
be physically distinguished. We assume there exists at least one such
functional \(A\) for which exact cancellation is forbidden by the Origin
Axiom.

\section{Statement of the Origin Axiom}

We now formulate the axiom in abstract terms. Let \(\mathcal{C}\) be a
configuration space and \(A:\mathcal{C}\to\mathbb{C}\) a chosen global
amplitude. Let \(A_\ast\in\mathbb{C}\) denote a distinguished reference value,
and let \(\epsilon>0\) be a small tolerance.

\medskip
\noindent
\textbf{Origin Axiom (informal).} \emph{Physical reality does not realise
configurations whose global amplitude \(A(C)\) lies in a forbidden
neighbourhood of \(A_\ast\).}

\medskip

Formally, we define the forbidden region
\begin{equation}
  \mathcal{D}_\epsilon(A_\ast) =
  \{ A\in\mathbb{C} \mid |A - A_\ast| < \epsilon \}.
\end{equation}
The Origin Axiom then asserts that the physically realised subset
\(\mathcal{C}_{\text{phys}}\subseteq\mathcal{C}\) satisfies
\begin{equation}
  \forall C\in\mathcal{C}_{\text{phys}}:\quad
  A(C) \notin \mathcal{D}_\epsilon(A_\ast).
\end{equation}

The simplest and most natural choice is \(A_\ast=0\), in which case
configurations with perfect global cancellation of the chosen amplitude are
excluded. The tolerance \(\epsilon\) expresses the idea that even extremely
small neighbourhoods of the origin are structurally disfavoured.

Several remarks are in order:
\begin{itemize}
  \item The axiom is intentionally agnostic about the microscopic dynamics.
  Local equations of motion may be standard; the restriction acts at the
  level of admissible global configurations.
  \item The axiom is also compatible with other conservation laws. The
  amplitude \(A\) need not be conserved; only its near--vanishing is forbidden.
  \item The magnitude of \(\epsilon\) is not fixed here. In concrete models it
  plays the role of a tunable scale of non--cancellation.
\end{itemize}

In companion work, the axiom is implemented by modifying the dynamics so that
whenever the system attempts to enter \(\mathcal{D}_\epsilon(A_\ast)\), it is
projected back to the boundary. This is a convenient numerical realisation;
conceptually, the axiom is a statement about which configurations occur at
all, not about how they are dynamically enforced.

\section{Consistency with known physics}

A structural rule that forbids certain global cancellations must be compatible
with the wealth of empirical evidence for systems that are locally and
globally neutral. We highlight a few consistency requirements.

\subsection{Ordinary neutrality is allowed}

Neutral atoms, neutral plasmas and globally neutral cosmologies must remain
permitted. The Origin Axiom therefore cannot simply outlaw all configurations
with vanishing conserved charges. Rather, it focuses on a particular global
amplitude \(A\) which need not coincide with any familiar conserved quantity.

For example, in the scalar toy universe the amplitude is the volume sum of a
complex field. A neutral atom has zero net electric charge, but its
corresponding scalar amplitude (if such a field exists) may be nonzero. The
axiom leaves all ordinary neutral systems untouched as long as they are not
exactly cancelling with respect to the selected amplitude.

\subsection{Small but nonzero amplitudes}

Observations suggest that many global quantities in our universe are small
but not exactly zero. For instance, the cosmological constant is tiny but
nonvanishing in natural units. The Origin Axiom naturally favours this sort
of situation: global amplitudes are generically pushed away from exact zero
into small but finite values. In concrete models, the parameter \(\epsilon\)
can be thought of as setting such a non--cancellation scale.

\subsection{Symmetry and gauge invariance}

Any proposed amplitude \(A\) must respect relevant symmetries. For example,
if a theory is invariant under gauge transformations or global phase
rotations, the definition of \(A\) should either be gauge invariant or
explicitly tied to a gauge--fixed description. In the toy models, we use a
simple complex scalar with no gauge redundancy, so this issue does not arise.
In more realistic theories, construction of a meaningful \(A\) would require
care.

\section{Toy models and concrete realisations}

The abstract formulation above becomes more tangible in explicit models. The
companion paper on the scalar toy universe studies a complex field on a
discrete three--torus where the global amplitude is simply the lattice sum
of the field values. There the axiom is implemented as a constraint
\(|A|\ge\epsilon\), enforced dynamically as a projection step when needed.

The main lessons from those simulations can be summarised as follows:
\begin{itemize}
  \item In both linear and nonlinear regimes, the constraint successfully
  keeps the universe away from global cancellation. The typical amplitude
  scale is set by \(\epsilon\).
  \item The energy evolution of the system is largely insensitive to the
  presence or absence of the constraint. This suggests that the axiom can be
  treated as a structural selection rule rather than a new local interaction.
  \item In simple one--dimensional twisted scalar models, the total vacuum
  energy is independent of a global twist angle. These cases act as null
  results, indicating that more structure is needed for global phase--like
  parameters to have observable energetic consequences.
\end{itemize}

From the perspective of the Origin Axiom programme, these toy models play two
roles. First, they test whether the non--cancelling rule can be implemented
without leading to obvious instabilities or contradictions. Second, they
provide a sandbox in which different choices of amplitude \(A\), reference
value \(A_\ast\) and tolerance \(\epsilon\) can be explored systematically.

\section{Outlook}

The Origin Axiom is deliberately modest in its formal content: it does not
specify a particular field content, interaction Lagrangian or cosmological
history. Instead, it proposes that the set of physically realised
configurations is a proper subset of the kinematically allowed ones,
selected by the requirement that an appropriate global amplitude never
vanishes.

Several open questions and directions follow:

\begin{itemize}
  \item \textbf{Which amplitude?} Identifying physically natural candidates
  for \(A\) in realistic field theories is a central task. Possibilities
  include weighted integrals over fields, currents or curvature scalars.
  \item \textbf{Relation to vacuum energy.} In what sense could a
  non--cancelling constraint help explain why certain contributions to vacuum
  energy do not cancel exactly? The toy models suggest qualitative parallels
  but no quantitative prediction yet.
  \item \textbf{Quantum formulation.} The present discussion is largely
  classical or semiclassical. A full quantum version of the Origin Axiom
  would require specifying how the forbidden region in amplitude space is
  represented in Hilbert space or path integrals.
  \item \textbf{Observational signatures.} Ultimately, the axiom is only
  interesting if it leads to testable deviations from standard expectations.
  Identifying such signatures---for instance, small residuals that cannot be
  tuned away by conventional symmetries---is an important part of future
  work.
\end{itemize}

In summary, the Origin Axiom is proposed as a structural principle: a simple
but nontrivial restriction on the global structure of physical configuration
space motivated by the incoherence of absolute nothingness. The scalar toy
universe and related models show that such a rule can be implemented in
concrete systems without immediate contradiction. Whether it plays a role in
our actual universe remains an open and intriguing question.

\end{document}
