\documentclass[11pt]{article}

\usepackage[a4paper,margin=2.5cm]{geometry}
\usepackage{amsmath,amssymb,amsfonts}
\usepackage{graphicx}
\usepackage{hyperref}
\usepackage{bm}

\title{Origin Axiom C:\\
The universe as a cancellation system\\
\large Act II: lattice sanity checks and null results}

\author{Drit\"ero Mehmetaj (Biri)\\[4pt]
\small \texttt{originaxiom/origin-axiom (GitHub)}}
\date{\today}

\begin{document}
\maketitle

\begin{abstract}
Papers A and B formulated the Origin Axiom as a structural non--cancellation rule,
implemented it in a minimal complex scalar field on a periodic lattice,
and demonstrated that a hard constraint $\lvert A(C)\rvert\ge \varepsilon$ can operate
without spoiling basic dynamics or energy conservation.
Paper~C reframes the same framework as a \emph{cancellation system}:
a playground where we look explicitly for places the axiom could have gone wrong,
or where it might already be secretly built into ordinary physics.
This Act~II collects the current battery of lattice sanity checks
and records the strongest statement we can honestly make at this stage:
for the class of tests implemented in \texttt{src/} and \texttt{notebooks/},
the non--cancelling rule behaves as a small, controllable perturbation and
does \emph{not} produce dramatic or unstable effects.
The goal of the paper is not to claim success,
but to make the ``null results'' legible and reproducible.
\end{abstract}

\section{Introduction: from axiom to cancellation system}
Paper~A motivates the Origin Axiom from the incoherence of absolute nothingness
and formalises it as a constraint on a global complex amplitude $A(C)$
over configuration space: physically realised configurations avoid a small
neighbourhood of a reference value $A^\ast$ (typically $A^\ast = 0$) and obey
\begin{equation}
  \lvert A(C)\rvert \ge \varepsilon,
\end{equation}
for some non--cancellation scale $\varepsilon>0$.
Paper~B implements this rule in a minimal complex scalar toy universe on a
periodic 3--torus and shows that, for small $\varepsilon$,
the constrained evolution remains numerically well--behaved.

The present work (Paper~C) takes a complementary viewpoint.
Instead of asking ``what phenomena does the axiom explain?''
we first ask the more basic question:
\emph{can we embed the rule into increasingly nontrivial systems without
breaking anything obvious?}
In other words, we treat the framework as a \emph{cancellation system} and
stress--test it with a set of deliberately modest lattice experiments.

Practically, this means:
\begin{itemize}
  \item We construct pairs of simulations: one purely standard (\emph{free}),
        one with a non--cancelling constraint (\emph{constrained})
        applied to the same underlying equations of motion.
  \item We fix parameters so that both runs live in a numerically safe regime.
  \item We measure quantities that would be especially sensitive to any
        hidden bias or instability: vacuum energies, mode frequencies,
        localisation fractions, and energy flow between coupled fields.
  \item We treat strong deviations as \emph{failures} of the current
        implementation and record them honestly.
\end{itemize}

This Act~II documents the tests we have completed so far,
all implemented in the public repository \texttt{originaxiom/origin-axiom}.
Each subsection below maps directly to a script in \texttt{src/}
and an analysis notebook in \texttt{notebooks/};
figures shown here are generated from cached outputs in
\texttt{data/processed/} and mirrored into \texttt{figures/}.

\section{Numerical testbed summary}
We briefly recap the ingredients that are common to all experiments:
\begin{itemize}
  \item Spatial discretisation: one-- or three--dimensional periodic lattices
        with $N$ sites per dimension.
  \item Time evolution: leapfrog / staggered--in--time scheme for scalar fields
        with mass terms and, where indicated, quartic self--interaction.
  \item Non--cancelling rule:
        we monitor a chosen global complex amplitude $A(t)$ and, whenever
        $\lvert A(t)\rvert < \varepsilon$, we project back onto
        the boundary $\lvert A\rvert=\varepsilon$ by a minimal rescaling
        of the relevant degrees of freedom.
  \item Diagnostics: total energy, mode spectra, localisation measures,
        and simple summary statistics of constraint hits.
\end{itemize}

For detailed derivations of the discrete equations of motion and baseline
stability analysis, we refer back to Paper~B.
Here we focus purely on the \emph{comparative} behaviour of free vs.\ constrained runs.

\section{Twisted 1D vacua: looking for hidden $\theta_\star$ structure}
\subsection{Plain twisted chain}
Our first class of tests probes the vacuum energy of a free massive scalar
on a 1D periodic chain with an imposed boundary twist.
The code lives in
\texttt{src/run\_1d\_twisted\_vacuum\_scan.py}, with analysis in
\texttt{notebooks/02\_1d\_twisted\_analysis.py}.

We consider $N=256$ sites, mass $m_0=0.1$, and a twist angle
$\theta_\star\in[0,2\pi]$ implemented as a phase on the boundary link.
For each $\theta_\star$ we diagonalise the lattice Hamiltonian numerically
and compute the (discretised) vacuum energy $E_0(\theta_\star)$.
The Origin Axiom would be in immediate tension with a strongly
$\theta_\star$--dependent ground state, since our framework is designed
to be as agnostic as possible to any micro--choice of ``twist''.

The result is a clean null test:
within numerical precision,
\begin{equation}
  E_0(\theta_\star) \approx \text{const.}
\end{equation}
over the full scan, with fluctuations consistent with the eigensolver tolerance.
The plot \texttt{twisted\_1d\_E0\_vs\_theta.png} shows a flat line;
the corresponding residual $\Delta E(\theta_\star)$ oscillates
around zero at the $10^{-13}$ level.
This reassures us that the lattice itself does not secretly imprint
a preferred twist---exactly as it should.

\subsection{Defected bond}
To stress the system slightly harder we repeat the scan with a single
defect bond of strength $0<\alpha<1$, implemented in
\texttt{src/run\_1d\_defected\_vacuum\_scan.py} and analysed in
\texttt{notebooks/02b\_1d\_defected\_twist\_analysis.py}.
The defect breaks translation invariance and introduces a local scale,
so any microscopic conspiracy between the twist and the defect
would show up here first.

Again, the vacuum energy $E_0(\theta_\star)$ is numerically flat,
with variations at the $10^{-13}$ level.
The non--cancelling rule is \emph{not} active in these tests;
their role is to confirm that our discretisation and numerics do not
accidentally manufacture a $\theta_\star$--dependent vacuum
that could later be misinterpreted as a signature of the axiom.

\section{Mode--by--mode tests: constrained oscillators}
The next step is to ask whether the non--cancelling rule distorts
individual lattice modes in a detectable way.
We couple the rule directly to a chosen Fourier mode in
\texttt{src/run\_mode\_spectrum\_with\_constraint.py}
with analysis in \texttt{notebooks/06\_mode\_spectrum\_analysis.py}.

We initialise a tiny sinusoidal perturbation in mode $k=1$
on a 1D chain and evolve it both with and without a constraint
on the global amplitude $A(t)$, using $\varepsilon=10^{-3}$.
From the time series we extract the dominant frequency
$\omega_{\rm num}$ and compare it to the analytic lattice dispersion relation.

The key observations are:
\begin{itemize}
  \item The free and constrained time series are visually almost indistinguishable;
        their Fourier power spectra peak at the same $\omega_{\rm num}$.
  \item The numerical frequency deviates slightly from the analytic one,
        but the deviation is identical in both runs and is attributable
        to the finite time step and windowing, not to the constraint.
  \item No extra sidebands or secular drifts appear when the non--cancelling
        rule is active at this small $\varepsilon$.
\end{itemize}

Within this limited setup the axiom behaves like a gentle projection
on the global configuration, not as a new dynamical force on individual modes.

\section{Localised bumps in 1D and 3D}
\subsection{1D propagation and quasi--localisation}
We then test whether the constraint can materially change the
spreading of a localised excitation.
The 1D experiment is implemented in
\texttt{src/run\_localized\_bump\_1d.py} and
\texttt{notebooks/07\_localized\_bump\_analysis\_1d.py}.

Initial conditions:
a Gaussian bump of amplitude $A=0.1$ and width $W=10$,
centred on a 1D lattice of $N=512$ sites.
We evolve for $T=2000$ time steps with and without the
non--cancelling rule (again at $\varepsilon=10^{-3}$)
and track the fraction of $\lvert\phi\rvert^2$
contained in a fixed ``central window'' around the origin.

The main findings:
\begin{itemize}
  \item In both runs the bump disperses and re--focuses in a sequence of
        quasi--recurrences set by the dispersion relation.
  \item The localisation fraction as a function of time is very similar
        in the free and constrained cases; the constrained curve
        sits slightly below the free one as expected, since a small
        amount of weight is continually nudged into a uniform background.
  \item No evidence of spontaneous self--trapping or anomalous long--lived
        localisation emerges at these parameters.
\end{itemize}

\subsection{3D bump with self--interaction}
To get closer to the intended ``toy universe'' we repeat the exercise
in 3D with a quartic self--interaction.
The code is in
\texttt{src/run\_localized\_bump\_3d.py} and
\texttt{notebooks/08\_localized\_bump\_analysis\_3d.py}.

Here we evolve a Gaussian bump of amplitude $A\simeq0.3$ and width $W\simeq4$
on a $40^3$ lattice, with mass $m_0=0.5$ and $\lambda_4=1$.
We again compare free vs.\ constrained evolutions and measure
the fraction of $\lvert\phi\rvert^2$ inside a central sphere.

Snapshots of the central $z$--slice (stored as
\texttt{localized\_bump\_3d\_slices.png}) show spherical shells
expanding and interfering in both cases.
The localisation fraction decays in an irregular but correlated way
between free and constrained runs.
At this stage there is no sign that the non--cancelling rule produces
oscillons, solitons, or other exotic long--lived structures on its own.

\subsection{Parameter scan}
To make sure we are not cherry--picking a benign corner of parameter space,
we perform a coarse scan over amplitudes, widths, self--couplings and
non--cancellation scales using
\texttt{src/run\_localized\_bump\_3d\_scan.py} and
\texttt{notebooks/09\_localized\_bump\_scan\_analysis\_3d.py}.

The scan currently covers:
\begin{itemize}
  \item amplitudes $A\in\{0.2,0.3\}$,
  \item widths $W\in\{3,5\}$,
  \item $\lambda_4\in\{0.5,1.0\}$,
  \item $\varepsilon\in\{10^{-3},5\times10^{-3}\}$.
\end{itemize}

For each point we evolve for a fixed time and record the final localisation
fraction in the central sphere, building 2D maps of localisation vs.\ width
and amplitude, with separate panels for free and constrained runs.

Within this coarse grid:
\begin{itemize}
  \item Regions of higher final localisation are shared between free and
        constrained systems; the constraint slightly lowers the absolute
        values but does not introduce qualitatively new behaviour.
  \item There is no obvious parameter island where the constrained system
        localises while the free system disperses completely, or vice versa.
\end{itemize}

\section{Two--field coupling: energy sharing as a probe}
\subsection{Homogeneous coupled fields}
A simple but sensitive test of the axiom is to let it act on a
\emph{combination} of fields instead of a single one.
We therefore consider two real scalar fields $\phi$ and $\chi$ on a 1D lattice,
with masses $m_1$ and $m_2$ and a bilinear coupling $g\,\phi\chi$.
The implementation is in
\texttt{src/run\_two\_field\_coupling\_1d.py} (or equivalently the scan
version in \texttt{src/run\_two\_field\_coupling\_scan.py}),
with analysis in
\texttt{notebooks/10\_two\_field\_coup\-ling\_analysis\_1d.py}.

We initialise a small homogeneous excitation in $\phi$ only and evolve both
the free system and one in which the non--cancelling rule acts on the
\emph{combined} mean field
\begin{equation}
  A(t) \propto \langle \phi(t)\rangle + i \langle \chi(t)\rangle .
\end{equation}

At large $g$ the coupled system becomes unstable and both mean fields
blow up; this is a feature of the underlying discretised equations,
not of the constraint itself, and we treat this as a parameter region
to avoid.
At more modest couplings (e.g.\ $g=0.02$ in the current runs)
the behaviour is much calmer:
\begin{itemize}
  \item Energy oscillates between the two fields as expected from
        coupled harmonic oscillators.
  \item The total energy is conserved to the same level in free and
        constrained runs; the constraint curve sits slightly above
        the free one but without secular drift.
  \item The mean fields remain small and oscillatory;
        constraint hits serve mainly to keep the combined mean away
        from exact cancellation, not to pump or drain energy.
\end{itemize}

\subsection{Two--field localised bump}
Finally we combine localisation and coupling in
\texttt{src/run\_two\_field\_bump\_1d.py} with analysis in
\texttt{notebooks/11\_two\_field\_bump\_analysis\_1d.py}.
We place a localised bump in $\phi$ on a 1D lattice,
leave $\chi$ initially at rest, and couple them with a moderate $g$.

Again we track a localisation fraction (now combining both fields)
in a central window as a function of time.
The free and constrained curves are extremely close;
the constrained system tends to lose a slightly larger fraction of its
central weight as time progresses, consistent with a very mild bias
towards spreading the combined amplitude away from perfect cancellation.

No qualitatively new attractors or long--lived composites appear in
this regime.
From the viewpoint of a cancellation system this is, again,
a null result---but an important one:
the rule can act on a coupled multi--field system without introducing
obvious pathologies.

\section{Discussion and roadmap}
The tests documented in this Act~II all point in the same direction:
within the scalar lattice frameworks explored so far, the non--cancelling
rule behaves as a small, tunable perturbation.
For sufficiently small $\varepsilon$,
\begin{itemize}
  \item basic vacuum properties (including twisted and defected chains)
        are unchanged within numerical precision;
  \item individual modes retain their frequencies and line shapes;
  \item localised bumps disperse and recur in much the same way with and
        without the constraint;
  \item energy sharing in modestly coupled two--field systems proceeds as
        expected, with total energy conserved to comparable accuracy.
\end{itemize}

These are not the spectacular signatures one might secretly hope for,
but they are exactly the kind of groundwork that any serious proposal
must survive.
They also clarify where \emph{not} to look: the simulations here suggest
that if the Origin Axiom has observable consequences, they are unlikely
to appear as wild instabilities or miraculous localisation at the level
of a simple scalar lattice.

The natural next steps, some of which are already sketched in
\texttt{docs/ROADMAP.md}, are:
\begin{itemize}
  \item Extend the cancellation--system viewpoint to sectors that more closely
        resemble realistic matter (multiple fields with different statistics,
        gauge structure, approximate symmetries).
  \item Explore versions of the rule where $A(C)$ is tied to genuinely
        global quantities (e.g.\ phase--twisted sums over sectors) rather
        than simple lattice means.
  \item Investigate whether the constraint can induce small but coherent
        biases in ensembles of initial conditions, potentially relevant to
        vacuum selection or cosmological initial data.
\end{itemize}

In that sense, Act~II is both a conclusion and a beginning:
it closes the loop on the first wave of sanity checks and prepares the
ground for more ambitious tests in which the axiom is allowed to ``touch''
structures closer to the real universe.

\end{document}
