\documentclass[11pt,a4paper]{article}

\usepackage[margin=1in]{geometry}
\usepackage{amsmath,amssymb,amsfonts}
\usepackage{hyperref}
\usepackage{physics}

\title{The Origin Axiom:\\
Non-Cancelling Existence and a Scalar Universe Program}
\author{Drit\"ero Mehmetaj}
\date{\today}

\begin{document}

\maketitle

\begin{abstract}
We propose the \emph{Origin Axiom}: reality exists because absolute nothingness is not a coherent possibility, and a fundamental non-cancelling phase twist $\theta_\ast$ forbids the universe from ever reaching a perfectly cancelling global state. We outline a $\theta_\ast$-agnostic scalar framework and a research program aimed at deriving structural properties of our universe---and eventually constraining $\theta_\ast$ itself---from this single non-cancellation principle. A first toy-universe model on a discrete three-torus, implemented and released as open-source code, provides a concrete testbed for the axiom.
\end{abstract}

\section{Motivation: Why ``Nothing'' Is Not an Option}

When we try to imagine \emph{absolute nothingness}, we usually smuggle in hidden structure: an empty space, a time before and after, a background in which nothing is happening. All of these already presuppose that something exists: dimensions, laws, or at least a space of possible states.

By ``absolute nothingness'' we mean the stronger notion: no space, no time, no fields, no laws, no relations, and no possibilities. As soon as we talk about such a ``nothing'' as a conceivable state---as something reality could have been---we have already made it part of a larger possibility space. It becomes an element in a bigger story (``the universe could have been nothing, but instead it is something''), which contradicts its own definition. In that sense, absolute nothingness is not a coherent element of the space of possibilities. Existence is not a special case inside a wider option set; existence is forced.

However, even if there cannot be a timeless, external ``nothing'', one might imagine a universe that \emph{evolves} into a perfectly cancelling state: every contribution matched by its opposite, leaving a final configuration that is, in effect, nothing. This would re-introduce a version of ``nothingness'' as a reachable point in the dynamics. To avoid this, we strengthen the claim: not only can absolute nothingness not exist ``outside'' reality, but there must be no allowed global state in which reality cancels itself exactly.

\section{The Origin Axiom: Non-Cancelling Existence}

We encode this strengthened requirement as the \emph{Origin Axiom}. Consider a description in which the universe, as a whole, can be associated with a global complex amplitude or phase---a point on the unit circle in the complex plane. Perfect cancellation would correspond to a special phase where reality and its ``negative'' exactly annihilate, analogous to adding $+1$ and $-1$.

\subsection*{Axiom}

\begin{quote}
\textbf{Origin Axiom (Non-Cancelling Existence).}
\begin{enumerate}
    \item \textbf{Impossibility of Nothing.} There is no admissible global state of reality corresponding to absolute nothingness; in particular, there is no physically realizable state whose total amplitude is null.
    \item \textbf{Irreducible Phase Twist.} The obstruction to such a null state can be represented by a fundamental phase $\theta_\ast \in \mathbb{R}$ such that the global configuration of reality can never realize a cancellation condition equivalent to
    \begin{equation}
        e^{i\theta_\ast} = -1,
    \end{equation}
    i.e.\ symbolically, the ``equation''
    \begin{equation}
        e^{i\theta_\ast} + 1 = 0
    \end{equation}
    has no solution within the space of \emph{physically attainable} global configurations.
\end{enumerate}
\end{quote}

Mathematically, the equation $e^{i\theta} + 1 = 0$ of course has solutions $\theta = (2k+1)\pi$ for integers $k$. The point of the axiom is not to deny this, but to state that none of the configurations that would globally realize such a cancelling phase are physically allowed. The forbidden configuration defines a kind of topological ``hole'' in the space of physical states: one can circle around it, approach it, but never occupy it. There is always a residual mismatch, a twist, a non-zero remainder.

In the concrete models considered here, this twist is carried by a complex scalar field $\Phi$ that lives on a simple microstructure. The global phase of $\Phi$ cannot be tuned into a configuration that cancels reality exactly. Local excitations can interfere, cancel approximately, and disperse, but there is no global trajectory in which the universe annihilates itself.

\section{A $\theta_\ast$-Agnostic Scalar Framework}

A central design choice is that the Origin Axiom is \emph{$\theta_\ast$-agnostic}. We do not assume a special numerical value---such as the golden ratio $\varphi$ or $\varphi^\varphi$---from the outset. Instead, we treat $\theta_\ast$ as a free parameter subject only to the non-cancelling constraint and ask which values are compatible with a universe that resembles ours.

The minimal framework consists of:
\begin{itemize}
    \item a simple microstructure (e.g.\ a discrete three-torus),
    \item a single complex scalar field $\Phi$ defined on that microstructure,
    \item local, wave-like dynamics for $\Phi$,
    \item and a global constraint (or equivalent term in the action) implementing the Origin Axiom by excluding a neighbourhood of the cancelling configuration.
\end{itemize}

For each choice of $\theta_\ast$ and local parameters, we can evolve $\Phi$ from suitable initial conditions and measure observables such as the global amplitude, energy, and the existence of long-lived structures. The qualitative question is: which ranges of $\theta_\ast$ yield ``worlds'' that are dynamically rich and structurally non-trivial, rather than trivial or pathological?

\section{The Origin Axiom Research Program}

The research program built around the axiom can be summarized as follows:
\begin{enumerate}
    \item \textbf{Formalization.} State the Origin Axiom sharply and express it as a constraint on an underlying scalar field theory, ideally as a term in the action that encodes a forbidden region in configuration space.

    \item \textbf{Toy Universes.} Construct simple, $\theta_\ast$-agnostic toy universes (such as a complex scalar on a discrete three-torus) in which the axiom is implemented explicitly, and study their dynamics both analytically and numerically.

    \item \textbf{Selection of $\theta_\ast$.} Impose internal consistency requirements (stability of the vacuum, bounded energy, existence of non-trivial excitations) and external requirements (compatibility with a universe that supports complex chemistry, long-lived structure, and a small positive vacuum energy). Investigate whether these constraints significantly narrow the allowed values of $\theta_\ast$.

    \item \textbf{Bridging to Known Physics.} Identify regimes in which the scalar dynamics can be mapped onto effective geometries and gauge structures reminiscent of general relativity and quantum field theory, and check whether the framework can naturally generate ingredients we currently treat as independent (vacuum energy scales, mass hierarchies, decoherence patterns, \dots).

    \item \textbf{Extensions.} If the framework proves viable at the level of fundamental physics, explore how the same non-cancelling logic constrains higher-level systems (thermodynamics, information, social and economic structures).
\end{enumerate}

Each step is designed to be falsifiable. In particular, the program will fail if no choice of microstructure and $\theta_\ast$ yields regimes that resemble known physics, or if the implementation of the axiom inevitably leads to pathologies (e.g.\ runaway instabilities or trivial frozen states).

\section{First Concrete Realization: A Toy Scalar Universe}

As a first concrete realization, we have implemented a toy universe consisting of a complex scalar field on a discrete three-torus with local, wave-like dynamics and a global constraint enforcing the Origin Axiom. The model is described in detail in a companion paper and released as open-source code; it serves as a testbed for exploring:
\begin{itemize}
    \item how the non-cancelling constraint influences global quantities such as the total amplitude and energy;
    \item whether the system exhibits long-lived, structured configurations;
    \item and how the qualitative behaviour depends on $\theta_\ast$.
\end{itemize}

This is not yet a realistic cosmology. Its role is to make the axiom mathematically concrete and to provide a controlled setting in which the consequences of non-cancelling existence can be studied.

\section{Discussion and Outlook}

The Origin Axiom attempts to move one level below the usual dynamical laws of physics. Instead of postulating equations of motion on an assumed stage, it begins from a restriction on what kinds of global states are even admissible: there is no absolute nothing, and there is no dynamical path by which reality cancels itself perfectly. The rest of the structure---spacetime, fields, particles, and effective laws---is conjectured to arise from the simplest scalar implementation of this non-cancellation.

The program is deliberately modest in what it claims now. It does not offer a complete theory of everything, but sketches a path:
\begin{enumerate}
    \item encode non-cancelling existence in a scalar theory;
    \item explore the space of toy universes generated by this axiom;
    \item test whether any of them resemble ours in a meaningful way.
\end{enumerate}

If this fails, the Origin Axiom is falsified as an organizing principle for fundamental physics. If it succeeds even partially, it may offer a new angle on the cosmological constant problem, the emergence of structure, and the deep connection between existence, irreversibility, and meaning.

\bibliographystyle{unsrt}
\bibliography{origin_axiom_refs}

\end{document}
