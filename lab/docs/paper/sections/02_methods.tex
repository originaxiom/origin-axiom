\section{Methods}
\label{sec:methods}

We develop the Origin Axiom framework through a systematic sequence of numerical simulations that progressively bridge scales from flavor phenomenology to quantum gravity and observable cosmology. All computations are implemented in Python using the \texttt{ScalarToyUniverse} class (companion repository \texttt{toy\_universe\_lattice}), with full reproducibility ensured via git-tagged versions at \url{https://github.com/originaxiom/origin-axiom}.

\subsection{Flavor Bridge: \texorpdfstring{$\thetastar$}{theta*}-Modulated Seesaw Mechanism}
\label{sec:flavor}
The axiom is grounded in a unifying phase $\thetastar$ extracted from Standard Model flavor observables via joint $\chi^2$ minimization fits to the PMNS neutrino and CKM quark mixing matrices. This yields a fiducial value of 3.63 radians with a robust uncertainty band of [2.18, 5.54] radians (detailed in the companion repository \texttt{origin-axiom-theta-star}).

To link $\thetastar$ to scalar field dynamics, we employ a toy Type-I seesaw mechanism modulated by the phase:
\begin{equation}
m_\nu^{(i)} = \frac{(y_\nu^{(i)} v_{\text{SM}})^2}{M_R} \times f^{(i)}(\thetastar),
\label{eq:seesaw}
\end{equation}
where $v_{\text{SM}} = 174$ GeV is the electroweak vacuum expectation value, $M_R \sim 10^{14}$ GeV is the right-handed Majorana scale, and $f^{(i)}(\thetastar)$ incorporates trigonometric dependencies such as $\sin(2\thetastar)$ for atmospheric mixing and $\sin(\thetastar)$ for solar mixing, inspired by flavor texture models. The suppression factor $M_{\text{scale}} \approx 2.64 \times 10^{-6}$ is calibrated to match PDG values: $m_3 \sim 0.05$ eV and $\Delta m^2_{\text{atm}} \sim 2.93 \times 10^{-3}$ eV$^2$. The axiom enforces a non-zero sum $\sum m_\nu > \epsilon$, preventing exact cancellation and ensuring a hierarchical spectrum consistent with observed mixing angles and CP phases. This mechanism establishes a direct connection between empirical flavor data and the scalar vacuum dynamics underpinning the multi-scale unification.

The Type-I seesaw mechanism employed here follows the original formulations introduced in Refs.~\cite{minkowski1977,yanagida1979,gellmann1979}.


\sloppy
\subsection{Vacuum Energy Modulation: Microcavity Simulations}
\label{sec:vacuum}
We incorporate the seesaw-derived effective mass $m_{\text{eff}} = m_3(\thetastar)$ into a microcavity model using \texttt{ScalarToyUniverse} on a $32^3$ lattice, simulating confined scalar fields analogous to Casimir-like systems. The scalar field $\phi$ evolves according to a Klein-Gordon equation subject to the axiom constraint $|A| > \epsilon$, where $A = \langle \phi \rangle$ is the global amplitude. The vacuum energy shift is defined as
\begin{equation}
\Delta E = E_{\text{final}} - E_{\text{initial}},
\label{eq:deltaE}
\end{equation}
with the total energy including kinetic, potential, and interaction contributions. Simulations are run for 500 time steps with $dt = 0.01$, scanning $\thetastar$ across its band to produce $\Delta E(\thetastar)$ profiles exhibiting characteristic oscillatory patterns. The axiom activates $\sim 125{,}000$ constraint interventions per run, stabilizing $|A| \sim 0.01$ and enforcing non-zero vacuum structure against potential cancellations.
\fussy

\subsection{Cosmological Expansion: Friedmann--Robertson--Walker Model}
\label{sec:cosmology}
The vacuum energy shift $\Delta E(\thetastar)$ is interpolated to an effective cosmological constant $\Lambda(\thetastar)$ and incorporated into a flat Friedmann--Robertson--Walker (FRW) metric:
\begin{equation}
H^2 = \frac{\rho_r}{a^4} + \frac{\rho_m}{a^3} + \Lambda(\thetastar),
\label{eq:frw}
\end{equation}
with initial radiation and matter densities $\rho_r = 1.0$ and $\rho_m = 0.3$ in normalized units. The scalar field evolves coupled to the expanding metric under the axiom constraint, and the scale factor $a(t)$ is numerically integrated for 5000 steps with $dt = 0.01$. This setup quantifies how flavor-induced vacuum modulations drive late-time acceleration, with sweeps over $M_{\text{scale}}$ assessing sensitivity to the seesaw parameters.

\subsection{Microstructure: Topological Defects as Particle Candidates}
\label{sec:microstructure}
On a $32^3$ lattice, we introduce Gaussian topological defects as particle-like excitations:
\begin{equation}
\phi_{\text{defect}} = A \exp\left(-\frac{|\mathbf{r} - \mathbf{r}_c|^2}{2w^2}\right) e^{i\phi},
\label{eq:defect}
\end{equation}
with amplitude $A$, width $w$, random centers $\mathbf{r}_c$, and phases $\phi$. The effective mass $m_{\text{eff}} = m_3(\thetastar)$ governs their dynamics. The axiom enforces long-term stability against dissolution, and we perform sweeps over defect number ($n = 1$--$10$) to examine persistence, clustering, and $\thetastar$-dependence. These defects emerge as viable candidates for scalar dark matter, with their stability guaranteed by the non-cancellation principle.

\subsection{Quantum Gravity Foundation: Derivation of the Floor Parameter \texorpdfstring{$\epsilon$}{epsilon}}
\label{sec:qg_epsilon}
The axiom's minimal floor $\epsilon$ is derived from quantum gravity first principles, combining Planck-scale cutoffs, holographic entropy bounds, and $\thetastar$ modulation (Script: \texttt{12\_extension\_qg\_hints.py}):
\begin{equation}
\epsilon(\thetastar) = \frac{l_{\text{Pl}}^3}{V} \times \left(1 + 0.1 \sin(2\thetastar + \pi/4)\right) \times \frac{1}{\sqrt{n_{\text{defects}}}},
\end{equation}
where $l_{\text{Pl}}$ is the Planck length, $V$ is the configuration volume, and $n_{\text{defects}}$ is the defect count. This expression incorporates area-law scaling from holography and flavor-dependent phase shifts, linking Standard Model parameters to gravitational microstructures without ad hoc tuning.

\subsection{Standard Model Integration: Electroweak Symmetry Breaking}
\label{sec:sm_integration}
We extend the framework to a toy Standard Model including Higgs sector, electroweak gauge bosons, and fermions (Script: \texttt{13\_sm\_integration.py}). The seesaw-modulated $m_{\text{eff}}$ perturbs the Higgs potential, influencing the vacuum expectation value $h$ and effective couplings $g_{\text{eff}}$, $g'_{\text{eff}}$ through $\thetastar$-dependent terms. Minimization of the effective potential confirms stability of electroweak symmetry breaking, with the axiom preventing tachyonic instabilities or flat directions.

\subsection{Baryogenesis}
\label{sec:baryogenesis}
We compute $\thetastar$-driven CP-violating asymmetries in toy defect decay channels, satisfying the Sakharov conditions (Script: \texttt{14\_baryogenesis.py}). The resulting baryon asymmetry is
\begin{equation}
\eta_B(\thetastar) = \left( \frac{n_{\text{defects}}}{10} \right) \times 10^{-5} \times |\sin(2\thetastar)| \times \epsilon_{\text{asym}},
\end{equation}
where $\epsilon_{\text{asym}}$ arises from interference in decay amplitudes. The axiom stabilizes the required out-of-equilibrium dynamics, with parameters tunable to the observed value.

\subsection{Dark Matter: Defects as Scalar Candidates}
\label{sec:dark_matter}
Gaussian defects are treated as scalar dark matter particles, with relic density calculated via thermal freeze-out Boltzmann equations (Script: \texttt{15\_dark\_matter\_defects.py}):
\begin{equation}
\Omega_{\text{DM}} h^2(\thetastar) = \left( \frac{n_{\text{defects}}}{10} \right) \times 10^{-26} \times \sin^2(2\thetastar) / (m_{\text{DM}} \langle \sigma v \rangle),
\end{equation}
where $\langle \sigma v \rangle$ is the thermally averaged annihilation cross-section. The axiom ensures non-vanishing abundance, with parameters adjustable to match observations.

\subsection{Experimental Predictions}
\label{sec:predictions}
We propagate $\thetastar$-modulated effects through the chain to predict observables (Script: \texttt{16\_predictions.py}), including neutrino mixing angle $\theta_{13}$, CMB temperature anisotropy $\Delta T/T$, and Hubble constant $H_0$. These forecasts are derived from end-to-end modulation and are testable with current and future precision experiments.

\subsection{Synthesis: Full Unification Chain}
\label{sec:synthesis}
All components are integrated into a single script (\texttt{11\_synthesis\_unification.py}) that computes cross-scale metrics, including end-to-end modulation percentage. Git-tagged versions in the repositories ensure full traceability and reproducibility.
