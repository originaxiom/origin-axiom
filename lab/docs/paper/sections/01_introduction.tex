\section{Introduction}
\label{sec:introduction}

\subsection{Philosophical and Physical Motivation}
The question ``Why is there something rather than nothing?'' has persisted across philosophy and physics for millennia. Classical physics permits a stable vacuum state of absolute nothingness, yet quantum mechanics reveals pervasive vacuum fluctuations, suggesting a non-zero baseline energy density. This tension between classical stability and quantum non-triviality motivates the Origin Axiom: the universe is structurally forbidden from perfect global cancellation, enforced on complex scalar amplitudes $A(\mathcal{C})$ over the space of all possible configurations $\mathcal{C}$.

Formally, the axiom imposes a strict lower bound on the modulus:
\begin{equation}
|A(\mathcal{C})| > \epsilon > 0,
\label{eq:axiom_bound}
\end{equation}
where $\epsilon$ is a minimal positive floor, derived from quantum gravity considerations (simulated as $\sim 10^{-12}$ in normalized units for computational purposes). This bound prevents $|A| \to 0$, rendering absolute nothingness impossible.

Philosophically, the axiom implies that existence is not contingent but inherent—a structural bias against nothingness that favors complexity and differentiation. Intellectually, it offers a principled alternative to fine-tuning arguments: rather than invoking anthropic selection or multiverses, the axiom posits that the universe must exhibit non-zero structure at all scales, as enforced by the underlying configuration space geometry.

\subsection{The Unifying Phase \texorpdfstring{$\thetastar$}{theta*}}
To anchor the axiom in empirical physics, we introduce a unifying phase $\thetastar$ extracted from Standard Model flavor observables. $\thetastar$ is determined via joint $\chi^2$ minimization fits to the PMNS neutrino mixing matrix and CKM quark mixing matrix, yielding a fiducial value of 3.63 radians with a robust uncertainty band of $[2.18, 5.54]$ radians (detailed in the companion repository \texttt{origin-axiom-theta-star}).

This phase $\thetastar$ serves as a universal bridge between flavor phenomenology and scalar dynamics. Simulations demonstrate that $\thetastar$-dependent effects propagate consistently across scales—from neutrino masses and vacuum energy modulation to cosmological expansion, microstructure stability, baryogenesis, dark matter relics, and quantum gravity-derived parameters—suggesting a deep underlying unity rooted in the Origin Axiom.

\subsection{Unification Roadmap}
We demonstrate the axiom's unifying power through a systematic chain of simulations:

\begin{itemize}
    \item \textbf{Flavor bridge} (Section \ref{sec:flavor}): $\thetastar$-modulated seesaw mechanism reproduces PDG neutrino masses and mixing.
    \item \textbf{Vacuum shift} (Section \ref{sec:vacuum}): Microcavity models reveal $\sim$2.2\% modulation in energy shift $\Delta E(\thetastar)$.
    \item \textbf{Cosmological expansion} (Section \ref{sec:cosmology}): FRW evolution with $\Lambda(\thetastar)$ yields $\sim$1\% acceleration in scale factor growth.
    \item \textbf{Microstructure} (Section \ref{sec:microstructure}): 3D lattices with defects exhibit $\thetastar$-dependent stability.
    \item \textbf{Quantum gravity foundation} (Section \ref{sec:qg_epsilon}): Derivation of $\epsilon(\thetastar)$ from Planck-scale and holographic principles.
    \item \textbf{Standard Model compatibility} (Section \ref{sec:sm_integration}): Integration with electroweak symmetry breaking and gauge coupling modulation.
    \item \textbf{Baryogenesis} (Section \ref{sec:baryogenesis}): $\thetastar$-driven CP asymmetry yielding baryon asymmetry $\eta_B$.
    \item \textbf{Dark matter} (Section \ref{sec:dark_matter}): Defects as scalar candidates with computed relic density $\Omega_{\text{DM}} h^2$.
    \item \textbf{Experimental predictions} (Section \ref{sec:predictions}): $\thetastar$-modulated observables including $\theta_{13}$, CMB $\Delta T/T$, and $H_0$.
    \item \textbf{Synthesis} (Section \ref{sec:synthesis}): Full chain reveals aligned patterns and emergent phenomena.
\end{itemize}

The framework is fully reproducible via open-source code at \url{https://github.com/originaxiom/origin-axiom} and \url{https://github.com/originaxiom/origin-axiom-theta-star}, with git-tagged versions ensuring traceability.
